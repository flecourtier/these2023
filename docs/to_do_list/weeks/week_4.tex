\textbf{Anciennes tâches :}

\begin{itemize}[label=$\square$]
	\item mettre à jour la documentation antora du stage
	\item ranger code du stage et push github ?
	\item faire sauvegarder sur disque dur tablette et pc fixe
	\item regarder proposition inria pc portable
	\item réinstaller environnement pytorch sur pc fixe (dépend de si je le gardes ?)
\end{itemize}

\textbf{Nouvelles tâches :}

\begin{itemize}[label=$\square$]
	\item Modifier la présentation du stage pour présentation Mimesis \textcolor{red}{$\rightarrow$ 12/12/2023}
	\item[\done] Organisation de la partie Correction avec sauvegarde des images - script qui lance la correction à partir d'un modèle donnée
	\item[\done] Faire récap semaine 3
	\item[\done] Push code pour la Semaine 3 sur github
	\item[\done] Préparer TP3 + cours 3 \textcolor{red}{$\rightarrow$ 27/10/2023}
	\item Lire article 2301.05187 sur les WIRE et 2302.04107
	\item[\done] Remettre en forme la partie excel ("create\_xlsx\_file.py")
	\begin{itemize}[label=\LARGE $\circ$]
		\item[\sdone] ajout des résultats de correction si existe ?
		\item[\sdone] griser les cellules qui sont différentes de la configuration précédente
		\item[\swontfix] génération d'un grand fichier qui regroupe tous les sous fichiers \textcolor{Green}{$\rightarrow$ je pense qu'on ne peut pas créer des feuilles pour Circle puis des sous-feuilles pour Poisson2D\_f..}		
	\end{itemize}
	\item regarder code Killian sur le recalage de la levelset et tester :
	\begin{itemize}[label=\LARGE $\circ$]
		\item sampling de n points sur le bord à une tolérance fixée puis recalage
		\item sampling de n points dans le carré puis recalage
	\end{itemize}
	$\rightarrow$ comparer le nombre d'itération et garder celui qui est le plus rapide
	\item régénération des modèles avec recalage de la levelset $\rightarrow$ utile uniquement dans le cas où on impose pas les conditions au bord de manière exacte
	\item Regarder méthode de Newton (proposé par Emmanuel par mail) et la tester ? - \href{https://www.mathweb.fr/euclide/methode-de-newton/}{Explication}
	\item[\done] faire un suivi hebdomadaire rapide avec les résultats (demandé par Michel)
	\item[\wontfix] récupérer les coordonnées des points au bord de $\Omega_h$ à partir de la sélection de cellule PhiFEM \textcolor{Green}{$\rightarrow$ n'était utile que pour le point suivant}
	\item entraînement du cas test du cercle sur 
	\begin{itemize}[label=\LARGE $\circ$]
		\item[\sdone] le carré tout entier
		\item un carré plus petit (on dirait que les plus grosses erreurs sont au bord du carré)
		\item[\swontfix] $\Omega_h$ - utilisation de MVP présenté dans l'article 2104.08426 pour la génération d'une fonction distance à $\Omega_h$ pour le sampling (ATTENTION : cette fonction distance n'est pas utilisé directement dans la loss du PINNs, elle sert juste à générer le domaine sur lequel on veut entraîner le modèle) \textcolor{Green}{$\rightarrow$ $\Omega_h$ varie en fonction du nombre de noeuds choisis, est-ce qu'on va le fixer ou est-ce qu'il varie ?}
		\item un cercle un peu plus grand (de rayon plus grand) 
	\end{itemize}
	\item essayer d'améliorer l'entraînement du cas du cercle sur le carré tout entier
	\item ajouter excel pour résultats avec recalage levelset
	\item dans le cas des erreurs PhiFEM calculée avec FEniCS, rajouter la projection sur un maillage conforme (maillage qui fit avec le bord, maillage FEM) afin d'avoir des erreurs sur $\Omega$ et pas $\Omega_h$
	\item Pour le script "run\_model.py":
	\begin{itemize}[label=\LARGE $\circ$]
		\item ajouter la possibilité de donner directement un nom de fichier de configuration et pas seulement un numéro ?
		\item vérifier le code (config+args fonctionne ?)
	\end{itemize}
	\item essayer de regarder à nouveau tricontourf pour plot mieux la fonction $\phi$ calculée par MVP sur $\Omega_h$
	\item[\done] vérification du code quand on fait varier $f$ \textcolor{Green}{$\rightarrow$ plage de paramètres donnée en argument de la classe mais pas utilisé}
	\item relancer des modèles avec f paramétrisé par S (car les résultats n'étaient pas bons)
	\item commencer à documenter le code avec sphinx/doxygen
	\item rajouter CI Github pour toute la partie rédaction
	\begin{itemize}[label=\LARGE $\circ$]
		\item correction à faire pour antora sur le rapport de stage (réutilisé ici)
		\item faire une page html à la main comme pour le stage sur lorenz (où y avait sphinx-doxygen-antora) pour pouvoir accèder aux "3 sites" (abstract/results/to\_do\_list (+ documentation des codes))
		\item rajouter la CI au compte Github
	\end{itemize}
	\item[\done] faire résumé résultat \textcolor{red}{$\rightarrow$ MEETING 30/10/2023} 
	\item faire abstract de la semaine 
	\item push tout le code sur github \textbf{vendredi}
\end{itemize}