\subsection*{Anciennes tâches - Semaine 9}

\textbf{Code :}
\begin{itemize}[label=$\square$] 
	\item[\previous{5}] trier les modifs dans le code ScimBa (pour pouvoir valider les issues)
	\item[\done] faire une étude du paramètre $\sigma$ (possibilité de choisir 2 $\sigma$) \textcolor{Green}{\previous{5}}
	\item[\previous{5}] afficher $\Delta \tilde{\phi}$ à la fin de l'entraînement (et comparer avec $f$)
	\item[\previous{5}] tester correction par addition avec IPP
	\item[\wontfix] Tester $\tilde{u}=\tilde{\phi}+\tilde{\phi}\tilde{C}$ au lieu de $\tilde{u}=\tilde{\phi}+\phi\tilde{C}$ pour la correction par addition \textcolor{Green}{\previous{5}}
	\item[\done] Tester sur une forme aléatoire (générée par le code de Killian) \textcolor{Green}{\previous{5}}
	\item[\previous{6}] vérifier les projections faites pour les plots des erreurs (cas FEM standard)
	\item[\previous{6}] (results Fig10+Fig11) : modifier colormap (comme plot dans scimba ?) + fixer l'échelle !
	\item[\previous{6}] merge branche develop dans ma branche (car ajout d'une fonction d'activation supplémentaire qu'il faudra peut-être que je teste)
	\item[\previous{6}] Cas $f$ qui varie : rajouter plus de paramètres (fréquence, phase à l'origine) 
	\item[\previous{6}] Retester avec solveur itératif type gradient conjugué (+ regarder ce qui est fait actuellement)
	\item[\previous{6}] Regarder \url{https://www.youtube.com/watch?v=G_hIppUWcsc} sur les PINNs
	\item[\previous{7}] revoir résultats du stage où $\Omega$ est un carré
	\item[\done] projeter dérivées 2ndes et 1ères sur $\Omega$ \textcolor{Green}{\previous{7}}
	\item[\previous{7}] pour la correction par addition plot : \\
	\begin{minipage}{0.48\linewidth}
		\begin{tabular}[\linewidth]{lc}
			$C_{theorique}$ & $\frac{u_{ex}-u_\theta}{\phi}$ \\
			$C$ & $C_{\phi-FEM}$ \\
			$\tilde{C}_{theorique}$ & $u_{ex}-u_\theta$ \\
			$\tilde{C}$ & $\tilde{C}_{\phi-FEM}=\phi C_{\phi-FEM}$ et $\tilde{C}_{FEM}$
		\end{tabular}
	\end{minipage}
	\begin{minipage}{0.48\linewidth}
		\begin{tabular}[\linewidth]{l}
			dérivées de $C_{theorique}$ \\
			dérivées de $C$ \\
			dérivées de $\tilde{C}_{theorique}$ \\
			dérivées de $\tilde{C}$
		\end{tabular}
	\end{minipage} \\
	sur $\Omega_h$ et projeté sur $\Omega$
	\item[\previous{7}] Pour Correction par addition avec FEM : augmenter le degré de $\tilde{C}$ (P2) et comparer avec FEM où $u$ de plus haut degré aussi (P2) $\rightarrow$ But : voir l'influence du degré sur le facteur
	\item[\previous{7}] Utiliser prédiction de $u_\theta$ sur un maillage conforme puis interpoler sur $\Omega_h$
	\item[\previous{7}] Correction avec $\phi$-FEM : prédiction sur un maillage conforme de $\Omega$ puis interpolation FEniCS sur $\Omega_h$
	\item[\previous{7}] Afficher $f$ et $\tilde{f}=f+\Delta u_\theta$ sur $\Omega$ et sur $\Omega_h$
\end{itemize}
\textbf{CI/Documentation :}
\begin{itemize}[label=$\square$] 
	\item[\previous{6}] Antora : pb avec couleur texte
	\item[\previous{7}] modifier fichier résultat (+antora) pour ordonner les résultats plutôt sous forme de rapport que sous forme de week
	\item[\previous{7}] penser rajouter la figure où on projette la solution sur $\omega$ dans le cas de l'entraînement sur le cercle
	\item[\previous{8}] Github Pages : sommaire pas affiché dans les pdf + pb affichage avec antora (fichier "convert\_latex\_to\_antora" à modifier)
	\item[\previous{8}] Documenter le code python (docstring)
	\item[\previous{8}] rajouter doc sphinx sur github + CI
	\item[\previous{8}] Antora : trier le script python convert\_latex\_to\_antora et le mettre au propre
	\item[\previous{8}] mettre au propre tous les résultats 
	\item[\previous{8}] rajouter : lancement de la CI uniquement quand docs est modifié
\end{itemize}
\textbf{Autre :}
\begin{itemize}[label=$\square$] 
	\item[\previous{4}] faire sauvegarder sur disque dur tablette et pc fixe
	\item[\previous{6}] Regarder formations amethis
	\item[\previous{7}] Rattraper Formation FIDLE - Seq 1
	\item[\previous{8}] Rattraper Formation FIDLE - Seq 2
	\item[\previous{8}] MOOC : intégrité scientifique
	\item[\previous{8}] VPN du bâtiment explora à tester !
\end{itemize}

\subsection*{Nouvelles tâches}

\textbf{Code :}
\begin{itemize}[label=$\square$] 
	\item[\previous{12}] Regarder pour les temps obtenus pour le assemble avec fenics (est-ce qu'il y a un préconditionneur qui est calculé avec le solve ?)
	\item[\previous{12}] Tester entraînement sur cercle plus grand (et pas sur le carré) puis correction avec et sans IPP !
	\item[\previous{12}] Tester entraînement avec loss = forme énergétique du problème de Poisson et correction avec et sans IPP !
	\item[\previous{12}] Rajouter test (où on voit que ça tourne $\Rightarrow$ petit modèle) et exemple dans scimba (où on voit que ça descend)
\end{itemize}
\textbf{CI/Documentation :}
\begin{itemize}[label=$\square$] 
	\item[\done] Finir la présentation pour la retraite Macaron/Tonus
	\item S'entraîner pour la présentation
	\item[\previous{15}] Faire un rapport avec les anciens résultats (jusqu'à la présentation)
\end{itemize}
\textbf{Autre :}
\begin{itemize}[label=$\square$] 
	\item[\textbf{\later}] Regarder VPN
	\item Finir der rattraper dernières formation Fidle
\end{itemize}
\textbf{Hebdomadaire :}
\begin{itemize}[label=$\square$] 
	\item Préparer TP1 + cours 1 \textcolor{red}{$\rightarrow$ 16/02/2021}
	\item faire abstract de la semaine 
	\item push tout le code sur github \textbf{vendredi}
\end{itemize}