\textbf{Code :}
\begin{itemize}[label=$\square$] 
	\item[\done] rajouter argument bash
	\item \textbf{revoir résultats du stage où $\Omega$ est un carré}
	\item afficher solution prédite par le PINNs entraîné sur le cercle où (x,y) est un sampling sur le carré tout entier
	\item projeter dérivées 2ndes et 1ères sur $\Omega$
	\item pour la correction par addition plot : \\
	\begin{minipage}{0.48\linewidth}
		\begin{tabular}[\linewidth]{lc}
			$C_{theorique}$ & $\frac{u_{ex}-u_\theta}{\phi}$ \\
			$C$ & $C_{\phi-FEM}$ \\
			$\tilde{C}_{theorique}$ & $u_{ex}-u_\theta$ \\
			$\tilde{C}$ & $\tilde{C}_{\phi-FEM}=\phi C_{\phi-FEM}$ et $\tilde{C}_{FEM}$
		\end{tabular}
	\end{minipage}
	\begin{minipage}{0.48\linewidth}
		\begin{tabular}[\linewidth]{l}
			dérivées de $C_{theorique}$ \\
			dérivées de $C$ \\
			dérivées de $\tilde{C}_{theorique}$ \\
			dérivées de $\tilde{C}$
		\end{tabular}
	\end{minipage} \\
	sur $\Omega_h$ et projeté sur $\Omega$
	\item Pour Correction par addition avec FEM : augmenter le degré de $\tilde{C}$ (P2) et comparer avec FEM où $u$ de plus haut degré aussi (P2) $\rightarrow$ But : voir l'influence du degré sur le facteur
	\item Utiliser prédiction de $u_\theta$ sur un maillage conforme puis interpoler sur $\Omega_h$
	\item \textbf{\textcolor{green}{noter idées réunion avec Michel}}
\end{itemize}
\textbf{CI/Documentation :}
\begin{itemize}[label=$\square$] 
	\item modifier fichier résultat (+antora) pour ordonner les résultats plutôt sous forme de rapport que sous forme de week
	\item penser rajouter la figure où on projette la solution sur $\omega$ dans le cas de l'entraînement sur le cercle
	\item \textbf{fixer pb CI pour exécution du script pdflatex ?}
\end{itemize}
\textbf{Autre :}
\begin{itemize}[label=$\square$] 
	\item[\done] Lire sujet CC1 (L2S3 Info) et faire des commentaires
	\item[\done] regarder PC bureau
	\item[\done] mail microsoft
	\item Notes formation Fidle
\end{itemize}
\textbf{Hebdomadaire :}
\begin{itemize}[label=$\square$] 
	\item[\wontfix] Préparer Meeting \textcolor{red}{$\rightarrow$ 20/11/2023}
	\item[\done] Préparer TP5 + cours 5 \textcolor{red}{$\rightarrow$ 17/11/2023}
	\item faire abstract de la semaine 
	\item push tout le code sur github \textbf{vendredi}
\end{itemize}