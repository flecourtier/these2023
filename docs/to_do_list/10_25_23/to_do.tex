\documentclass{article}

\usepackage[left=2cm,right=2cm,top=2cm,bottom=2cm]{geometry}
\usepackage{enumitem,amssymb}
\usepackage{pifont}
\newcommand{\done}{\rlap{$\square$}{\raisebox{2pt}{\large\hspace{1pt}\ding{51}}}%
	\hspace{-2.5pt}}
\newcommand{\wontfix}{\rlap{$\square$}{\large\hspace{1pt}\ding{55}}}
\usepackage{xcolor}

\usepackage{hyperref}
\hypersetup{
	colorlinks=true,
	linkcolor=blue,
	filecolor=magenta,      
	urlcolor=cyan,
	pdfpagemode=FullScreen,
}

\begin{document}
	\begin{center}
		\Large\textbf{{TO DO LIST} - 25 Octobre 2023}
	\end{center}
	
	\textbf{Anciennes tâches :}
	
	\begin{itemize}[label=$\square$]
		\item mettre à jour la documentation antora du stage
		\item ranger code du stage et push github ?
		\item faire sauvegarder sur disque dur tablette et pc fixe
		\item regarder proposition inria pc portable
		\item réinstaller environnement pytorch sur pc fixe (dépend de si je le gardes ?)
	\end{itemize}
	
	\textbf{Nouvelles tâches :}
	
	\begin{itemize}[label=$\square$]
		\item Modifier la présentation du stage pour présentation Mimesis \textcolor{red}{$\rightarrow$ 12/12/2023}
		\item Organisation de la partie Correction avec sauvegarde des images - script qui lance la correction à partir d'un modèle donnée
		\item Faire récap semaine 3
		\item Push code pour la Semaine 3 sur github
		\item Préparer TP3 + cours 3 \textcolor{red}{$\rightarrow$ 27/10/2023}
		\item Lire article 2301.05187 sur les WIRE
		\item Remettre en forme la partie excel ("create\_xlsx\_file.py")
		\begin{itemize}
			\item ajout des résultats de correction si existe ?
			\item griser les cellules qui sont différentes de la configuration précédente
			\item génération d'un grand fichier qui regroupe tous les sous fichiers
		\end{itemize}
		\item regarder code Killian sur le recalage de la levelset et tester :
		\begin{itemize}
			\item sampling de n points sur le bord à une tolérance fixée puis recalage
			\item sampling de n points dans le carré puis recalage
		\end{itemize}
		$\rightarrow$ comparer le nombre d'itération et garder celui qui est le plus rapide \\
		+ régénération des modèles avec loss au bord
		\item Regarder méthode de Newton (proposé par Emmanuel par mail) et la tester ? - \href{https://www.mathweb.fr/euclide/methode-de-newton/}{Explication}
		\item faire un suivi hebdomadaire rapide avec les résultats (demandé par Michel)
		\item entraînement du cas test du cercle sur 
		\begin{itemize}
			\item le carré tout entier
			\item $\Omega_h$ - utilisation de MVP présenté dans l'article 2104.08426 pour la génération d'une fonction distance à $\Omega_h$ pour le sampling (ATTENTION : cette fonction distance n'est pas utilisé directement dans la loss du PINNs, elle sert juste à générer le domaine sur lequel on veut entraîner le modèle)
			\item un cercle un peu plus grand (de rayon plus grand) 
		\end{itemize}
		\item dans le cas des erreurs PhiFEM calculée avec FEniCS, rajouter la pprojection sur un maillage conforme (maillage qui fit avec le bord, maillage FEM) afin d'avoir des erreurs sur $\Omega$ et pas $\Omega_h$
		\item Pour le script "run\_model.py":
		\begin{itemize}
			\item ajouter la possibilité de donner directement un nom de fichier de configuration et pas seulement un numéro ?
			\item vérifier le code (config+args fonctionne ?)
		\end{itemize}
		\item essayer de regarder à nouveau tricontourf pour plot mieux la fonction $\phi$ calculée par MVP sur $\Omega_h$
		\item vérification du code quand on fait varier $f$
	\end{itemize}
\end{document}