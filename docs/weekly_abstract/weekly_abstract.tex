\documentclass[french]{article}
\usepackage[T1]{fontenc}
\usepackage[utf8]{inputenc}
\usepackage[french]{babel}
\usepackage{amsmath}
\usepackage{mathtools}
\usepackage{color}
\usepackage[svgnames,dvipsnames]{xcolor} 
\usepackage{soul}
\usepackage{amssymb}
\usepackage{enumitem}
\usepackage{multicol}
\usepackage[left=2cm,right=2cm,top=2cm,bottom=2cm]{geometry}
\newcommand{\mathcolorbox}[2]{\colorbox{#1}{$\displaystyle #2$}}
\usepackage{pifont}
\usepackage{pst-all}
\usepackage{pstricks}
\usepackage{delarray}
\usepackage{setspace}
\usepackage{graphicx}
\usepackage{hyperref}
\usepackage{nicematrix}
\usepackage{listings}
\usepackage{float}

\hypersetup{
	colorlinks=true,
	linkcolor=blue,
	filecolor=magenta,      
	urlcolor=cyan,
	pdfpagemode=FullScreen,
}

\usepackage{amsthm}
\newtheorem*{Rem}{Remarque}

\newenvironment{conclusion}[1]{%
	\begin{center}\normalfont\textbf{Conclusion}\end{center}
	\begin{quotation} #1 \end{quotation}
}{%
	\vspace{1cm}
}

\newcommand\pythonstyle{\lstset{
	language=Python,
	basicstyle=\ttm,
	morekeywords={self},              % Add keywords here
	keywordstyle=\ttb\color{deepblue},
	emph={MyClass,__init__},          % Custom highlighting
	emphstyle=\ttb\color{deepred},    % Custom highlighting style
	stringstyle=\color{deepgreen},
	frame=tb,                         % Any extra options here
	showstringspaces=false
}}

\lstdefinestyle{Cpp}{
	language=C++,
	tabsize=3,
	basicstyle=\ttfamily,
	keywordstyle=\color{blue}\ttfamily,
	stringstyle=\color{red}\ttfamily,
	commentstyle=\color{green}\ttfamily,
	morecomment=[l][\color{magenta}]{\#}
}

\lstdefinestyle{Python}{
	language=Python,
	tabsize=3,
	basicstyle=\ttfamily,
	keywordstyle=\color{blue}\ttfamily,
	stringstyle=\color{red}\ttfamily,
	commentstyle=\color{green}\ttfamily,
	morecomment=[l][\color{magenta}]{\#}
}

\lstset{style=Cpp}

\setlength\parindent{0pt}


\usepackage{fontawesome}

\begin{document}
	LECOURTIER Frédérique \hfill \today
	\begin{center}
		\Large\textbf{{Abstracts}}
	\end{center}

\section{Semaine 1 : 02/10/2023 - 06/10/2023}
	\textbf{Réunions :}
	\begin{enumerate}[label=\textbullet]
		\item \textit{Lundi matin} -  Présentation de Hugo Talbot sur la méthodes des éléments finis
	\end{enumerate}
	\textbf{Fait durant la semaine :}
	\begin{enumerate}[label=\textbullet]
		\item modification du rapport de stage avec les remarques de Michel
		\item lecture de l'article 2104.08426 : "Exact imposition of boundary conditions with distance functions in physics-informed deep neural networks"; lecture jusqu'à la page 23, il ne reste plus que les résultats numérique
		\item reproduction de certains résultats de l'article, notamment : cal:cul de la fonction distance sur un segment et un triangle (2 méthodes)
	\end{enumerate}

	\textbf{A faire :}
	\begin{enumerate}[label=\textbullet]
		\item réécouter vocal réunion et prendre des notes clairs de ce qu'on me demande !
		\item essayer de calculer une distance \textit{signée}
		\item reproduire certains des résultats avec le PINNs présentés dans l'article
		\item récupérer repo git ScimBa et regarder les issues !
	\end{enumerate}

\section{Semaine 2 : 09/10/2023 - 13/10/2023}
	\textbf{Réunions :}
	\begin{enumerate}[label=\textbullet]
		\item \textit{Mardi matin} -  Réunion d'équipe - Preséntation de pablo
		\item \textit{Vendredi matin} - TP d'Informatique L2S3
	\end{enumerate}
	\textbf{Fait durant la semaine :}
	\begin{enumerate}[label=\textbullet]
		\item sampling dans Scimba dans un domaine créé par une fonction distance signée (SD) et sampling sur le bord
		\item entraînement du PINNs à apprendre $u$ et comparaison en apprenant $w$ -> application de la correction par addition avec FEM et $\phi$-FEM sur le cercle
		\item organisation du code :
		\begin{itemize}
			\item création d'un document latex pour expliquer le problème considéré
			\item homogénéisation du code (pas de copies des paramètres, des fonctions...)
			\item création d'un script python qui permette de lancer le PINNs avec différentes configurations (paramètres en arguments, sauvegarde du modèle)
			\item création d'un script python qui permette de créer un tableur qui regroupe toutes les configurations choisies
		\end{itemize} 
	\end{enumerate}
	
	\textbf{A faire :}
	\begin{enumerate}[label=\textbullet]
		\item ajout des images résultats dans le fichier excel (training ?)
		\item organisation de la partie correction avec sauvegarde des images
		\item reproduire certains des résultats avec le PINNs présentés dans l'article ?
		\item continuer lecture article 2104.08426
	\end{enumerate}
\end{document}