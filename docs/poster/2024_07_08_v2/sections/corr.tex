\begin{columns}

    \column{0.5}

    \block{How can we improve PINNs prediction ? \\ \textnormal{\Large Using FEM-type methods}}{
        \vspace{-20pt}
        \begin{center}
            \begin{tcolorbox}[
                colback=color1!50, % Couleur de fond de la boîte
                colframe=color2, % Couleur du cadre de la boîte
                arc=2mm, % Rayon de l'arrondi des coins
                boxrule=2pt, % Épaisseur du cadre de la boîte
                breakable, enhanced jigsaw,
                width=\linewidth
                ]            
                \textbf{Additive approach.} Considering $u_{\theta}$ as the prediction of our PINNs for the Poisson problem, the correction problem consists in writing the solution as
                \begin{equation*}
                    \tilde{u}=u_{\theta}+\underset{\textcolor{orange}{\ll 1}}{\fcolorbox{orange}{color1!50}{$\tilde{C}$}}
                \end{equation*}

                \vspace{-20pt}

                and searching $\tilde{C}: \Omega \rightarrow \mathbb{R}^d$ such that
                \begin{equation}
                    \left\{\begin{aligned}
                        -\Delta \tilde{C}&=\tilde{f}, \; &&\text{in } \Omega, \\
                        \tilde{C}&=0, \; &&\text{on } \Gamma,
                    \end{aligned}\right. \label{corr_add} \tag{$\mathcal{P}^{+}$}
                \end{equation}

                \vspace{-10pt}

                with $\tilde{f}=f+\Delta u_{\theta}$.
            \end{tcolorbox}
    
            \begin{tcolorbox}[
                colback=color1!50, % Couleur de fond de la boîte
                colframe=color2, % Couleur du cadre de la boîte
                arc=2mm, % Rayon de l'arrondi des coins
                boxrule=2pt, % Épaisseur du cadre de la boîte
                breakable, enhanced jigsaw,
                width=\linewidth
                ]            
                \textbf{Poisson problem on Square.}
                
                \vspace{10pt}

                \ding{217} Considering homogeneous Dirichlet BC ($g=0$) and $\Omega=[-0.5\pi,0.5\pi]^2$. \\
                \ding{217} Analytical levelset function : $\quad \phi(x,y)=(x-0.5\pi)(x+0.5\pi)(y-0.5\pi)(y+0.5\pi)$ \\
                \ding{217} Analytical solution :               
                \vspace{-8pt}
                \begin{equation*}
                    u_{ex}(x,y)=\exp\left(-\frac{(x-\mu_1)^2+(y-\mu_2)^2}{2}\right)\sin(2x)\sin(2y)
                \end{equation*} 
                with $\mu_1,\mu_2\in[-0.5,0.5]$ (\textbf{\fcolorbox{color1!50}{color1}{parametric}}). 
            \end{tcolorbox}

            \begin{tcolorbox}[
                colback=color1!50, % Couleur de fond de la boîte
                colframe=color2, % Couleur du cadre de la boîte
                arc=2mm, % Rayon de l'arrondi des coins
                boxrule=2pt, % Épaisseur du cadre de la boîte
                breakable, enhanced jigsaw,
                width=\linewidth
                ]            
                \textbf{Theoretical results.} Considering $u_{\theta}$ as the prediction of our PINNs.

                \hypersetup{citecolor=white}

                \begin{center}
                    \begin{mytheo}{\cite{ours_2024}}{add}
                        We denote $u$ the solution of the Poisson problem and $u_h$ the discrete solution of the correction problem (\ref{corr_add}) with $V_h$ a $\mathbb{P}_k$ Lagrange space. Thus
                        \begin{equation*}
                            || u-u_h ||_0 \lesssim \fcolorbox{orange}{color1!30}{$\frac{|u-u_{\theta}|_{H^{k+1}}}{|u|_{H^{k+1}}}$} h^{k+1} |u|_{H^{k+1}}
                        \end{equation*}
                        \vspace{-5pt}
                        \hspace{485pt} \begin{minipage}{0.2\linewidth}
                            \large \textbf{\textcolor{orange}{$C_{\text{gain}}$}}
                        \end{minipage}
                    \end{mytheo}
                \end{center}

                \hypersetup{citecolor=color2}

                \textit{Remark :} The constant $C_{\text{gain}}$ shows that the closer the prior is to the solution, the lower the error constant associated with the method.
            \end{tcolorbox}
        \end{center}	
    }

    \usecolorstyle{bibStyle}
    \useblockstyle{Default}

	\block{
		\vspace{-40pt}
        \AtNextBibliography{\small}
		\printbibliography[heading=none]
	}

    \usecolorstyle{myColorStyle}
    \useblockstyle{TornOut}

    \column{0.5}

    \block{Results \textnormal{- Improve prediction}}{
        \vspace{-20pt}
        \begin{center}
            \begin{tcolorbox}[
                colback=color1!50, % Couleur de fond de la boîte
                colframe=color2, % Couleur du cadre de la boîte
                arc=2mm, % Rayon de l'arrondi des coins
                boxrule=2pt, % Épaisseur du cadre de la boîte
                breakable, enhanced jigsaw,
                width=\linewidth
                ]            
                \textbf{Theoretical results.} Taking $\mu_1=0.05$, $\mu_2=0.22$.

                \vspace{10pt}

                \begin{center}
                    \pgfimage[width=0.75\linewidth]{images/corr/cvg_case1.png}
                \end{center}
                
                \normalsize
                \textit{Remark :} We note N the number of nodes in each direction of the square \big(Total : $N^2$\big).
            \end{tcolorbox}
    
            \begin{tcolorbox}[
                colback=color1!50, % Couleur de fond de la boîte
                colframe=color2, % Couleur du cadre de la boîte
                arc=2mm, % Rayon de l'arrondi des coins
                boxrule=2pt, % Épaisseur du cadre de la boîte
                breakable, enhanced jigsaw,
                width=\linewidth
                ]            
                \textbf{Gains on error using additive approach.}

                Considering a set of $n_p=50$ parameters : $\left\{\big(\mu_1^{(1)},\mu_2^{(1)}\big),\dots,\big(\mu_1^{(n_p)},\mu_2^{(n_p)}\big)\right\}$.
	
                \vspace{5pt}
                
                \hspace{20pt}\begin{minipage}{0.05\linewidth}
                    \normalsize
                    \flushright
                    \rotatebox[origin=b]{90}{\textbf{Solution $\mathbb{P}_1$}} 
                \end{minipage}
                \begin{minipage}{0.8\linewidth}
                    \centering
                    \pgfimage[width=0.9\linewidth]{images/corr/gains_P1.png}
                \end{minipage} 
            \end{tcolorbox}

            \begin{tcolorbox}[
                colback=color1!50, % Couleur de fond de la boîte
                colframe=color2, % Couleur du cadre de la boîte
                arc=2mm, % Rayon de l'arrondi des coins
                boxrule=2pt, % Épaisseur du cadre de la boîte
                breakable, enhanced jigsaw,
                width=\linewidth
                ]            
                \textbf{Time/error ratio.} Training time for PINNs : $t_{PINNs}\approx 240s$.

                \vspace{10pt}

                \ding{217} \textbf{At a given precision, how long does each method take to solve 1 problem ?}

                \begin{center}
                    \pgfimage[width=0.45\linewidth]{images/corr/time_error_tab.png}
                \end{center}

                \vspace{10pt}

                \ding{217} \textbf{How many parameters $n_p$ to make our method faster than FEM ?}

                \vspace{10pt}

                \begin{minipage}{0.48\linewidth}
                    Total time of Additive approach :
                    \vspace{-5pt}
                    $$Tot_{Add}=t_{PINNs}+n_p t_{Add}$$
                \end{minipage}
                \begin{minipage}{0.48\linewidth}
                    Total time of FEM :
                    \vspace{-5pt}
                    $$Tot_{FEM}=n_p t_{FEM}$$
                \end{minipage}

                \vspace{10pt}

                Let's suppose we want to achieve an \textbf{\fcolorbox{color1!50}{color1}{error of $1e-3$}}.
                
                \vspace{-10pt}

                \begin{equation*}
                    Tot_{Add}<Tot_{FEM} \quad \Rightarrow \quad n_p > \frac{t_{PINNs}}{t_{FEM}-t_{Add}}\approx 5.61\quad  \Rightarrow \quad \fcolorbox{orange}{color1!50}{$n_p=6$}
                \end{equation*}

                \vspace{10pt}
            \end{tcolorbox}
        \end{center}
    }

    % \usecolorstyle{bibStyle}
    % \useblockstyle{Default}

	% \block{
	% 	\vspace{-40pt}
	% 	\footnotesize
	% 	\printbibliography[heading=none]
	% }

\end{columns}