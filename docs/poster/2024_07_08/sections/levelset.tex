\block{How to deal with complex geometry in PINNs ?}{
    \warning \textbf{In practice :} Not so easy ! We need to find \textbf{\fcolorbox{color1!50}{color1}{how to sample in the geometry}}.
    \vspace{10pt}

    \begin{center}
        \begin{minipage}{0.48\linewidth}
            \centering
            \begin{tcolorbox}[
                colback=color1!50, % Couleur de fond de la boîte
                colframe=color2, % Couleur du cadre de la boîte
                arc=2mm, % Rayon de l'arrondi des coins
                boxrule=2pt, % Épaisseur du cadre de la boîte
                breakable, enhanced jigsaw,
                width=\linewidth
                ]            
                \textbf{Approach by levelset.} \textcolor{red}{CITE}
    
                \vspace{10pt}
    
                \begin{minipage}{0.38\linewidth}
                    \centering
                    \pgfimage[width=0.7\linewidth]{images/levelset/levelset.png}
                \end{minipage}
                \begin{minipage}{0.58\linewidth}
                    \textbf{\textit{Advantages :}} \\
                    \ding{217} Sample is easy in this case. \\
                    \ding{217} Allow to impose in hard the BC (no more $J_{\text{bc}}$) :
                    \vspace{-5pt}
                    \begin{equation*}
                        u_\theta(X)=\phi(X)w_\theta(X)+g(X)
                    \end{equation*}
                    with $\phi$ a levelset function and $w_\theta$ a NN.
                \end{minipage}
            \end{tcolorbox}
    
            \begin{tcolorbox}[
                colback=color1!50, % Couleur de fond de la boîte
                colframe=color2, % Couleur du cadre de la boîte
                arc=2mm, % Rayon de l'arrondi des coins
                boxrule=2pt, % Épaisseur du cadre de la boîte
                breakable, enhanced jigsaw,
                width=\linewidth
                ]            
                \textbf{Levelset considered.} A regularized Signed Distance Function (SDF).
    
                \begin{tcolorbox}[
                    colback=color1!30, % Couleur de fond de la boîte
                    colframe=color2, % Couleur du cadre de la boîte
                    arc=2mm, % Rayon de l'arrondi des coins
                    boxrule=0.5pt, % Épaisseur du cadre de la boîte
                    breakable, enhanced jigsaw,
                    width=\linewidth
                    ]
                    
                    \textbf{\fcolorbox{color1!50}{color1}{Eikonal equation.}} If we have a boundary domain $\Gamma$, the SDF is solution to:
                    
                    \begin{minipage}{0.7\linewidth}
                        \hspace{350pt}
                        $\left\{\begin{aligned}
                            &||\nabla\phi(X)||=1, \; X\in\mathcal{O} \\
                            &\phi(X)=0, \; X\in\Gamma \\
                            &\nabla\phi(X)=n, \; X\in\Gamma
                        \end{aligned}\right.$
                    \end{minipage}
                    \begin{minipage}{0.25\linewidth}
                        \centering
                        \pgfimage[width=0.7\linewidth]{images/levelset/points_normals.png}
                    \end{minipage}
                    
                    with $\mathcal{O}$ a box which contains $\Omega$ completely and $n$ the exterior normal to $\Gamma$.
                \end{tcolorbox}
                
                \textbf{How to do that ?} with a PINNs \textcolor{red}{CITE}, by adding the following regularization term
                \vspace{-5pt}
                \begin{equation*}
                    J_{\text{reg}} = \int_\mathcal{O} |\Delta\phi|^2.
                \end{equation*} 
            \end{tcolorbox}
        \end{minipage}	
        \qquad
        \begin{minipage}{0.48\linewidth}
            \centering
            \begin{tcolorbox}[
                colback=color1!50, % Couleur de fond de la boîte
                colframe=color2, % Couleur du cadre de la boîte
                arc=2mm, % Rayon de l'arrondi des coins
                boxrule=2pt, % Épaisseur du cadre de la boîte
                breakable, enhanced jigsaw,
                width=\linewidth
                ]            
                \textbf{Result :} \textbf{\fcolorbox{color1!50}{color1}{Levelset learning.}}
                
                \vspace{2pt}
    
                \begin{minipage}{0.48\linewidth}
                    \centering
                    \pgfimage[width=0.8\linewidth]{images/levelset/cat_levelset_loss.png}
                \end{minipage} 
                \begin{minipage}{0.48\linewidth}
                    \centering
                    \pgfimage[width=0.8\linewidth]{images/levelset/cat_levelset.png}
                \end{minipage} 
            \end{tcolorbox}
    
            \begin{tcolorbox}[
                colback=color1!50, % Couleur de fond de la boîte
                colframe=color2, % Couleur du cadre de la boîte
                arc=2mm, % Rayon de l'arrondi des coins
                boxrule=2pt, % Épaisseur du cadre de la boîte
                breakable, enhanced jigsaw,
                width=\linewidth
                ]            
                \textbf{Result :} \textbf{\fcolorbox{color1!50}{color1}{Poisson on Cat.}}
    
                \ding{217} Solving (\ref{edp}) with $f=1$ (non parametric) and homogeneous Dirichlet BC ($g= 0$). \\
                \ding{217} Looking for $u_\theta = \phi w_\theta$ with $\phi$ the levelset learned.
    
                \vspace{2pt}
    
                \begin{minipage}{0.48\linewidth}
                    \centering
                    \pgfimage[width=0.8\linewidth]{images/levelset/cat_poisson_loss.png}
                \end{minipage} 
                \begin{minipage}{0.48\linewidth}
                    \centering
                    \pgfimage[width=0.9\linewidth]{images/levelset/cat_poisson.png}
                \end{minipage} 
            \end{tcolorbox}
        \end{minipage}
    \end{center}

}

% \node (manote){
\note[rotate=8, width = 11cm, targetoffsetx=26cm, targetoffsety=15cm, roundedcorners=30, linewidth=1pt]{No mesh, so easy to go on complex geometry !}
\node[below left=0cm and 12cm] at (topright) {\includegraphics[width=2cm]{images/levelset/speaking.png}};
% };

% \begin{scope}[shift=(manote.south west), x={($0.1*(manote.south east)-0.1*(manote.south west)$)},
%      y={($0.1*(manote.north west)-0.1*(manote.south west)$)}]

%      \draw[lightgray, step=1](manote.south west) grid (manote.north east);

% \end{scope}

    
