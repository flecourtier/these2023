\begin{frame}[label={lastslide}]{Conclusion}
	\textbf{Current progress :}
	\begin{itemize}[\ding{217}]
		\item Levelset learning works on complex geometries
		
		\textit{Advantage :} enables “exact” imposition of BC in PINNs
		
		\item Additive approach works on simple geometries
		
		\textit{Advantage (compared with standard FEM) :}
		\begin{itemize}[-]
			\item More accurate solution (smaller error) 
			\item Better execution time
		\end{itemize}
	\end{itemize}
	
	\textbf{Perspectives :}
	\begin{itemize}[\ding{217}]
		\item Working on parametric models for Levelset learning
		\item Combine the 2 axis to improve NN predictions on complex geometries \refappendix{frame:cat}
		\item Use $\phi$-FEM (fictitious domain method) to improve NN predictions
		
		\textit{Advantage :} The levelset learned by PINNs can be used in $\phi$-FEM
		\item Start considering 3D cases
	\end{itemize}
\end{frame}

\begin{frame}{Supplementary work I}
	\small
	\vspace{-5pt}
	\begin{tcolorbox}[
		skin=bicolor,
		colback=other, % Couleur de fond de la boîte
		colbacklower=other!20!white,
		title={Teaching at the university},
%		colframe=white, % Couleur du cadre de la boîte
		arc=2mm, % Rayon de l'arrondi des coins
		boxrule=0.5pt, % Épaisseur du cadre de la boîte
		breakable, enhanced jigsaw,
		width=\linewidth,
		opacityback=0.1,
		]
		\begin{itemize}[\textcolor{other}{$\blacktriangleright$}]
			\item 16h of Computer Science Practical Work (Python) - L2S3
			\item 34h of Computer Science Practical Work (C++) - L3S6
		\end{itemize}
	\end{tcolorbox}

	\begin{tcolorbox}[
		skin=bicolor,
		colback=other, % Couleur de fond de la boîte
		colbacklower=other!20!white,
		title={Formations (Total : $\approx 65h$)},
%		colframe=white, % Couleur du cadre de la boîte
		arc=2mm, % Rayon de l'arrondi des coins
		boxrule=0.5pt, % Épaisseur du cadre de la boîte
		breakable, enhanced jigsaw,
		width=\linewidth,
		opacityback=0.1
		]
		
		\begin{itemize}[\textcolor{other}{$\blacktriangleright$}]
			\item "Charte de déontologie des métiers de la Recherche" (OBLIGATORY)
			\item MOOC Bordeaux - "Intégrité scientifique dans les métiers de la recherche" (OBLIGATORY)
			\item "Enseigner et apprendre (public : mission enseignement)"
			\item "Gérer ses ressources bibliographiques avec Zotero"
			\item 3 Workshops on EDP at IRMA
			\item 19 Remote Sessions ($\approx$ 40h) - "Formation Introduction au Deep Leraning" (FIDLE)
		\end{itemize} 
	\end{tcolorbox}
\end{frame}

\begin{frame}{Supplementary work II}
	\small
	\vspace{-10pt}
	
	\begin{tcolorbox}[
		skin=bicolor,
		colback=other, % Couleur de fond de la boîte
		colbacklower=other!20!white,
		title={Talks},
		%		colframe=white, % Couleur du cadre de la boîte
		arc=2mm, % Rayon de l'arrondi des coins
		boxrule=0.5pt, % Épaisseur du cadre de la boîte
		breakable, enhanced jigsaw,
		width=\linewidth,
		opacityback=0.1
		]
		
		\begin{itemize}[\textcolor{other}{$\blacktriangleright$}]
			\item Team meeting (Mimesis) - December 12, 2023 - \href{https://flecourtier.github.io/these2023/these2023/1.0.3/_attachments/presentation/2023_12_12.pdf}{"Development of hybrid finite element/neural network methods to help create digital surgical twins"}
			\item Retreat (Macaron/Tonus) - February 6, 2024 \\
			\href{https://flecourtier.github.io/these2023/these2023/1.0.3/_attachments/presentation/2024_02_06.pdf}{"Mesh-based methods and physically informed learning"}
			\item Exama project, WP2 reunion - March 26, 2024 \\
			\href{https://flecourtier.github.io/these2023/these2023/1.0.3/_attachments/presentation/2024_03_26.pdf}{"How to work with complex geometries in PINNs ?"}
		\end{itemize}
	\end{tcolorbox}

	\vspace{-7pt}

	\begin{tcolorbox}[
		skin=bicolor,
		colback=other, % Couleur de fond de la boîte
		colbacklower=other!20!white,
		title={Publications},
		%		colframe=white, % Couleur du cadre de la boîte
		arc=2mm, % Rayon de l'arrondi des coins
		boxrule=0.5pt, % Épaisseur du cadre de la boîte
		breakable, enhanced jigsaw,
		width=\linewidth,
		opacityback=0.1
		]
		
		\begin{itemize}[\textcolor{other}{$\blacktriangleright$}]
			\item \textbf{Lecourtier}, Victorion, Barucq, Duprez, Faucher, Franck, Lleras, and Michel-Dansac. Enhanced finite element methods
			using neural networks. in progress.
		\end{itemize}
	\end{tcolorbox}

	\vspace{-7pt}
	
	\begin{tcolorbox}[
		skin=bicolor,
		colback=other, % Couleur de fond de la boîte
		colbacklower=other!20!white,
		title={Coming soon...},
		%		colframe=white, % Couleur du cadre de la boîte
		arc=2mm, % Rayon de l'arrondi des coins
		boxrule=0.5pt, % Épaisseur du cadre de la boîte
		breakable, enhanced jigsaw,
		width=\linewidth,
		opacityback=0.1
		]
		
		\begin{itemize}[\textcolor{other}{$\blacktriangleright$}]
			\item July 8 - 12, 2024 - Poster for a Workshop on Scientific Machine Learning (\href{https://irma.math.unistra.fr/~micheldansac/SciML2024/participants.html}{SciML 2024})
		\end{itemize}
	\end{tcolorbox}
\end{frame}

