\section{\appendixname~\theappendixframenumber~: Standard FEM}\labelappendixframe{frame:fem}

\begin{frame}{\appendixname~\theappendixframenumber~: General Idea}
	\textbf{Variational Problem :} $\quad\text{Find } u\in V \; | \; a(u,v)=l(v), \;\forall v\in V$
	
	with $V$ - Hilbert space, $a$ - bilinear form, $l$ - linear form.
	
	\vspace{10pt}
	
	\begin{minipage}[t]{0.76\linewidth}
		\textbf{Approach Problem :} $\quad \text{Find } u_h\in V_h \; | \; a(u_h,v_h)=l(v_h), \;\forall v_h\in V_h$
		
		with $\textbullet$ $u_h\in V_h$ an approximate solution of $u$, 
		
		$\textbullet V_h\subset V, \; dim V_h=N_h<\infty, \; (\forall h>0)$ 
		
		$\Rightarrow$ Construct a piecewise continuous functions space
		\vspace{-5pt}
		\begin{equation*}
			V_h:=P_{C,h}^k=\{v_h\in C^0(\bar{\Omega}), \forall K\in\mathcal{T}_h, {v_h}_{|K}\in\mathbb{P}_k\}
		\end{equation*}
		
		where $\mathbb{P}_k$ is the vector space of polynomials of total degree $\le k$.
	\end{minipage} \hfill \begin{minipage}[t][][b]{0.2\linewidth}
		\vspace{-5pt}
		\centering
		\pgfimage[width=0.9\linewidth]{images/appendix/fem/FEM_triangle_mesh.png}
		
		\footnotesize
		$\mathcal{T}_h = \left\{K_1,\dots,K_{N_e}\right\}$
		
		\tiny
		($N_e$ : number of elements)
	\end{minipage}
	
	\vspace{10pt}
	
	Finding an approximation of the PDE solution $\Rightarrow$ solving the following linear system:
	\begin{equation*}
		AU=b
	\end{equation*}
	with
	\begin{equation*}
		A=(a(\varphi_i,\varphi_j))_{1\le i,j\le N_h}, \quad U=(u_i)_{1\le i\le N_h} \quad \text{and} \quad b=(l(\varphi_j))_{1\le j\le N_h}
	\end{equation*}
	where $(\varphi_1,\dots,\varphi_{N_h})$ is a basis of $V_h$.
\end{frame}
\addtocounter{appendixframenumber}{1}

\section{\appendixname~\theappendixframenumber~:$\phi$-FEM}\labelappendixframe{frame:phifem}

\begin{frame}{\appendixname~\theappendixframenumber~: Problem}
	Let $u=\phi w+g$ such that
	$$\left\{\begin{aligned}
		-\Delta u &= f, \; \text{in } \Omega, \\
		u&=g, \; \text{on } \Gamma, \\
	\end{aligned}\right.$$
	where $\phi$ is the level-set function and $\Omega$ and $\Gamma$ are given by :
	\begin{center}
		\pgfimage[width=0.5\linewidth]{images/appendix/phifem/PhiFEM_level_set.png}
	\end{center}
	The level-set function $\phi$ is supposed to be known on $\mathbb{R}^d$ and sufficiently smooth. \\
	For instance, the signed distance to $\Gamma$ is a good candidate.
	
	\vspace{5pt}
	
	\footnotesize
	\textit{Remark :} Thanks to $\phi$ and $g$, the boundary conditions are respected.
\end{frame}

\begin{frame}{\appendixname~\theappendixframenumber~: Fictitious domain}
	\setstretch{0.5}
	
	\vspace{10pt}
	
	\begin{center}
		\begin{minipage}{0.43\linewidth}
			\centering
			\pgfimage[width=\linewidth]{images/appendix/phifem/PhiFEM_domain.png}
		\end{minipage} \hfill
		\begin{minipage}{0.1\linewidth}
			\centering
			\pgfimage[width=\linewidth]{images/appendix/phifem/PhiFEM_fleche.png} 
		\end{minipage} \hfill
		\begin{minipage}{0.43\linewidth}
			\centering
			\pgfimage[width=\linewidth]{images/appendix/phifem/PhiFEM_domain_considered.png}
		\end{minipage}
	\end{center}
	
	\begin{enumerate}[\ding{217}]
		\item $\phi_h$ : approximation of $\phi$ \\ 
		\item $\Gamma_h=\{\phi_h=0\}$ : approximate boundary of $\Gamma$
		\item $\Omega_h$ : computational mesh
		\item $\partial\Omega_h$ : boundary of $\Omega_h$ ($\partial\Omega_h \ne \Gamma_h$)
	\end{enumerate}	
	
	\footnotesize
	\; \\
	\textit{Remark :} $n_{vert}$ will denote the number of vertices in each direction
\end{frame}

\begin{frame}{\appendixname~\theappendixframenumber~: Facets and Cells sets}
	
	\vspace{15pt}
	
	\begin{center}
		\begin{minipage}{0.48\linewidth}
			\centering
			\pgfimage[width=\linewidth]{images/appendix/phifem/PhiFEM_boundary_cells.png}
		\end{minipage} \hfill
		\begin{minipage}{0.48\linewidth}
			\centering
			\pgfimage[width=\linewidth]{images/appendix/phifem/PhiFEM_boundary_edges.png}
		\end{minipage}
	\end{center}
	
	\begin{enumerate}[\ding{217}]
		\item $\mathcal{T}_h^\Gamma$ : mesh elements cut by $\Gamma_h$
		\item $\mathcal{F}_h^\Gamma$ : collects the interior facets of $\mathcal{T}_h^\Gamma$ \\
		(either cut by $\Gamma_h$ or belonging to a cut mesh element)
	\end{enumerate}
	
	% \begin{minipage}{0.6\linewidth}
		%     \begin{enumerate}[\ding{217}]
			%         \item $\mathcal{T}_h^\Gamma\subset \mathcal{T}_h$ : contains the mesh elements cut by $\Gamma_h$, i.e. 
			%         \begin{equation*}
				%             \mathcal{T}_h^\Gamma=\left\{T\in\mathcal{T}_h:T\cap\Gamma_h\ne\emptyset\right\},
				%         \end{equation*}
			%         \item $\Omega_h^\Gamma$ : domain covered by the $\mathcal{T}_h^\Gamma$ mesh, i.e.
			%         \begin{equation*}
				%             \Omega_h^\Gamma=\left(\cup_{T\in\mathcal{T}_h^\Gamma}T\right)^O
				%         \end{equation*}
			%         \item $\mathcal{F}_h^\Gamma$ : collects the interior facets of $\mathcal{T}_h$ either cut by $\Gamma_h$ or belonging to a cut mesh element, i.e.
			%         \begin{align*}
				%             \mathcal{F}_h^\Gamma=\left\{E\;(\text{an internal facet of } \mathcal{T}_h) \text{ such that }\right. \\
				%             \left. \exists T\in \mathcal{T}_h:T\cap\Gamma_h\ne\emptyset \text{ and } E\in\partial T\right\}
				%         \end{align*}
			%     \end{enumerate}
		% \end{minipage}
\end{frame}

\begin{frame}{\appendixname~\theappendixframenumber~: Poisson problem}
	\textbf{Approach Problem :} Find $w_h\in V_h^{(k)}$ such that 
	$$a_h(w_h,v_h) = l_h(v_h) \quad \forall v_h \in V_h^{(k)}$$
	where
	$$a_h(w,v)=\int_{\Omega_h} \nabla (\phi_h w) \cdot \nabla (\phi_h v) - \int_{\partial\Omega_h} \frac{\partial}{\partial n}(\phi_h w)\phi_h v+\fcolorbox{blue}{white}{$G_h(w,v)$},$$
	$$l_h(v)=\int_{\Omega_h} f \phi_h v + \fcolorbox{blue}{white}{$G_h^{rhs}(v)$} \qquad \qquad \color{blue}\text{Stabilization terms}$$
	and 
	$$V_h^{(k)}=\left\{v_h\in H^1(\Omega_h):v_{h|_T}\in\mathbb{P}_k(T), \; \forall T\in\mathcal{T}_h\right\}.$$
	For the non homogeneous case, we replace
	$$u=\phi w \quad \rightarrow \quad u=\phi w+g$$ 
	by supposing that $g$ is currently given over the entire $\Omega_h$.
\end{frame}

\begin{frame}{\appendixname~\theappendixframenumber~: Stabilization terms}
	\begin{center}
		\centering
		\pgfimage[width=\linewidth]{images/appendix/phifem/PhiFEM_stab_terms.png}
	\end{center}
	\small
	\underline{1st term :} ensure continuity of the solution by penalizing gradient jumps. \\
	$\rightarrow$ Ghost penalty [Burman, 2010] \\
	\underline{2nd term :} require the solution to verify the strong form on $\Omega_h^\Gamma$. \\
	\normalsize
	\textbf{Purpose :} 
	\begin{enumerate}[\ding{217}]
		\item reduce the errors created by the "fictitious" boundary 
		\item ensure the correct condition number of the finite element matrix
		\item restore the coercivity of the bilinear scheme
	\end{enumerate}
\end{frame}
\addtocounter{appendixframenumber}{1}

\section{Other results}

\subsection{Poisson on Bean}\labelappendixframe{frame:bean}

\begin{frame}{\appendixname~\theappendixframenumber~: Learn a levelset}
	\vspace{-10pt}
	\begin{tcolorbox}[
		colback=other, % Couleur de fond de la boîte
		colframe=other, % Couleur du cadre de la boîte
		arc=2mm, % Rayon de l'arrondi des coins
		boxrule=0.5pt, % Épaisseur du cadre de la boîte
		breakable, enhanced jigsaw,
		width=\linewidth,
		opacityback=0.1
		]
		
		If we have a boundary domain $\Gamma$, the SDF is solution to the Eikonal equation:
		
		\begin{minipage}{\linewidth}
			\centering
			$\left\{\begin{aligned}
				&||\nabla\phi(X)||=1, \; X\in\mathcal{O} \\
				&\phi(X)=0, \; X\in\Gamma \\
				&\nabla\phi(X)=n, \; X\in\Gamma
			\end{aligned}\right.$
		\end{minipage}
		
		with $\mathcal{O}$ a box which contains $\Omega$ completely and $n$ the exterior normal to $\Gamma$.
	\end{tcolorbox}
	
	\textbf{How make that ?} with a PINNs \cite{clemot_neural_2023} by \textcolor{orange}{adding a term to regularize}.
	\vspace{-5pt}
	\begin{equation*}
		J_{reg} = \int_\mathcal{O} |\Delta\phi|^2
	\end{equation*}

	\begin{minipage}{0.32\linewidth}
		\centering
		\pgfimage[width=\linewidth]{images/appendix/bean/bean_levelset_loss.png}
	\end{minipage} 
	\begin{minipage}{0.32\linewidth}
		\centering
		\pgfimage[width=\linewidth]{images/appendix/bean/bean_levelset.png}
	\end{minipage} 
	\begin{minipage}{0.32\linewidth}
		\centering
		\pgfimage[width=\linewidth]{images/appendix/bean/bean_levelset_bc.png}
	\end{minipage}
\end{frame}

\begin{frame}{\appendixname~\theappendixframenumber~: Poisson 2D}
	\ding{217} Solving the \textcolor{orange}{Poisson problem} with $f=1$ and homogeneous Dirichlet BC. \\
	\ding{217} Looking for $u_\theta = \phi w_\theta$ with $\phi$ the levelset learned. 
	
	\begin{center}
		\begin{minipage}{0.32\linewidth}
			\textbf{Sampling}
			
			\vspace{-8pt}
			\centering
			\pgfimage[width=0.6\linewidth]{images/appendix/bean/bean_levelset_sampling.png}
		\end{minipage} \qquad 
		\begin{minipage}{0.32\linewidth}
			\centering
			\pgfimage[width=\linewidth]{images/appendix/bean/bean_poisson_loss.png}
		\end{minipage} 
	\end{center}
	
	\begin{center}
		\pgfimage[width=0.8\linewidth]{images/appendix/bean/bean_poisson_sol.png}
	\end{center}
\end{frame}
\addtocounter{appendixframenumber}{1}

\subsection{Additive approach on Cat}\labelappendixframe{frame:cat}

\begin{frame}{\appendixname~\theappendixframenumber~: Add on Cat}
	\begin{center}
		\pgfimage[width=0.9\linewidth]{images/appendix/cat/cat.jpg}
	\end{center}
\end{frame}
\addtocounter{appendixframenumber}{1}

\subsection{Multiplicative approach}\labelappendixframe{frame:mult}

\begin{frame}{\appendixname~\theappendixframenumber~: Multiplicative approach}
	\vspace{-5pt}
	
	\textbf{Correct by multiplying :} Considering $u_{NN}$ as the prediction of our PINNs for (\ref{edp}), we define
	\begin{equation*}
		u_M = u_{NN}+M
	\end{equation*}
	with $M$ a constant chosen so that $u_M>0$, called the enhancement constant.
	
	Thus, the correction problem consists in writing the solution as
	\begin{equation*}
		\tilde{u}=u_M\times\underset{\textcolor{red}{\approx 1}}{\fcolorbox{red}{white}{$\tilde{C}$}}
	\end{equation*}
	
	\begin{minipage}{\linewidth}
		\setstretch{0.5}
		and searching $\tilde{C}: \Omega \rightarrow \mathbb{R}^d$ such that
		\begin{equation*}
			\left\{\begin{aligned}
				L(u_M\tilde{C})&=f, \; &&\text{in } \Omega, \\
				\tilde{C}&=1, \; &&\text{on } \Gamma.
			\end{aligned}\right. \label{corr_add}
		\end{equation*}
	\end{minipage}
\end{frame}
\addtocounter{appendixframenumber}{1}

%\subsection{Some 3D results}\labelappendixframe{frame:results3D}
%
%\begin{frame}{\appendixname~\theappendixframenumber~: Results}
%	TODO
%\end{frame}
%\addtocounter{appendixframenumber}{1}

%\subsection{Degree of PINNs evaluation}\labelappendixframe{frame:degree}
%
%\begin{frame}{\appendixname~\theappendixframenumber~: Results}
%	TODO
%\end{frame}
%\addtocounter{appendixframenumber}{1}