\subsection{Complex geometries}

\begin{frame}{Learn a regular levelset}		
    \vspace{-10pt}
    \hypersetup{
		citecolor=white,
	}

    \begin{mytheo}{\footnotesize\citep{clemot_neural_2023}\normalsize}{fem}
		If we have a boundary domain $\Gamma$, the SDF is solution to the Eikonal equation:
		
		\begin{minipage}{0.7\linewidth}
			\hspace{100pt}
			$\left\{\begin{aligned}
				&||\nabla\phi(X)||=1, \; X\in\mathcal{O} \\
				&\phi(X)=0, \; X\in\Gamma \\
				&\nabla\phi(X)=n, \; X\in\Gamma
			\end{aligned}\right.$
		\end{minipage}
		\begin{minipage}{0.25\linewidth}
			\centering
			\pgfimage[width=0.7\linewidth]{images/newlines/levelset/points_normals.png}
		\end{minipage}
		
		with $\mathcal{O}$ a box which contains $\Omega$ completely and $n$ the exterior normal to $\Gamma$.
	\end{mytheo}

    \hypersetup{
        citecolor=other,
    }

    \vspace{5pt}

    \textbf{Objective:} Move on to complex geometries by using a levelset function to

    \begin{itemize}
        \item Sample points in the domain $\Omega$ for the PINN training.
        \item Impose exactly the boundary condition in PINN \citep{Sukumar_2022}.
    \end{itemize}

    \vspace{5pt}

	\textbf{How to learn a regular levelset ?} with a PINN by \textcolor{orange}{adding a regularization term},
	\vspace{-5pt}
	\begin{equation*}
		J_{reg} = \int_\mathcal{O} |\Delta\phi|^2,
	\end{equation*}
    and a sample of boundary points that considers the \textcolor{orange}{curvature} of $ \Gamma$. \filledstar

    % Curvature
\end{frame}

\begin{frame}{Numerical results}		
    \begin{figure}[!ht] \centering
		\includegraphics[width=\linewidth]{images/newlines/levelset/EikonalBean.png}

        \vspace{10pt}

		\includegraphics[width=\linewidth]{images/newlines/levelset/boundary.png}
	\end{figure}
\end{frame}

\subsection{\filledstar Posteriori error estimates}

\begin{frame}{Adaptive mesh refinement}	
    \textbf{Adaptive refinement loop} using Dorfler marking strategy. \refappendix{frame:amr} %(residual estimator)
    
    \begin{center}
        \textbf{Standard FEM}
        \vspace{2pt}

        \includegraphics[width=0.2\linewidth]{images/newlines/mesh/fem/u_h.png}
        \quad
        \includegraphics[width=0.2\linewidth]{images/newlines/mesh/fem/eta_h.png}
        \quad
        \includegraphics[width=0.17\linewidth]{images/newlines/mesh/fem/marking.png}
        \qquad
        \includegraphics[width=0.17\linewidth]{images/newlines/mesh/fem/refined.png}
    \end{center}

    \vspace{-10pt}
    $\cdots\hspace{1pt}\longrightarrow\hspace{8pt}
    \text{SOLVE}\hspace{18pt}\longrightarrow\hspace{6pt}
    \text{ESTIMATE}\hspace{8pt}\longrightarrow\hspace{14pt}
    \text{MARK}\hspace{14pt}\longrightarrow\hspace{8pt}
    \text{REFINE}\hspace{4pt}\longrightarrow\hspace{1pt}
    \cdots$

    \hspace{45pt}$\text{on }u_h\hspace{55pt}\eta_{res,T}$

    \vspace{8pt}
    \textbf{Local residual estimator (in $L^2$ norm):} Let $T$ be a cell of $\mathcal{T}_h$ .

    \vspace{-8pt}
    $$\eta_{res,T}^2 = h_T^4 \|\Delta u_h + f_h\|_{L^2(T)}^2 + \frac{1}{2} \sum_{E \in \partial T} h_E^2 \|[\nabla u_h\cdot n]\|_{L^2(E)}^2$$
    with $h_\bullet$ the size of $\bullet$ and considering the Poisson problem.

    % Considering the Poisson problem with Dirichlet boundary conditions.

    % (en précisant que c'est le coût du solve qui est le plus important)
\end{frame}

\begin{frame}[noframenumbering]{Adaptive mesh refinement}	
    \textbf{Adaptive refinement loop} using Dorfler marking strategy.
    
    \begin{center}
        \textcolor{red}{\textbf{Additive Approach}}
        \vspace{2pt}

        \includegraphics[width=0.2\linewidth]{images/newlines/mesh/add/p_h.png}
        \quad
        \includegraphics[width=0.2\linewidth]{images/newlines/mesh/add/eta_h_add.png}
        \quad
        \includegraphics[width=0.17\linewidth]{images/newlines/mesh/add/marking_add.png}
        \qquad
        \includegraphics[width=0.17\linewidth]{images/newlines/mesh/add/refined_add.png}
    \end{center}

    \vspace{-10pt}
    $\cdots\hspace{1pt}\longrightarrow\hspace{8pt}
    \text{SOLVE}\hspace{18pt}\longrightarrow\hspace{6pt}
    \text{ESTIMATE}\hspace{8pt}\longrightarrow\hspace{14pt}
    \text{MARK}\hspace{14pt}\longrightarrow\hspace{8pt}
    \text{REFINE}\hspace{4pt}\longrightarrow\hspace{1pt}
    \cdots$

    \hspace{45pt}$\text{on }\textcolor{red}{p_h^+}\hspace{55pt}\eta_{res,T}$

    \vspace{8pt}
    \textbf{Local residual estimator (in $L^2$ norm):} Let $T$ be a cell of $\mathcal{T}_h$ .

    \vspace{-8pt}
    $$\eta_{res,T}^2 = h_T^4 \|\textcolor{red}{\big((\Delta u_\theta)_h + \Delta p_h^+\big) + f_h}\|_{L^2(T)}^2 + \frac{1}{2} \sum_{E \in \partial T} h_E^2 \|\textcolor{red}{[\nabla p_h^+\cdot n]}\|_{L^2(E)}^2$$
    with $h_\bullet$ the size of $\bullet$ and considering the Poisson problem.
\end{frame}

\begin{frame}[noframenumbering]{Adaptive mesh refinement}	
    \textbf{Adaptive refinement loop} using Dorfler marking strategy.
    
    \begin{center}
        \textbf{Additive Approach \textcolor{red}{- No resolution}}
        \vspace{2pt}

        \includegraphics[width=0.2\linewidth]{images/newlines/mesh/addnet/u_theta_h.png}
        \quad
        \includegraphics[width=0.2\linewidth]{images/newlines/mesh/addnet/eta_h_addnet.png}
        \quad
        \includegraphics[width=0.17\linewidth]{images/newlines/mesh/addnet/marking_addnet.png}
        \qquad
        \includegraphics[width=0.17\linewidth]{images/newlines/mesh/addnet/refined_addnet.png}
    \end{center}

    \vspace{-10pt}
    $\cdots\longrightarrow
    \textcolor{red}{\text{INTERPOLATE}}\longrightarrow\hspace{6pt}
    \text{ESTIMATE}\hspace{8pt}\longrightarrow\hspace{14pt}
    \text{MARK}\hspace{14pt}\longrightarrow\hspace{8pt}
    \text{REFINE}\hspace{4pt}\longrightarrow\hspace{1pt}
    \cdots$

    \hspace{55pt}$\textcolor{red}{u_\theta}\hspace{55pt}\eta_{res,T}$

    \vspace{8pt}
    \textbf{Local residual estimator (in $L^2$ norm):} Let $T$ be a cell of $\mathcal{T}_h$ .

    \vspace{-8pt}
    $$\eta_{res,T}^2 = h_T^4 \|\textcolor{red}{(\Delta u_\theta)_h + f_h}\|_{L^2(T)}^2$$
    with $h_\bullet$ the size of $\bullet$ and considering the Poisson problem.
\end{frame}

\begin{frame}{Numerical results}	
    TODO

    Points à travailler :
    \begin{itemize}
        \item Interpoler uniquement les nouveaux points.
        \item Cout du passage sur GPU.
        \item Erreur résiduel ne correspond pas vraiment à l'erreur additive ($u-u_\theta$), travail sur la courbure plutôt quer sur l'erreur résiduelle.
    \end{itemize}

\end{frame}

\subsection{\filledstar Non linear PDEs}