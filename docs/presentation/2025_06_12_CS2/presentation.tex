%% Requires compilation with XeLaTeX or LuaLaTeX
\documentclass[compress,10pt,xcolor={table,dvipsnames},t]{beamer} %aspectratio=169
\usetheme{diapo}
\usepackage{amsmath}
\DeclareMathOperator*{\argmax}{arg\,max}
\DeclareMathOperator*{\argmin}{argmin}
\usepackage{xparse} %for \NewDocumentEnvironment
\usepackage{amssymb}
\usepackage{xcolor}
\usepackage[bottom]{footmisc}
\usepackage{multirow}
\usepackage{setspace}
\usepackage{caption}
\usepackage{array,multirow,makecell}
\usepackage{pifont}
\usepackage{tikz}
\usepackage{paralist}
\usepackage{appendixnumberbeamer}
%\usepackage[style=authoryear,sorting=nyt,doi=false,url=false,maxbibnames=99,date=year]{biblatex}
\usepackage[square]{natbib}
\bibliographystyle{plainnat}
\usepackage{etoolbox}
% box colorée dans équation
\usepackage[most]{tcolorbox}
\usepackage{tikz}
\usepackage{soul}
% pour l'indicatrice
\usepackage{dsfont}
\usepackage{cancel}
\usepackage{booktabs}
\usepackage{bm}
% pour indentation des itemize
% \usepackage{enumitem}

\setcellgapes{1pt}
\setlength{\parindent}{0pt}
\makegapedcells
\newcolumntype{R}[1]{>{\raggedleft\arraybackslash }b{#1}}
\newcolumntype{L}[1]{>{\raggedright\arraybackslash }b{#1}}
\newcolumntype{C}[1]{>{\centering\arraybackslash }b{#1}}
\renewcommand*{\bibfont}{\footnotesize}
\useoutertheme[subsection=false]{miniframes}
\makeatletter
%\patchcmd{\slideentry}{\advance\beamer@xpos by1\relax}{}{}{}
\def\beamer@subsectionentry#1#2#3#4#5{\advance\beamer@xpos by1\relax}%
\makeatother
\setbeamercolor*{mini frame}{fg=bulles,bg=bulles}
\hypersetup{
	colorlinks=true,
	urlcolor=blue,
	citecolor=other,
	linkcolor=title,
}

\title[PhiFEM]{Development of hybrid finite element/neural network methods to help create digital surgical twins}
\subtitle{2nd CSI}

\author{%
    Michel Duprez\inst{1}, 
    Emmanuel Franck\inst{2}, 
    \textbf{Frédérique Lecourtier}\inst{1} and
    Vanessa Lleras\inst{3}
}

\institute{%
	\inst{1} Project-Team MIMESIS, Inria, Strasbourg, France \\
    \inst{2} Project-Team MACARON, Inria, Strasbourg, France \\
    \inst{3} IMAG, University of Montpellier, Montpellier, France
}

\date{June 12, 2025}

\allowbreak

% u_chapeau (chapeau en couleur)
\usepackage{accents}
\newcommand{\uchapeau}[1]{\accentset{\textcolor{red}{\wedge}}{#1}}
\newcommand{\refappendix}[1]{\tikz[baseline=(char.base)]{\node[framednumber] (char) {\hyperlink{#1}{\small \textcolor{white}{Appendix \ref*{#1}}}};}}
% \newcommand{\refsubappendix}[1]{\tikz[baseline=(char.base)]{\node[framednumber] (char) {\hyperlink{#1.maj}{\small \textcolor{white}{Appendix \ref*{#1}}}};}}

\tikzset{
	framednumber/.style={
		draw=appendix,% Couleur de la bordure
		fill=appendix, % Couleur de fond
		rounded corners, % Coins arrondis
		inner sep=2pt,  % Espace intérieur
	}
}

%% numérotation et label des appendix
\newcounter{appendixframenumber}
\setcounter{appendixframenumber}{0}
\newcounter{subappendixframenumber}
\setcounter{subappendixframenumber}{1}

\makeatletter
\newcommand{\labelappendixframe}[1]{%
	\protected@write\@auxout{}{%
		\string\newlabel{#1}{{\theappendixframenumber}{\thepage}}%
	}%
	\hypertarget{#1}{}
}	
\makeatother

% Ce compteur temporaire stockera "x.y" (1.1, 1.2, etc.) pour les sous-appendices
\makeatletter
\newcommand{\labelsubappendixframe}[1]{%
	\edef\@currentlabel{\theappendixframenumber.\thesubappendixframenumber}%
	\label{#1}%
}
\makeatother

\newcommand{\appendixsection}[1]{%
	\addtocounter{appendixframenumber}{1}%
	\section{\appendixname~\theappendixframenumber~: #1}
}

\NewDocumentEnvironment{appendixframe}{mo+b} 
{%
	% Optionnal : noframenumbering
	\IfNoValueTF{#2}
	{\begin{frame}{A\theappendixframenumber.\thesubappendixframenumber~– #1}}
	{\begin{frame}[#2]{A\theappendixframenumber.\thesubappendixframenumber~– #1}}
		#3
	\end{frame}
}{}

% barre en couleur terme dans équation
\newcommand\Ccancel[2][black]{\renewcommand\CancelColor{\color{#1}}\cancel{#2}}

% chifrre romain dans le texte
\makeatletter
\newcommand*{\rom}[1]{\expandafter\@slowromancap\romannumeral #1@}
\makeatother

% warning
\newcommand{\warning}{{\fontencoding{U}\fontfamily{futs}\selectfont\char 49\relax}}

\newcommand{\insertsectionheadSubtitle}{}

\newtcbtheorem{mytheo}{Theorem}{colback=other, % Couleur de fond de la boîte
	colframe=other, % Couleur du cadre de la boîte
	arc=2mm, % Rayon de l'arrondi des coins
	boxrule=0.5pt, % Épaisseur du cadre de la boîte
	breakable, enhanced jigsaw,
	width=\linewidth,
	opacityback=0.1
	}{th}

\newcommand*{\footcite}[1]{\footnote[frame,1]{\citep{#1}}}

% star command
\newcommand{\filledstar}{\textcolor{Goldenrod}{\ding{72}}\hspace{-8pt}\ding{73}\,}

\usepackage{amssymb}
\usepackage{mathtools}
\usepackage{pgfplots}
\usepackage{pgfplotstable}
% \usepackage{filecontents}
\usepackage{datatool}
\usepackage{fp}
\usetikzlibrary{backgrounds}

\pgfplotsset{
    compat=newest,
}
\pgfplotsset{
    smaller labels/.style={
        label style={font=\footnotesize},
        tick label style={font=\footnotesize}
    }
}
\tikzset{font=\small}
\usetikzlibrary{
    fpu,
    fixedpointarithmetic,
    babel,
    external,
    arrows.meta,
    plotmarks,
    positioning,
    angles,
    quotes,
    intersections,
    calc,
    spy,
    decorations.pathreplacing,
    matrix,
    fit,
}
\usepgfplotslibrary{fillbetween}

% Define colors
\definecolor{femcolor}{RGB}{51, 138, 55} %Green (27,158,119)
\definecolor{addcolor}{RGB}{217,95,2} %Orange
\definecolor{addsobcolor}{RGB}{199,39,34} %Red (sob or other)
\definecolor{multcolor3}{RGB}{117,112,179} %Purple 
\definecolor{multcolor100}{RGB}{0,0,0} %Black (+ empty marker)
\definecolor{multcolor0weak}{RGB}{49, 73, 181} %Blue
\definecolor{multcolor0strong}{RGB}{49, 181, 161} %Cyan

% Define line styles according to the method 
% FEM : solid
% Add : dashed
% Mult : dotted

% Define marker styles according to the degree
% P1 : square
% P2 : circle
% P3 : triangle

%________________ error lines (by Ricardo Costa) ________________

% argument 1: slopes (e.g. {4,6})
% argument 2: x position of the bottom left corner
% argument 3: y position of the bottom left corner
% argument 4: x length

\makeatletter

\newcommand{\printslopeinv}[4]{
    \tikzset{fixed point arithmetic}
    % get arguments
    \def\nero@printslope@orderlist{#1}
    \edef\nero@printslope@xpos{#2}
    \edef\nero@printslope@ypos{#3}
    \edef\nero@printslope@width{#4}
    % get points position
    \pgfmathparse{\nero@printslope@xpos+\nero@printslope@width}
    \edef\nero@printslope@px{\pgfmathresult}
    \edef\nero@printslope@py{\nero@printslope@ypos}
    \edef\nero@printslope@qx{\pgfmathresult}
    \edef\nero@printslope@ry{\nero@printslope@ypos}
    \foreach \nero@printslope@order in {#1}{
        \pgfmathparse{
        ((\nero@printslope@px/\nero@printslope@xpos)^(\nero@printslope@order))*\nero@printslope@ypos}
        \edef\nero@printslope@qy{\pgfmathresult}
            \edef\nero@aux1{\noexpand\draw[line width=0.6pt]
            (axis cs:\nero@printslope@xpos,\nero@printslope@ypos)
            -- (axis cs:\nero@printslope@qx,\nero@printslope@qy)
            -- (axis cs:\nero@printslope@px,\nero@printslope@py);}
        \nero@aux1
        % slope label
        \pgfmathparse{10^((ln(\nero@printslope@ry)+ln(\nero@printslope@qy))/(ln(10)*2))}
        \edef\nero@printslope@labelpos{\pgfmathresult}
        \edef\nero@aux2{\noexpand\node[anchor=west] at
            (axis cs:\nero@printslope@qx,\nero@printslope@labelpos)
            {\noexpand\tiny \nero@printslope@order};}
        \nero@aux2
        \global\edef\nero@printslope@ry{\nero@printslope@qy}
    }
    % base line
    \draw[line width=0.6pt] (axis cs:\nero@printslope@xpos,\nero@printslope@ypos)
        |- (axis cs:\nero@printslope@px,\nero@printslope@py);
    % label of base line
    \pgfmathparse{10^((ln(\nero@printslope@px)+ln(\nero@printslope@xpos))/(ln(10)*2))}
    \edef\nero@printslope@labelpos{\pgfmathresult}
    \node[anchor=north] at (axis cs:\nero@printslope@labelpos,\nero@printslope@ypos) {\tiny 1};
}

\makeatother

\newlength{\plotwidth}
\setlength{\plotwidth}{0.54\textwidth}
\newlength{\plotheight}
\setlength{\plotheight}{0.4\textwidth}

\gdef\iterator{0}

\newenvironment{cvgh}[4]{
    \begin{tikzpicture}
        \edef\filename{#1}
        \edef\legendcolumns{#2}
        \edef\slopes{#3}
        \edef\ypos{#4}

        % Read the CSV file into a table
        \pgfplotstableread[col sep=comma]{\filename}\datatable

        % Obtenir le second élément
        \pgfmathtruncatemacro{\secondrow}{1} % Index de la dernière ligne
        \pgfplotstablegetelem{\secondrow}{h}\of\datatable
        \pgfmathsetmacro{\second}{\pgfplotsretval} % Dernière valeur de h_rounded

        % Obtenir le premier élément
        \pgfmathtruncatemacro{\firstrow}{0} % Index de l'avant-dernière ligne
        \pgfplotstablegetelem{\firstrow}{h}\of\datatable
        \pgfmathsetmacro{\first}{\pgfplotsretval} % Avant-dernière valeur de h_rounded

        % Calculer la différence entre les deux
        \pgfmathsetmacro{\diff}{\first - \second}

        %update iterator
        \pgfmathtruncatemacro{\iterator}{\iterator+1}

        \begin{loglogaxis}[
            smaller labels,
            name = left_plot,
            % axis lines
            axis lines = left,
            enlarge x limits={abs=10pt},
            enlarge y limits={abs=10pt},
            % axis x line shift = -5pt,
            axis y line shift = -5pt,
            % labels
			xmode=log,
            xlabel = {$h$},
            ylabel = {\rotatebox{270}{$L^2$}},
            xlabel style={at={(ticklabel* cs:1.01)},anchor=west},
            ylabel style={at={(ticklabel* cs:1.01)},anchor=west},
            % ticks and labels
            xtick=data,
            xticklabels from table={\datatable}{h},
            width=\plotwidth, height=\plotheight,
            mark options={solid, scale=1},
            grid = major,
            legend columns=\legendcolumns,
            legend to name=leg:legendFEMCORR_\iterator,
            legend image post style={mark options={solid, scale=1},xscale=0.8},
        ]
        \expandafter\printslopeinv\expandafter{\slopes}{\second}{\ypos}{\diff}
    }
    {
        \end{loglogaxis}
        \node[yshift=-20pt] at (left_plot.outer south) {\pgfplotslegendfromname{leg:legendFEMCORR_\iterator}};

    \end{tikzpicture}
}

\newenvironment{cvghline}[4]{
    \begin{tikzpicture}
        \edef\filename{#1}
        \edef\legendcolumns{#2}
        \edef\slopes{#3}
        \edef\ypos{#4}

        % Read the CSV file into a table
        \pgfplotstableread[col sep=comma]{\filename}\datatable

        % Obtenir le second élément
        \pgfmathtruncatemacro{\secondrow}{1} % Index de la dernière ligne
        \pgfplotstablegetelem{\secondrow}{h}\of\datatable
        \pgfmathsetmacro{\second}{\pgfplotsretval} % Dernière valeur de h_rounded

        % Obtenir le premier élément
        \pgfmathtruncatemacro{\firstrow}{0} % Index de l'avant-dernière ligne
        \pgfplotstablegetelem{\firstrow}{h}\of\datatable
        \pgfmathsetmacro{\first}{\pgfplotsretval} % Avant-dernière valeur de h_rounded

        % Calculer la différence entre les deux
        \pgfmathsetmacro{\diff}{\first - \second}

        %update iterator
        \pgfmathtruncatemacro{\iterator}{\iterator+1}

        \begin{loglogaxis}[
            smaller labels,
            name = left_plot,
            % axis lines
            axis lines = left,
            enlarge x limits={abs=10pt},
            enlarge y limits={abs=10pt},
            % axis x line shift = -5pt,
            axis y line shift = -5pt,
            % labels
			xmode=log,
            xlabel = {$h$},
            ylabel = {\rotatebox{270}{$L^2$}},
            xlabel style={at={(ticklabel* cs:1.01)},anchor=west},
            ylabel style={at={(ticklabel* cs:1.01)},anchor=west},
            % ticks and labels
            xtick=data,
            xticklabels from table={\datatable}{h},
            width=\plotwidth, height=\plotheight,
            mark options={solid, scale=1},
            grid = major,
            legend columns=\legendcolumns,
            legend to name=leg:legendFEMCORR_\iterator,
            legend image post style={mark options={solid, scale=1},xscale=0.8},
        ]
        \expandafter\printslopeinv\expandafter{\slopes}{\second}{\ypos}{\diff}
    }
    {
        \end{loglogaxis}
        \node[yshift=-20pt] at (left_plot.outer south) {\pgfplotslegendfromname{leg:legendFEMCORR_\iterator}};
        
        \draw[black, line width=0.4mm] (0.1,1.9) -- (4.7,1.9) node[anchor=west, xshift=2pt] {$e$};

    \end{tikzpicture}
}

\newcommand{\cvgFEMCorrAlldeg}[3]{
    \edef\fem{#1}
    \edef\add{#2}

    \begin{cvgh}{\fem}{3}{2,3,4}{#3}
        % Complete the legend
        \addlegendentry{\,FEM $\mathbb{P}_1$\;}
        \addlegendentry{\,FEM $\mathbb{P}_2$\;}
        \addlegendentry{\,FEM $\mathbb{P}_3$\;}
        \addlegendentry{\,Add $\mathbb{P}_1$\;}
        \addlegendentry{\,Add $\mathbb{P}_2$\;}
        \addlegendentry{\,Add $\mathbb{P}_3$\;}

        % Plot FEM
        \addplot [style={solid}, mark=square*, mark size=2, color=femcolor, line width=0.8pt ]
        table [x=h, y=P1, col sep=comma]
            {\fem};
        
        \addplot [style={solid}, mark=*, mark size=2, color=femcolor, line width=0.8pt ]
        table [x=h, y=P2, col sep=comma]
            {\fem};
        
        \addplot [style={solid}, mark=triangle*, mark size=2, color=femcolor, line width=0.8pt ]
        table [x=h, y=P3, col sep=comma]
            {\fem};

        % Plot Add
        \addplot [style={dashed}, mark=square*, mark size=2, color=addcolor, line width=0.8pt ]
        table [x=h, y=P1, col sep=comma]
            {\add};

        \addplot [style={dashed}, mark=*, mark size=2, color=addcolor, line width=0.8pt ]
        table [x=h, y=P2, col sep=comma]
            {\add};

        \addplot [style={dashed}, mark=triangle*, mark size=2, color=addcolor, line width=0.8pt ]
        table [x=h, y=P3, col sep=comma]
            {\add};

    \end{cvgh}
}

\newcommand{\cvgFEMCorrAlldegLine}[3]{
    \edef\fem{#1}
    \edef\add{#2}

    \begin{cvghline}{\fem}{3}{2,3,4}{#3}
        % Complete the legend
        \addlegendentry{\,FEM $\mathbb{P}_1$\;}
        \addlegendentry{\,FEM $\mathbb{P}_2$\;}
        \addlegendentry{\,FEM $\mathbb{P}_3$\;}
        \addlegendentry{\,Add $\mathbb{P}_1$\;}
        \addlegendentry{\,Add $\mathbb{P}_2$\;}
        \addlegendentry{\,Add $\mathbb{P}_3$\;}

        % Plot FEM
        \addplot [style={solid}, mark=square*, mark size=2, color=femcolor, line width=0.8pt ]
        table [x=h, y=P1, col sep=comma]
            {\fem};
        
        \addplot [style={solid}, mark=*, mark size=2, color=femcolor, line width=0.8pt ]
        table [x=h, y=P2, col sep=comma]
            {\fem};
        
        \addplot [style={solid}, mark=triangle*, mark size=2, color=femcolor, line width=0.8pt ]
        table [x=h, y=P3, col sep=comma]
            {\fem};

        % Plot Add
        \addplot [style={dashed}, mark=square*, mark size=2, color=addcolor, line width=0.8pt ]
        table [x=h, y=P1, col sep=comma]
            {\add};

        \addplot [style={dashed}, mark=*, mark size=2, color=addcolor, line width=0.8pt ]
        table [x=h, y=P2, col sep=comma]
            {\add};

        \addplot [style={dashed}, mark=triangle*, mark size=2, color=addcolor, line width=0.8pt ]
        table [x=h, y=P3, col sep=comma]
            {\add};

    \end{cvghline}
}

\newcommand{\cvgFEMCorrMultOnedeg}[6]{
    \edef\fem{#1}
    \edef\femsec{#2}
    \edef\add{#3}
    \edef\mult{#4}
    \edef\multHundred{#5}

    \begin{cvgh}{\fem}{3}{2,3}{#6}
        % Complete the legend
        \addlegendentry{\,FEM $\mathbb{P}_1$\;}
        \addlegendentry{\,Mult $\mathbb{P}_1$ (M=3)\;}
        \addlegendentry{\,Add $\mathbb{P}_1$\;}
        \addlegendentry{\,FEM $\mathbb{P}_2$\;}
        \addlegendentry{\,Mult $\mathbb{P}_1$ (M=100)\;}

        % Plot the data
        \addplot [style={solid}, mark=square*, mark size=2, color=femcolor, line width=0.8pt ]
        table [x=h, y=err, col sep=comma]
            {\fem};

        \addplot [style={dotted}, mark=square*, mark size=2, color=multcolor3, line width=1.0pt ]
        table [x=h, y=err, col sep=comma]
            {\mult};

        \addplot [style={dashed}, mark=square*, mark size=2, color=addcolor, line width=0.8pt ]
        table [x=h, y=err, col sep=comma]
            {\add};

        \addplot [style={solid}, mark=*, mark size=2, color=femcolor, line width=0.8pt ]
        table [x=h, y=err, col sep=comma]
            {\femsec};

        \addplot [style={dotted}, mark=square, mark size=2, color=multcolor100, line width=1.0pt ]
        table [x=h, y=err, col sep=comma]
            {\multHundred};
    \end{cvgh}
}
\documentclass{article}

\usepackage{amssymb}
\usepackage{mathtools}
%\usepackage[scale=0.8]{geometry}
\usepackage{pgfplots}
\usepackage{pgfplotstable}
\usepackage{filecontents}
\usepackage{datatool}
\usepackage{fp}
\pgfplotsset{
    compat=newest,
}
\pgfplotsset{
    smaller labels/.style={
        label style={font=\footnotesize},
        tick label style={font=\footnotesize}
    }
}
\tikzset{font=\small}
\usetikzlibrary{
    fpu,
    fixedpointarithmetic,
    babel,
    external,
    arrows.meta,
    plotmarks,
    positioning,
    angles,
    quotes,
    intersections,
    calc,
    spy,
    decorations.pathreplacing,
    matrix,
    fit,
}
\usepgfplotslibrary{fillbetween}

\definecolor{graph_1}{RGB}{117,112,179}
\definecolor{graph_2}{RGB}{217,95,2}
\definecolor{graph_3}{RGB}{27,158,119}
\definecolor{graph_4}{RGB}{231,41,138}
\definecolor{fill_topo}{RGB}{191,191,191}


%________________ error lines (by Ricardo Costa) ________________

% argument 1: slopes (e.g. {4,6})
% argument 2: x position of the bottom left corner
% argument 3: y position of the bottom left corner
% argument 4: x length

\makeatletter

% print slope on graphic
\newcommand{\printslope}[4]{
   \tikzset{fixed point arithmetic}
   % get arguments
   \def\nero@printslope@orderlist{#1}
   \edef\nero@printslope@xpos{#2}
   \edef\nero@printslope@ypos{#3}
   \edef\nero@printslope@width{#4}
   % get points position
   \pgfmathparse{\nero@printslope@xpos+\nero@printslope@width}
   \edef\nero@printslope@px{\pgfmathresult}
   \edef\nero@printslope@py{\nero@printslope@ypos}
   \edef\nero@printslope@qx{\nero@printslope@xpos}
   \edef\nero@printslope@ry{\nero@printslope@ypos}
   \foreach \nero@printslope@order in {#1}{
      \pgfmathparse{
      ((\nero@printslope@px/\nero@printslope@xpos)^(\nero@printslope@order))*\nero@printslope@ypos}
      \edef\nero@printslope@qy{\pgfmathresult}
      % print slope line
      \edef\nero@aux1{\noexpand\draw[line width=0.6pt]
         (axis cs:\nero@printslope@xpos,\nero@printslope@ry)
         -- (axis cs:\nero@printslope@qx,\nero@printslope@qy)
         -- (axis cs:\nero@printslope@px,\nero@printslope@py);}
      \nero@aux1
      % slope label
      \pgfmathparse{10^((ln(\nero@printslope@ry)+ln(\nero@printslope@qy))/(ln(10)*2))}
      \edef\nero@printslope@labelpos{\pgfmathresult}
      \edef\nero@aux2{\noexpand\node[anchor=east] at
         (axis cs:\nero@printslope@qx,\nero@printslope@labelpos)
         {\noexpand\tiny \nero@printslope@order};}
      \nero@aux2
      \global\edef\nero@printslope@ry{\nero@printslope@qy}
   }
   % base line
   \draw[line width=0.6pt] (axis cs:\nero@printslope@xpos,\nero@printslope@ypos)
      |- (axis cs:\nero@printslope@px,\nero@printslope@py);
   % label of base line
   \pgfmathparse{10^((ln(\nero@printslope@px)+ln(\nero@printslope@xpos))/(ln(10)*2))}
   \edef\nero@printslope@labelpos{\pgfmathresult}
   %\node[anchor=north] at (axis cs:\nero@printslope@labelpos,\nero@printslope@ypos) {\tiny 1};
}

\makeatother

%________________ error lines (by Ricardo Costa) ________________

% \setlength\textwidth{5.125in}

% \newlength{\plotwidth}
% \setlength{\plotwidth}{0.5\textwidth}
% \newlength{\plotheight}
% \setlength{\plotheight}{0.3333333\textwidth}
\newlength{\plotwidth}
\setlength{\plotwidth}{0.45\textwidth}
\newlength{\plotheight}
\setlength{\plotheight}{0.3\textwidth}
% \newlength{\plotwidth}
% \setlength{\plotwidth}{0.4\textwidth}
% \newlength{\plotheight}
% \setlength{\plotheight}{0.266\textwidth}

\usepackage{tabularx}

\newcommand{\gainstable}[1]{
	\pgfplotstabletypeset[
	col sep=comma,
	every head row/.style={
		before row={\toprule[1.pt]
			& \multicolumn{4}{c}{\textbf{Gains on PINNs}} &
			\multicolumn{4}{c}{\textbf{Gains on FEM}} \\
			\cmidrule(lr){2-5} \cmidrule(lr){6-9}
		}, 
		after row=\cmidrule(lr){1-1} \cmidrule(lr){2-5} \cmidrule(lr){6-9}},
	every last row/.style={after row=\bottomrule[1.pt]},
	columns/N/.style={
		column name=\textbf{N}%,
		%			postproc cell content/.append style={
			%				/pgfplots/table/@cell content/.add={$\fontfamily{pag}\selectfont}{$}
			%			}
	},
	columns/min_PINNs/.style={column name=\textbf{min},fixed},
	columns/max_PINNs/.style={column name=\textbf{max},fixed},
	columns/mean_PINNs/.style={column name=\textbf{mean},fixed},
	columns/std_PINNs/.style={column name=\textbf{std},fixed},
	columns/min_FEM/.style={column name=\textbf{min},fixed},
	columns/max_FEM/.style={column name=\textbf{max},fixed},
	columns/mean_FEM/.style={column name=\textbf{mean},fixed},
	columns/std_FEM/.style={column name=\textbf{std},fixed},
	columns={N,min_PINNs,max_PINNs,mean_PINNs,std_PINNs,min_FEM,max_FEM,mean_FEM,std_FEM},
	precision=2
	]{#1}
	

	
}

\usepackage{float}
\usepackage{subcaption}
\usepackage[english]{babel}

% Set page size and margins
% Replace `letterpaper' with`a4paper' for UK/EU standard size
\usepackage[letterpaper,top=2cm,bottom=2cm,left=3cm,right=3cm,marginparwidth=1.75cm]{geometry}

\usepackage{booktabs}

\begin{document}	
	\gainstable{data/gains_table_case1_degree1.csv}
	
	\vspace{20pt}
	
	\gainstable{data/gains_table_case1_degree2.csv}
	
	\vspace{20pt}
	
	\gainstable{data/gains_table_case1_degree3.csv}
\end{document}


\documentclass[border=1mm]{standalone}
\usepackage[utf8]{inputenc}
\usepackage[english]{babel}
\usepackage{tikz}
% \usepackage[margin=1cm]{geometry}

\usepackage{booktabs}
\usepackage{pgfplotstable}
\usepackage{xcolor}
\usepackage{amsmath}

% gains pour tous les q
\newcommand{\coststableallq}[1]{
    \pgfplotstabletypeset[
        col sep=comma,
        every head row/.style={
        before row={\toprule[1.pt]
        & & \multicolumn{2}{c}{\textbf{$N_\text{dofs}$}} \\
		\cmidrule(lr){3-4}
        },
        after row=\cmidrule(lr){1-1} \cmidrule(lr){2-2} \cmidrule(lr){3-4}},
        every last row/.style={after row=\bottomrule[1.pt]},
        every nth row={2}{before row=\cmidrule(lr){1-1} \cmidrule(lr){2-2} \cmidrule(lr){3-4}},
		columns/q/.style={column name=\textbf{k}},
        columns/e/.style={column name=\textbf{e},sci},
		columns/FEM_dofs/.style={column name=\textbf{FEM},fixed},
        columns/Add_dofs/.style={column name=\textbf{Add},fixed,
            postproc cell content/.append style={
                /pgfplots/table/@cell content/.add={\color{red}}{},
            }
        },
        columns={q,e,FEM_dofs,Add_dofs},
        precision=2
    ]{#1}
}

\begin{document}
    \coststableallq{TabDoFs_case1_v1_param1.csv}
\end{document}

\begin{document}
	\nocite{*}
	
	\renewcommand{\inserttotalframenumber}{\pageref{lastslide}}
	
	{\setbeamertemplate{footline}{} 
		\BackgroundTitle	
		\begin{frame}
			\maketitle
		\end{frame}
	}
	\addtocounter{framenumber}{-1} 	
	
	\AtBeginSection[]{
		{\setbeamertemplate{footline}{}
			\begin{frame}
				\vfill
				\centering
				\begin{beamercolorbox}[sep=5pt,shadow=true,rounded=true]{subtitle}
					\usebeamerfont{title}\insertsectionhead\par%
					\vspace{0.5cm} % Ajustez l'espacement selon vos besoins
					% \usebeamerfont{classic}\usebeamercolor[fg]{classic}\insertsectionheadSubtitle
				\end{beamercolorbox}
				%\tableofcontents[sectionstyle=hide,subsectionstyle=show]
				
				%subsectionstyle=⟨style for current subsection⟩/⟨style for other subsections in current section⟩/⟨style for subsections in other sections⟩
				\tableofcontents[sectionstyle=hide,subsectionstyle=show/show/hide]
				\vfill
				\begin{beamercolorbox}[sep=5pt,shadow=true,rounded=true]{subtitle}
					\usebeamerfont{classic}\usebeamercolor[fg]{classic}\insertsectionheadSubtitle
				\end{beamercolorbox}
			\end{frame}
		}
		\addtocounter{framenumber}{-1} 
	}
	
	\AtBeginSubsection[]{
		{\setbeamertemplate{footline}{}
			\begin{frame}
				\vfill
				\centering
				\begin{beamercolorbox}[sep=5pt,shadow=true,rounded=true]{subtitle}
					\usebeamerfont{title}\insertsectionhead\par%
					\vspace{0.5cm} 
				\end{beamercolorbox}
				\tableofcontents[sectionstyle=hide,subsectionstyle=show/shaded/hide]
				\vfill
			\end{frame}
		}
		\addtocounter{framenumber}{-1} 
	}
	
	\Background

	% insert table of contents
	% {\setbeamertemplate{footline}{}
	% \begin{frame}{Table of contents}
	% 	\tableofcontents[sectionstyle=show,subsectionstyle=shaded]
	% \end{frame}
	% }

	\section*{Introduction}
	\begin{frame}{Scientific context}
	\begin{minipage}{0.78\linewidth}
		\textbf{Context :} Create real-time digital twins of an organ (e.g. liver).
	\end{minipage}
	\begin{minipage}{0.18\linewidth}
		\vspace{-20pt}
		\includegraphics[width=0.95\linewidth]{images/intro/liver.png}
	\end{minipage}
	
	\vspace{1pt}
	\textbf{Objective :} Develop an hybrid \fcolorbox{red}{white}{finite element} / \fcolorbox{orange}{white}{neural network} method.
	
	\vspace{1pt}
	\small
	\hspace{130pt} \begin{minipage}{0.14\linewidth}
		\textcolor{red}{accurate}
	\end{minipage} \hspace{8pt} \begin{minipage}{0.3\linewidth}
		\textcolor{orange}{quick + parameterized}
	\end{minipage}

	\normalsize
	\vspace{5pt}
	\textbf{Parametric linear elliptic PDE :}
	For one or several  $\bm{\mu}\in \mathcal{M}$, find $u: \Omega\to \mathbb{R}$ such that
	\begin{equation*}
		% \label{eq:ob_pde}
		\mathcal{L}\big(u;\bm{x},\bm{\mu}\big) = f(\bm{x},\bm{\mu}),
	\end{equation*}
	where $\mathcal{L}$ is the parametric differential operator defined  by
	\begin{equation*}
		\mathcal{L}(\cdot;\bm{x},\bm{\mu}) : u \mapsto R(\bm{x},\bm{\mu}) u + C(\bm{\mu}) \cdot \nabla u - \frac{1}{\text{Pe}} \nabla \cdot (D(\bm{x},\bm{\mu}) \nabla u),
	\end{equation*}
	and some Dirichlet, Neumann or Robin BC (which can also depend on $\bm{\mu}$).
	
	\footnotesize
	\begin{table}[ht!]
		\centering
		\begin{tabular}{c|c}
			$\Omega$ & Spatial domain \\
			$d$ & Spatial dimension \\
			$\bm{x}=(x_1,\dots,x_d)$ & Spatial coordinates \\
			\hline
			$\mathcal{M}$ & Parameter space \\
			$p$ & Number of parameters \\
			$\bm{\mu}=(\mu_1,\ldots,\mu_p)$ & Parameter vector \\
		\end{tabular} \hspace{10pt}
		\begin{tabular}{c|c}
			$f$ & Right-hand side \\
			$R$ & Reaction coefficient \\
			$C$ & Convection coefficient \\
			$D$ & Diffusion matrix \\
			Pe & Péclet number \\
		\end{tabular}
	\end{table}
\end{frame}

\begin{frame}{Pipeline of the Enriched FEM}
	\begin{figure}[!ht]
		\centering
		\includegraphics[width=0.7\linewidth]{images/intro/pipeline/offline_v2.pdf}

		\includegraphics[width=0.7\linewidth]{images/intro/pipeline/online_v2.pdf}
	\end{figure}

	\textbf{Correction :} Enriched continuous Lagrange finite element approximation spaces
	using the PINN prediction.
\end{frame}

\begin{frame}{Physics-Informed Neural Networks}
	\textbf{Standard PINNs :} Find the optimal weights $\theta^\star$ that satisfy
	\begin{equation}
		\label{eq:opt_pb}
		\theta^\star = \argmin_{\theta}	\big( \omega_r \; J_r(\theta) + \omega_b \; J_b(\theta) \big),
	\end{equation}
	with the residual loss function and the boundary loss function defined by
	\begin{equation*}
		J_r(\theta) =
		\int_{\mathcal{M}}\int_{\Omega}
		\big| \mathcal{L}\big(u_\theta(\bm{x},\bm{\mu});\bm{x},\bm{\mu}\big)-f(\bm{x},\bm{\mu}) \big|^2 d\bm{x} d\bm{\mu},
	\end{equation*}
	\begin{equation*}
		J_b(\theta) =
		\int_{\mathcal{M}}\int_{\partial \Omega} \big| u_\theta(\bm{x},\bm{\mu}) - g(\bm{x},\bm{\mu}) \big|^2 d\bm{x} d\bm{\mu},
	\end{equation*}
	where $u_\theta$ is a neural network, $g$ is the Dirichlet BC. In \eqref{eq:opt_pb}, the weights $\omega_r$ and $\omega_b$ (hyperparameters) are used to balance the different terms of the loss function.

	\vspace{5pt}
	\textbf{Monte-Carlo method :} Discretize the cost functions by random process.
\end{frame}

\begin{frame}[noframenumbering]{Physics-Informed Neural Networks}
	\textbf{\textcolor{red}{Improved} PINNs\footcite{LagLikFot1998,FraMicNav2024} :} Find the optimal weights $\theta^\star$ that satisfy
	\begin{equation}
		\label{eq:opt_pb_nobc}
		\theta^\star = \argmin_{\theta}	\big( \omega_r \; J_r(\theta) + \Ccancel[red]{\omega_b \; J_b(\theta)} \big),
	\end{equation}
	with $\omega_r=1$ and the residual loss function defined by
	\begin{equation*}
		J_r(\theta) =
		\int_{\mathcal{M}}\int_{\Omega}
		\big| \mathcal{L}\big(u_\theta(\bm{x},\bm{\mu});\bm{x},\bm{\mu}\big)-f(\bm{x},\bm{\mu}) \big|^2 d\bm{x} d\bm{\mu},
	\end{equation*}
	\begin{minipage}{0.7\linewidth}
		where $u_\theta$ is a neural network defined by
		\begin{equation*}
			\textcolor{red}{u_{\theta}(\bm{x},\bm{\mu}) = \varphi(\bm{x}) w_{\theta}(\bm{x},\bm{\mu}) + g(\bm{x},\bm{\mu}),}
		\end{equation*}
		with $\varphi$ a level-set function, $w_\theta$ a NN and $g$ the Dirichlet BC. 
	\end{minipage}
	\begin{minipage}{0.28\linewidth}
		\vspace{-15pt}
		\includegraphics[width=0.95\linewidth]{images/intro/levelset.png}
	\end{minipage}

	\vspace{5pt}
	\textbf{Monte-Carlo method :} Discretize the residual cost function by random process.
	\vspace{15pt}
\end{frame}


\begin{frame}{Finite Element Method}
	TODO
\end{frame}
	
	
	\renewcommand{\insertsectionheadSubtitle}{This section is based on \citep{ours_2025}.}
	\section{Enriched finite element method}
	\begin{frame}{Idea}
	\vspace{-20pt}
	\begin{figure}[htb]
		\centering
		\resizebox{\textwidth}{!}{%
			\begin{tikzpicture}
				\node at (0,0.8) {1 Geometry + 1 Force};
				\node[draw=none, inner sep=0pt] at (0,0) {\includegraphics[width=2cm]{images/correction/objective_onegeom_onefct.png}};
				\node at (0,-1) {$\begin{aligned}[t]
						\; \phi \quad \text{\small and} \quad &f \\
						\; (\text{\small and} \quad &g)
					\end{aligned}$};
				
				\draw[->, title, line width=1.5pt] (1.7,0.1) -- (2.7,0.1);
				
				\node[align=center] at (4,1) {Get PINNs prediction};
				\node[draw=none, inner sep=0pt] at (4,0.1) {\includegraphics[width=1.4cm]{images/correction/objective_pinns.jpg}};
				\node at (4,-0.8) {\fcolorbox{blue}{white}{$u_{NN}=\phi w_{NN}+g$}};
				\node at (4,-1.3) {\textcolor{blue}{$u_{NN}=g$ on $\Gamma$}};
				
				% Ajouter une flèche entre les deux rectangles
				\draw[->, title, line width=1.5pt] (5.2,0.1) -- (6.2,0.1);
				
				\node[align=center] at (7.8,1) {Correct prediction \\ with FEM};
				\node[draw=none, inner sep=0pt] at (7.8,-0.1) {\includegraphics[width=2.5cm]{images/correction/objective_corr.png}};		
				\node at (7.8,-1) {$u_{NN}\rightarrow\tilde{u}=u_{NN}+\tilde{C}$};
			\end{tikzpicture} 
		}%
	\end{figure}
	
	\vspace{-5pt}
	
	\textbf{Correct by adding :} Considering $u_{NN}$ as the prediction of our PINNs for (\ref{edp}), the correction problem consists in writing the solution as
	\begin{equation*}
		\tilde{u}=u_{NN}+\underset{\textcolor{red}{\ll 1}}{\fcolorbox{red}{white}{$\tilde{C}$}}
	\end{equation*}
	
	\vspace{-8pt}
	\begin{minipage}{\linewidth}
		\setstretch{0.5}
		and searching $\tilde{C}: \Omega \rightarrow \mathbb{R}^d$ such that
		\begin{equation*}
			\left\{\begin{aligned}
				L(\tilde{C})&=\tilde{f}, \; &&\text{in } \Omega, \\
				\tilde{C}&=0, \; &&\text{on } \Gamma,
			\end{aligned}\right. %\tag{$\mathcal{C}_{+}$} %\label{corr_add}
		\end{equation*}
		with $\tilde{f}=f-L(u_{NN})$. \refappendix{frame:fem}
	\end{minipage}
\end{frame}

\begin{frame}{Poisson on Square}
	Solving the \textcolor{orange}{Poisson problem} with homogeneous Dirichlet BC. \\
	\ding{217} \textbf{Domain :} $\Omega=[−0.5\pi,0.5\pi]^2$ \\
	\ding{217} \textbf{Analytical levelset function :}
	\small
	\begin{equation*}
		\phi(x,y)=(x-0.5\pi)(x+0.5\pi)(y-0.5\pi)(y+0.5\pi)
	\end{equation*} 
	\ding{217} \textbf{Analytical solution :}
	\small
	
	\vspace{-8pt}
	\begin{equation*}
		u_{ex}(x,y)=\exp\left(−\frac{(x-\mu_1)^2+(y-\mu_2)^2}{2}\right)\sin(2x)\sin(2y)
	\end{equation*} 
	\normalsize
	with $\mu_1,\mu_2\in[-0.5,0.5]$. 
	
	\vspace{8pt}
	Taking $\mu_1=0.05,\mu_2=0.22$, the solution is given by
	\begin{minipage}{0.68\linewidth}
		\centering
		\pgfimage[width=\linewidth]{images/correction/poisson_sol.png}
	\end{minipage}
	\begin{minipage}{0.28\linewidth}
		\flushright
		\pgfimage[width=0.9\linewidth]{images/correction/poisson_loss.png}
	\end{minipage}
\end{frame}

\begin{frame}{Theoretical results}
	TODO
	$\mu_1=0.05,\mu_2=0.22$
	\begin{center}
		\pgfimage[width=0.5\linewidth]{images/correction/theoretical.png}
	\end{center}
\end{frame}

\begin{frame}{Gains using our approach}	
	\vspace{10pt}
	
	\hspace{20pt}\begin{minipage}{0.05\linewidth}
		\footnotesize
		\rotatebox[origin=b]{90}{\textbf{Solution $\mathbb{P}_1$}} 
	\end{minipage}
	\begin{minipage}{0.8\linewidth}
		\centering
		\pgfimage[height=1.7cm]{images/correction/gains_P1.png}
	\end{minipage} 

	\vspace{5pt}

	\hspace{20pt}\begin{minipage}{0.05\linewidth}
		\footnotesize
		\rotatebox[origin=b]{90}{\textbf{Solution $\mathbb{P}_2$}} 
	\end{minipage}
	\begin{minipage}{0.8\linewidth}
		\centering
		\pgfimage[height=1.7cm]{images/correction/gains_P2.png}
	\end{minipage} 

	\vspace{5pt}

	\hspace{20pt}\begin{minipage}{0.05\linewidth}
		\footnotesize
		\rotatebox[origin=b]{90}{\textbf{Solution $\mathbb{P}_3$}} 
	\end{minipage}
	\begin{minipage}{0.8\linewidth}
		\centering
		\pgfimage[height=1.7cm]{images/correction/gains_P3.png}
	\end{minipage} 
\end{frame}

\begin{frame}{Time/Precision}	
	TODO
\end{frame}

	\renewcommand{\insertsectionheadSubtitle}{}

	\section{Supplementary work}
	\begin{frame}{Supplementary work I}
	\small
	\vspace{-10pt}
	\begin{tcolorbox}[
		skin=bicolor,
		colback=other, % Couleur de fond de la boîte
		colbacklower=other!20!white,
		title={Teaching},
		colframe=title, % Couleur du cadre de la boîte
		arc=2mm, % Rayon de l'arrondi des coins
		boxrule=0.5pt, % Épaisseur du cadre de la boîte
		breakable, enhanced jigsaw,
		width=\linewidth,
		opacityback=0.1,
		]

		\begin{itemize}[\textcolor{other}{$\blacktriangleright$}]
			\item 2024/2025 :
			\begin{itemize}[\textcolor{other}{$\blacktriangleright$}]
                \item $64$h of Computer Science Practical Work - L1S2 and L2S3 (Python) / L3S6 (C++)
                \item $3$ days supervising a group of high school girls in RJMI \\
                ("Rendez-vous des Jeunes Mathématiciennes et Informaticiennes")
            \end{itemize}
            \item 2023/2024 : $50$h of Computer Science Practical Work - L2S3 (Python) / L3S6 (C++)
		\end{itemize}
	\end{tcolorbox}

    \vspace{-7pt}
	\begin{tcolorbox}[
		skin=bicolor,
		colback=other, % Couleur de fond de la boîte
		colbacklower=other!20!white,
		title={Training courses (Total : $176$h$35$)},
		colframe=title, % Couleur du cadre de la boîte
		arc=2mm, % Rayon de l'arrondi des coins
		boxrule=0.5pt, % Épaisseur du cadre de la boîte
		breakable, enhanced jigsaw,
		width=\linewidth,
		opacityback=0.1
		]
		
		\begin{itemize}[\textcolor{other}{$\blacktriangleright$}]
			\item A dozen seminars organized by IRMA ($\approx 10h$)
            \item 1 Deep Learning introductory course - FIDLE ($\approx 40h$)
            \item 2 workshops on Scientific Machine Learning ($\approx 2\times 21h$)
            \item 1 summer school on "New Trend in computing" ($\approx 27h$)
            \item several cross-disciplinary courses - Methodology, scientific English, etc. ($\approx 58h$)
		\end{itemize} 
	\end{tcolorbox}
\end{frame}

\begin{frame}{Supplementary work II}
    \small 

	\begin{tcolorbox}[
		skin=bicolor,
		colback=other, % Couleur de fond de la boîte
		colbacklower=other!20!white,
		title={Talks},
		colframe=title, % Couleur du cadre de la boîte
		arc=2mm, % Rayon de l'arrondi des coins
		boxrule=0.5pt, % Épaisseur du cadre de la boîte
		breakable, enhanced jigsaw,
		width=\linewidth,
		opacityback=0.1
		]
		
		\begin{itemize}[\textcolor{other}{$\blacktriangleright$}]
            \item \textbf{\href{https://icosahom2025.org/}{ICOSAHOM 2025}, Montréal} - July 2025 \textit{(Coming soon...)} \\
            "Enriching continuous Lagrange finite element approximation spaces using neural networks" 
            \item \textbf{\href{https://dte_aicomas_2025.iacm.info/organizers}{DTE \& AICOMAS 2025}, Paris} - February 20, 2025 \\
            \href{https://flecourtier.github.io/these2023/these2023/1.0.3/_attachments/presentation/2025_02_20.pdf}{"Combining Finite Element Methods and Neural Networks to Solve Elliptic Problems on 2D Geometries"}
			\item \textbf{Exama project, WP2 reunion} - March 26, 2024 \\
			\href{https://flecourtier.github.io/these2023/these2023/1.0.3/_attachments/presentation/2024_03_26.pdf}{"How to work with complex geometries in PINNs ?"}
			\item \textbf{Retreat (Macaron/Tonus)} - February 6, 2024 \\
			\href{https://flecourtier.github.io/these2023/these2023/1.0.3/_attachments/presentation/2024_02_06.pdf}{"Mesh-based methods and physically informed learning"}
			\item \textbf{Team meeting (Mimesis)} - December 12, 2023 \\
            \href{https://flecourtier.github.io/these2023/these2023/1.0.3/_attachments/presentation/2023_12_12.pdf}{"Development of hybrid finite element/neural network methods to help create digital surgical twins"}
		\end{itemize}
	\end{tcolorbox}
\end{frame}

\begin{frame}{Supplementary work III}
    \small
	% \vspace{-7pt}
	\begin{tcolorbox}[
		skin=bicolor,
		colback=other, % Couleur de fond de la boîte
		colbacklower=other!20!white,
		title={Posters},
		colframe=title, % Couleur du cadre de la boîte
		arc=2mm, % Rayon de l'arrondi des coins
		boxrule=0.5pt, % Épaisseur du cadre de la boîte
		breakable, enhanced jigsaw,
		width=\linewidth,
		opacityback=0.1
		]
		
        \begin{itemize}[\textcolor{other}{$\blacktriangleright$}]
            \item \textbf{\href{https://www.mate.polimi.it/events/EMS-TAG-SciML-25/index.php}{EMS-TAG-SciML 2025}, Milan} - March 24, 2025 - \href{https://flecourtier.github.io/these2023/these2023/1.0.3/_attachments/poster/2025_03_24.pdf}{"Enriching continuous Lagrange finite element approximation spaces using neural networks"} 
            \item \textbf{\href{https://cjc-ma2024.sciencesconf.org/program?lang=fr}{CJC-MA 2024}, Lyon} - October 29, 2024 - \href{https://flecourtier.github.io/these2023/these2023/1.0.3/_attachments/poster/2024_10_24.pdf}{"Combining Finite Element Methods and Neural Networks to Solve Elliptic Problems on 2D Geometries"}
            \item \textbf{MSII poster day, Strasbourg} - October 24, 2024 \\
			\item \textbf{\href{https://irma.math.unistra.fr/~micheldansac/SciML2024/participants.html}{SciML 2024}, Strasbourg} - July 08, 2024 \\
		\end{itemize}
	\end{tcolorbox}

    \normalsize
    \begin{tcolorbox}[
		skin=bicolor,
		colback=other, % Couleur de fond de la boîte
		colbacklower=other!20!white,
		title={Publications},
		colframe=title, % Couleur du cadre de la boîte
		arc=2mm, % Rayon de l'arrondi des coins
		boxrule=0.5pt, % Épaisseur du cadre de la boîte
		breakable, enhanced jigsaw,
		width=\linewidth,
		opacityback=0.1
		]
		
		\begin{itemize}[\textcolor{other}{$\blacktriangleright$}]
			\item Enriching continuous lagrange finite element approximation spaces using neural networks. \textit{\small(submitted in February 2025, M2AN journal)} \\
			H. Barucq, M. Duprez, F. Faucher, E. Franck, \textbf{F. Lecourtier}, V. Lleras, V. Michel-Dansac, and N. Victorion.
		\end{itemize}
	\end{tcolorbox}
	
\end{frame}



	\section*{Conclusion}

	\begin{frame}{Conclusion}
		TODO
	\end{frame}
	
	{\setbeamertemplate{footline}{} 	
	\begin{frame}{References}
		\Large
		\bibliography{biblio}
	\end{frame}
	}
	\addtocounter{framenumber}{-1} 
	
	% \section{Appendix}
	
	\appendix
	
	\appendixsection{Finite element method (FEM)}\labelappendixframe{frame:appfem}

\begin{appendixframe}{Construction of the unknown vector}\labelappendixframe{frame:basis}

	Considering $(\phi_i)_{i=1}^{N_u}$, $(\psi_j)_{j=1}^{N_p}$ and $(\eta_k)_{k=1}^{N_T}$ the basis functions of the finite element spaces $V_h^{\, 0}$, $Q_h$ and $W_h$ respectively, we can write the discrete solutions as:
	\begin{equation*}
		\bm{u}_h(\bm{x}) = \sum_{i=1}^{N_u} \begin{pmatrix}
			u_i \\
			v_i
		\end{pmatrix} \phi_i(\bm{x}), \quad p_h(\bm{x}) = \sum_{j=1}^{N_p} p_j \psi_j(\bm{x}) \quad \text{and} \quad T_h(\bm{x}) = \sum_{k=1}^{N_T} T_k \eta_k(\bm{x}),
	\end{equation*}	
	with the unknown vectors for velocity, pressure and temperature defined by

	\vspace{-5pt}
	$$\vec{u} = \big(u_i\big)_{i=1}^{N_u} \in \mathbb{R}^{N_u}, \quad \vec{v} = \big(v_i\big)_{i=1}^{N_u} \in \mathbb{R}^{N_u},$$
	$$\vec{p} = \big(p_j\big)_{j=1}^{N_p} \in \mathbb{R}^{N_p} \; \text{ and } \; \vec{T} = \big(T_k\big)_{k=1}^{N_T} \in \mathbb{R}^{N_T}.$$

	\vspace{5pt}
	Considering $N_h = 2N_u + N_p + N_T$, we can define the global vector of unknowns as:
	\begin{equation*}
		\vec{U} = \big(\vec{u}, \vec{v}, \vec{p}, \vec{T}) \in \mathbb{R}^{N_h}.
	\end{equation*}
	and $F:\mathbb{R}^{N_h} \to \mathbb{R}^{N_h}$ the nonlinear operator associated to the weak formulation \eqref{eq:weak_pb}.
\end{appendixframe}

\appendixsection{PINN Initialization / Additive approach}\labelappendixframe{frame:comp}


\begin{appendixframe}{Comparison of the 2 approaches}
	Taking $U_\theta$ and $C_h^+$ in the same space, we have :
	$$F_\theta(\vec{C})=F(\vec{U}_\theta+\vec{C}),$$
	with $\vec{C}$ the correction vector and $\vec{U}_\theta$ the PINN vector (PINN evaluation at the dofs), both of size $N_h$.

	The first iteration of the additive approach :
	$$ F_\theta(\vec{C}^{(0)}) + F_\theta'(\vec{C}^{(0)}) \delta^{(1)} = 0 $$
	becomes (as $C^{(0)}=0$) :
	$$F(\vec{U}_\theta) + F'(\vec{U}_\theta)\delta^{(1)}=0,$$
	which is equivalent as the standard method with the PINN initialization.
\end{appendixframe}

%% AUTRES

\appendixsection{Results - Linear problem}\labelappendixframe{frame:linear}

\begin{appendixframe}{Problem considered} 
	\textbf{Problem statement:} Consider the Poisson problem with Dirichlet BC:
	\vspace{-5pt}
	\begin{equation*}
		\left\{
		\begin{aligned}
			-\Delta u & = f, \; &  & \text{in } \; \Omega \times \mathcal{M}, \\
			u         & =0, \;  &  & \text{on } \; \partial\Omega \times \mathcal{M},
		\end{aligned}
		\right.
		% \label{eq:Lap2D}\tag{$\mathcal{P}$}
	\end{equation*}

	\vspace{-5pt}
	with $\Omega=[-0.5 \pi, 0.5 \pi]^2$ and $\mathcal{M}=[-0.5,0.5]^2$ ($p=2$ parameters).
		
	\vspace{8pt}
	\textbf{Analytical solution :}

	\vspace{-5pt}
	\begin{equation*}
		% \label{eq:analytical_solution_Lap2D}
		u(\bm{x},\bm{\mu})=\exp\left(-\frac{(x-\mu_1)^2+(y-\mu_2)^2}{2}\right)\sin(2 x)\sin(2 y).
	\end{equation*}

	\vspace{12pt}
	\textbf{PINN training:} MLP of 5 layers; LBFGs optimizer (5000 epochs). \\
	Imposing the Dirichlet BC exactly in the PINN with the levelset $\varphi$ defined by
	$$\varphi(\bm{x})=(x+0.5\pi)(x-0.5\pi)(y+0.5\pi)(y-0.5\pi).$$
	
	\small\vspace{4pt}
	Training time : less than 1 hour on a laptop GPU.
\end{appendixframe}

\begin{appendixframe}{Numerical results}
	\hspace{-5pt}\begin{minipage}[t]{0.46\linewidth}
		\textbf{Error estimates :} 1 set of parameters.
		$$\bm{\mu}^{(1)}=(0.05, 0.22) $$
		\vspace{-35pt}
		\begin{figure}[H]
			\cvgFEMCorrAlldeg{images/appendix/linear_results/cvg/FEM_case1_v1_param1.csv}{images/appendix/linear_results/cvg/Corr_case1_v1_param1.csv}{1e-10}
		\end{figure}
	\end{minipage} \qquad \small
	\begin{minipage}[t]{0.48\linewidth}
	\end{minipage}
\end{appendixframe}

\begin{appendixframe}{Numerical results}[noframenumbering]
	\hspace{-5pt}\begin{minipage}[t]{0.46\linewidth}
		\textbf{Error estimates :} 1  set of parameters.
		$$\bm{\mu}^{(1)}=(0.05, 0.22) $$
		\vspace{-35pt}
		\begin{figure}[H]
			\cvgFEMCorrAlldeg{images/appendix/linear_results/cvg/FEM_case1_v1_param1.csv}{images/appendix/linear_results/cvg/Corr_case1_v1_param1.csv}{1e-10}
		\end{figure}
	\end{minipage} \qquad \small
	\begin{minipage}[t]{0.48\linewidth}
		\textbf{Gains achieved :} $n_p=50$ sets of parameters.
		$$\mathcal{S}=\left\{\bm{\mu}^{(1)},\dots,\bm{\mu}^{(n_p)}\right\}$$
		\vspace{-15pt}
		\begin{table}[H]
			\gainstableallq{images/appendix/linear_results/gains/Tab_stats_case1_v1.csv}
		\end{table}

		\normalsize\centering\vspace{-20pt}
		$$N=20$$

		\vspace{-5pt}
		Gain : $\| u-u_h\|_{L^2} / \| u-u_h^+\|_{L^2}$ \\
		
		\small\vspace{8pt}
		Cartesian mesh : $N^2$ nodes.
	\end{minipage}
\end{appendixframe}

\begin{appendixframe}{Numerical results}[noframenumbering]
	\hspace{-5pt}\begin{minipage}[t]{0.46\linewidth}
		\textbf{Error estimates :} 1 set of parameters.
		$$\bm{\mu}^{(1)}=(0.05, 0.22) $$
		\vspace{-35pt}
		\begin{figure}[H]
			\cvgFEMCorrAlldegLine{images/appendix/linear_results/cvg/FEM_case1_v1_param1.csv}{images/appendix/linear_results/cvg/Corr_case1_v1_param1.csv}{1e-10}
		\end{figure}
	\end{minipage} \qquad \small
	\begin{minipage}[t]{0.48\linewidth}
		\textbf{$N_\text{dofs}$ required to reach the same error $e$ :}

		\vspace{10pt}
		\begin{table}[H]
			\centering
			\coststableallq{images/appendix/linear_results/costs/TabDoFs_case1_v1_param1.csv}
		\end{table}
	\end{minipage}
\end{appendixframe}

\appendixsection{Data-driven vs Physics-Informed training}\labelappendixframe{frame:datavspinns}

\begin{appendixframe}{Problem considered}
	\textbf{Problem statement:} Consider the Poisson problem in 1D with Dirichlet BC:
	\vspace{-5pt}
	\begin{equation*}
		\left\{
		\begin{aligned}
			-\partial_{xx} u & = f, \; &  & \text{in } \; \Omega \times \mathcal{M}, \\
			u         & = 0, \;  &  & \text{on } \; \partial\Omega \times \mathcal{M},
		\end{aligned}
		\right.
		% \label{eq:Lap2D}\tag{$\mathcal{P}$}
	\end{equation*}

	\vspace{-5pt}
	with $\Omega=[0,1]^2$ and $\mathcal{M}=[0,1]^3$ ($p=3$ parameters).
		
	\vspace{4pt}
	\textbf{Analytical solution :} $\quad u(x;\bm{\mu})=\mu_1\sin(2\pi x)+\mu_2\sin(4\pi x)+\mu_3\sin(6\pi x) \,.$

	\vspace{4pt}
	\textbf{Construction of two priors:} MLP of 6 layers; Adam optimizer (10000 epochs). \\
	Imposing the Dirichlet BC exactly in the PINN with $\varphi(x)=x(x-1)$.

	\begin{itemize}
		\item \textbf{Physics-informed training:} $N_\text{col}=5000$ collocation points.
		$$J_r(\theta) \simeq
			\frac{1}{N_\text{col}} \sum_{i=1}^{N_\text{col}} \big| \partial_{xx}u_\theta(\bm{x}_\text{col}^{(i)};\bm{\mu}_\text{col}^{(i)}\big) + f\big(\bm{x}_\text{col}^{(i)};\bm{\mu}_\text{col}^{(i)}\big) \big|^2.$$
	
		\item \textbf{Data-driven training:}  $N_\text{data}=5000$ data.
		$$J_\text{data}(\theta) =
		\frac{1}{N_\text{data}}
		\sum_{i=1}^{N_\text{data}} \big| u_\theta^\text{data}(\bm{x}_\text{data}^{(i)};\bm{\mu}_\text{data}^{(i)}) - u(\bm{x}_\text{data}^{(i)};\bm{\mu}_\text{data}^{(i)}) \big|^2.$$
	\end{itemize}
\end{appendixframe}

\begin{appendixframe}{Priors derivatives}
	\vspace{-10pt}
	$$\bm{\mu}^{(1)}=(0.3,0.2,0.1)$$
	\begin{figure}[ht!]
		\centering
		\includegraphics[width=\linewidth]{images/appendix/datavspinns/standalone_solutions_and_errors_PINN.pdf}
	\end{figure}
	
	\begin{figure}[ht!]
		\centering
		\includegraphics[width=\linewidth]{images/appendix/datavspinns/standalone_solutions_and_errors_NN.pdf}
	\end{figure}
\end{appendixframe}

\begin{appendixframe}{Additive approach in $\mathbb{P}_1$}
	\vspace{-2pt}
	\textbf{1 set of parameters:} $\quad \bm{\mu}^{(1)}=(0.3,0.2,0.1)$
	
	\begin{table}[H]
		\centering
		\gainsbothNN{images/appendix/datavspinns/FEM_param1.csv}{images/appendix/datavspinns/compare_gains_param1.csv}
	\end{table}

	\vspace{6pt}
	\textbf{50 set of parameters:}

	\begin{table}[H]
		\centering
		\gainstableMult{images/appendix/datavspinns/Tab_stats_case1_degree1.csv}
	\end{table}

	\footnotesize
	$N$ : Nodes.
\end{appendixframe}
	
\end{document}
