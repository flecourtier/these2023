%% Requires compilation with XeLaTeX or LuaLaTeX
\documentclass[compress,10pt,xcolor={table,dvipsnames},t]{beamer}
\usetheme{diapo}
%\usepackage{hyperref}
\usepackage{amsmath}
\usepackage{amssymb}
\usepackage{xcolor}
\usepackage[bottom]{footmisc}
\usepackage{multirow}
\usepackage{setspace}
\usepackage{caption}
\usepackage{array,multirow,makecell}
\usepackage{pifont}
%\usepackage[colorlinks=true]{hyperref}
\usepackage{tikz}
\usepackage{paralist}
\usepackage{appendixnumberbeamer}
\usepackage[backend=biber,style=numeric,sorting=nyt,doi=false,url=false]{biblatex}
\usepackage{etoolbox}
% box colorée dans équation
\usepackage[most]{tcolorbox}
\usepackage{tikz}
\usepackage{soul}
% pour l'indicatrice
\usepackage{dsfont}
\usepackage{cancel}


\setcellgapes{1pt}
\setlength{\parindent}{0pt}
\makegapedcells
\newcolumntype{R}[1]{>{\raggedleft\arraybackslash }b{#1}}
\newcolumntype{L}[1]{>{\raggedright\arraybackslash }b{#1}}
\newcolumntype{C}[1]{>{\centering\arraybackslash }b{#1}}
\renewcommand*{\bibfont}{\scriptsize}
% Supprimer "In:" pour les articles
\renewbibmacro{in:}{}
% Supprimer les champs d'eprints
\AtEveryBibitem{\clearfield{arxiv}}
\AtEveryBibitem{\clearfield{eprint}}
\AtEveryBibitem{\clearfield{note}}
\AtEveryBibitem{\clearfield{eprintclass}}
\AtEveryBibitem{\clearfield{eprinttype}}
% Supprimer les URL
\ExecuteBibliographyOptions{url=false}
% Charger votre fichier de bibliographie
\addbibresource{biblio.bib}
\useoutertheme[subsection=false]{miniframes}
\makeatletter
%\patchcmd{\slideentry}{\advance\beamer@xpos by1\relax}{}{}{}
\def\beamer@subsectionentry#1#2#3#4#5{\advance\beamer@xpos by1\relax}%
\makeatother
\setbeamercolor*{mini frame}{fg=bulles,bg=bulles}
\hypersetup{
	colorlinks=true,
	urlcolor=blue,
	citecolor=blue,
	linkcolor=title,
}

\title[PhiFEM]{Mesh-based methods and physically informed learning}
\subtitle{Macaron/Tonus retreat presentation}
\authors[LECOURTIER Frédérique]
\supervisors[DUPREZ Michel, FRANCK Emmanuel, LLERAS Vanessa]
\date{February 6-7, 2024}

\allowbreak

% u_chapeau (chapeau en couleur)
\usepackage{accents}
\newcommand{\uchapeau}[1]{\accentset{\textcolor{red}{\wedge}}{#1}}
\newcommand{\refappendix}[1]{\tikz[baseline=(char.base)]{\node[framednumber] (char) {\hyperlink{#1}{\small \textcolor{white}{Appendix \ref*{#1}}}};}}

\definecolor{appendix}{RGB}{180, 189, 138}
\tikzset{
	framednumber/.style={
		draw=appendix,% Couleur de la bordure
		fill=appendix, % Couleur de fond
		rounded corners, % Coins arrondis
		inner sep=2pt,  % Espace intérieur
	}
}

% numérotation et label des appendix
\newcounter{appendixframenumber}
\setcounter{appendixframenumber}{1}

\makeatletter
\newcommand{\labelappendixframe}[1]{%
	\protected@write\@auxout{}{%
		\string\newlabel{#1}{{\theappendixframenumber}{\thepage}}%
	}%
	\hypertarget{#1}{}
}	
\makeatother

%\newenvironment{appendixframe}[2][]{%
%	\begin{frame}[#1]{\appendixname~\theappendixframenumber~: #2}%
%	}{%
%	\end{frame}
%	\addtocounter{appendixframenumber}{1}
%}

% barre en couleur terme dans équation
\newcommand\Ccancel[2][black]{\renewcommand\CancelColor{\color{#1}}\cancel{#2}}

\begin{document}
	\nocite{*}
	
	\renewcommand{\inserttotalframenumber}{\pageref{lastslide}}
	
	{\setbeamertemplate{footline}{} 
		\begin{frame}
			\maketitle
		\end{frame}
	}
	\addtocounter{framenumber}{-1} 
	
	\AtBeginSection[]{
		{\setbeamertemplate{footline}{}
			\begin{frame}
				\vfill
				\centering
				\begin{beamercolorbox}[sep=5pt,shadow=true,rounded=true]{subtitle}
					\usebeamerfont{title}\insertsectionhead\par%
				\end{beamercolorbox}
				%\tableofcontents[sectionstyle=hide,subsectionstyle=show]
				
				%subsectionstyle=⟨style for current subsection⟩/⟨style for other subsections in current section⟩/⟨style for subsections in other sections⟩
				\tableofcontents[sectionstyle=hide,subsectionstyle=show/show/hide]
				\vfill
			\end{frame}
		}
		\addtocounter{framenumber}{-1} 
	}
	
	\AtBeginSubsection[]{
		{\setbeamertemplate{footline}{}
			\begin{frame}
				\vfill
				\centering
				\begin{beamercolorbox}[sep=5pt,shadow=true,rounded=true]{subtitle}
					\usebeamerfont{title}\insertsectionhead\par%
				\end{beamercolorbox}
				\tableofcontents[sectionstyle=hide,subsectionstyle=show/shaded/hide]
				\vfill
			\end{frame}
		}
		\addtocounter{framenumber}{-1} 
	}
	
	\section{Introduction}
	\begin{frame}{Scientific context}
	\begin{minipage}{0.78\linewidth}
		\textbf{Context :} Create real-time digital twins of an organ (e.g. liver).
	\end{minipage}
	\begin{minipage}{0.18\linewidth}
		\vspace{-20pt}
		\includegraphics[width=0.95\linewidth]{images/intro/liver.png}
	\end{minipage}
	
	\vspace{1pt}
	\textbf{Objective :} Develop an hybrid \fcolorbox{red}{white}{finite element} / \fcolorbox{orange}{white}{neural network} method.
	
	\vspace{1pt}
	\small
	\hspace{130pt} \begin{minipage}{0.14\linewidth}
		\textcolor{red}{accurate}
	\end{minipage} \hspace{8pt} \begin{minipage}{0.3\linewidth}
		\textcolor{orange}{quick + parameterized}
	\end{minipage}

	\normalsize
	\vspace{5pt}
	\textbf{Parametric linear elliptic PDE :}
	For one or several  $\bm{\mu}\in \mathcal{M}$, find $u: \Omega\to \mathbb{R}$ such that
	\begin{equation*}
		% \label{eq:ob_pde}
		\mathcal{L}\big(u;\bm{x},\bm{\mu}\big) = f(\bm{x},\bm{\mu}),
	\end{equation*}
	where $\mathcal{L}$ is the parametric differential operator defined  by
	\begin{equation*}
		\mathcal{L}(\cdot;\bm{x},\bm{\mu}) : u \mapsto R(\bm{x},\bm{\mu}) u + C(\bm{\mu}) \cdot \nabla u - \frac{1}{\text{Pe}} \nabla \cdot (D(\bm{x},\bm{\mu}) \nabla u),
	\end{equation*}
	and some Dirichlet, Neumann or Robin BC (which can also depend on $\bm{\mu}$).
	
	\footnotesize
	\begin{table}[ht!]
		\centering
		\begin{tabular}{c|c}
			$\Omega$ & Spatial domain \\
			$d$ & Spatial dimension \\
			$\bm{x}=(x_1,\dots,x_d)$ & Spatial coordinates \\
			\hline
			$\mathcal{M}$ & Parameter space \\
			$p$ & Number of parameters \\
			$\bm{\mu}=(\mu_1,\ldots,\mu_p)$ & Parameter vector \\
		\end{tabular} \hspace{10pt}
		\begin{tabular}{c|c}
			$f$ & Right-hand side \\
			$R$ & Reaction coefficient \\
			$C$ & Convection coefficient \\
			$D$ & Diffusion matrix \\
			Pe & Péclet number \\
		\end{tabular}
	\end{table}
\end{frame}

\begin{frame}{Pipeline of the Enriched FEM}
	\begin{figure}[!ht]
		\centering
		\includegraphics[width=0.7\linewidth]{images/intro/pipeline/offline_v2.pdf}

		\includegraphics[width=0.7\linewidth]{images/intro/pipeline/online_v2.pdf}
	\end{figure}

	\textbf{Correction :} Enriched continuous Lagrange finite element approximation spaces
	using the PINN prediction.
\end{frame}

\begin{frame}{Physics-Informed Neural Networks}
	\textbf{Standard PINNs :} Find the optimal weights $\theta^\star$ that satisfy
	\begin{equation}
		\label{eq:opt_pb}
		\theta^\star = \argmin_{\theta}	\big( \omega_r \; J_r(\theta) + \omega_b \; J_b(\theta) \big),
	\end{equation}
	with the residual loss function and the boundary loss function defined by
	\begin{equation*}
		J_r(\theta) =
		\int_{\mathcal{M}}\int_{\Omega}
		\big| \mathcal{L}\big(u_\theta(\bm{x},\bm{\mu});\bm{x},\bm{\mu}\big)-f(\bm{x},\bm{\mu}) \big|^2 d\bm{x} d\bm{\mu},
	\end{equation*}
	\begin{equation*}
		J_b(\theta) =
		\int_{\mathcal{M}}\int_{\partial \Omega} \big| u_\theta(\bm{x},\bm{\mu}) - g(\bm{x},\bm{\mu}) \big|^2 d\bm{x} d\bm{\mu},
	\end{equation*}
	where $u_\theta$ is a neural network, $g$ is the Dirichlet BC. In \eqref{eq:opt_pb}, the weights $\omega_r$ and $\omega_b$ (hyperparameters) are used to balance the different terms of the loss function.

	\vspace{5pt}
	\textbf{Monte-Carlo method :} Discretize the cost functions by random process.
\end{frame}

\begin{frame}[noframenumbering]{Physics-Informed Neural Networks}
	\textbf{\textcolor{red}{Improved} PINNs\footcite{LagLikFot1998,FraMicNav2024} :} Find the optimal weights $\theta^\star$ that satisfy
	\begin{equation}
		\label{eq:opt_pb_nobc}
		\theta^\star = \argmin_{\theta}	\big( \omega_r \; J_r(\theta) + \Ccancel[red]{\omega_b \; J_b(\theta)} \big),
	\end{equation}
	with $\omega_r=1$ and the residual loss function defined by
	\begin{equation*}
		J_r(\theta) =
		\int_{\mathcal{M}}\int_{\Omega}
		\big| \mathcal{L}\big(u_\theta(\bm{x},\bm{\mu});\bm{x},\bm{\mu}\big)-f(\bm{x},\bm{\mu}) \big|^2 d\bm{x} d\bm{\mu},
	\end{equation*}
	\begin{minipage}{0.7\linewidth}
		where $u_\theta$ is a neural network defined by
		\begin{equation*}
			\textcolor{red}{u_{\theta}(\bm{x},\bm{\mu}) = \varphi(\bm{x}) w_{\theta}(\bm{x},\bm{\mu}) + g(\bm{x},\bm{\mu}),}
		\end{equation*}
		with $\varphi$ a level-set function, $w_\theta$ a NN and $g$ the Dirichlet BC. 
	\end{minipage}
	\begin{minipage}{0.28\linewidth}
		\vspace{-15pt}
		\includegraphics[width=0.95\linewidth]{images/intro/levelset.png}
	\end{minipage}

	\vspace{5pt}
	\textbf{Monte-Carlo method :} Discretize the residual cost function by random process.
	\vspace{15pt}
\end{frame}


\begin{frame}{Finite Element Method}
	TODO
\end{frame}
	
	\section{Mesh-based methods}
	\subsection{Encoding/Decoding}

\begin{frame}{Encoding/Decoding - FEMs}
	\begin{itemize}[\textbullet]
		\item \textbf{Decoding :} Linear combination of piecewise polynomial function $\varphi_i$.
		\begin{equation*}
			u_\theta(x)=\mathcal{D}(\theta_u)(x) = \sum_{i=1}^{N}(\theta_u)_i\varphi_i(x)
		\end{equation*}
		$\Rightarrow$ linear decoding $\Rightarrow$ approximation space $V_N$ = vectorial space \\
		$\Rightarrow$ existence and uniqueness of the orthogonal projector
		\item \textbf{Encoding :} Optimization process.
		\begin{equation*}
			\displaystyle \theta_f=\mathcal{E}(f)=\argmin_{\theta\in\mathbb{R}^N}\int_\Omega ||f_\theta(x)-f(x)||^2 dx
		\end{equation*}
		$\Leftrightarrow$ Orthogonal projection on vector space $V_N=Vect\{\varphi_1,\dots,\varphi_N\}$.
		\begin{equation*}
			\theta_f=\mathcal{E}(f)=M^{-1}b(f)
		\end{equation*}
		with $M_{ij}=\int_\Omega \varphi_i(x)\varphi_j(x)$ and $b_i(f)=\int_\Omega \varphi_i(x)f(x)$. \refappendix{frame:encoding_fems} 
	\end{itemize}
\end{frame}

\subsection{Approximation}

\begin{frame}{Approximation}
	\textbf{Idea :} Project a certain form of the equation onto the vector space $V_N$. \\
	We introduce the residual inside $\Omega$ and on the boundary $\partial\Omega$ defined by
	\begin{equation*}
		R_{in}(v)=L(v) - f \qquad \text{and} \qquad R_{bc}(v)=v-g
	\end{equation*}
	
	\vspace{5pt}
	
	\textbf{Discretization :} Degrees of freedom problem (which also has a unique solution)
	\begin{center}
		$\displaystyle u=\argmin_{v\in H_1^0(\Omega)} J(v) \quad \longrightarrow \quad \theta_u=\argmin_{\theta\in \mathbb{R}^N} J(\theta) $
	\end{center}
	with $J$ a functional to minimize.
	
	\vspace{5pt}
	
	\textbf{Variants :} Depends on the problem form used for projection.
	
	\begin{center}
		\begin{tabular}{c|c}
			\textbf{Symmetric spatial PDE} & \textbf{Any type of PDE} \\
			Problem - Energetic form & Problem - Least-square form \\
			Galerkin projection & Galerkin Least-square projection
		\end{tabular}
	\end{center}
\end{frame}

\begin{frame}{Energetic form}
	\textbf{Discrete Minimization Problem :}
	\begin{equation}
		\displaystyle u_\theta(x)=\argmin_{v\in V_N} J(v), \qquad J(v)=J_{in}(v)+J_{bc}(v)\label{minpb_galerkin}
	\end{equation}
	with 
	\begin{equation*}
		J_{in}(v)=\frac{1}{2}\int_\Omega L(v)v - \int_\Omega fv  \qquad \text{and} \qquad J_{bc}(v)=\frac{1}{2}\int_{\partial\Omega} R_{bc}(v)^2
	\end{equation*}

	\footnotesize	
	\textit{Remark :} This form of the problem is due to the Lax-Milgram theorem as $a$ is symmetrical.
	\normalsize
	
	\footnotesize
	\begin{center}
		\begin{tcolorbox}[
			colback=white, % Couleur de fond de la boîte
			colframe=other, % Couleur du cadre de la boîte
			arc=2mm, % Rayon de l'arrondi des coins
			boxrule=0.5pt, % Épaisseur du cadre de la boîte
			breakable, enhanced jigsaw,
			width=0.8\linewidth
			]
			
			\textbf{Discrete Minimization Problem \eqref{minpb_galerkin} $\Leftrightarrow$ PDE \eqref{edp} :}
			
			\centering
			$\nabla_v \; J_{in}(v)=R_{in}(v) \; , \; \nabla_v \; J_{bc}(v)=R_{bc}(v) \qquad $ \refappendix{frame:minpb_galerkin} 
			
			\vspace{5pt}
			
			\begin{minipage}{0.1\linewidth}
				\centering
				$u_\theta$ sol \\
				of \eqref{minpb_galerkin}
			\end{minipage} $\Leftrightarrow \; \nabla_{u_\theta} \; J(u_\theta)=0 \; \Leftrightarrow \; \left\{\begin{aligned}
				&R_{in}(u_\theta)=0 \; \text{in} \; \Omega \\
				&u_\theta=g \; \text{on} \; \partial\Omega
			\end{aligned}\right. \; \Leftrightarrow$ \begin{minipage}{0.15\linewidth}
				\centering
				$u_\theta$ approx \\
				sol of \eqref{edp}
			\end{minipage}
		
			\vspace{5pt}
			
			\begin{minipage}{0.1\linewidth}
				\centering
				\textbf{Discrete} \\
				\textbf{min pb}
			\end{minipage} \; \hspace{160pt} \; \begin{minipage}{0.1\linewidth}
				\centering
				\textbf{PDE}
			\end{minipage}
		\end{tcolorbox}
	\end{center}
\end{frame}

\begin{frame}{Galerkin Projection}
	\textbf{DoFs minimization Problem :}
	\begin{equation}
		\displaystyle \theta_u=\argmin_{\theta\in\mathbb{R}^N} J(\theta), \qquad J(\theta)=J_{in}(\theta)=\frac{1}{2}\int_\Omega L(v_\theta)v_\theta - \int_\Omega fv_\theta \label{minpb_galerkin_discret}
	\end{equation}
%	with 
%	\begin{equation*}
%		
%	\end{equation*}
	
	\footnotesize	
	\textit{Remark :} Here, we are only interested in the minimisation problem on $\Omega$.
	
	\normalsize	
	
	\textbf{Galerkin projection :} Consists in resolving
	\begin{equation}
		\langle R_{in}(u_\theta(x)),\varphi_i\rangle_{L^2}=0, \qquad \forall i\in\{1,\dots,N\}\label{galerkin_proj}
	\end{equation}

	\footnotesize
	\begin{center}
		\begin{tcolorbox}[
			colback=white, % Couleur de fond de la boîte
			colframe=other, % Couleur du cadre de la boîte
			arc=2mm, % Rayon de l'arrondi des coins
			boxrule=0.5pt, % Épaisseur du cadre de la boîte
			breakable, enhanced jigsaw,
			width=\linewidth
			]
			
			\textbf{Galerkin Projection \eqref{galerkin_proj} $\Leftrightarrow$ PDE \eqref{edp} :}
			
			\centering
			$\nabla_\theta \; J(\theta)=\left(\int_\Omega R_{in}(v_\theta)\varphi_i\right)_{i=1,\dots,N} \qquad $ \refappendix{frame:galerkin_proj} 
			
			\vspace{5pt}
			
			\begin{minipage}{0.15\linewidth}
				\centering
				$u_\theta$ approx \\
				sol of \eqref{edp}
			\end{minipage} $\; \Leftrightarrow \;$	\begin{minipage}{0.1\linewidth}
				\centering
				$u_\theta$ sol \\
				of \eqref{minpb_galerkin}
			\end{minipage} $\; \Leftrightarrow \;$	\begin{minipage}{0.1\linewidth}
				\centering
				$\theta_u$ sol \\
				of \eqref{minpb_galerkin_discret}
			\end{minipage} $\Leftrightarrow \; \nabla_\theta \; J(\theta)=0 \; \Leftrightarrow$ \begin{minipage}{0.1\linewidth}
				\centering
				$u_\theta$ sol \\
				of \eqref{galerkin_proj}
			\end{minipage}
		
			\vspace{5pt}
		
			\begin{minipage}{0.1\linewidth}
				\centering
				\textbf{PDE}
			\end{minipage} \; \hspace{15pt} \; \begin{minipage}{0.1\linewidth}
				\centering
				\textbf{Discrete} \\
				\textbf{min pb}
			\end{minipage} \; \hspace{10pt} \; \begin{minipage}{0.1\linewidth}
				\centering
				\textbf{DoFs} \\
				\textbf{min pb}
			\end{minipage} \; \hspace{60pt} \; \begin{minipage}{0.1\linewidth}
				\centering
				\textbf{Galerkin} \\
				\textbf{projection}
			\end{minipage}
		\end{tcolorbox}
	\end{center}
\end{frame}

\begin{frame}{Least-Square form}
	\textbf{Discrete Minimization Problem :}
	\begin{equation*}
		\displaystyle u_\theta(x)=\argmin_{v\in V_N} J(v), \qquad J(v)=J_{in}(v)+J_{bc}(v)\label{minpb_leastsquare}
	\end{equation*}
	with 
	\begin{equation*}
		J_{in}(v)=\frac{1}{2}\int_\Omega R_{in}(v)^2  \qquad \text{and} \qquad J_{bc}(v)=\frac{1}{2}\int_{\partial\Omega} R_{bc}(v)^2
	\end{equation*}
	
%	\footnotesize
%	\begin{center}
%		\begin{tcolorbox}[
%			colback=white, % Couleur de fond de la boîte
%			colframe=other, % Couleur du cadre de la boîte
%			arc=2mm, % Rayon de l'arrondi des coins
%			boxrule=0.5pt, % Épaisseur du cadre de la boîte
%			breakable, enhanced jigsaw,
%			width=\linewidth
%			]
%			
%			\textbf{Minimization Problem \eqref{minpb_leastsquare} $\Leftrightarrow$ PDE \eqref{edp} :}
%			
%			\centering
%			$\nabla_v \; J(v)=L(R(v))\mathds{1}_\Omega+(v-g)\mathds{1}_{\partial\Omega} \qquad $ \refappendix{frame:minpb_leastsquare} 
%			
%			\vspace{5pt}
%			
%			\begin{minipage}{0.1\linewidth}
%				\centering
%				$u_\theta$ sol \\
%				of \eqref{minpb_leastsquare}
%			\end{minipage} $\Leftrightarrow \; \nabla_{u_\theta} \; J(u_\theta)=0 \; \Leftrightarrow \; \left\{\begin{aligned}
%				&L(R(u_\theta))=0 \; \text{in} \; \Omega \\
%				&R(u_\theta)=0 \; \text{on} \; \partial\Omega
%			\end{aligned}\right. \; \Leftrightarrow \; R(u_\theta)=0 \; \Leftrightarrow$ \begin{minipage}{0.1\linewidth}
%				\centering
%				$u_\theta$ sol \\
%				of \eqref{edp}
%			\end{minipage}
%			
%			\vspace{5pt}
%			
%			\begin{minipage}{0.1\linewidth}
%				\centering
%				\textbf{Min pb}
%			\end{minipage} \; \hspace{210pt} \; \begin{minipage}{0.1\linewidth}
%				\centering
%				\textbf{PDE}
%			\end{minipage}
%		\end{tcolorbox}
%	\end{center}
%\end{frame}
%
%\begin{frame}{Least-Square Galerkin Projection}
	\textbf{DoFs minimization Problem :}
	\begin{equation*}
		\displaystyle \theta_u=\argmin_{\theta\in\mathbb{R}^N} J(\theta), \qquad J(\theta)=J_{in}(\theta)=\frac{1}{2}\int_\Omega (L(v_\theta) - f)^2 \label{minpb_leastsquare_discret}
	\end{equation*}
	%	with 
	%	\begin{equation*}
		%		
		%	\end{equation*}
	
	\textbf{Least-Square Galerkin projection :} Consists in resolving
	\begin{equation*}
		\langle R_{in}(u_\theta(x)),(\nabla_\theta R_{in}(u_\theta(x)))_i\rangle_{L^2}=0, \qquad \forall i\in\{1,\dots,N\}\label{leastsquare_proj}
	\end{equation*}
	
%	\footnotesize
%	\begin{center}
%		\begin{tcolorbox}[
%			colback=white, % Couleur de fond de la boîte
%			colframe=other, % Couleur du cadre de la boîte
%			arc=2mm, % Rayon de l'arrondi des coins
%			boxrule=0.5pt, % Épaisseur du cadre de la boîte
%			breakable, enhanced jigsaw,
%			width=\linewidth
%			]
%			
%			\textbf{Least-Square Galerkin Projection \eqref{leastsquare_proj} $\Leftrightarrow$ PDE \eqref{edp} :}
%			
%			\centering
%			$\nabla_\theta \; J(\theta)=\left(\int_\Omega L(R_{in}(v_\theta))\varphi_i\right)_{i=1,\dots,N} \qquad $ \refappendix{frame:leastsquare_proj} 
%			
%			\vspace{5pt}
%			
%			\begin{minipage}{0.1\linewidth}
%				\centering
%				$u_\theta$ sol \\
%				of \eqref{edp}
%			\end{minipage} $\; \Leftrightarrow \;$	\begin{minipage}{0.1\linewidth}
%				\centering
%				$u_\theta$ sol \\
%				of \eqref{minpb_leastsquare}
%			\end{minipage} $\; \Leftrightarrow \;$	\begin{minipage}{0.1\linewidth}
%				\centering
%				$\theta_u$ sol \\
%				of \eqref{minpb_leastsquare_discret}
%			\end{minipage} $\Leftrightarrow \; \nabla_\theta \; J(\theta)=0 \; \Leftrightarrow$ \begin{minipage}{0.1\linewidth}
%				\centering
%				$u_\theta$ sol \\
%				of \eqref{leastsquare_proj}
%			\end{minipage}
%			
%			\vspace{5pt}
%			
%			\begin{minipage}{0.1\linewidth}
%				\centering
%				\textbf{PDE}
%			\end{minipage} $\; \quad \;$ \begin{minipage}{0.1\linewidth}
%				\centering
%				\textbf{Min pb}
%			\end{minipage} $\; \quad \;$ \begin{minipage}{0.1\linewidth}
%				\centering
%				\textbf{Discrete} \\
%				\textbf{min pb}
%			\end{minipage} \; \hspace{60pt} \; \begin{minipage}{0.15\linewidth}
%				\centering
%				\textbf{LS Galerkin} \\
%				\textbf{projection}
%			\end{minipage}
%		\end{tcolorbox}
%	\end{center}
\end{frame}

\begin{frame}{Steps Decomposition - FEMs}
	\begin{center}
		\renewcommand{\arraystretch}{1.5}
		\begin{tabular}{|c|c|c|c|}
			\hline
			\textbf{Encoding} & \multicolumn{2}{c|}{\textbf{Approximation}} & \textbf{Decoding} \\
			\hline
			$f \; \rightarrow \theta_f$ & \multicolumn{2}{c|}{$\theta_f \; \rightarrow \theta_u$} & $\theta_u \; \rightarrow u_\theta$ \\
			\hline
			\multirow{3}{*}{$\begin{aligned}
				\theta_f&=\mathcal{E}(f) \\
				&=M^{-1}b(f)
			\end{aligned}$} & Galerkin & LS Galerkin & \multirow{3}{*}{$\begin{aligned}
				u_\theta(x)&=\mathcal{D}(\theta_u)(x) \\
				&=\sum_{i=1}^N (\theta_u)_i\varphi_i
			\end{aligned}$} \\
			 & \small $\langle R(u_\theta),\varphi_i\rangle_{L^2}=0$ & \small $\langle R(u_\theta),(\nabla_\theta R(u_\theta))_i\rangle_{L^2}=0$ & \\
			\cline{2-3}
			 & \multicolumn{2}{c|}{$A\theta_u=B$} & \\
			 \hline
		\end{tabular}
	\end{center}

	\footnotesize
	\textit{\textbf{Example :}} Galerkin projection.
	
	\begin{minipage}{0.48\linewidth}
		For $i\in\{1,\dots,N\}$,
		\begin{align*}
			\langle R(u_\theta),\varphi_i\rangle_{L^2}&=0 \\
			\iff \quad \int_\Omega L(u_\theta)\varphi_i &= \int_\Omega f\varphi_i \\
			\iff \quad \sum_{j=1}^N(\theta_u)_j \int_\Omega \varphi_i L(\varphi_j) &= \int_\Omega f\varphi_i
		\end{align*}
	\end{minipage}
	\begin{minipage}{0.48\linewidth}
		\begin{equation*}
			A\theta_u=B \; \text{with}
		\end{equation*}
		\begin{equation*}
			A_{i,j} = \int_\Omega \varphi_i L(\varphi_j) \quad \text{,} \quad B_i =  \int_\Omega f\varphi_i
		\end{equation*}
	\end{minipage}
\end{frame}
	
	\section{Physically Informed Learning}
	\subsection{Encoding/Decoding}

\begin{frame}{Encoding/Decoding - NNs}
	\hl{A compléter !}
\end{frame}

\begin{frame}{Non-Linear Decoder - Advantages}
	\hl{A compléter !}
\end{frame}

\subsection{Approximation}

\begin{frame}{Approximation}
	\hl{A compléter !}
\end{frame}

\begin{frame}{Deep-Ritz}
	\hl{A compléter !}
\end{frame}

\begin{frame}{Standard PINNs}
	\hl{A compléter !}
\end{frame}

\begin{frame}{In practice...}
	\hl{A compléter !}
\end{frame}

\begin{frame}{Steps Decomposition - NNs}
	\begin{center}
		\renewcommand{\arraystretch}{1.5}
		\begin{tabular}{|c|c|c|c|}
			\hline
			\textbf{Encoding} & \multicolumn{2}{c|}{\textbf{Approximation}} & \textbf{Decoding} \\
			\hline
			$f \; \rightarrow \theta_f$ & \multicolumn{2}{c|}{$\theta_f \; \rightarrow \theta_u$} & $\theta_u \; \rightarrow u_\theta$ \\
			\hline
			\multirow{3}{*}{$\begin{aligned}
					\theta_f&=\mathcal{E}(f) \\
					&=M^{-1}b(f)
				\end{aligned}$} & Galerkin & LS Galerkin & \multirow{3}{*}{$\begin{aligned}
					u_\theta(x)&=\mathcal{D}_\theta(x) \\
					&=\sum_{i=1}^N (\theta_u)_i\varphi_i
				\end{aligned}$} \\
			& \small $\langle R(u_\theta),\varphi_i\rangle_{L^2}=0$ & \small $\langle R(u_\theta),(\nabla_\theta R(u_\theta))_i\rangle_{L^2}=0$ & \\
			\cline{2-3}
			& \multicolumn{2}{c|}{$A\theta_u=B$} & \\
			\hline
		\end{tabular}
	\end{center}
	
	\hl{A compléter !}
\end{frame}


	
	\section{Our hybrid method}
	\begin{frame}{$\phi$-FEM Method}
	\textbf{Main ideas :} \hspace{30pt} \refappendix{frame:phifem}  \small
	
	\begin{minipage}[t]{0.48\linewidth}
		\begin{itemize}[\textbullet]
			\item Domain defined by a LevelSet Function $\phi$.
		\end{itemize}
		\centering
		\pgfimage[width=0.6\linewidth]{images/hybrid/PhiFEM_level_set.png}
	\end{minipage} \hfill
	\begin{minipage}[t]{0.48\linewidth}
		\begin{itemize}[\textbullet]
			\item We are looking for $w$ such that $u=\phi w+g$. \\
			Thus, the decoder is written as
			\begin{equation*}
				u_\theta(x)=\mathcal{D}_{\theta_w}(x) = \phi(x)\sum_{i=1}^{N}(\theta_w)_i\varphi_i+g(x)
			\end{equation*}
		\end{itemize}
	\end{minipage}

	\begin{itemize}[\textbullet]
		\item Mesh of a fictitious domain containing $\Omega$.
	\end{itemize}
	\begin{center}
		\begin{minipage}{0.43\linewidth}
			\centering
			\pgfimage[width=\linewidth]{images/more/PhiFEM_domain.png}
		\end{minipage} \hfill
		\begin{minipage}{0.1\linewidth}
			\centering
			\pgfimage[width=\linewidth]{images/more/PhiFEM_fleche.png} 
		\end{minipage} \hfill
		\begin{minipage}{0.43\linewidth}
			\centering
			\pgfimage[width=\linewidth]{images/more/PhiFEM_domain_considered.png}
		\end{minipage}
	\end{center}
\end{frame}

\begin{frame}{Impose exact BC in PINNs}
	Considering the least squares form of our PDE, we impose the exact boundary conditions by writing our solution as
	\begin{equation*}
		u_\theta=\phi w_\theta + g
	\end{equation*}
	where $w_\theta$ is our decoder (defined by a neural network such as an MLP).
	
	We then consider the same minimization problem by removing the cost function associated with the boundary
	\begin{equation*}
		\displaystyle \theta_u=\argmin_{\theta\in\mathbb{R}^N} J_{in}(\theta)+\Ccancel[red]{J_{bc}(\theta)}
	\end{equation*}
	with 
	\begin{equation*}
		J_{in}(\theta)=\frac{1}{2}\int_\Omega (L(\phi w_\theta + g) - f)^2  \qquad \text{and} \qquad \Ccancel[red]{J_{bc}(\theta)=\frac{1}{2}\int_{\partial\Omega} (v_\theta-g)^2}
	\end{equation*}	
%	\vspace{-20pt}
%	\begin{figure}[htb]
%		\hspace{-105pt}
%		\begin{tikzpicture}
%			\draw[->, blue, line width=1pt] (0,1) -- (0.15,0.3);
%		\end{tikzpicture} 
%	\end{figure}
%	\begin{equation*}
%		J_{in}(\theta)=\frac{1}{2}\int_\Omega (L(\phi w_\theta + g) - f)^2
%	\end{equation*}	
	\textbf{Connection :} \qquad $\phi$-FEM \hspace{5pt} // \hspace{5pt} Exact BC in PINNs
\end{frame}
	
	\section{Conclusion} %perspectives
	
	\begin{frame}{Conclusion - What has been seen }
		\begin{itemize}[\textbullet]
			\item "Physical Informed Learning" methods are simply an extension of classic numerical methods such as FEM, where the decoder belongs to a variety (whose properties are different from those of vector spaces).
			\item These approaches have real advantages in high dimensions, particularly in the context of parametric PDEs.
			\item Moreover, as they are mesh-free methods, they have a major advantage in the context of complex geometries.
		\end{itemize}
	\end{frame}

	\begin{frame}[label={lastslide}]{Conclusion - Our hybrid approach}
		\textbf{Interest of our approach:} 
		\begin{itemize}[\textbullet]
			\item It combines
			\begin{itemize}[\ding{217}]
				\item Speed of neural networks in predicting a solution
				\item Precision of FEM methods to correct and certify the prediction of the neural network (which can be completely wrong, on an unknown dataset for example)
			\end{itemize}
			\item In the context of complex geometry (or in application domains such as real-time or shape optimisation), like NNs, $\phi$-FEM makes it possible to avoid mesh (re-)generation.
		\end{itemize}
		
		\textbf{Current results:}
		\begin{itemize}[\textbullet]
			\item Encouraging results on simple geometries \refappendix{frame:results}
			\item Difficulties on complex geometries - Importance of the regularity of the LevelSet function \\
			$\rightarrow$ Next step: learning levelset functions (Eikonal equation)
		\end{itemize}
	\end{frame}

	\begin{frame}
		\vfill
		\centering
		\LARGE Thank you !
		\vfill
	\end{frame}
	
%	\section{Bibliography}
	
	{\setbeamertemplate{footline}{} 
		\begin{frame}{Bibliography}
			\small
			% \vspace{30pt}
			% \setstretch{0.2}
			% \AtNextBibliography{\small}
			\printbibliography[heading=none]
		\end{frame}
	}
	\addtocounter{framenumber}{-1} 
	
	\appendix
	
	\section{Mesh-based methods}

\begin{frame}{\appendixname~\theappendixframenumber~: Encoding - FEMs}\labelappendixframe{frame:encoding_fems}
	We want to project $f$ onto the vector subspace $V_N$ so that $f_\theta = p_{V_N}(f)$ \\	
	then $\forall i \in \{1,\dots,N\}$, we have
	\begin{align*}
		&\quad \langle f_\theta - f, \varphi_i \rangle = 0 \\
		\iff &\quad \langle f_\theta, \varphi_i \rangle = \langle f, \varphi_i \rangle \quad  \\
		\iff &\quad \sum_{j=1}^N(\theta_f)_j \langle \varphi_j, \varphi_i\rangle = \langle f, \varphi_i \rangle \\ 
		\iff &\quad M \theta_f = b(f) \\
		\iff &\quad \theta_f = M^{-1} b(f)
	\end{align*}	
	with 
	\begin{align*}
		M_{ij} &= \langle \varphi_i, \varphi_j\rangle = \int_{\Omega} \varphi_i(x) \varphi_j(x) \, dx \\
		b_i(f) &= \langle f, \varphi_i \rangle = \int_{\Omega} f(x) \varphi_i(x) \, dx
	\end{align*}	
\end{frame}
\addtocounter{appendixframenumber}{1}

\begin{frame}[allowframebreaks]{\appendixname~\theappendixframenumber~: Energetic form} \labelappendixframe{frame:minpb_galerkin}
	Let's compute the gradient of $J$ with respect to $v$ with
	\begin{equation*}
		J(v)=J_{in}(v)+J_{bc}(v)=\left(\frac{1}{2}\int_\Omega L(v)v - \int_\Omega fv\right) + \left(\frac{1}{2}\int_\Omega R_{bc}(v)^2\right)
	\end{equation*}

	\begin{itemize}[\textbullet]
		\item First, let's calculate the differential of $J_{in}$ with respect to $v$.
		\begin{align*}
			J_{in}(v+\epsilon h)=\frac{1}{2} \int_{\Omega} (A\nabla(v+\epsilon h)) \cdot \nabla(v+\epsilon h) + c(v+\epsilon h)^2 - \int_{\Omega} f(v+\epsilon h)
		\end{align*}
		
		By bilinearity of the scalar product and by symmetry of $A$, we finally obtain
		\begin{equation*}
			\mathcal{D}J_{in}(v)\cdot h = \lim_{\epsilon\rightarrow 0}\frac{J_{in}(v+\epsilon h)-J_{in}(v)}{\epsilon} = \int_{\Omega} (-\nabla\cdot(A\nabla v) + cv - f)h
		\end{equation*}
		
		And thus
		\begin{equation*}
			\nabla_v \; J_{in}(v) = L(v) - f = R_{in}(v)
		\end{equation*}
	
		\newpage
		
		\item In the same way, we can compute the differential of $J_{bc}$ with respect to $v$.
		\begin{align*}
			J_{bc}(v+\epsilon h)=\frac{1}{2} \int_{\Omega} v^2+2\epsilon vh +\epsilon^2 h^2 - 2vg - 2\epsilon hg+g^2
		\end{align*}
		
		Then
		\begin{align*}
			\mathcal{D}J_{bc}(v)\cdot h =  \lim_{\epsilon\rightarrow 0}\frac{J_{bc}(v+\epsilon h)-J_{bc}(v)}{\epsilon} = \int_{\Omega} v^2 - hg
		\end{align*}
		
		And thus
		\begin{align*}
			\nabla_v \; J_{bc}(v) = (v-g) = R_{bc}(v) 
		\end{align*}
	\end{itemize}
	
	Finally
	\begin{equation*}
		\nabla_v \; J(v) = \nabla_v \; J_{i}(v) + \nabla_v \; J_{bc}(v) = R(v)
	\end{equation*}
\end{frame}
\addtocounter{appendixframenumber}{1}

\begin{frame}{\appendixname~\theappendixframenumber~: Galerkin Projection}\labelappendixframe{frame:galerkin_proj}
	Let's compute the gradient of $J$ with respect to $\theta$ with
	\begin{equation*}
		J(\theta)=J_{in}(\theta)=\frac{1}{2}\int_\Omega L(u_\theta)v_\theta - \int_\Omega fv_\theta
	\end{equation*}
	First, we define
	\begin{equation*}
		v_\theta=\sum_{i=1}^{N} \theta_i \varphi_i=\theta\cdot\varphi \qquad \text{and} \qquad v_{\theta+\epsilon h}=(\theta+\epsilon h)\cdot\varphi=v_\theta+\epsilon v_h
	\end{equation*}
	Then since $A$ is symmetric
	\begin{equation*}
		\mathcal{D}J(\theta)\cdot h =\int_\Omega R(v_\theta)v_h =\sum_{i=1}^N h_i\int_\Omega R(v_\theta)\varphi_i
	\end{equation*}
	Finally
	\begin{align*}
		\nabla_\theta \; J(\theta) = \left(\int_\Omega R(v_\theta)\varphi_i\right)_{i=1,\dots,N}
	\end{align*}
\end{frame}
\addtocounter{appendixframenumber}{1}

\begin{frame}[allowframebreaks]{\appendixname~\theappendixframenumber~: Least-Square form}\labelappendixframe{frame:minpb_leastsquare}
	Let's compute the gradient of $J$ with respect to $v$ with
	\begin{equation*}
		J(v)=J_{in}(v)+J_{bc}(v)=\left(\frac{1}{2}\int_\Omega R_{in}(v)^2\right)=\left(\frac{1}{2}\int_\Omega R_{bc}(v)^2\right)
	\end{equation*}
	\begin{itemize}[\textbullet]
		\item First, let's calculate the differential of $J_{in}$ with respect to $v$.
		\begin{align*}
			\mathcal{D}J_{in}(v)\cdot h &= \langle \nabla\cdot(A\nabla h), \nabla\cdot(A\nabla v) - cv +f \rangle+\langle ch, -\nabla\cdot(A\nabla v) + cv - f \rangle \\ 
			&= -\langle \nabla\cdot(A\nabla h), R_{in}(v) \rangle+\langle ch, R_{in}(v)\rangle \\ 
			&= \langle -\nabla\cdot(A\nabla R_{in}(v))+cR_{in}(v), h \rangle \\
			&= \langle L(R_{in}(v)), h \rangle
		\end{align*}
		And thus		
		\begin{equation*}
			\nabla_v \; J_{in}(v) = L(R_{in}(v))
		\end{equation*}
	
		\newpage
		
		\item In the same way, we can compute the differential of $J_{bc}$ with respect to $v$.
		\begin{align*}
			J_{bc}(v+\epsilon h)=\frac{1}{2} \int_{\Omega} v^2+2\epsilon vh +\epsilon^2 h^2 - 2vg - 2\epsilon hg+g^2
		\end{align*}
		
		Then
		\begin{align*}
			\mathcal{D}J_{bc}(v)\cdot h =  \lim_{\epsilon\rightarrow 0}\frac{J_{bc}(v+\epsilon h)-J_{bc}(v)}{\epsilon} = \int_{\Omega} v^2 - hg
		\end{align*}
		
		And thus
		\begin{align*}
			\nabla_v \; J_{bc}(v) = (v-g) = R_{bc}(v) 
		\end{align*}
	\end{itemize}
	
	Finally
	\begin{equation*}
		\nabla_v \; J(v) = L(R(v))\mathds{1}_\Omega + (v-g)\mathds{1}_{\partial\Omega}
	\end{equation*}
	
\end{frame}
\addtocounter{appendixframenumber}{1}

\begin{frame}{\appendixname~\arabic{appendixframenumber}~: LS Galerkin Projection}\labelappendixframe{frame:leastsquare_proj}
	Let's compute the gradient of $J$ with respect to $\theta$ with
	\begin{equation*}
		J(\theta)=J_{in}(\theta)=\frac{1}{2}\int_\Omega (L(u_\theta) - f)^2
	\end{equation*}
	First, we define
	\begin{equation*}
		v_\theta=\sum_{i=1}^{N} \theta_i \varphi_i=\theta\cdot\varphi \qquad \text{and} \qquad v_{\theta+\epsilon h}=(\theta+\epsilon h)\cdot\varphi=v_\theta+\epsilon v_h
	\end{equation*}
	Then since $A$ is symmetric
	\begin{equation*}
		\mathcal{D}J(\theta)\cdot h = \int_\Omega L(R(v_\theta))v_h = \sum_{i=1}^N h_i\int_\Omega L(R(v_\theta))\varphi_i
	\end{equation*}
	Finally
	\begin{align*}
		\nabla_\theta J(\theta) = \left(\int_\Omega L(R(v_\theta))\varphi_i\right)_{i=1,\dots,N}
	\end{align*}
\end{frame}
\addtocounter{appendixframenumber}{1}

\section{Physically Informed Learning}

\begin{frame}{\appendixname~\theappendixframenumber~: ADAM Method}\labelappendixframe{frame:adam}
	\hl{A compléter !}
\end{frame}
\addtocounter{appendixframenumber}{1}
	
\end{document}
