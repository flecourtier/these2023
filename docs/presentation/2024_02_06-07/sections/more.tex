\section{Mesh-based methods}

\begin{frame}{\appendixname~\theappendixframenumber~: Encoding - FEMs}\labelappendixframe{frame:encoding_fems}
	We want to project $f$ onto the vector subspace $V_N$ so that $f_\theta = p_{V_N}(f)$ \\	
	then $\forall i \in \{1,\dots,N\}$, we have
	\begin{align*}
		&\quad \langle f_\theta - f, \varphi_i \rangle = 0 \\
		\iff &\quad \langle f_\theta, \varphi_i \rangle = \langle f, \varphi_i \rangle \quad  \\
		\iff &\quad \sum_{j=1}^N(\theta_f)_j \langle \varphi_j, \varphi_i\rangle = \langle f, \varphi_i \rangle \\ 
		\iff &\quad M \theta_f = b(f) \\
		\iff &\quad \theta_f = M^{-1} b(f)
	\end{align*}	
	with 
	\begin{align*}
		M_{ij} &= \langle \varphi_i, \varphi_j\rangle = \int_{\Omega} \varphi_i(x) \varphi_j(x) \, dx \\
		b_i(f) &= \langle f, \varphi_i \rangle = \int_{\Omega} f(x) \varphi_i(x) \, dx
	\end{align*}	
\end{frame}
\addtocounter{appendixframenumber}{1}

\begin{frame}[allowframebreaks]{\appendixname~\theappendixframenumber~: Energetic form} \labelappendixframe{frame:minpb_galerkin}
	Let's compute the gradient of $J$ with respect to $v$ with
	\begin{equation*}
		J(v)=J_{in}(v)+J_{bc}(v)=\left(\frac{1}{2}\int_\Omega L(v)v - \int_\Omega fv\right) + \left(\frac{1}{2}\int_{\partial\Omega} R_{bc}(v)^2\right)
	\end{equation*}

	\begin{itemize}[\textbullet]
		\item First, let's calculate the differential of $J_{in}$ with respect to $v$.
		\begin{align*}
			J_{in}(v+\epsilon h)=\frac{1}{2} \int_{\Omega} (A\nabla(v+\epsilon h)) \cdot \nabla(v+\epsilon h) + c(v+\epsilon h)^2 - \int_{\Omega} f(v+\epsilon h)
		\end{align*}
		
		By bilinearity of the scalar product and by symmetry of $A$, we finally obtain
		\begin{equation*}
			\mathcal{D}J_{in}(v)\cdot h = \lim_{\epsilon\rightarrow 0}\frac{J_{in}(v+\epsilon h)-J_{in}(v)}{\epsilon} = \int_{\Omega} (-\nabla\cdot(A\nabla v) + cv - f)h
		\end{equation*}
		
		And thus
		\begin{equation*}
			\nabla_v \; J_{in}(v) = L(v) - f = R_{in}(v)
		\end{equation*}
	
		\newpage
		
		\item In the same way, we can compute the differential of $J_{bc}$ with respect to $v$.
		\begin{align*}
			J_{bc}(v+\epsilon h)=\frac{1}{2} \int_{\partial\Omega} v^2+2\epsilon vh +\epsilon^2 h^2 - 2vg - 2\epsilon hg+g^2
		\end{align*}
		
		Then
		\begin{align*}
			\mathcal{D}J_{bc}(v)\cdot h =  \lim_{\epsilon\rightarrow 0}\frac{J_{bc}(v+\epsilon h)-J_{bc}(v)}{\epsilon} = \int_{\partial\Omega} (v-g)h
		\end{align*}
		
		And thus
		\begin{align*}
			\nabla_v \; J_{bc}(v) = (v-g) = R_{bc}(v) 
		\end{align*}
	\end{itemize}
	
	Finally
	\begin{equation*}
		\nabla_v \; J(v) = \nabla_v \; J_{i}(v) + \nabla_v \; J_{bc}(v) = R(v)
	\end{equation*}
\end{frame}
\addtocounter{appendixframenumber}{1}

\begin{frame}{\appendixname~\theappendixframenumber~: Galerkin Projection}\labelappendixframe{frame:galerkin_proj}
	Let's compute the gradient of $J$ with respect to $\theta$ with
	\begin{equation*}
		J(\theta)=J_{in}(\theta)=\frac{1}{2}\int_\Omega L(u_\theta)v_\theta - \int_\Omega fv_\theta
	\end{equation*}
	First, we define
	\begin{equation*}
		v_\theta=\sum_{i=1}^{N} \theta_i \varphi_i=\theta\cdot\varphi \qquad \text{and} \qquad v_{\theta+\epsilon h}=(\theta+\epsilon h)\cdot\varphi=v_\theta+\epsilon v_h
	\end{equation*}
	Then since $A$ is symmetric
	\begin{equation*}
		\mathcal{D}J(\theta)\cdot h =\int_\Omega R(v_\theta)v_h =\sum_{i=1}^N h_i\int_\Omega R(v_\theta)\varphi_i
	\end{equation*}
	Finally
	\begin{align*}
		\nabla_\theta \; J(\theta) = \left(\int_\Omega R(v_\theta)\varphi_i\right)_{i=1,\dots,N}
	\end{align*}
\end{frame}
\addtocounter{appendixframenumber}{1}

\begin{frame}[allowframebreaks]{\appendixname~\theappendixframenumber~: Least-Square form}\labelappendixframe{frame:minpb_leastsquare}
	Let's compute the gradient of $J$ with respect to $v$ with
	\begin{equation*}
		J(v)=J_{in}(v)+J_{bc}(v)=\left(\frac{1}{2}\int_\Omega R_{in}(v)^2\right)+\left(\frac{1}{2}\int_{\partial\Omega} R_{bc}(v)^2\right)
	\end{equation*}
	\begin{itemize}[\textbullet]
		\item First, let's calculate the differential of $J_{in}$ with respect to $v$.
		\begin{align*}
			\mathcal{D}J_{in}(v)\cdot h &= \langle \nabla\cdot(A\nabla h), \nabla\cdot(A\nabla v) - cv +f \rangle+\langle ch, -\nabla\cdot(A\nabla v) + cv - f \rangle \\ 
			&= -\langle \nabla\cdot(A\nabla h), R_{in}(v) \rangle+\langle ch, R_{in}(v)\rangle \\ 
			&= \langle -\nabla\cdot(A\nabla R_{in}(v))+cR_{in}(v), h \rangle \\
			&= \langle L(R_{in}(v)), h \rangle
		\end{align*}
		And thus		
		\begin{equation*}
			\nabla_v \; J_{in}(v) = L(R_{in}(v))
		\end{equation*}
	
		\newpage
		
		\item In the same way, we can compute the differential of $J_{bc}$ with respect to $v$.
		\begin{align*}
			J_{bc}(v+\epsilon h)=\frac{1}{2} \int_{\partial\Omega} v^2+2\epsilon vh +\epsilon^2 h^2 - 2vg - 2\epsilon hg+g^2
		\end{align*}
		
		Then
		\begin{align*}
			\mathcal{D}J_{bc}(v)\cdot h =  \lim_{\epsilon\rightarrow 0}\frac{J_{bc}(v+\epsilon h)-J_{bc}(v)}{\epsilon} = \int_{\partial\Omega} (v-g)h
		\end{align*}
		
		And thus
		\begin{align*}
			\nabla_v \; J_{bc}(v) = (v-g) = R_{bc}(v) 
		\end{align*}
	\end{itemize}
	
	Finally
	\begin{equation*}
		\nabla_v \; J(v) = L(R(v))\mathds{1}_\Omega + (v-g)\mathds{1}_{\partial\Omega}
	\end{equation*}
	
\end{frame}
\addtocounter{appendixframenumber}{1}

\begin{frame}{\appendixname~\arabic{appendixframenumber}~: LS Galerkin Projection}\labelappendixframe{frame:leastsquare_proj}
	Let's compute the gradient of $J$ with respect to $\theta$ with
	\begin{equation*}
		J(\theta)=J_{in}(\theta)=\frac{1}{2}\int_\Omega (L(u_\theta) - f)^2
	\end{equation*}
	First, we define
	\begin{equation*}
		v_\theta=\sum_{i=1}^{N} \theta_i \varphi_i=\theta\cdot\varphi \qquad \text{and} \qquad v_{\theta+\epsilon h}=(\theta+\epsilon h)\cdot\varphi=v_\theta+\epsilon v_h
	\end{equation*}
	Then since $A$ is symmetric
	\begin{equation*}
		\mathcal{D}J(\theta)\cdot h = \int_\Omega L(R(v_\theta))v_h = \sum_{i=1}^N h_i\int_\Omega L(R(v_\theta))\varphi_i
	\end{equation*}
	Finally
	\begin{align*}
		\nabla_\theta J(\theta) = \left(\int_\Omega L(R(v_\theta))\varphi_i\right)_{i=1,\dots,N}
	\end{align*}
\end{frame}
\addtocounter{appendixframenumber}{1}

\section{Physically Informed Learning}

\begin{frame}{\appendixname~\theappendixframenumber~: ADAM Method}\labelappendixframe{frame:adam}
	Adam = Adaptive Moment Estimation" - combine les idées du moment et de RMSProp.

	\begin{align*}
		&1: \qquad m \leftarrow \frac{\beta_1 m +  (1-\beta_1) \nabla f_{x}}{1-\beta_1^T}\\
		&2: \qquad s \leftarrow \frac{\beta_2 s +  (1-\beta_2) \nabla^2 f_{x}}{1-\beta_2^T}\\
		&3: \qquad x \leftarrow x-  \ell \frac{m }{\sqrt{s+\epsilon}} 
	\end{align*}

	with 
	\begin{itemize}[\textbullet]
		\item $T$ the number of iteration (starting at 1) 
		\item $\epsilon$ a smoothing paramete
		\item $\beta_i \in ]0,1[$ which convergence quickly to 0. 
	\end{itemize}	
\end{frame}
\addtocounter{appendixframenumber}{1}

\section{Our hybrid method}

\subsection{\appendixname~\theappendixframenumber~: $\phi$-FEM Method}\labelappendixframe{frame:phifem}

\begin{frame}{\appendixname~\theappendixframenumber~: Problem}
	Let $u=\phi w+g$ such that
	$$\left\{\begin{aligned}
		-\Delta u &= f, \; \text{in } \Omega, \\
		u&=g, \; \text{on } \Gamma, \\
	\end{aligned}\right.$$
	where $\phi$ is the level-set function and $\Omega$ and $\Gamma$ are given by :
	\begin{center}
		\pgfimage[width=0.5\linewidth]{images/more/PhiFEM_level_set.png}
	\end{center}
	The level-set function $\phi$ is supposed to be known on $\mathbb{R}^d$ and sufficiently smooth. \\
	For instance, the signed distance to $\Gamma$ is a good candidate.
	
	\vspace{5pt}
	
	\footnotesize
	\textit{Remark :} Thanks to $\phi$ and $g$, the conditions on the boundary are respected.
\end{frame}

\begin{frame}{\appendixname~\theappendixframenumber~: Fictitious domain}
	\setstretch{0.5}
	
	\vspace{10pt}
	
	\begin{center}
		\begin{minipage}{0.43\linewidth}
			\centering
			\pgfimage[width=\linewidth]{images/more/PhiFEM_domain.png}
		\end{minipage} \hfill
		\begin{minipage}{0.1\linewidth}
			\centering
			\pgfimage[width=\linewidth]{images/more/PhiFEM_fleche.png} 
		\end{minipage} \hfill
		\begin{minipage}{0.43\linewidth}
			\centering
			\pgfimage[width=\linewidth]{images/more/PhiFEM_domain_considered.png}
		\end{minipage}
	\end{center}
	
	\begin{enumerate}[\ding{217}]
		\item $\phi_h$ : approximation of $\phi$ \\ 
		\item $\Gamma_h=\{\phi_h=0\}$ : approximate boundary of $\Gamma$
		\item $\Omega_h$ : computational mesh
		\item $\partial\Omega_h$ : boundary of $\Omega_h$ ($\partial\Omega_h \ne \Gamma_h$)
	\end{enumerate}	
	
	% \begin{minipage}{0.6\linewidth}
		%     \begin{enumerate}[\ding{217}]
			%         \item $\mathcal{O}$ : fictitious domain such that $\Omega\subset\mathcal{O}$
			%         \item $\mathcal{T}_h^\mathcal{O}$ : simple quasi-uniform mesh on $\mathcal{O}$
			%         \item $\phi_h=I_{h,\mathcal{O}}^{(l)}(\phi)\in V_{h,\mathcal{O}}^{(l)}$ : approximation of $\phi$ \\ 
			%         with $I_{h,\mathcal{O}}^{(l)}$ the standard Lagrange interpolation operator on
			%         $$V_{h,\mathcal{O}}^{(l)}=\left\{v_h\in H^1(\mathcal{O}):v_{h|_T}\in\mathbb{P}_l(T) \;  \forall T\in\mathcal{T}_h^\mathcal{O}\right\}$$
			%         \item $\Gamma_h=\{\phi_h=0\}$ : approximate boundary of $\Gamma$
			%         \item $\mathcal{T}_h$ : sub-mesh of $\mathcal{T}_h^\mathcal{O}$ defined by
			%         $$\mathcal{T}_h=\left\{T\in \mathcal{T}_h^\mathcal{O}:T\cap\{\phi_h<0\}\ne\emptyset\right\}$$
			%         \item $\Omega_h$ : domain covered by the $\mathcal{T}_h$ mesh defined by
			%         $$\Omega_h=\left(\cup_{T\in\mathcal{T}_h}T\right)^O$$
			%         ($\partial\Omega_h$ its boundary)
			%     \end{enumerate}			
		% \end{minipage}
	
	\footnotesize
	\; \\
	\textit{Remark :} $n_{vert}$ will denote the number of vertices in each direction for $\mathcal{O}$
\end{frame}

\begin{frame}{\appendixname~\theappendixframenumber~: Facets and Cells sets}
	
	\vspace{15pt}
	
	\begin{center}
		\begin{minipage}{0.48\linewidth}
			\centering
			\pgfimage[width=\linewidth]{images/more/PhiFEM_boundary_cells.png}
		\end{minipage} \hfill
		\begin{minipage}{0.48\linewidth}
			\centering
			\pgfimage[width=\linewidth]{images/more/PhiFEM_boundary_edges.png}
		\end{minipage}
	\end{center}
	
	\begin{enumerate}[\ding{217}]
		\item $\mathcal{T}_h^\Gamma$ : mesh elements cut by $\Gamma_h$
		\item $\mathcal{F}_h^\Gamma$ : collects the interior facets of $\mathcal{T}_h^\Gamma$ \\
		(either cut by $\Gamma_h$ or belonging to a cut mesh element)
	\end{enumerate}
	
	% \begin{minipage}{0.6\linewidth}
		%     \begin{enumerate}[\ding{217}]
			%         \item $\mathcal{T}_h^\Gamma\subset \mathcal{T}_h$ : contains the mesh elements cut by $\Gamma_h$, i.e. 
			%         \begin{equation*}
				%             \mathcal{T}_h^\Gamma=\left\{T\in\mathcal{T}_h:T\cap\Gamma_h\ne\emptyset\right\},
				%         \end{equation*}
			%         \item $\Omega_h^\Gamma$ : domain covered by the $\mathcal{T}_h^\Gamma$ mesh, i.e.
			%         \begin{equation*}
				%             \Omega_h^\Gamma=\left(\cup_{T\in\mathcal{T}_h^\Gamma}T\right)^O
				%         \end{equation*}
			%         \item $\mathcal{F}_h^\Gamma$ : collects the interior facets of $\mathcal{T}_h$ either cut by $\Gamma_h$ or belonging to a cut mesh element, i.e.
			%         \begin{align*}
				%             \mathcal{F}_h^\Gamma=\left\{E\;(\text{an internal facet of } \mathcal{T}_h) \text{ such that }\right. \\
				%             \left. \exists T\in \mathcal{T}_h:T\cap\Gamma_h\ne\emptyset \text{ and } E\in\partial T\right\}
				%         \end{align*}
			%     \end{enumerate}
		% \end{minipage}
\end{frame}

\begin{frame}{\appendixname~\theappendixframenumber~: Poisson problem}
	\textbf{Approach Problem :} Find $w_h\in V_h^{(k)}$ such that 
	$$a_h(w_h,v_h) = l_h(v_h) \quad \forall v_h \in V_h^{(k)}$$
	where
	$$a_h(w,v)=\int_{\Omega_h} \nabla (\phi_h w) \cdot \nabla (\phi_h v) - \int_{\partial\Omega_h} \frac{\partial}{\partial n}(\phi_h w)\phi_h v+\fcolorbox{blue}{white}{$G_h(w,v)$},$$
	$$l_h(v)=\int_{\Omega_h} f \phi_h v + \fcolorbox{blue}{white}{$G_h^{rhs}(v)$} \qquad \qquad \color{blue}\text{Stabilization terms}$$
	and 
	$$V_h^{(k)}=\left\{v_h\in H^1(\Omega_h):v_{h|_T}\in\mathbb{P}_k(T), \; \forall T\in\mathcal{T}_h\right\}.$$
	For the non homogeneous case, we replace
	$$u=\phi w \quad \rightarrow \quad u=\phi w+g$$ 
	by supposing that $g$ is currently given over the entire $\Omega_h$.
\end{frame}

\begin{frame}{\appendixname~\theappendixframenumber~: Stabilization terms}
	\begin{center}
		\centering
		\pgfimage[width=\linewidth]{images/more/PhiFEM_stab_terms.png}
	\end{center}
	\small
	\underline{1st term :} ensure continuity of the solution by penalizing gradient jumps. \\
	$\rightarrow$ Ghost penalty [Burman, 2010] \\
	\underline{2nd term :} require the solution to verify the strong form on $\Omega_h^\Gamma$. \\
	\normalsize
	\textbf{Purpose :} 
	\begin{enumerate}[\ding{217}]
		\item reduce the errors created by the "fictitious" boundary 
		\item ensure the correct condition number of the finite element matrix
		\item restore the coercivity of the bilinear scheme
	\end{enumerate}
\end{frame}
\addtocounter{appendixframenumber}{1}

\subsection{\appendixname~\theappendixframenumber~: Results}\labelappendixframe{frame:results}

\begin{frame}{\appendixname~\theappendixframenumber~: Results}
	\hl{A compléter !}
\end{frame}
\addtocounter{appendixframenumber}{1}
