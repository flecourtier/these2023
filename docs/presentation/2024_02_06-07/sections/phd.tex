\begin{frame}{Explanation}
    \textbf{Context :} Need $u_\theta\in\mathbb{P}^k$ with $k$ of high degree

    \begin{center}
        \begin{minipage}{0.28\linewidth}
            \centering
            FNO \\
            (on a regular grid) 
        \end{minipage} $\rightarrow$ \begin{minipage}{0.35\linewidth}
            \centering
            NN which can predict \\
            solution at any point
        \end{minipage}
    \end{center}

    \textbf{Solutions :}

    \vspace{2pt}
    
    \begin{minipage}{0.48\linewidth}
        \; \\
        \textbf{1. MLP} - Multi-Layer Perceptron \\
        (= Fully connected)

        \centering
        \pgfimage[width=0.9\linewidth]{images/phd/MLP_schema.png}

        \raggedright
        \textit{Problem :} As the prediction is injected into an FEM solver, the accuracy of the derivatives is very important.
    \end{minipage} \quad
    \begin{minipage}{0.48\linewidth}
        \textbf{2. PINNs} - MLP with a physical loss
        \begin{center}
            $loss = mse(\Delta (\phi(x_i,y_i)w_{\theta,i})+f_i)$

            \vspace{1.5pt}
        
            \pgfimage[width=0.5\linewidth]{images/phd/PINNs_explanation.png}
        \end{center}

        with $(x_i,y_i)\in\mathcal{O}$.

        \textit{Remark :} We impose exact boundary conditions. \\
    \end{minipage}
\end{frame}

\begin{frame}{PINNs Training}
    We consider the solution on the circle defined in \\eqref{sol3} and defined by
    \begin{equation*}
        u_{ex}(x,y)=\phi(x,y)\sin(x)\exp(y)
    \end{equation*}
    We train a PINNs with 4 layers of 20 neurons over 10000 epochs (with $n_{pts}=2000$ points selected uniformly over $\mathcal{O}$).

    \centering
    \pgfimage[width=0.9\linewidth]{images/phd/solution_config0.png}

    {\fontencoding{U}\fontfamily{futs}\selectfont\char 49\relax} We consider a single problem ($f$ fixed) on a single geometry ($\phi$ fixed).

    \raggedright    
    $||u_{ex}-u_\theta||_{0,\Omega}^{(rel)}\approx 2.81e-3$
\end{frame}

\begin{frame}{Derivatives}
    \; \\
    
    \centering
    \pgfimage[width=0.75\linewidth]{images/phd/derivatives_x.png}
\end{frame}

\begin{frame}{Correction by addition}
    $u_\theta\in\mathbb{P}^{10} \; \rightarrow \; \tilde{u}\in\mathbb{P}^1$

    \begin{minipage}{0.5\linewidth}
        \centering
        \pgfimage[width=\linewidth]{images/phd/time_precision.png}
    
        \raggedright
        FEM / $\phi$-FEM : $n_{vert}\in\{8,16,32,64,128\}$
        
        Corr : $n_{vert}\in\{5,10,15,20,25,30\}$
    \end{minipage} \quad
    \begin{minipage}{0.46\linewidth}
        \centering
        \pgfimage[width=\linewidth]{images/phd/results_time_1e-4.png}

        \small\raggedright
        \begin{enumerate}[\textbullet]
        	\item \textbf{mesh} - FEM : construct the mesh \\
        	($\phi$-FEM : construct cell/facet sets) \\
        	\item \textbf{u\_PINNs} - get $u_\theta$ in $\mathbb{P}^{10}$ freedom degrees \\
        	\item \textbf{assemble} - assemble the FE matrix \\
        	\item \textbf{solve} - resolve the linear system
        \end{enumerate}
    \end{minipage}

    \small
    \textit{Remark :} The stabilisation parameter $\sigma$ of the $\phi$-FEM method has a major impact on the error obtained.
\end{frame}