\subsection{Correction Methods}

% \begin{frame}{Correction Methods}
%     We are given $u_\theta$ the FNO prediction (for the problem under consideration).
    
%     \begin{minipage}{0.48\linewidth}
%         \textbf{By adding :}
    
%         We will consider
%         \begin{equation*}
%             \tilde{u}=u_\theta+\tilde{C}
%         \end{equation*}

%         We want $\tilde{C}: \Omega \rightarrow \mathbb{R}^d$ such that
%         \begin{equation}
%             \left\{\begin{aligned}
%                 -\Delta \tilde{C}&=\tilde{f}, \; &&\text{on } \Omega, \\
%                 \tilde{C}&=0, \; &&\text{in } \Gamma.
%             \end{aligned}\right. \tag{$\mathcal{C}_{+}$} %\label{corr_add}
%         \end{equation}
%         with $\tilde{f}=f+\Delta u_\theta$ and $\tilde{C}=\phi C$ for the $\phi$-FEM method.
        
%         \small
%         \textit{Remark :} In practice, it may be useful to integrate by parts the term containing $\Delta u_\theta$.
%     \end{minipage} \quad
%     \begin{minipage}{0.48\linewidth}
%         \textbf{By multiplying :}
        
%         \small
%         \textcolor{white}{Find $\hat{u} : \Omega \rightarrow \mathbb{R}^d$ such that
% 		\begin{equation}
% 			\left\{
% 			\begin{aligned}
% 				-\Delta \hat{u} = f, \; &&\text{in } \; \Omega, \\
% 				\hat{u}=g+m, \; &&\text{on } \; \Gamma,
% 			\end{aligned}
% 			\right. \tag{$\mathcal{P}^\mathcal{M}$} \label{pb_reh}
% 		\end{equation}
% 		with $\hat{u}=u+m$ ($m$ a constant).}

%         \normalsize
%         We will consider
%         \begin{equation*}
%             \tilde{u}=u_\theta C
%         \end{equation*}
%         \textcolor{white}{with $\hat{u_\theta}=u_\theta+m$.}
        
%         We want $C: \Omega \rightarrow \mathbb{R}^d$ such that
%         \begin{equation*}
%             \left\{\begin{aligned}
%                 &-\Delta (u_\theta C)=f, \; &&\text{on } \Omega, \\
%                 &C=1, \; &&\text{on } \; \Gamma.
%             \end{aligned}\right. \tag{$\mathcal{C}_\times$} %\label{corr_mult}
%         \end{equation*}
%     \end{minipage}
% \end{frame}

\begin{frame}{Correction Methods}
    We are given $u_\theta$ the FNO prediction (for the problem under consideration).
    
    \begin{minipage}{0.48\linewidth}
        \textbf{By adding :}
    
        We will consider
        \begin{equation*}
            \tilde{u}=u_\theta+\fcolorbox{blue}{white}{$\tilde{C}$}\textcolor{blue}{\approx 0}
        \end{equation*}

        We want $\tilde{C}: \Omega \rightarrow \mathbb{R}^d$ such that
        \begin{equation}
            \left\{\begin{aligned}
                -\Delta \tilde{C}&=\tilde{f}, \; &&\text{in } \Omega, \\
                \tilde{C}&=0, \; &&\text{on } \Gamma.
            \end{aligned}\right. \tag{$\mathcal{C}_{+}$} %\label{corr_add}
        \end{equation}
        with $\tilde{f}=f+\Delta u_\theta$ and $\tilde{C}=\phi C$ for the $\phi$-FEM method.
        
        \small
        \textit{Remark :} In practice, it may be useful to integrate by parts the term containing $\Delta u_\theta$.
    \end{minipage} \quad
    \begin{minipage}{0.48\linewidth}
        \textbf{By multiplying :}
        
        \small
        \textcolor{white}{Find $\hat{u} : \Omega \rightarrow \mathbb{R}^d$ such that
		\begin{equation}
			\left\{
			\begin{aligned}
				-\Delta \hat{u} = f, \; &&\text{in } \; \Omega, \\
				\hat{u}=g+m, \; &&\text{on } \; \Gamma,
			\end{aligned}
			\right. \tag{$\mathcal{P}^\mathcal{M}$} %\label{pb_reh}
		\end{equation}
		with $\hat{u}=u+m$ ($m$ a constant).}

        \normalsize
        We will consider
        \begin{equation*}
            \tilde{u}=u_\theta \fcolorbox{blue}{white}{$C$}\textcolor{blue}{\approx 1}
        \end{equation*}
        \textcolor{white}{with $\hat{u_\theta}=u_\theta+m$.}
        
        We want $C: \Omega \rightarrow \mathbb{R}^d$ such that
        \begin{equation*}
            \left\{\begin{aligned}
                &-\Delta (u_\theta C)=f, \; &&\text{on } \Omega, \\
                &C=1, \; &&\text{on } \; \Gamma.
            \end{aligned}\right. \tag{$\mathcal{C}_\times$} \label{corr_mult}
        \end{equation*}
    \end{minipage}
\end{frame}

\begin{frame}[noframenumbering]{Correction Methods}
    We are given $u_\theta$ the FNO prediction (for the problem under consideration).
    
    \begin{minipage}{0.48\linewidth}
        \textbf{By adding :}
    
        We will consider
        \begin{equation*}
            \tilde{u}=u_\theta+\tilde{C}
        \end{equation*}

        We want $\tilde{C}: \Omega \rightarrow \mathbb{R}^d$ such that
        \begin{equation}
            \left\{\begin{aligned}
                -\Delta \tilde{C}&=\tilde{f}, \; &&\text{in } \Omega, \\
                \tilde{C}&=0, \; &&\text{on } \Gamma.
            \end{aligned}\right. \tag{$\mathcal{C}_{+}$} \label{corr_add}
        \end{equation}
        with $\tilde{f}=f+\Delta u_\theta$ and $\tilde{C}=\phi C$ for the $\phi$-FEM method.
        
        \small
        \textit{Remark :} In practice, it may be useful to integrate by parts the term containing $\Delta u_\theta$.
    \end{minipage} \quad
    \begin{minipage}{0.48\linewidth}
        \textbf{By multiplying \textcolor{red}{- elevated problem} :}
        
        \small
        \textcolor{red}{Find $\hat{u} : \Omega \rightarrow \mathbb{R}^d$ such that
		\begin{equation}
			\left\{
			\begin{aligned}
				-\Delta \hat{u} = f, \; &&\text{in } \; \Omega, \\
				\hat{u}=g+m, \; &&\text{on } \; \Gamma,
			\end{aligned}
			\right. \tag{$\mathcal{P}^{\mathcolor{red}{\mathcal{M}}}$} \label{pb_reh}
		\end{equation}
		with $\hat{u}=u+m$ ($m$ a constant).}

        \normalsize
        We will consider
        \begin{equation*}
            \tilde{u}=\uchapeau{u_\theta} C
        \end{equation*}
        \textcolor{red}{with $\hat{u_\theta}=u_\theta+m$.}
        
        We want $C: \Omega \rightarrow \mathbb{R}^d$ such that
        \begin{equation*}
            \left\{\begin{aligned}
                &-\Delta (\uchapeau{u_\theta} C)=f, \; &&\text{in } \Omega, \\
                &C=1, \; &&\text{on } \; \Gamma.
            \end{aligned}\right. \tag{$\mathcal{C}_\times^{\mathcolor{red}{\mathcal{M}}}$} \label{corr_mult_reh}
        \end{equation*}
    \end{minipage}
\end{frame}

\subsection{Results - with FNO}

\begin{frame}{Explanation}
    \vspace{5pt}
    \begin{minipage}{0.48\linewidth}
        \textbf{Train a FNO :}
        
        \pgfimage[width=\linewidth]{images/internship/FNO/Training.png}
    \end{minipage} \quad
    \begin{minipage}{0.48\linewidth}
        \textbf{Correct the predictions of the FNO :}
        
        \pgfimage[width=\linewidth]{images/internship/FNO/Correction.png}
    \end{minipage}

    \vspace{15pt}
    \textbf{Some important points on the FNO :}
    \begin{enumerate}[\ding{217}]        
        \item widely used in PDE solving and constitute an active field of research
        \item FNO are Neural Operator networks : Unlike standard neural networks, which learn using inputs and outputs of fixed dimensions, neural operators \textbf{learn operators, which are mappings between spaces of functions}.
        \item \textbf{Mesh resolution independent :} can be evaluated at almost any data resolution without the need for retraining
    \end{enumerate}
\end{frame}

\begin{frame}{Correction on a FNO prediction - $\phi$-FEM}
    We consider an unknown solution on the circle with $f$ Gaussian \eqref{sol1}, $n_{vert}=63$, $n_{data}=1000$ (including validation sample) and $n_{test}=100$. 

    \vspace{10pt}
    \begin{minipage}{0.38\linewidth}
        Training on 4000 epochs (bs=64,lr=0.01) :
        \centering
        \pgfimage[width=0.9\linewidth]{images/internship/FNO/misfits_f_gaussienne.png}
    \end{minipage}
    \begin{minipage}{0.58\linewidth}
        Correction with the different methods :
        \centering
        \pgfimage[width=\linewidth]{images/internship/FNO/corr_boxplot.png}
    \end{minipage}	
    
    \vspace{5pt}
    \footnotesize
    \textbf{Remark :} We should try to reduce the resolution for correction, maybe we will gain in the time-to-error ratio.
\end{frame}

\subsection{Other results}

% \begin{frame}{Precision of the prediction - FEM}    
%     We consider the trigonometric solution on the circle \eqref{sol2} with

%     \begin{minipage}{0.4\linewidth}
%         $\; \textbullet$ $u_{ex}$ : the exact solution of \eqref{pb_initial}
        
%         \textcolor{white}{$\; \textbullet$ $u_\theta$ : a disturbed solution of }
        
%         $(S,f,\varphi)=(0.5,1,0)$
        
%         \textcolor{white}{$(S_p,f_p,\varphi_p)=(0.5,2,0)$}
        
%         \textbf{Exact solution :} $u_\theta=u_{ex}\in\mathbb{P}^{10}$

%         Correction with FEM ($n_{vert}=32$) :
        
%         \pgfimage[width=\linewidth]{images/internship/precision/exact_sol/exact_fem_circle_nvert32_fleche}
%     \end{minipage}
    
% \end{frame}

% \begin{frame}[noframenumbering]{Precision of the prediction - FEM}    
%     We consider the trigonometric solution on the circle \eqref{sol2} with

%     \begin{minipage}{0.4\linewidth}
%         $\; \textbullet$ $u_{ex}$ : the exact solution of \eqref{pb_initial}
        
%         \textcolor{white}{$\; \textbullet$ $u_\theta$ : a disturbed solution of }
        
%         $(S,f,\varphi)=(0.5,1,0)$
        
%         \textcolor{white}{$(S_p,f_p,\varphi_p)=(0.5,2,0)$}
        
%         \textbf{Exact solution :} $u_\theta=u_{ex}\in\mathbb{P}^{10}$

%         Correction with FEM ($n_{vert}=32$) :
        
%         \pgfimage[width=\linewidth]{images/internship/precision/exact_sol/exact_fem_circle_nvert32_fleche}

%         \textcolor{red}{Correction with FEM ($n_{vert}=100$) :}
        
%         \pgfimage[width=\linewidth]{images/internship/precision/exact_sol/exact_fem_circle_nvert100_colorbox}
%     \end{minipage}	
% \end{frame}

% \begin{frame}[noframenumbering]{Precision of the prediction - FEM}    
%     We consider the trigonometric solution on the circle \eqref{sol2} with

%     \begin{minipage}{0.4\linewidth}
%         $\; \textbullet$ $u_{ex}$ : the exact solution of \eqref{pb_initial}
        
%         \textcolor{red}{$\; \textbullet$ $u_\theta$ : a disturbed solution}
        
%         $(S,f,\varphi)=(0.5,1,0)$
        
%         \textcolor{red}{$(S_p,f_p,\varphi_p)=(0.5,2,0)$}
        
%         \textbf{Exact solution :} $u_\theta=u_{ex}\in\mathbb{P}^{10}$
        
%         Correction with FEM ($n_{vert}=32$) :

%         \centering
%         \pgfimage[width=0.9\linewidth]{images/internship/precision/exact_sol/exact_fem_circle_nvert32}

%         \raggedright
%         Correction with FEM ($n_{vert}=100$) :

%         \centering
%         \pgfimage[width=0.9\linewidth]{images/internship/precision/exact_sol/exact_fem_circle_nvert100}
%     \end{minipage} \quad
%     \fcolorbox{red}{white}{\begin{minipage}{0.54\linewidth}
%         \textbf{Disturbed solution :} $u_\theta=u_{ex}+\epsilon P\in\mathbb{P}^{k}$
        
%         with $\epsilon$ a real number and $P$ the perturbation.
        
%         Correction by adding with FEM ($n_{vert}=32$) :

%         \centering
%         \pgfimage[width=0.8\linewidth]{images/internship/precision/disturbed_sol/disturbed_fem_circle_nvert32_add}

%         \raggedright
%         Results for $k=10$ :
        
%         \pgfimage[width=\linewidth]{images/internship/precision/disturbed_sol/norms_P10_line}

%         with $\tilde{u}$ the solution obtained with the correction by addition, apply on $u_\theta$. 
%     \end{minipage}}

%     \footnotesize
%     \textit{Remark :} In practice, the expression of $u_{ex}$ will be chosen as the expression of $P$ (with the perturbation parameters $(S_p,f_p,\varphi_p)$). Consider $\varphi_p=0$ so that $P=0$ on $\Gamma$ and therefore $\tilde{\phi}=0$ on $\Gamma$.
% \end{frame}

\begin{frame}{Precision of the prediction - FEM}    
    We consider the trigonometric solution on the circle \eqref{sol2} with
    $$u_{ex}(x,y)=S\sin\left(8\pi f\left((x-0.5)^2+(y-0.5)^2\right)+\varphi\right)$$
    with $S=0.5$ and $\varphi=0$.

    \textbf{Exact solution :} Testing different correction methods for different frequencies.

    $$u_\theta=u_{ex}\in\mathbb{P}^{10} \; \rightarrow \; \tilde{u}\in\mathbb{P}^1$$

    \begin{center}     
        Correction with FEM ($n_{vert}=100$) :
        
        \pgfimage[width=0.6\linewidth]{images/internship/precision/exact_sol/exact_fem_circle_nvert100_colorbox}
    \end{center}
\end{frame}

\begin{frame}{Precision of the prediction - FEM}  
    We consider $(S,f,\varphi)=(0.5,1,0)$.

    \textbf{Disturbed solution :} Testing different $\epsilon$ and different degree $k$.
    $$u_\theta=u_{ex}+\epsilon P\in\mathbb{P}^{k} \; \rightarrow \; \tilde{u}\in\mathbb{P}^1$$
    with $\epsilon$ a real number and $P$ a perturbation.
    
    \begin{minipage}{0.58\linewidth}        
        Correction \eqref{corr_add} with FEM ($n_{vert}=32$) :

        \centering
        \pgfimage[width=\linewidth]{images/internship/precision/disturbed_sol/disturbed_fem_circle_nvert32_add}
    \end{minipage} \hfill
    \begin{minipage}{0.38\linewidth}        
        \vspace{-5pt}
        Results for $k=10$ :

        \centering
        \pgfimage[width=0.5\linewidth]{images/internship/precision/disturbed_sol/norms_P10}
    \end{minipage}

    \footnotesize
    \textit{Remark :} $P(x,y)=S_p\sin\left(8\pi f_p\left((x-0.5)^2+(y-0.5)^2\right)+\varphi_p\right)$ with $(S_p,f_p,\varphi_p)=(0.5,2,0)$
\end{frame}

\begin{frame}{Theoretical results - FEM}
    \textbf{Correction by multiplication on the elevated problem :} We consider 
        
    \begin{itemize}
        \item \textcolor{red}{$\hat{u_{ex}}=u_{ex}+m$} : the exact solution of \eqref{pb_reh}
        \item \textcolor{red}{$\hat{u_\theta} = u_\theta+m$} : a disturbed solution of \eqref{pb_reh}.
        \item $\tilde{u_h}=\hat{u_\theta}C_h$ : the approximate solution of \eqref{corr_mult_reh}
    \end{itemize}

    % We show that $\qquad \left|\left|\hat{u_{ex}}-\tilde{u_h}\right|\right|_0\le ch^{k+1}||\hat{u_\theta}||_\infty\left|C\right|_{k+1}$
        
    \begin{minipage}{0.48\linewidth}
        \setstretch{0.5}
        \textbf{1.} When m tends to infinity : 
            \begin{center}
                solution of \eqref{corr_mult_reh} $\rightarrow$ solution of \eqref{corr_add}
            \end{center}

            \small
            \textbf{Results :} $n_{vert}=32$, $\epsilon=0.001$ \\
            
            \centering
            \pgfimage[width=0.6\linewidth]{images/internship/theo_results/first/fig_fem_circle.png}
    \end{minipage} \;
    \begin{minipage}{0.48\linewidth}
        \setstretch{0.5}
        \textbf{2.} For $m$ sufficiently large : $C_{ex}=\hat{u_{ex}}/\hat{u_\theta}$
            \begin{equation*}
                \left|\left|C_{ex}-C_h\right|\right|_{0,\Omega}\le ch^{k+1}\epsilon\left|\left|P''\right|\right|_{0,\Omega}
            \end{equation*}

            \small
            \textbf{Results :} $n_{vert}=32$, $\epsilon=0.001$, $f_p=2$ \\
    
            \centering
            \pgfimage[width=0.6\linewidth]{images/internship/theo_results/second/fig_fem_circle.png}
    \end{minipage}
    
    
\end{frame}