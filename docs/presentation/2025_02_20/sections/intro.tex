\begin{frame}{Scientific context}
	\begin{minipage}{0.78\linewidth}
		\textbf{Context :} Create real-time digital twins of an organ (e.g. liver).
	\end{minipage}
	\begin{minipage}{0.18\linewidth}
		\vspace{-20pt}
		\includegraphics[width=0.95\linewidth]{images/intro/liver.png}
	\end{minipage}
	
	\vspace{3pt}
	\textbf{Objective :} Develop an hybrid \fcolorbox{red}{white}{finite element} / \fcolorbox{orange}{white}{neural network} method.
	
	\vspace{1pt}
	\small
	\hspace{130pt} \begin{minipage}{0.14\linewidth}
		\textcolor{red}{accurate}
	\end{minipage} \hspace{8pt} \begin{minipage}{0.3\linewidth}
		\textcolor{orange}{quick + parameterized}
	\end{minipage}

	\vspace{5pt}
	\textbf{Parametric linear elliptic PDE :}
	For one or several  $\bm{\mu}\in \mathcal{M}$, find $u: \Omega\to \mathbb{R}$ such that
	\begin{equation*}
		% \label{eq:ob_pde}
		\mathcal{L}\big(u;\bm{x},\bm{\mu}\big) = f(\bm{x},\bm{\mu}),
	\end{equation*}
	where $\mathcal{L}$ is the parametric differential operator defined  by
	\begin{equation*}
		\mathcal{L}(\cdot;\bm{x},\bm{\mu}) : u \mapsto R(\bm{x},\bm{\mu}) u + C(\bm{\mu}) \cdot \nabla u - \frac{1}{\text{Pe}} \nabla \cdot (D(\bm{x},\bm{\mu}) \nabla u).
	\end{equation*}
	\begin{table}[ht!]
		\centering
		\begin{tabular}{c|c}
			$\Omega$ & Spatial domain \\
			$d$ & Spatial dimension \\
			$\bm{x}=(x_1,\dots,x_d)$ & Spatial coordinates \\
			\hline
			$\mathcal{M}$ & Parameter space \\
			$p$ & Number of parameters \\
			$\bm{\mu}=(\mu_1,\ldots,\mu_p)$ & Parameter vector \\
		\end{tabular} \hspace{10pt}
		\begin{tabular}{c|c}
			$f$ & Right-hand side \\
			$R$ & Reaction coefficient \\
			$C$ & Convection coefficient \\
			$D$ & Diffusion matrix \\
			Pe & Péclet number \\
		\end{tabular}
	\end{table}
\end{frame}

\begin{frame}{Pipeline of the Enriched FEM}
	\begin{figure}[!ht]
		\centering
		\includegraphics[width=0.7\linewidth]{images/intro/pipeline/offline_v2.pdf}

		\includegraphics[width=0.7\linewidth]{images/intro/pipeline/online_v2.pdf}
	\end{figure}

	\textbf{Correction :} Enriched continuous Lagrange finite element approximation spaces
	using the PINN prediction.
\end{frame}

\begin{frame}{Physics-Informed Neural Networks}
	TODO
\end{frame}

\begin{frame}{Finite Element Method}
	TODO
\end{frame}