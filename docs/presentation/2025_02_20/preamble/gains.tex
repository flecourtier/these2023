\usepackage{booktabs}
\usepackage{xcolor}

% gains pour tous les q
\newcommand{\gainstableallq}[1]{
    \pgfplotstabletypeset[
        col sep=comma,
        every head row/.style={
        before row={\toprule[1.pt]
        & \multicolumn{3}{c}{\textbf{Gains in $L^2$ rel error}} \\
		& \multicolumn{3}{c}{\textbf{of our method w.r.t. FEM}} \\
		\cmidrule(lr){2-4}
        },
        after row=\cmidrule(lr){1-1} \cmidrule(lr){2-4}},
        every last row/.style={after row=\bottomrule[1.pt]},
        every nth row={1}{before row=\cmidrule(lr){1-1} \cmidrule(lr){2-4}},
		columns/q/.style={column name=\textbf{k}},
        columns/min_FEM/.style={column name=\textbf{min},fixed},
        columns/max_FEM/.style={column name=\textbf{max},fixed},
        columns/mean_FEM/.style={column name=\textbf{mean},fixed,
            postproc cell content/.append style={
                /pgfplots/table/@cell content/.add={\color{red}}{},
            }
        },
        columns={q,min_FEM,max_FEM,mean_FEM},
        precision=2
    ]{#1}
}

