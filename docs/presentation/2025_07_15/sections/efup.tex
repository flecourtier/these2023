\begin{frame}{Enriched space using PINN} %\footnote[frame,1]{The $\bm{\mu}$ parameter is fixed in the FE resolution.}}	
    Considering the PINN prior $U_\theta = (\bm{u}_\theta, p_\theta, T_\theta)$, we define the \textcolor{darkred}{mixed finite element space additively enriched} by the PINN as follows:
    
    \begin{center}
        \fcolorbox{darkred}{white}{$M_h^+ = \left\{U_h^+ = U_\theta + C_h^+, \quad C_h^+ \in M_h^{\, 0}\right\}$}
    \end{center}

    with $M_h^{\, 0}=[V_h^{\, 0}]^2 \times Q_h \times W_h^0$,
    $U_h^+ = (\bm{u}_h^+, p_h^+, T_h^+) \in M_h^+$ and $C_h^+ = (\bm{C}_{h,\bm{u}}^+, C_{h,p}^+, C_{h,T}^+)$.

    \vspace{8pt}

    We can then define the three finite element subspaces of $M_h^+$ as follows:
    \begin{minipage}{0.6\linewidth}
        \vspace{-15pt}
        \begin{align*}
            \bm{V}_h^+ &= \left\{\bm{u}_h^+ = \bm{u}_\theta + \bm{C}_{h,\bm{u}}^+, \; \bm{C}_{h,\bm{u}}^+ \in [V_h^{\, 0}]^2\right\}, \\
            Q_h^+ &= \left\{p_h^+ = p_\theta + C_{h,p}^+, \; C_{h,p}^+ \in Q_h\right\}, \\
            W_h^+ &= \left\{T_h^+ = T_\theta + C_{h,T}^+, \; C_{h,T}^+ \in W_h^{\, 0}\right\},
        \end{align*}
        where $\bm{C}_{h,\bm{u}}^+$, $C_{h,p}^+$ and $C_{h,T}^+$ becomes the unknowns.
        
        \vspace{5pt}
        \hl{à ajouter : dans quoi vit $U_\theta$ ?}
    \end{minipage}
    \begin{minipage}{0.38\linewidth}
        \centering
        \vspace{5pt}
        \pgfimage[width=0.8\linewidth]{images/efup/correction/correction.pdf}
    \end{minipage}
\end{frame}


\begin{frame}{Weak formulation - Additive approach}
    \textbf{Weak problem :} Find $C_h^+=(\textcolor{Cyan}{\bm{C}_{h,\bm{u}}^+}, \textcolor{blue}{C_{h,p}^+}, \textcolor{ForestGreen}{C_{h,T}^+}) \in M_h^{\, 0}$ s.t., \; $\forall (\bm{v}_h, q_h, w_h) \in M_h^{\, 0}$,

    \vspace{-4pt}
    \footnotesize
    \begin{equation}
        \label{eq:weak_pb_add}
        \hspace{-8pt}\begin{aligned}
            &\int_\Omega \big[(\textcolor{darkred}{\bm{u}_\theta} \cdot \nabla)\textcolor{darkred}{\bm{u}_\theta} + (\textcolor{darkred}{\bm{u}_\theta} \cdot \nabla)\textcolor{Cyan}{\bm{C}_{h,\bm{u}}^+} + (\textcolor{Cyan}{\bm{C}_{h,\bm{u}}^+} \cdot \nabla)\textcolor{darkred}{\bm{u}_\theta} + (\textcolor{Cyan}{\bm{C}_{h,\bm{u}}^+} \cdot \nabla)\textcolor{Cyan}{\bm{C}_{h,\bm{u}}^+} \big] \cdot \bm{v_h} \, d\bm{x} \\
            &\hspace{20pt} +\mu \left(\int_\Omega  \nabla \textcolor{darkred}{\bm{u}_\theta} : \nabla \bm{v}_h \, d\bm{x} + \int_\Omega \nabla \textcolor{Cyan}{\bm{C}_{h,\bm{u}}^+} : \nabla \bm{v}_h \, d\bm{x}\right) + \left(\int_\Omega \nabla \textcolor{darkred}{p_\theta} \cdot \bm{v}_h \, d\bm{x} - \int_\Omega \textcolor{blue}{C_{h,p}^+} \nabla \cdot \bm{v}_h \, d\bm{x}\right)\\
            &\hspace{50pt} - g \int_\Omega (1 + \beta (\textcolor{darkred}{T_\theta} + \textcolor{ForestGreen}{C_{h,T}^+})) \bm{e}_y \cdot \bm{v}_h \, d\bm{x} = 0, \,\text{\footnotesize (momentum)}  \\
            &\int_\Omega q_h \, \big[\nabla \cdot \textcolor{darkred}{\bm{u}_\theta} + \nabla \cdot \textcolor{Cyan}{\bm{C}_{h,\bm{u}}^+}\big] \, d\bm{x} \, + \, 10^{-4} \int_\Omega q_h \, (\textcolor{darkred}{p_\theta} + \textcolor{blue}{C_{h,p}^+}) \, d\bm{x} = 0, \; \text{\footnotesize (incompressibility + penal)} \\
            & \int_\Omega \big[ \textcolor{darkred}{\bm{u}_\theta} \cdot \nabla \textcolor{darkred}{T_\theta} + \textcolor{darkred}{\bm{u}_\theta} \cdot \nabla \textcolor{ForestGreen}{C_{h,T}^+} + \textcolor{Cyan}{\bm{C}_{h,\bm{u}}^+} \cdot \nabla \textcolor{darkred}{T_\theta} + \textcolor{Cyan}{\bm{C}_{h,\bm{u}}^+} \cdot \nabla \textcolor{ForestGreen}{C_{h,T}^+} \big] w_h \, d\bm{x} \\
            & \hspace{40pt} + k_f \left(\int_\Omega \nabla \textcolor{darkred}{T_\theta} \cdot \nabla w_h \; d\bm{x} + \int_\Omega \nabla \textcolor{ForestGreen}{C_{h,T}^+} \cdot \nabla w_h \, d\bm{x} \, w_h \, d\bm{s}\right) = 0, \, \text{\footnotesize (energy)}
        \end{aligned}
        \tag{$\mathcal{P}_h^+$}
    \end{equation}

    \vspace{5pt}
    with \textcolor{darkred}{$U_\theta = (\bm{u}_\theta, p_\theta, T_\theta)$} the PINN prior and some modified boundary conditions.
\end{frame}

\begin{frame}{Newton method - Additive approach}
    \vspace{-5pt}
    We want to solve the non linear system: %\hfill \tiny $N_h$ : number of degrees of freedom.

    \normalsize
    \vspace{-10pt}
    \begin{equation*}
        % \label{eq:nonlinear}
        F_\theta(\vec{C}) = 0 
    \end{equation*}

    \vspace{-2pt}
    with $F_\theta:\mathbb{R}^{N_h} \to \mathbb{R}^{N_h}$ the non linear operator associated to the weak problem \eqref{eq:weak_pb_add} and $\vec{C}\in \mathbb{R}^{N_h}$ the correction vector (unknown).

	\setcounter{algocf}{1}
    \begin{center}
        \small
        \begin{minipage}{0.9\linewidth}
            \begin{algorithm}[H]
                \SetAlgoLined
                \caption{Newton algorithm \citep{newton_accel_2025}}
                \textbf{Initialization step:} set $\vec{C}^{(0)} = \textcolor{darkred}{0}$\;
                \For{\( n \ge 0 \)}{
                    Solve the linear system \( F_\theta(\vec{C}^{(n)}) + F_\theta'(\vec{C}^{(n)}) \delta^{(n+1)} = 0 \) for \( \delta^{(n+1)} \)\;
                    Update \( \vec{C}^{(n+1)} = \vec{C}^{(n)} + \delta^{(n+1)} \)\;
                }
            \end{algorithm}
        \end{minipage}
    \end{center}
    
    \vspace{3pt}
    \textbf{Advantage compared to DeepPhysics\footnote[frame,1]{The additive approach is exactly the same as DeepPhysics if we take $U_\theta$ in the same space as $C_h^+$.}:} \refappendix{frame:comp}

    \vspace{-2pt}
    \begin{center}
        \textcolor{darkred}{$u_\theta$ is not required to live in the same discrete space as $C_h^+$}.
    \end{center}
    \vspace{8pt}
\end{frame}