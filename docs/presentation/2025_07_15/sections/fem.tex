\begin{frame}{Newton method}
	TODO
    % We want to solve the non linear system: \hfill \tiny $N_h$ : number of degrees of freedom.

    % \normalsize
    % \vspace{-10pt}
    % \begin{equation}
    %     \label{eq:nonlinear}
    %     F(u) = 0 
    % \end{equation}

    % \vspace{-2pt}
    % with $F:\mathbb{R}^{N_h} \to \mathbb{R}^{N_h}$ a non linear operator and $u\in\mathbb{R}^{N_h}$ the unknown vector.

    % \begin{center}
    %     \small
    %     \begin{minipage}{0.9\linewidth}
    %         \begin{algorithm}[H]
    %             \SetAlgoLined
    %             \caption{Newton's method to solve \eqref{eq:nonlinear} \citep{newton_accel_2025}}
    %             \textbf{Initialization step:} set $u^{(0)} = u_0$\;
    %             \For{\( k \ge 0 \)}{
    %                 Solve the linear system \( F(u^{(k)}) + F'(u^{(k)}) \delta^{(k+1)} = 0 \) for \( \delta^{(k+1)} \)\;
    %                 Update \( u^{(k+1)} = u^{(k)} + \delta^{(k+1)} \)\;
    %             }
    %         \end{algorithm}
    %     \end{minipage}
    % \end{center}

    % \vspace{5pt}
    % \textbf{Standard version:} \\
    % Initialization with a constant value $u_0$. For instance, $u_0=1$.

    % \vspace{5pt}
    % \textbf{DeepPhysics version:} \citep{odot_deepphysics_2021} \\
    % Initialization with a PINN solution $u_0=u_\theta$.
\end{frame}

\begin{frame}{Finite Element Methods\footcite{Ern2004TheoryAP}}	
	TODO
	% \textbf{Variational Problem :} 
	% \begin{equation}
	% 	\label{eq:weakform}
	% 	\text{Find } u_h\in V_h^0 \;\text{such that}, \forall v_h\in V_h^0, a(u_h,v_h)=l(v_h),
	% 	\tag{$\mathcal{P}_h$}
	% \end{equation}
	% \vspace{1pt}
	% with $h$ the characteristic mesh size, $a$ and $l$ the bilinear and linear forms given by
	% \vspace{-3pt}
	% \begin{equation*}
	% 	a(u_h,v_h)=
	% 	\frac{1}{\text{Pe}} \int_{\Omega}D \nabla u_h \cdot  \nabla v_h+
	% 	\int_{\Omega} R \, u_h \, v_h  +
	% 	\int_{\Omega} v_h \, C \cdot \nabla u_h, \quad l(v_h)=\int_{\Omega} f \, v_h,
	% \end{equation*}

	% \begin{minipage}[t]{0.7\linewidth}
	% 	\vspace{-3pt}
	% 	and $V_h^0$ the finite element space defined by
	% 	\vspace{-3pt}
	% 	\begin{equation*}
	% 		% \label{eq:Vh}
	% 		V_h^0 = \left\{v_h\in C^0(\Omega),\; \forall K\in \mathcal{T}_h,\; v_h\vert_{K}\in\mathbb{P}_k,v_h\vert_{\partial\Omega}=0\right\},
	% 	\end{equation*}
		
	% 	\vspace{-3pt}
	% 	where $\mathbb{P}_k$ is the space of polynomials of degree at most $k$.
		
	% 	\vspace{10pt}
	% 	\textbf{Linear system :} Let $(\phi_1,\dots,\phi_{N_h})$ a basis of $V_h^0$.
	% \end{minipage} \qquad \begin{minipage}[t][][b]{0.2\linewidth}
	% 	% \vspace{-5pt}
	% 	\centering
	% 	\pgfimage[width=0.9\linewidth]{images/appendix/std_method/FEM_triangle_mesh.png}
		
	% 	\footnotesize
	% 	$\mathcal{T}_h = \left\{K_1,\dots,K_{N_e}\right\}$
		
	% 	\tiny
	% 	($N_e$ : number of elements)
	% \end{minipage}
	
	% \vspace{-5pt}
	% Find $U\in\mathbb{R}^{N_h}$ such that \hspace{40pt} $AU=b$
	
	% with 
	% \begin{equation*}
	% 	A=\big(a(\phi_i,\phi_j)\big)_{1\le i,j\le N_h} \quad \text{and} \quad b=\big(l(\phi_j)\big)_{1\le j\le N_h}.
	% \end{equation*}
	% \vspace{-3pt}
\end{frame}