\begin{frame}{Discrete weak formulation}
	\vspace{-2pt}
	We consider a mixed finite element space \; \fcolorbox{darkred}{white}{$M_h = [V_h^{\, 0}]^2 \times Q_h \times W_h$} \; and
	
	\vspace{-4pt}
	\begin{center}
		\begin{tabular}{ccccccccl}
		\uncover<0>{\footnotesize \big(dim$(V_h^{\, 0})=N_u$\big)} \qquad & $\bm{u}_h$ & $\in$ & $[V_h^{\, 0}]^2$ & $\subset$ & $[H^1_0(\Omega)]^2$ & : & $\mathbb{P}_2$ & \multirow{2}{*}{$\left. \rule{0pt}{1.7em} \right\} \;$ \footnotesize (Taylor–Hood spaces)} \\
		\uncover<0>{\footnotesize \big(dim$(Q_h)=N_p$\big)} \qquad & $p_h$ & $\in$ & $Q_h$ & $\subset$ & $L^2_0(\Omega)$ & : & $ \mathbb{P}_1$ & \\ 
		\uncover<0>{\footnotesize \big(dim$(W_h)=N_T$\big)} \qquad & $T_h$ & $\in$ & $W_h$ & $\subset$ & $W$ & : & $\mathbb{P}_2$ & 
		\end{tabular}
	\end{center}

	with $\;W = \{w\in H^1(\Omega), \; w\vert_{x=-1}=1, \; w\vert_{x=1}=-1\}$.

	\vspace{5pt}

	\uncover<2>{\textbf{Weak problem :} Find $U_h=(\bm{u}_h, p_h, T_h) \in M_h$ s.t., \; $\forall (\bm{v}_h, q_h, w_h) \in M_h^{\, 0} $,

	\vspace{-4pt}
	\footnotesize
	\begin{equation}
		\label{eq:weak_pb}
		\begin{aligned}
			&\int_\Omega (\bm{u}_h \cdot \nabla)\bm{u}_h \cdot \bm{v}_h \, d\bm{x} + \mu \int_\Omega \nabla \bm{u}_h : \nabla \bm{v}_h \, d\bm{x} \\
			&\hspace{50pt} - \int_\Omega p_h \, \nabla \cdot \bm{v}_h \, d\bm{x} - g \int_\Omega (1 + \beta T_h) \bm{e}_y \cdot \bm{v}_h \, d\bm{x} = 0, \qquad\text{\footnotesize (momentum)} \\
			&\int_\Omega q_h \, \nabla \cdot \bm{u}_h \, d\bm{x} \only<2>{\textcolor{darkred}{\, + \, 10^{-4} \int_\Omega q_h \, p_h \, d\bm{x}}} = 0, \qquad\text{\footnotesize (incompressibility \only<2>{\textcolor{darkred}{+ pressure penalization}})}\\
			&\int_\Omega (\bm{u}_h \cdot \nabla T_h) \, w_h \, d\bm{x} + \int_\Omega k_f \nabla T_h \cdot \nabla w_h \, d\bm{x} = 0,  \qquad\text{\footnotesize (energy)}
			% \epsilon \int_\Omega q \, p \, dx = 0
		\end{aligned}
		\tag{$\mathcal{P}_h$}
	\end{equation}
	
	\vspace{5pt}
	where $M_h^{\, 0} = [V_h^{\, 0}]^2 \times Q_h \times W_h^{\, 0}$ with $W_h^{\, 0} \subset \{w \in H^1[\Omega], \; w\vert_{x=\pm 1}=0\}$.}
\end{frame}

\begin{frame}{Newton method}
	We consider the following three parameters:
	$$\bm{\mu}^{(1)} = (0.1,0.1), \; \bm{\mu}^{(2)} = (0.05,0.05) \; \text{and} \; \bm{\mu}^{(3)} = (0.01,0.01).$$

	Denoting $N_h$ the dimension of $M_h$, we want to solve the non linear system: %\hfill \footnotesize $N_h$ : dimension of $M_h$.

    \normalsize
    \vspace{-10pt}
    \begin{equation*}
        % \label{eq:nonlinear}
        F(\vec{U}_k) = 0 
    \end{equation*}

    with $F:\mathbb{R}^{N_h} \to \mathbb{R}^{N_h}$ a non linear operator and $\vec{U}_k\in \mathbb{R}^{N_h}$ the unknown vector associated to the $k$-th parameter $\bm{\mu}^{(k)}$ ($k=1,2,3$). \quad\refappendix{frame:basis}

	\setcounter{algocf}{0}
    \begin{center}
        \small
        \begin{minipage}{0.9\linewidth}
            \begin{algorithm}[H]
                \SetAlgoLined
                \caption{Newton algorithm} % \citep{newton_accel_2025}}
                \textbf{Initialization step:} set $\vec{U}_k^{(0)} = \only<1>{\vec{U}_{k,0}}\only<2>{\textcolor{darkred}{\vec{U}_{k,0}}}$\;
                \For{\( n \ge 0 \)}{
                    Solve the linear system \( F(\vec{U}_k^{(n)}) + F'(\vec{U}_k^{(n)}) \delta_k^{(n+1)} = 0 \) for \( \delta_k^{(n+1)} \)\;
                    Update \( \vec{U}_k^{(n+1)} = \vec{U}_k^{(n)} + \delta_k^{(n+1)} \)\;
                }
            \end{algorithm}
        \end{minipage}
    \end{center}
	\uncover<2>{\textcolor{darkred}{How to initialize the Newton solver?}}
\end{frame}

\begin{frame}{3 types of initialization}
	\begin{itemize}
		\item \textbf{Natural :} \only<2-4>{Using constant or linear function.}
		
		\only<2>{Considering a fixed parameter with $k\in\{1,2,3\}$, we can use the following initialization:	
		$$\vec{U}_{k,0} = \big(\vec{0}, \vec{0}, \vec{0}, \vec{T}_0\big)$$
		where for $i=1,\ldots,\text{dim}(W_h)$,
		$$(\vec{T}_0)_i = g(\bm{x}^{(i)}) = 1 - (x^{(i)}+1)$$
		with $\bm{x}^{(i)}=\big(x^{(i)},y^{(i)}\big)$ the $i$-th dofs coordinates of $W_h$.}
		
		\item \textbf{PINN :} \only<3-4>{Using PINN prediction. \\	
		(UNet : \citep{odot_deepphysics_2021} ; FNO : \citep{newton_accel_2025})} \\
		\only<3>{Considering a fixed parameter with $k\in\{1,2,3\}$, we can use the following initialization for $i=1,\ldots,N_h$,
		$$\big(\vec{U}_{k,0}\big)_i = U_\theta(\bm{x}^{(i)},\bm{\mu}^{(k)})$$
		with $\bm{x}^{(i)}=\big(x^{(i)},y^{(i)}\big)$ the $i$-th dofs coordinates of $M_h$ and $U_\theta$ the PINN.}
		
		\item \textbf{Continuation method :} \only<4>{Using a coarse FE solution of a simpler parameter.}
		
		\only<4>{\begin{itemize}
			\item We consider a fixed parameter with $k\in\{2,3\}$.
			\item We consider a coarse grid ($16\times 16$ grid) and compute the FE solution of \eqref{eq:weak_pb} for the parameter $\bm{\mu}^{(k-1)}$.
			\item We interpolate the coarse solution to the current mesh.
			\item We use it as an initialization for the Newton method, i.e.
			$$\vec{U}_{k,0} = \big(\vec{u}_{k-1}, \vec{v}_{k-1}, \vec{p}_{k-1}, \vec{T}_{k-1}\big)$$
			where $\vec{u}_{k-1}$, $\vec{v}_{k-1}$, $\vec{p}_{k-1}$ and $\vec{T}_{k-1}$ are the FE solutions for the parameter $\bm{\mu}^{(k-1)}$.
		\end{itemize}
		}
	\end{itemize}
\end{frame}