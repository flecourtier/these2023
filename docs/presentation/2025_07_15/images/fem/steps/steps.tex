\documentclass[11pt]{article}

% Encodage
\usepackage[utf8]{inputenc}
\usepackage[T1]{fontenc}
\usepackage[french]{babel}

% Mathématiques
\usepackage{amsmath, amssymb, amsfonts}
\usepackage{mathtools}

% Marges
\usepackage[a4paper, margin=2.5cm]{geometry}

% Pour les vecteurs en gras
\usepackage{bm}

% Pour les symboles des ensembles R, etc.
\usepackage{dsfont}

% Pour les figures (si besoin de TikZ plus tard)
\usepackage{graphicx}
\usepackage{tikz}

% Style
\usepackage{enumitem}
\setlist[itemize]{leftmargin=1.5em}

% Liens cliquables
\usepackage[colorlinks=true, linkcolor=blue, urlcolor=blue]{hyperref}

\title{Méthode de Newton pour Navier-Stokes \footnotesize(proposé par ChatGPT)}
\author{}
\date{}


\begin{document}

\maketitle

\textbf{Objectif} : résoudre le système stationnaire de Navier-Stokes avec flottabilité et gravité, en formulant les équations sous forme faible, puis en les discrétisant et en résolvant par la méthode de Newton.

\section*{1. Formulation faible continue}

On considère les formes faibles suivantes :\\

\begin{itemize}
  \item Incompressibilité :
  \[
  a_{\text{inc}}(\mathbf{u}, q) = \int_\Omega (\nabla \cdot \mathbf{u}) \, q \, dx
  \]
  \item Quantité de mouvement :
  \[
  a_{\text{mom}}(\mathbf{u}, p, T; \mathbf{v}) = \int_\Omega \left[ (\mathbf{u} \cdot \nabla)\mathbf{u} \cdot \mathbf{v} + \mu \nabla \mathbf{u} : \nabla \mathbf{v} - p \nabla \cdot \mathbf{v} - g(\beta T + 1)\mathbf{e}_y \cdot \mathbf{v} \right] dx
  \]
  \item Énergie thermique :
  \[
  a_{\text{ener}}(\mathbf{u}, T; s) = \int_\Omega \left[ (\mathbf{u} \cdot \nabla T) \, s + k_f \nabla T \cdot \nabla s \right] dx
  \]
\end{itemize}

\section*{2. Espaces discrets}

On choisit des espaces d’éléments finis compatibles :

\begin{itemize}
  \item Vitesse : \( V_h \subset [H^1]^2 \), typiquement \( \mathbb{P}_2 \)
  \item Pression : \( Q_h \subset L^2 \), typiquement \( \mathbb{P}_1 \)
  \item Température : \( W_h \subset H^1 \), souvent \( \mathbb{P}_1 \)
\end{itemize}

\section*{3. Bases discrètes}

On construit des bases finies :

\begin{itemize}
  \item \( \{ \phi_i^{(u_x)} \}_{i=1}^{N_u} \), \( \{ \phi_i^{(u_y)} \}_{i=1}^{N_u} \) pour \( \mathbf{u} \)
  \item \( \{ \psi_j \}_{j=1}^{N_p} \) pour \( p \)
  \item \( \{ \eta_k \}_{k=1}^{N_T} \) pour \( T \)
\end{itemize}

\section*{4. Représentation des inconnues}

Les champs sont approchés par :
\[
\mathbf{u}_h = \sum_{i=1}^{N_u} u_i^{(x)} \phi_i^{(u_x)} + u_i^{(y)} \phi_i^{(u_y)}, \quad
p_h = \sum_{j=1}^{N_p} p_j \psi_j, \quad
T_h = \sum_{k=1}^{N_T} T_k \eta_k
\]

Le vecteur global d’inconnues est :
\[
u = \begin{bmatrix}
\mathbf{u} \\
p \\
T
\end{bmatrix}
\in \mathbb{R}^{2N_u + N_p + N_T}
\]

\section*{5. Linéarisation}

Les formes \( a_{\text{mom}} \) et \( a_{\text{ener}} \) sont non linéaires. On linéarise autour de \( u^{(k)} \), en construisant :

\begin{itemize}
  \item Le résidu \( F(u^{(k)}) \)
  \item La jacobienne \( F'(u^{(k)}) \)
\end{itemize}

\section*{6. Assemblage de \( F(u) \)}

On construit \( F(u) \in \mathbb{R}^{2N_u + N_p + N_T} \) en testant les formes discrètes avec les fonctions de base :

\[
F(u) =
\begin{bmatrix}
F_{\text{mom}}(\mathbf{u}, p, T) \in \mathbb{R}^{2N_u} \\
F_{\text{inc}}(\mathbf{u}) \in \mathbb{R}^{N_p} \\
F_{\text{ener}}(\mathbf{u}, T) \in \mathbb{R}^{N_T}
\end{bmatrix}
\]

\section*{7. Méthode de Newton}

À chaque itération \( k \), on effectue :

\begin{align*}
\text{Résolution : } & \quad F'(u^{(k)}) \delta^{(k+1)} = -F(u^{(k)}) \\
\text{Mise à jour : } & \quad u^{(k+1)} = u^{(k)} + \delta^{(k+1)}
\end{align*}

---

\textbf{Remarque} : Ce cadre est valable pour de nombreuses équations couplées en mécanique des fluides ou thermique, et l’opérateur \( F \) représente simplement le résidu discret des équations testées par les bases.


\end{document}