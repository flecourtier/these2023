%% Requires compilation with XeLaTeX or LuaLaTeX
\documentclass[compress,10pt,xcolor={table,dvipsnames},t]{beamer} %aspectratio=169
\usetheme{diapo}
\usepackage{amsmath}
\DeclareMathOperator*{\argmax}{arg\,max}
\DeclareMathOperator*{\argmin}{argmin}
\usepackage{xparse} %for \NewDocumentEnvironment
\usepackage{amssymb}
\usepackage{xcolor}
\usepackage[bottom]{footmisc}
\usepackage{multirow}
\usepackage{setspace}
\usepackage{caption}
\usepackage{array,multirow,makecell}
\usepackage{pifont}
\usepackage{tikz}
\usepackage{paralist}
\usepackage{appendixnumberbeamer}
%\usepackage[style=authoryear,sorting=nyt,doi=false,url=false,maxbibnames=99,date=year]{biblatex}
\usepackage[square]{natbib}
\bibliographystyle{plainnat}
\usepackage{etoolbox}
% box colorée dans équation
\usepackage[most]{tcolorbox}
\usepackage{tikz}
\usepackage{soul}
% pour l'indicatrice
\usepackage{dsfont}
\usepackage{cancel}
\usepackage{booktabs}
\usepackage{bm}
% pour indentation des itemize
% \usepackage{enumitem}

\setcellgapes{1pt}
\setlength{\parindent}{0pt}
\makegapedcells
\newcolumntype{R}[1]{>{\raggedleft\arraybackslash }b{#1}}
\newcolumntype{L}[1]{>{\raggedright\arraybackslash }b{#1}}
\newcolumntype{C}[1]{>{\centering\arraybackslash }b{#1}}
\renewcommand*{\bibfont}{\footnotesize}
\useoutertheme[subsection=false]{miniframes}
\makeatletter
%\patchcmd{\slideentry}{\advance\beamer@xpos by1\relax}{}{}{}
\def\beamer@subsectionentry#1#2#3#4#5{\advance\beamer@xpos by1\relax}%
\makeatother
\setbeamercolor*{mini frame}{fg=bulles,bg=bulles}
\hypersetup{
	colorlinks=true,
	urlcolor=blue,
	citecolor=other,
	linkcolor=title,
}

\title[PhiFEM]{Enriching continuous Lagrange finite element approximation spaces using neural networks}
\subtitle{ICOSAHOM 2025}

\author{%
	Michel Duprez\inst{1}, 
	Emmanuel Franck\inst{2},
	\textbf{Frédérique Lecourtier}\inst{1} and
    Vanessa Lleras\inst{3}
}


\institute{%
	\inst{1} Project-Team MIMESIS, Inria, Strasbourg, France \\
    \inst{2} Project-Team MACARON, Inria, Strasbourg, France \\
    \inst{3} IMAG, University of Montpellier, Montpellier, France
}

\date{July 15, 2025}

\allowbreak

% u_chapeau (chapeau en couleur)
\usepackage{accents}
\newcommand{\uchapeau}[1]{\accentset{\textcolor{red}{\wedge}}{#1}}
\newcommand{\refappendix}[1]{\tikz[baseline=(char.base)]{\node[framednumber] (char) {\hyperlink{#1}{\small \textcolor{white}{Appendix \ref*{#1}}}};}}
% \newcommand{\refsubappendix}[1]{\tikz[baseline=(char.base)]{\node[framednumber] (char) {\hyperlink{#1.maj}{\small \textcolor{white}{Appendix \ref*{#1}}}};}}

\tikzset{
	framednumber/.style={
		draw=appendix,% Couleur de la bordure
		fill=appendix, % Couleur de fond
		rounded corners, % Coins arrondis
		inner sep=2pt,  % Espace intérieur
	}
}

%% numérotation et label des appendix
\newcounter{appendixframenumber}
\setcounter{appendixframenumber}{0}
\newcounter{subappendixframenumber}
\setcounter{subappendixframenumber}{1}

\makeatletter
\newcommand{\labelappendixframe}[1]{%
	\protected@write\@auxout{}{%
		\string\newlabel{#1}{{\theappendixframenumber}{\thepage}}%
	}%
	\hypertarget{#1}{}
}	
\makeatother

% Ce compteur temporaire stockera "x.y" (1.1, 1.2, etc.) pour les sous-appendices
\makeatletter
\newcommand{\labelsubappendixframe}[1]{%
	\edef\@currentlabel{\theappendixframenumber.\thesubappendixframenumber}%
	\label{#1}%
}
\makeatother

\newcommand{\appendixsection}[1]{%
	\addtocounter{appendixframenumber}{1}%
	\section{\appendixname~\theappendixframenumber~: #1}%\labelappendixframe{frame:#2}%
	\setcounter{subappendixframenumber}{1}% Réinitialiser le compteur des sous-appendices
}

\NewDocumentEnvironment{subappendixframe}{mo+b} 
{%
	% Optionnal : noframenumbering
	\IfNoValueTF{#2}
	{\begin{frame}{A\theappendixframenumber.\thesubappendixframenumber~– #1}}
	{\begin{frame}[#2]{A\theappendixframenumber.\thesubappendixframenumber~– #1}}
		#3
	\end{frame}
}{}

\NewDocumentEnvironment{appendixframe}{mo+b} 
{%
	% Optionnal : noframenumbering
	\IfNoValueTF{#2}
	{\begin{frame}{A\theappendixframenumber~– #1}}
	{\begin{frame}[#2]{A\theappendixframenumber~– #1}}
		#3
	\end{frame}
}{}

% barre en couleur terme dans équation
\newcommand\Ccancel[2][black]{\renewcommand\CancelColor{\color{#1}}\cancel{#2}}

% chifrre romain dans le texte
\makeatletter
\newcommand*{\rom}[1]{\expandafter\@slowromancap\romannumeral #1@}
\makeatother

% warning
\newcommand{\warning}{{\fontencoding{U}\fontfamily{futs}\selectfont\char 49\relax}}

\newcommand{\insertsectionheadSubtitle}{}

\newtcbtheorem{mytheo}{Theorem}{colback=other, % Couleur de fond de la boîte
	colframe=other, % Couleur du cadre de la boîte
	arc=2mm, % Rayon de l'arrondi des coins
	boxrule=0.5pt, % Épaisseur du cadre de la boîte
	breakable, enhanced jigsaw,
	width=\linewidth,
	opacityback=0.1
	}{th}

\newcommand*{\footcite}[1]{\footnote[frame,1]{\citep{#1}}}

% star command
\newcommand{\filledstar}{\textcolor{Goldenrod}{\ding{72}}\hspace{-8pt}\ding{73}\,}

% algorithm
\usepackage[ruled,vlined]{algorithm2e}
\usepackage{multirow}

% define color : darkred : red!85!black
\definecolor{darkred}{rgb}{0.9, 0, 0}


\usepackage{amssymb}
\usepackage{mathtools}
\usepackage{pgfplots}
\usepackage{pgfplotstable}
% \usepackage{filecontents}
\usepackage{datatool}
\usepackage{fp}
\usetikzlibrary{backgrounds}

\pgfplotsset{
    compat=newest,
}
\pgfplotsset{
    smaller labels/.style={
        label style={font=\footnotesize},
        tick label style={font=\footnotesize}
    }
}
\tikzset{font=\small}
\usetikzlibrary{
    fpu,
    fixedpointarithmetic,
    babel,
    external,
    arrows.meta,
    plotmarks,
    positioning,
    angles,
    quotes,
    intersections,
    calc,
    spy,
    decorations.pathreplacing,
    matrix,
    fit,
}
\usepgfplotslibrary{fillbetween}

% Define colors
\definecolor{femcolor}{RGB}{51, 138, 55} %Green (27,158,119)
\definecolor{addcolor}{RGB}{217,95,2} %Orange
\definecolor{addsobcolor}{RGB}{199,39,34} %Red (sob or other)
\definecolor{multcolor3}{RGB}{117,112,179} %Purple 
\definecolor{multcolor100}{RGB}{0,0,0} %Black (+ empty marker)
\definecolor{multcolor0weak}{RGB}{49, 73, 181} %Blue
\definecolor{multcolor0strong}{RGB}{49, 181, 161} %Cyan

% Define line styles according to the method 
% FEM : solid
% Add : dashed
% Mult : dotted

% Define marker styles according to the degree
% P1 : square
% P2 : circle
% P3 : triangle

%________________ error lines (by Ricardo Costa) ________________

% argument 1: slopes (e.g. {4,6})
% argument 2: x position of the bottom left corner
% argument 3: y position of the bottom left corner
% argument 4: x length

\makeatletter

\newcommand{\printslopeinv}[4]{
    \tikzset{fixed point arithmetic}
    % get arguments
    \def\nero@printslope@orderlist{#1}
    \edef\nero@printslope@xpos{#2}
    \edef\nero@printslope@ypos{#3}
    \edef\nero@printslope@width{#4}
    % get points position
    \pgfmathparse{\nero@printslope@xpos+\nero@printslope@width}
    \edef\nero@printslope@px{\pgfmathresult}
    \edef\nero@printslope@py{\nero@printslope@ypos}
    \edef\nero@printslope@qx{\pgfmathresult}
    \edef\nero@printslope@ry{\nero@printslope@ypos}
    \foreach \nero@printslope@order in {#1}{
        \pgfmathparse{
        ((\nero@printslope@px/\nero@printslope@xpos)^(\nero@printslope@order))*\nero@printslope@ypos}
        \edef\nero@printslope@qy{\pgfmathresult}
            \edef\nero@aux1{\noexpand\draw[line width=0.6pt]
            (axis cs:\nero@printslope@xpos,\nero@printslope@ypos)
            -- (axis cs:\nero@printslope@qx,\nero@printslope@qy)
            -- (axis cs:\nero@printslope@px,\nero@printslope@py);}
        \nero@aux1
        % slope label
        \pgfmathparse{10^((ln(\nero@printslope@ry)+ln(\nero@printslope@qy))/(ln(10)*2))}
        \edef\nero@printslope@labelpos{\pgfmathresult}
        \edef\nero@aux2{\noexpand\node[anchor=west] at
            (axis cs:\nero@printslope@qx,\nero@printslope@labelpos)
            {\noexpand\tiny \nero@printslope@order};}
        \nero@aux2
        \global\edef\nero@printslope@ry{\nero@printslope@qy}
    }
    % base line
    \draw[line width=0.6pt] (axis cs:\nero@printslope@xpos,\nero@printslope@ypos)
        |- (axis cs:\nero@printslope@px,\nero@printslope@py);
    % label of base line
    \pgfmathparse{10^((ln(\nero@printslope@px)+ln(\nero@printslope@xpos))/(ln(10)*2))}
    \edef\nero@printslope@labelpos{\pgfmathresult}
    \node[anchor=north] at (axis cs:\nero@printslope@labelpos,\nero@printslope@ypos) {\tiny 1};
}

\makeatother

\newlength{\plotwidth}
\setlength{\plotwidth}{0.54\textwidth}
\newlength{\plotheight}
\setlength{\plotheight}{0.4\textwidth}

\gdef\iterator{0}

\newenvironment{cvgh}[4]{
    \begin{tikzpicture}
        \edef\filename{#1}
        \edef\legendcolumns{#2}
        \edef\slopes{#3}
        \edef\ypos{#4}

        % Read the CSV file into a table
        \pgfplotstableread[col sep=comma]{\filename}\datatable

        % Obtenir le second élément
        \pgfmathtruncatemacro{\secondrow}{1} % Index de la dernière ligne
        \pgfplotstablegetelem{\secondrow}{h}\of\datatable
        \pgfmathsetmacro{\second}{\pgfplotsretval} % Dernière valeur de h_rounded

        % Obtenir le premier élément
        \pgfmathtruncatemacro{\firstrow}{0} % Index de l'avant-dernière ligne
        \pgfplotstablegetelem{\firstrow}{h}\of\datatable
        \pgfmathsetmacro{\first}{\pgfplotsretval} % Avant-dernière valeur de h_rounded

        % Calculer la différence entre les deux
        \pgfmathsetmacro{\diff}{\first - \second}

        %update iterator
        \pgfmathtruncatemacro{\iterator}{\iterator+1}

        \begin{loglogaxis}[
            smaller labels,
            name = left_plot,
            % axis lines
            axis lines = left,
            enlarge x limits={abs=10pt},
            enlarge y limits={abs=10pt},
            % axis x line shift = -5pt,
            axis y line shift = -5pt,
            % labels
			xmode=log,
            xlabel = {$h$},
            ylabel = {\rotatebox{270}{$L^2$}},
            xlabel style={at={(ticklabel* cs:1.01)},anchor=west},
            ylabel style={at={(ticklabel* cs:1.01)},anchor=west},
            % ticks and labels
            xtick=data,
            xticklabels from table={\datatable}{h},
            width=\plotwidth, height=\plotheight,
            mark options={solid, scale=1},
            grid = major,
            legend columns=\legendcolumns,
            legend to name=leg:legendFEMCORR_\iterator,
            legend image post style={mark options={solid, scale=1},xscale=0.8},
        ]
        \expandafter\printslopeinv\expandafter{\slopes}{\second}{\ypos}{\diff}
    }
    {
        \end{loglogaxis}
        \node[yshift=-20pt] at (left_plot.outer south) {\pgfplotslegendfromname{leg:legendFEMCORR_\iterator}};

    \end{tikzpicture}
}

\newenvironment{cvghline}[4]{
    \begin{tikzpicture}
        \edef\filename{#1}
        \edef\legendcolumns{#2}
        \edef\slopes{#3}
        \edef\ypos{#4}

        % Read the CSV file into a table
        \pgfplotstableread[col sep=comma]{\filename}\datatable

        % Obtenir le second élément
        \pgfmathtruncatemacro{\secondrow}{1} % Index de la dernière ligne
        \pgfplotstablegetelem{\secondrow}{h}\of\datatable
        \pgfmathsetmacro{\second}{\pgfplotsretval} % Dernière valeur de h_rounded

        % Obtenir le premier élément
        \pgfmathtruncatemacro{\firstrow}{0} % Index de l'avant-dernière ligne
        \pgfplotstablegetelem{\firstrow}{h}\of\datatable
        \pgfmathsetmacro{\first}{\pgfplotsretval} % Avant-dernière valeur de h_rounded

        % Calculer la différence entre les deux
        \pgfmathsetmacro{\diff}{\first - \second}

        %update iterator
        \pgfmathtruncatemacro{\iterator}{\iterator+1}

        \begin{loglogaxis}[
            smaller labels,
            name = left_plot,
            % axis lines
            axis lines = left,
            enlarge x limits={abs=10pt},
            enlarge y limits={abs=10pt},
            % axis x line shift = -5pt,
            axis y line shift = -5pt,
            % labels
			xmode=log,
            xlabel = {$h$},
            ylabel = {\rotatebox{270}{$L^2$}},
            xlabel style={at={(ticklabel* cs:1.01)},anchor=west},
            ylabel style={at={(ticklabel* cs:1.01)},anchor=west},
            % ticks and labels
            xtick=data,
            xticklabels from table={\datatable}{h},
            width=\plotwidth, height=\plotheight,
            mark options={solid, scale=1},
            grid = major,
            legend columns=\legendcolumns,
            legend to name=leg:legendFEMCORR_\iterator,
            legend image post style={mark options={solid, scale=1},xscale=0.8},
        ]
        \expandafter\printslopeinv\expandafter{\slopes}{\second}{\ypos}{\diff}
    }
    {
        \end{loglogaxis}
        \node[yshift=-20pt] at (left_plot.outer south) {\pgfplotslegendfromname{leg:legendFEMCORR_\iterator}};
        
        \draw[black, line width=0.4mm] (0.1,1.9) -- (4.7,1.9) node[anchor=west, xshift=2pt] {$e$};

    \end{tikzpicture}
}

\newcommand{\cvgFEMCorrAlldeg}[3]{
    \edef\fem{#1}
    \edef\add{#2}

    \begin{cvgh}{\fem}{3}{2,3,4}{#3}
        % Complete the legend
        \addlegendentry{\,FEM $\mathbb{P}_1$\;}
        \addlegendentry{\,FEM $\mathbb{P}_2$\;}
        \addlegendentry{\,FEM $\mathbb{P}_3$\;}
        \addlegendentry{\,Add $\mathbb{P}_1$\;}
        \addlegendentry{\,Add $\mathbb{P}_2$\;}
        \addlegendentry{\,Add $\mathbb{P}_3$\;}

        % Plot FEM
        \addplot [style={solid}, mark=square*, mark size=2, color=femcolor, line width=0.8pt ]
        table [x=h, y=P1, col sep=comma]
            {\fem};
        
        \addplot [style={solid}, mark=*, mark size=2, color=femcolor, line width=0.8pt ]
        table [x=h, y=P2, col sep=comma]
            {\fem};
        
        \addplot [style={solid}, mark=triangle*, mark size=2, color=femcolor, line width=0.8pt ]
        table [x=h, y=P3, col sep=comma]
            {\fem};

        % Plot Add
        \addplot [style={dashed}, mark=square*, mark size=2, color=addcolor, line width=0.8pt ]
        table [x=h, y=P1, col sep=comma]
            {\add};

        \addplot [style={dashed}, mark=*, mark size=2, color=addcolor, line width=0.8pt ]
        table [x=h, y=P2, col sep=comma]
            {\add};

        \addplot [style={dashed}, mark=triangle*, mark size=2, color=addcolor, line width=0.8pt ]
        table [x=h, y=P3, col sep=comma]
            {\add};

    \end{cvgh}
}

\newcommand{\cvgFEMCorrAlldegLine}[3]{
    \edef\fem{#1}
    \edef\add{#2}

    \begin{cvghline}{\fem}{3}{2,3,4}{#3}
        % Complete the legend
        \addlegendentry{\,FEM $\mathbb{P}_1$\;}
        \addlegendentry{\,FEM $\mathbb{P}_2$\;}
        \addlegendentry{\,FEM $\mathbb{P}_3$\;}
        \addlegendentry{\,Add $\mathbb{P}_1$\;}
        \addlegendentry{\,Add $\mathbb{P}_2$\;}
        \addlegendentry{\,Add $\mathbb{P}_3$\;}

        % Plot FEM
        \addplot [style={solid}, mark=square*, mark size=2, color=femcolor, line width=0.8pt ]
        table [x=h, y=P1, col sep=comma]
            {\fem};
        
        \addplot [style={solid}, mark=*, mark size=2, color=femcolor, line width=0.8pt ]
        table [x=h, y=P2, col sep=comma]
            {\fem};
        
        \addplot [style={solid}, mark=triangle*, mark size=2, color=femcolor, line width=0.8pt ]
        table [x=h, y=P3, col sep=comma]
            {\fem};

        % Plot Add
        \addplot [style={dashed}, mark=square*, mark size=2, color=addcolor, line width=0.8pt ]
        table [x=h, y=P1, col sep=comma]
            {\add};

        \addplot [style={dashed}, mark=*, mark size=2, color=addcolor, line width=0.8pt ]
        table [x=h, y=P2, col sep=comma]
            {\add};

        \addplot [style={dashed}, mark=triangle*, mark size=2, color=addcolor, line width=0.8pt ]
        table [x=h, y=P3, col sep=comma]
            {\add};

    \end{cvghline}
}

\newcommand{\cvgFEMCorrMultOnedeg}[6]{
    \edef\fem{#1}
    \edef\femsec{#2}
    \edef\add{#3}
    \edef\mult{#4}
    \edef\multHundred{#5}

    \begin{cvgh}{\fem}{3}{2,3}{#6}
        % Complete the legend
        \addlegendentry{\,FEM $\mathbb{P}_1$\;}
        \addlegendentry{\,Mult $\mathbb{P}_1$ (M=3)\;}
        \addlegendentry{\,Add $\mathbb{P}_1$\;}
        \addlegendentry{\,FEM $\mathbb{P}_2$\;}
        \addlegendentry{\,Mult $\mathbb{P}_1$ (M=100)\;}

        % Plot the data
        \addplot [style={solid}, mark=square*, mark size=2, color=femcolor, line width=0.8pt ]
        table [x=h, y=err, col sep=comma]
            {\fem};

        \addplot [style={dotted}, mark=square*, mark size=2, color=multcolor3, line width=1.0pt ]
        table [x=h, y=err, col sep=comma]
            {\mult};

        \addplot [style={dashed}, mark=square*, mark size=2, color=addcolor, line width=0.8pt ]
        table [x=h, y=err, col sep=comma]
            {\add};

        \addplot [style={solid}, mark=*, mark size=2, color=femcolor, line width=0.8pt ]
        table [x=h, y=err, col sep=comma]
            {\femsec};

        \addplot [style={dotted}, mark=square, mark size=2, color=multcolor100, line width=1.0pt ]
        table [x=h, y=err, col sep=comma]
            {\multHundred};
    \end{cvgh}
}
\documentclass{article}

\usepackage{amssymb}
\usepackage{mathtools}
%\usepackage[scale=0.8]{geometry}
\usepackage{pgfplots}
\usepackage{pgfplotstable}
\usepackage{filecontents}
\usepackage{datatool}
\usepackage{fp}
\pgfplotsset{
    compat=newest,
}
\pgfplotsset{
    smaller labels/.style={
        label style={font=\footnotesize},
        tick label style={font=\footnotesize}
    }
}
\tikzset{font=\small}
\usetikzlibrary{
    fpu,
    fixedpointarithmetic,
    babel,
    external,
    arrows.meta,
    plotmarks,
    positioning,
    angles,
    quotes,
    intersections,
    calc,
    spy,
    decorations.pathreplacing,
    matrix,
    fit,
}
\usepgfplotslibrary{fillbetween}

\definecolor{graph_1}{RGB}{117,112,179}
\definecolor{graph_2}{RGB}{217,95,2}
\definecolor{graph_3}{RGB}{27,158,119}
\definecolor{graph_4}{RGB}{231,41,138}
\definecolor{fill_topo}{RGB}{191,191,191}


%________________ error lines (by Ricardo Costa) ________________

% argument 1: slopes (e.g. {4,6})
% argument 2: x position of the bottom left corner
% argument 3: y position of the bottom left corner
% argument 4: x length

\makeatletter

% print slope on graphic
\newcommand{\printslope}[4]{
   \tikzset{fixed point arithmetic}
   % get arguments
   \def\nero@printslope@orderlist{#1}
   \edef\nero@printslope@xpos{#2}
   \edef\nero@printslope@ypos{#3}
   \edef\nero@printslope@width{#4}
   % get points position
   \pgfmathparse{\nero@printslope@xpos+\nero@printslope@width}
   \edef\nero@printslope@px{\pgfmathresult}
   \edef\nero@printslope@py{\nero@printslope@ypos}
   \edef\nero@printslope@qx{\nero@printslope@xpos}
   \edef\nero@printslope@ry{\nero@printslope@ypos}
   \foreach \nero@printslope@order in {#1}{
      \pgfmathparse{
      ((\nero@printslope@px/\nero@printslope@xpos)^(\nero@printslope@order))*\nero@printslope@ypos}
      \edef\nero@printslope@qy{\pgfmathresult}
      % print slope line
      \edef\nero@aux1{\noexpand\draw[line width=0.6pt]
         (axis cs:\nero@printslope@xpos,\nero@printslope@ry)
         -- (axis cs:\nero@printslope@qx,\nero@printslope@qy)
         -- (axis cs:\nero@printslope@px,\nero@printslope@py);}
      \nero@aux1
      % slope label
      \pgfmathparse{10^((ln(\nero@printslope@ry)+ln(\nero@printslope@qy))/(ln(10)*2))}
      \edef\nero@printslope@labelpos{\pgfmathresult}
      \edef\nero@aux2{\noexpand\node[anchor=east] at
         (axis cs:\nero@printslope@qx,\nero@printslope@labelpos)
         {\noexpand\tiny \nero@printslope@order};}
      \nero@aux2
      \global\edef\nero@printslope@ry{\nero@printslope@qy}
   }
   % base line
   \draw[line width=0.6pt] (axis cs:\nero@printslope@xpos,\nero@printslope@ypos)
      |- (axis cs:\nero@printslope@px,\nero@printslope@py);
   % label of base line
   \pgfmathparse{10^((ln(\nero@printslope@px)+ln(\nero@printslope@xpos))/(ln(10)*2))}
   \edef\nero@printslope@labelpos{\pgfmathresult}
   %\node[anchor=north] at (axis cs:\nero@printslope@labelpos,\nero@printslope@ypos) {\tiny 1};
}

\makeatother

%________________ error lines (by Ricardo Costa) ________________

% \setlength\textwidth{5.125in}

% \newlength{\plotwidth}
% \setlength{\plotwidth}{0.5\textwidth}
% \newlength{\plotheight}
% \setlength{\plotheight}{0.3333333\textwidth}
\newlength{\plotwidth}
\setlength{\plotwidth}{0.45\textwidth}
\newlength{\plotheight}
\setlength{\plotheight}{0.3\textwidth}
% \newlength{\plotwidth}
% \setlength{\plotwidth}{0.4\textwidth}
% \newlength{\plotheight}
% \setlength{\plotheight}{0.266\textwidth}

\usepackage{tabularx}

\newcommand{\gainstable}[1]{
	\pgfplotstabletypeset[
	col sep=comma,
	every head row/.style={
		before row={\toprule[1.pt]
			& \multicolumn{4}{c}{\textbf{Gains on PINNs}} &
			\multicolumn{4}{c}{\textbf{Gains on FEM}} \\
			\cmidrule(lr){2-5} \cmidrule(lr){6-9}
		}, 
		after row=\cmidrule(lr){1-1} \cmidrule(lr){2-5} \cmidrule(lr){6-9}},
	every last row/.style={after row=\bottomrule[1.pt]},
	columns/N/.style={
		column name=\textbf{N}%,
		%			postproc cell content/.append style={
			%				/pgfplots/table/@cell content/.add={$\fontfamily{pag}\selectfont}{$}
			%			}
	},
	columns/min_PINNs/.style={column name=\textbf{min},fixed},
	columns/max_PINNs/.style={column name=\textbf{max},fixed},
	columns/mean_PINNs/.style={column name=\textbf{mean},fixed},
	columns/std_PINNs/.style={column name=\textbf{std},fixed},
	columns/min_FEM/.style={column name=\textbf{min},fixed},
	columns/max_FEM/.style={column name=\textbf{max},fixed},
	columns/mean_FEM/.style={column name=\textbf{mean},fixed},
	columns/std_FEM/.style={column name=\textbf{std},fixed},
	columns={N,min_PINNs,max_PINNs,mean_PINNs,std_PINNs,min_FEM,max_FEM,mean_FEM,std_FEM},
	precision=2
	]{#1}
	

	
}

\usepackage{float}
\usepackage{subcaption}
\usepackage[english]{babel}

% Set page size and margins
% Replace `letterpaper' with`a4paper' for UK/EU standard size
\usepackage[letterpaper,top=2cm,bottom=2cm,left=3cm,right=3cm,marginparwidth=1.75cm]{geometry}

\usepackage{booktabs}

\begin{document}	
	\gainstable{data/gains_table_case1_degree1.csv}
	
	\vspace{20pt}
	
	\gainstable{data/gains_table_case1_degree2.csv}
	
	\vspace{20pt}
	
	\gainstable{data/gains_table_case1_degree3.csv}
\end{document}


\documentclass[border=1mm]{standalone}
\usepackage[utf8]{inputenc}
\usepackage[english]{babel}
\usepackage{tikz}
% \usepackage[margin=1cm]{geometry}

\usepackage{booktabs}
\usepackage{pgfplotstable}
\usepackage{xcolor}
\usepackage{amsmath}

% gains pour tous les q
\newcommand{\coststableallq}[1]{
    \pgfplotstabletypeset[
        col sep=comma,
        every head row/.style={
        before row={\toprule[1.pt]
        & & \multicolumn{2}{c}{\textbf{$N_\text{dofs}$}} \\
		\cmidrule(lr){3-4}
        },
        after row=\cmidrule(lr){1-1} \cmidrule(lr){2-2} \cmidrule(lr){3-4}},
        every last row/.style={after row=\bottomrule[1.pt]},
        every nth row={2}{before row=\cmidrule(lr){1-1} \cmidrule(lr){2-2} \cmidrule(lr){3-4}},
		columns/q/.style={column name=\textbf{k}},
        columns/e/.style={column name=\textbf{e},sci},
		columns/FEM_dofs/.style={column name=\textbf{FEM},fixed},
        columns/Add_dofs/.style={column name=\textbf{Add},fixed,
            postproc cell content/.append style={
                /pgfplots/table/@cell content/.add={\color{red}}{},
            }
        },
        columns={q,e,FEM_dofs,Add_dofs},
        precision=2
    ]{#1}
}

\begin{document}
    \coststableallq{TabDoFs_case1_v1_param1.csv}
\end{document}

\begin{document}
	% \nocite{*}
	
	\renewcommand{\inserttotalframenumber}{\pageref{lastslide}}
	
	{\setbeamertemplate{footline}{} 
		\BackgroundTitle
		\begin{frame}[plain]
			\maketitle
		\end{frame}
	}
	\addtocounter{framenumber}{-1} 	
	
	\AtBeginSection[]{
		{\setbeamertemplate{footline}{}
			\begin{frame}
				\vfill
				\centering
				\begin{beamercolorbox}[sep=5pt,shadow=true,rounded=true]{subtitle}
					\usebeamerfont{title}\insertsectionhead\par%
					\vspace{0.5cm} % Ajustez l'espacement selon vos besoins
					% \usebeamerfont{classic}\usebeamercolor[fg]{classic}\insertsectionheadSubtitle
				\end{beamercolorbox}
				%\tableofcontents[sectionstyle=hide,subsectionstyle=show]
				
				%subsectionstyle=⟨style for current subsection⟩/⟨style for other subsections in current section⟩/⟨style for subsections in other sections⟩
				\tableofcontents[sectionstyle=hide,subsectionstyle=show/show/hide]
				\vfill
				\begin{beamercolorbox}[sep=5pt,shadow=true,rounded=true]{subtitle}
					\usebeamerfont{classic}\usebeamercolor[fg]{classic}\insertsectionheadSubtitle
				\end{beamercolorbox}
			\end{frame}
		}
		\addtocounter{framenumber}{-1} 
	}
	
	\AtBeginSubsection[]{
		{\setbeamertemplate{footline}{}
			\begin{frame}
				\vfill
				\centering
				\begin{beamercolorbox}[sep=5pt,shadow=true,rounded=true]{subtitle}
					\usebeamerfont{title}\insertsectionhead\par%
					\vspace{0.5cm} 
				\end{beamercolorbox}
				\tableofcontents[sectionstyle=hide,subsectionstyle=show/shaded/hide]
				\vfill
			\end{frame}
		}
		\addtocounter{framenumber}{-1} 
	}
	
	\Background
	
	\section*{Introduction}
	\begin{frame}{Scientific context}
	\begin{minipage}{0.78\linewidth}
		\textbf{Context :} Create real-time digital twins of an organ (e.g. liver).
	\end{minipage}
	\begin{minipage}{0.18\linewidth}
		\vspace{-20pt}
		\includegraphics[width=0.95\linewidth]{images/intro/liver.png}
	\end{minipage}
	
	\vspace{1pt}
	\textbf{Objective :} Develop an hybrid \fcolorbox{red}{white}{finite element} / \fcolorbox{orange}{white}{neural network} method.
	
	\vspace{1pt}
	\small
	\hspace{130pt} \begin{minipage}{0.14\linewidth}
		\textcolor{red}{accurate}
	\end{minipage} \hspace{8pt} \begin{minipage}{0.3\linewidth}
		\textcolor{orange}{quick + parameterized}
	\end{minipage}

	\normalsize
	\vspace{5pt}
	\textbf{Parametric linear elliptic PDE :}
	For one or several  $\bm{\mu}\in \mathcal{M}$, find $u: \Omega\to \mathbb{R}$ such that
	\begin{equation*}
		% \label{eq:ob_pde}
		\mathcal{L}\big(u;\bm{x},\bm{\mu}\big) = f(\bm{x},\bm{\mu}),
	\end{equation*}
	where $\mathcal{L}$ is the parametric differential operator defined  by
	\begin{equation*}
		\mathcal{L}(\cdot;\bm{x},\bm{\mu}) : u \mapsto R(\bm{x},\bm{\mu}) u + C(\bm{\mu}) \cdot \nabla u - \frac{1}{\text{Pe}} \nabla \cdot (D(\bm{x},\bm{\mu}) \nabla u),
	\end{equation*}
	and some Dirichlet, Neumann or Robin BC (which can also depend on $\bm{\mu}$).
	
	\footnotesize
	\begin{table}[ht!]
		\centering
		\begin{tabular}{c|c}
			$\Omega$ & Spatial domain \\
			$d$ & Spatial dimension \\
			$\bm{x}=(x_1,\dots,x_d)$ & Spatial coordinates \\
			\hline
			$\mathcal{M}$ & Parameter space \\
			$p$ & Number of parameters \\
			$\bm{\mu}=(\mu_1,\ldots,\mu_p)$ & Parameter vector \\
		\end{tabular} \hspace{10pt}
		\begin{tabular}{c|c}
			$f$ & Right-hand side \\
			$R$ & Reaction coefficient \\
			$C$ & Convection coefficient \\
			$D$ & Diffusion matrix \\
			Pe & Péclet number \\
		\end{tabular}
	\end{table}
\end{frame}

\begin{frame}{Pipeline of the Enriched FEM}
	\begin{figure}[!ht]
		\centering
		\includegraphics[width=0.7\linewidth]{images/intro/pipeline/offline_v2.pdf}

		\includegraphics[width=0.7\linewidth]{images/intro/pipeline/online_v2.pdf}
	\end{figure}

	\textbf{Correction :} Enriched continuous Lagrange finite element approximation spaces
	using the PINN prediction.
\end{frame}

\begin{frame}{Physics-Informed Neural Networks}
	\textbf{Standard PINNs :} Find the optimal weights $\theta^\star$ that satisfy
	\begin{equation}
		\label{eq:opt_pb}
		\theta^\star = \argmin_{\theta}	\big( \omega_r \; J_r(\theta) + \omega_b \; J_b(\theta) \big),
	\end{equation}
	with the residual loss function and the boundary loss function defined by
	\begin{equation*}
		J_r(\theta) =
		\int_{\mathcal{M}}\int_{\Omega}
		\big| \mathcal{L}\big(u_\theta(\bm{x},\bm{\mu});\bm{x},\bm{\mu}\big)-f(\bm{x},\bm{\mu}) \big|^2 d\bm{x} d\bm{\mu},
	\end{equation*}
	\begin{equation*}
		J_b(\theta) =
		\int_{\mathcal{M}}\int_{\partial \Omega} \big| u_\theta(\bm{x},\bm{\mu}) - g(\bm{x},\bm{\mu}) \big|^2 d\bm{x} d\bm{\mu},
	\end{equation*}
	where $u_\theta$ is a neural network, $g$ is the Dirichlet BC. In \eqref{eq:opt_pb}, the weights $\omega_r$ and $\omega_b$ (hyperparameters) are used to balance the different terms of the loss function.

	\vspace{5pt}
	\textbf{Monte-Carlo method :} Discretize the cost functions by random process.
\end{frame}

\begin{frame}[noframenumbering]{Physics-Informed Neural Networks}
	\textbf{\textcolor{red}{Improved} PINNs\footcite{LagLikFot1998,FraMicNav2024} :} Find the optimal weights $\theta^\star$ that satisfy
	\begin{equation}
		\label{eq:opt_pb_nobc}
		\theta^\star = \argmin_{\theta}	\big( \omega_r \; J_r(\theta) + \Ccancel[red]{\omega_b \; J_b(\theta)} \big),
	\end{equation}
	with $\omega_r=1$ and the residual loss function defined by
	\begin{equation*}
		J_r(\theta) =
		\int_{\mathcal{M}}\int_{\Omega}
		\big| \mathcal{L}\big(u_\theta(\bm{x},\bm{\mu});\bm{x},\bm{\mu}\big)-f(\bm{x},\bm{\mu}) \big|^2 d\bm{x} d\bm{\mu},
	\end{equation*}
	\begin{minipage}{0.7\linewidth}
		where $u_\theta$ is a neural network defined by
		\begin{equation*}
			\textcolor{red}{u_{\theta}(\bm{x},\bm{\mu}) = \varphi(\bm{x}) w_{\theta}(\bm{x},\bm{\mu}) + g(\bm{x},\bm{\mu}),}
		\end{equation*}
		with $\varphi$ a level-set function, $w_\theta$ a NN and $g$ the Dirichlet BC. 
	\end{minipage}
	\begin{minipage}{0.28\linewidth}
		\vspace{-15pt}
		\includegraphics[width=0.95\linewidth]{images/intro/levelset.png}
	\end{minipage}

	\vspace{5pt}
	\textbf{Monte-Carlo method :} Discretize the residual cost function by random process.
	\vspace{15pt}
\end{frame}


\begin{frame}{Finite Element Method}
	TODO
\end{frame}

	
	\renewcommand{\insertsectionheadSubtitle}{The PINN is parametrized by the $\bm{\mu}$ parameter.}
	\section{Parametric Physics-Informed Neural Network (PINN)}
	\subsection{Encoding/Decoding}

\begin{frame}{Encoding/Decoding - NNs}
	\hl{A compléter !}
\end{frame}

\begin{frame}{Non-Linear Decoder - Advantages}
	\hl{A compléter !}
\end{frame}

\subsection{Approximation}

\begin{frame}{Approximation}
	\hl{A compléter !}
\end{frame}

\begin{frame}{Deep-Ritz}
	\hl{A compléter !}
\end{frame}

\begin{frame}{Standard PINNs}
	\hl{A compléter !}
\end{frame}

\begin{frame}{In practice...}
	\hl{A compléter !}
\end{frame}

\begin{frame}{Steps Decomposition - NNs}
	\begin{center}
		\renewcommand{\arraystretch}{1.5}
		\begin{tabular}{|c|c|c|c|}
			\hline
			\textbf{Encoding} & \multicolumn{2}{c|}{\textbf{Approximation}} & \textbf{Decoding} \\
			\hline
			$f \; \rightarrow \theta_f$ & \multicolumn{2}{c|}{$\theta_f \; \rightarrow \theta_u$} & $\theta_u \; \rightarrow u_\theta$ \\
			\hline
			\multirow{3}{*}{$\begin{aligned}
					\theta_f&=\mathcal{E}(f) \\
					&=M^{-1}b(f)
				\end{aligned}$} & Galerkin & LS Galerkin & \multirow{3}{*}{$\begin{aligned}
					u_\theta(x)&=\mathcal{D}_\theta(x) \\
					&=\sum_{i=1}^N (\theta_u)_i\varphi_i
				\end{aligned}$} \\
			& \small $\langle R(u_\theta),\varphi_i\rangle_{L^2}=0$ & \small $\langle R(u_\theta),(\nabla_\theta R(u_\theta))_i\rangle_{L^2}=0$ & \\
			\cline{2-3}
			& \multicolumn{2}{c|}{$A\theta_u=B$} & \\
			\hline
		\end{tabular}
	\end{center}
	
	\hl{A compléter !}
\end{frame}


	\renewcommand{\insertsectionheadSubtitle}{}

	\renewcommand{\insertsectionheadSubtitle}{The $\bm{\mu}$ parameter is fixed in the FE resolution.}
	\section{Finite element method (FEM)}
	\begin{frame}{Discrete weak formulation}
	\vspace{-2pt}
	We consider a mixed finite element space \; \fcolorbox{darkred}{white}{$M_h = [V_h^{\, 0}]^2 \times Q_h \times W_h$} \; and
	
	\vspace{-4pt}
	\begin{center}
		\begin{tabular}{ccccccccl}
		\uncover<0>{\footnotesize \big(dim$(V_h^{\, 0})=N_u$\big)} \qquad & $\bm{u}_h$ & $\in$ & $[V_h^{\, 0}]^2$ & $\subset$ & $[H^1_0(\Omega)]^2$ & : & $\mathbb{P}_2$ & \multirow{2}{*}{$\left. \rule{0pt}{1.7em} \right\} \;$ \footnotesize (Taylor–Hood spaces)} \\
		\uncover<0>{\footnotesize \big(dim$(Q_h)=N_p$\big)} \qquad & $p_h$ & $\in$ & $Q_h$ & $\subset$ & $L^2_0(\Omega)$ & : & $ \mathbb{P}_1$ & \\ 
		\uncover<0>{\footnotesize \big(dim$(W_h)=N_T$\big)} \qquad & $T_h$ & $\in$ & $W_h$ & $\subset$ & $W$ & : & $\mathbb{P}_2$ & 
		\end{tabular}
	\end{center}

	with $\;W = \{w\in H^1(\Omega), \; w\vert_{x=-1}=1, \; w\vert_{x=1}=-1\}$.

	\vspace{5pt}

	\uncover<2>{\textbf{Weak problem :} Find $U_h=(\bm{u}_h, p_h, T_h) \in M_h$ s.t., \; $\forall (\bm{v}_h, q_h, w_h) \in M_h^{\, 0} $,

	\vspace{-4pt}
	\footnotesize
	\begin{equation}
		\label{eq:weak_pb}
		\begin{aligned}
			&\int_\Omega (\bm{u}_h \cdot \nabla)\bm{u}_h \cdot \bm{v}_h \, d\bm{x} + \mu \int_\Omega \nabla \bm{u}_h : \nabla \bm{v}_h \, d\bm{x} \\
			&\hspace{50pt} - \int_\Omega p_h \, \nabla \cdot \bm{v}_h \, d\bm{x} - g \int_\Omega (1 + \beta T_h) \bm{e}_y \cdot \bm{v}_h \, d\bm{x} = 0, \qquad\text{\footnotesize (momentum)} \\
			&\int_\Omega q_h \, \nabla \cdot \bm{u}_h \, d\bm{x} \only<2>{\textcolor{darkred}{\, + \, 10^{-4} \int_\Omega q_h \, p_h \, d\bm{x}}} = 0, \qquad\text{\footnotesize (incompressibility \only<2>{\textcolor{darkred}{+ pressure penalization}})}\\
			&\int_\Omega (\bm{u}_h \cdot \nabla T_h) \, w_h \, d\bm{x} + \int_\Omega k_f \nabla T_h \cdot \nabla w_h \, d\bm{x} = 0,  \qquad\text{\footnotesize (energy)}
			% \epsilon \int_\Omega q \, p \, dx = 0
		\end{aligned}
		\tag{$\mathcal{P}_h$}
	\end{equation}
	
	\vspace{5pt}
	where $M_h^{\, 0} = [V_h^{\, 0}]^2 \times Q_h \times W_h^{\, 0}$ with $W_h^{\, 0} \subset \{w \in H^1[\Omega], \; w\vert_{x=\pm 1}=0\}$.}
\end{frame}

\begin{frame}{Newton method}
	We consider the following three parameters:
	$$\bm{\mu}^{(1)} = (0.1,0.1), \; \bm{\mu}^{(2)} = (0.05,0.05) \; \text{and} \; \bm{\mu}^{(3)} = (0.01,0.01).$$

	Denoting $N_h$ the dimension of $M_h$, we want to solve the non linear system: %\hfill \footnotesize $N_h$ : dimension of $M_h$.

    \normalsize
    \vspace{-10pt}
    \begin{equation*}
        % \label{eq:nonlinear}
        F(\vec{U}_k) = 0 
    \end{equation*}

    with $F:\mathbb{R}^{N_h} \to \mathbb{R}^{N_h}$ a non linear operator and $\vec{U}_k\in \mathbb{R}^{N_h}$ the unknown vector associated to the $k$-th parameter $\bm{\mu}^{(k)}$ ($k=1,2,3$). \quad\refappendix{frame:basis}

	\setcounter{algocf}{0}
    \begin{center}
        \small
        \begin{minipage}{0.9\linewidth}
            \begin{algorithm}[H]
                \SetAlgoLined
                \caption{Newton algorithm} % \citep{newton_accel_2025}}
                \textbf{Initialization step:} set $\vec{U}_k^{(0)} = \only<1>{\vec{U}_{k,0}}\only<2>{\textcolor{darkred}{\vec{U}_{k,0}}}$\;
                \For{\( n \ge 0 \)}{
                    Solve the linear system \( F(\vec{U}_k^{(n)}) + F'(\vec{U}_k^{(n)}) \delta_k^{(n+1)} = 0 \) for \( \delta_k^{(n+1)} \)\;
                    Update \( \vec{U}_k^{(n+1)} = \vec{U}_k^{(n)} + \delta_k^{(n+1)} \)\;
                }
            \end{algorithm}
        \end{minipage}
    \end{center}
	\uncover<2>{\textcolor{darkred}{How to initialize the Newton solver?}}
\end{frame}

\begin{frame}{3 types of initialization}
	\begin{itemize}
		\item \textbf{Natural :} \only<2-4>{Using constant or linear function.}
		
		\only<2>{Considering a fixed parameter with $k\in\{1,2,3\}$, we can use the following initialization:	
		$$\vec{U}_{k,0} = \big(\vec{0}, \vec{0}, \vec{0}, \vec{T}_0\big)$$
		where for $i=1,\ldots,\text{dim}(W_h)$,
		$$(\vec{T}_0)_i = g(\bm{x}^{(i)}) = 1 - (x^{(i)}+1)$$
		with $\bm{x}^{(i)}=\big(x^{(i)},y^{(i)}\big)$ the $i$-th dofs coordinates of $W_h$.}
		
		\item \textbf{PINN :} \only<3-4>{Using PINN prediction. \\	
		(UNet : \citep{odot_deepphysics_2021} ; FNO : \citep{newton_accel_2025})} \\
		\only<3>{Considering a fixed parameter with $k\in\{1,2,3\}$, we can use the following initialization for $i=1,\ldots,N_h$,
		$$\big(\vec{U}_{k,0}\big)_i = U_\theta(\bm{x}^{(i)},\bm{\mu}^{(k)})$$
		with $\bm{x}^{(i)}=\big(x^{(i)},y^{(i)}\big)$ the $i$-th dofs coordinates of $M_h$ and $U_\theta$ the PINN.}
		
		\item \textbf{Continuation method :} \only<4>{Using a coarse FE solution of a simpler parameter.}
		
		\only<4>{\begin{itemize}
			\item We consider a fixed parameter with $k\in\{2,3\}$.
			\item We consider a coarse grid ($16\times 16$ grid) and compute the FE solution of \eqref{eq:weak_pb} for the parameter $\bm{\mu}^{(k-1)}$.
			\item We interpolate the coarse solution to the current mesh.
			\item We use it as an initialization for the Newton method, i.e.
			$$\vec{U}_{k,0} = \big(\vec{u}_{k-1}, \vec{v}_{k-1}, \vec{p}_{k-1}, \vec{T}_{k-1}\big)$$
			where $\vec{u}_{k-1}$, $\vec{v}_{k-1}$, $\vec{p}_{k-1}$ and $\vec{T}_{k-1}$ are the FE solutions for the parameter $\bm{\mu}^{(k-1)}$.
		\end{itemize}
		}
	\end{itemize}
\end{frame}
	\renewcommand{\insertsectionheadSubtitle}{}

	\section{Enriched finite element method using PINN}
	\begin{frame}{Newton method - Additive approach}
    TODO
    % \vspace{5pt}
    % We want to solve the non linear system: \hfill \tiny $N_h$ : number of degrees of freedom.

    % \normalsize
    % \vspace{-10pt}
    % \begin{equation}
    %     F(\textcolor{red}{p_+ + u_\theta}) = 0 \tag{1}
    % \end{equation}

    % \vspace{-2pt}
    % with $F:\mathbb{R}^{N_h} \to \mathbb{R}^{N_h}$ a non linear operator and $\textcolor{red}{p_+}\in\mathbb{R}^{N_h}$ the unknown vector.

    % \begin{center}
    %     \small
    %     \begin{minipage}{0.9\linewidth}
    %         \begin{algorithm}[H]
    %             \SetAlgoLined
    %             \caption{\textcolor{red}{Additive approach} to solve \eqref{eq:nonlinear} }
    %             \textbf{Initialization step:} set \textcolor{red}{$p_+^{(0)} = 0$}\;
    %             \For{\( k \ge 0 \)}{
    %                 Solve the linear system \( F(\textcolor{red}{p_+^{(k)}+u_\theta}) + F'(\textcolor{red}{p_+^{(k)}+u_\theta}) \delta^{(k+1)} = 0 \) for \( \delta^{(k+1)} \)\;
    %                 Update \( \textcolor{red}{p_+^{(k+1)}} = \textcolor{red}{p_+^{(k)}} + \delta^{(k+1)} \)\;
    %             }
    %         \end{algorithm}
    %     \end{minipage}
    % \end{center}

    % \textbf{Advantage compared to DeepPhysics:}

    % \begin{center}
    % $u_\theta$ is not required to live in the same space as $p_+$.
    % \end{center}
\end{frame}


% \begin{frame}{Additive approach}
% 	\textbf{Variational Problem :} Let $u_{\theta} \in H^{k+1}(\Omega)\cap H^1_0(\Omega)$.
	
% 	\vspace{-5pt}
% 	\begin{equation}
% 		\label{eq:weakplus}
% 		\text{Find } p_h^+ \in V_h^0 \text{ such that}, \forall v_h \in V_h^0, a(p_h^+,v_h) = l(v_h) - a(u_{\theta},v_h),\tag{$\mathcal{P}_h^+$}
% 	\end{equation}
	
% 	\vspace{5pt}
% 	\begin{minipage}[t]{0.6\linewidth}
% 		with the \textcolor{red}{enriched trial space $V_h^+$} defined by
% 		\begin{equation*}
% 			V_h^+ = \left\{
% 			u_h^+= u_{\theta} + p_h^+, \quad p_h^+ \in V_h^0
% 			\right\}.
% 		\end{equation*}
	
% 		\vspace{20pt}
	
% 		\textbf{General Dirichlet BC :} If $u=g$ on $\partial \Omega$, then
% 		\[
% 			p_h^+ = g - u_{\theta} \text{\quad on } \partial \Omega,
% 		\]
% 		with $u_\theta$ the PINN prior. 
% 	\end{minipage} \qquad \begin{minipage}[t][][b]{0.28\linewidth}
% 		\vspace{-15pt}
% 		\centering
% 		\pgfimage[width=\linewidth]{images/correction/correction.pdf}
% 	\end{minipage}
% \end{frame}
	% \begin{frame}{Idea}
	\vspace{-20pt}
	\begin{figure}[htb]
		\centering
		\resizebox{\textwidth}{!}{%
			\begin{tikzpicture}
				\node at (0,0.8) {1 Geometry + 1 Force};
				\node[draw=none, inner sep=0pt] at (0,0) {\includegraphics[width=2cm]{images/correction/objective_onegeom_onefct.png}};
				\node at (0,-1) {$\begin{aligned}[t]
						\; \phi \quad \text{\small and} \quad &f \\
						\; (\text{\small and} \quad &g)
					\end{aligned}$};
				
				\draw[->, title, line width=1.5pt] (1.7,0.1) -- (2.7,0.1);
				
				\node[align=center] at (4,1) {Get PINNs prediction};
				\node[draw=none, inner sep=0pt] at (4,0.1) {\includegraphics[width=1.4cm]{images/correction/objective_pinns.jpg}};
				\node at (4,-0.8) {\fcolorbox{blue}{white}{$u_{NN}=\phi w_{NN}+g$}};
				\node at (4,-1.3) {\textcolor{blue}{$u_{NN}=g$ on $\Gamma$}};
				
				% Ajouter une flèche entre les deux rectangles
				\draw[->, title, line width=1.5pt] (5.2,0.1) -- (6.2,0.1);
				
				\node[align=center] at (7.8,1) {Correct prediction \\ with FEM};
				\node[draw=none, inner sep=0pt] at (7.8,-0.1) {\includegraphics[width=2.5cm]{images/correction/objective_corr.png}};		
				\node at (7.8,-1) {$u_{NN}\rightarrow\tilde{u}=u_{NN}+\tilde{C}$};
			\end{tikzpicture} 
		}%
	\end{figure}
	
	\vspace{-5pt}
	
	\textbf{Correct by adding :} Considering $u_{NN}$ as the prediction of our PINNs for (\ref{edp}), the correction problem consists in writing the solution as
	\begin{equation*}
		\tilde{u}=u_{NN}+\underset{\textcolor{red}{\ll 1}}{\fcolorbox{red}{white}{$\tilde{C}$}}
	\end{equation*}
	
	\vspace{-8pt}
	\begin{minipage}{\linewidth}
		\setstretch{0.5}
		and searching $\tilde{C}: \Omega \rightarrow \mathbb{R}^d$ such that
		\begin{equation*}
			\left\{\begin{aligned}
				L(\tilde{C})&=\tilde{f}, \; &&\text{in } \Omega, \\
				\tilde{C}&=0, \; &&\text{on } \Gamma,
			\end{aligned}\right. %\tag{$\mathcal{C}_{+}$} %\label{corr_add}
		\end{equation*}
		with $\tilde{f}=f-L(u_{NN})$. \refappendix{frame:fem}
	\end{minipage}
\end{frame}

\begin{frame}{Poisson on Square}
	Solving the \textcolor{orange}{Poisson problem} with homogeneous Dirichlet BC. \\
	\ding{217} \textbf{Domain :} $\Omega=[−0.5\pi,0.5\pi]^2$ \\
	\ding{217} \textbf{Analytical levelset function :}
	\small
	\begin{equation*}
		\phi(x,y)=(x-0.5\pi)(x+0.5\pi)(y-0.5\pi)(y+0.5\pi)
	\end{equation*} 
	\ding{217} \textbf{Analytical solution :}
	\small
	
	\vspace{-8pt}
	\begin{equation*}
		u_{ex}(x,y)=\exp\left(−\frac{(x-\mu_1)^2+(y-\mu_2)^2}{2}\right)\sin(2x)\sin(2y)
	\end{equation*} 
	\normalsize
	with $\mu_1,\mu_2\in[-0.5,0.5]$. 
	
	\vspace{8pt}
	Taking $\mu_1=0.05,\mu_2=0.22$, the solution is given by
	\begin{minipage}{0.68\linewidth}
		\centering
		\pgfimage[width=\linewidth]{images/correction/poisson_sol.png}
	\end{minipage}
	\begin{minipage}{0.28\linewidth}
		\flushright
		\pgfimage[width=0.9\linewidth]{images/correction/poisson_loss.png}
	\end{minipage}
\end{frame}

\begin{frame}{Theoretical results}
	TODO
	$\mu_1=0.05,\mu_2=0.22$
	\begin{center}
		\pgfimage[width=0.5\linewidth]{images/correction/theoretical.png}
	\end{center}
\end{frame}

\begin{frame}{Gains using our approach}	
	\vspace{10pt}
	
	\hspace{20pt}\begin{minipage}{0.05\linewidth}
		\footnotesize
		\rotatebox[origin=b]{90}{\textbf{Solution $\mathbb{P}_1$}} 
	\end{minipage}
	\begin{minipage}{0.8\linewidth}
		\centering
		\pgfimage[height=1.7cm]{images/correction/gains_P1.png}
	\end{minipage} 

	\vspace{5pt}

	\hspace{20pt}\begin{minipage}{0.05\linewidth}
		\footnotesize
		\rotatebox[origin=b]{90}{\textbf{Solution $\mathbb{P}_2$}} 
	\end{minipage}
	\begin{minipage}{0.8\linewidth}
		\centering
		\pgfimage[height=1.7cm]{images/correction/gains_P2.png}
	\end{minipage} 

	\vspace{5pt}

	\hspace{20pt}\begin{minipage}{0.05\linewidth}
		\footnotesize
		\rotatebox[origin=b]{90}{\textbf{Solution $\mathbb{P}_3$}} 
	\end{minipage}
	\begin{minipage}{0.8\linewidth}
		\centering
		\pgfimage[height=1.7cm]{images/correction/gains_P3.png}
	\end{minipage} 
\end{frame}

\begin{frame}{Time/Precision}	
	TODO
\end{frame}

	% \subsection{Complex geometries}

\begin{frame}{Learn a regular levelset}		
    \vspace{-10pt}
    \hypersetup{
		citecolor=white,
	}

    \begin{mytheo}{\footnotesize\citep{clemot_neural_2023}\normalsize}{fem}
		If we have a boundary domain $\Gamma$, the SDF is solution to the Eikonal equation:
		
		\begin{minipage}{0.7\linewidth}
			\hspace{100pt}
			$\left\{\begin{aligned}
				&||\nabla\phi(X)||=1, \; X\in\mathcal{O} \\
				&\phi(X)=0, \; X\in\Gamma \\
				&\nabla\phi(X)=n, \; X\in\Gamma
			\end{aligned}\right.$
		\end{minipage}
		\begin{minipage}{0.25\linewidth}
			\centering
			\pgfimage[width=0.7\linewidth]{images/newlines/levelset/points_normals.png}
		\end{minipage}
		
		with $\mathcal{O}$ a box which contains $\Omega$ completely and $n$ the exterior normal to $\Gamma$.
	\end{mytheo}

    \hypersetup{
        citecolor=other,
    }

    \vspace{5pt}

    \textbf{Objective:} Move on to complex geometries by using a levelset function to

    \begin{itemize}
        \item Sample points in the domain $\Omega$ for the PINN training.
        \item Impose exactly the boundary condition in PINN \citep{Sukumar_2022}.
    \end{itemize}

    \vspace{5pt}

	\textbf{How to learn a regular levelset ?} with a PINN by \textcolor{orange}{adding a regularization term},
	\vspace{-5pt}
	\begin{equation*}
		J_{reg} = \int_\mathcal{O} |\Delta\phi|^2,
	\end{equation*}
    and a sample of boundary points that considers the \textcolor{orange}{curvature} of $ \Gamma$. \filledstar

    % Curvature
\end{frame}

\begin{frame}{Numerical results}		
    \begin{figure}[!ht] \centering
		\includegraphics[width=\linewidth]{images/newlines/levelset/EikonalBean_curvature.png}

        % \vspace{10pt}

		\includegraphics[width=\linewidth]{images/newlines/levelset/boundary_curvature.png}
	\end{figure}

    % \vspace{-5pt}
    % TODO : Ajouter résultats "Poisson on Bean" + Mettre au propre les images. 
\end{frame}

\subsection{\filledstar A posteriori error estimates}

\begin{frame}{Problem considered} 
	\textbf{Problem statement:} Considering the \textcolor{red}{Poisson problem with Dirichlet BC}:
	\vspace{-5pt}
	\begin{equation*}
		\left\{
		\begin{aligned}
			-\Delta u & = f, \; &  & \text{in } \; \Omega \times \mathcal{M}, \\
			u         & = 0, \;  &  & \text{on } \; \Gamma \times \mathcal{M},
		\end{aligned}
		\right.
		% \label{eq:Lap2DMixed}\tag{$\mathcal{P}$}
	\end{equation*}

	with $\Omega=[-0.5\pi,0.5\pi]^2$ and $\mathcal{M}=[-0.5,0.5]^2$ ($p=2$).
	
	\vspace{2pt}
	\textbf{Analytical solution :}

	\vspace{-12pt}
	\begin{equation*}
		% \label{eq:analytical_solution_Lap2D}
		u(\bm{x};\bm{\mu})= \exp\left(-\frac{(x-\mu_1)^2+(y-\mu_2)^2}{2(0.15)^2}\right)\sin(2x)\sin(2y).
	\end{equation*}

	\vspace{2pt}
	\small
	\textbf{PINN training:} Imposing Dirichlet BC exactly in the PINN.

	\vspace{8pt}
\end{frame}

\begin{frame}{Adaptive mesh refinement}	
    \textbf{Adaptive refinement loop} using Dorfler marking strategy. \refappendix{frame:amr} %(residual estimator)
    
    \begin{center}
        \textbf{Standard FEM}
        \vspace{2pt}

        \includegraphics[width=0.2\linewidth]{images/newlines/mesh/explications/fem/u_h.png}
        \quad
        \includegraphics[width=0.2\linewidth]{images/newlines/mesh/explications/fem/eta_h.png}
        \quad
        \includegraphics[width=0.17\linewidth]{images/newlines/mesh/explications/fem/marking.png}
        \qquad
        \includegraphics[width=0.17\linewidth]{images/newlines/mesh/explications/fem/refined.png}
    \end{center}

    \vspace{-10pt}
    $\cdots\hspace{1pt}\longrightarrow\hspace{8pt}
    \text{SOLVE}\hspace{18pt}\longrightarrow\hspace{6pt}
    \text{ESTIMATE}\hspace{8pt}\longrightarrow\hspace{14pt}
    \text{MARK}\hspace{14pt}\longrightarrow\hspace{8pt}
    \text{REFINE}\hspace{4pt}\longrightarrow\hspace{1pt}
    \cdots$

    \hspace{45pt}$\text{on }u_h\hspace{55pt}\eta_{res,T}$

    \vspace{8pt}
    \textbf{Local residual estimator (in $L^2$ norm):} Let $T$ be a cell of $\mathcal{T}_h$ .

    \vspace{-8pt}
    % $$\eta_{res,T}^2 = h_T^4 \|\Delta u_h + f_h\|_{L^2(T)}^2 + \frac{1}{2} \sum_{E \in \partial T} h_E^2 \|[\nabla u_h\cdot n]\|_{L^2(E)}^2$$
    $$\eta_{res,T}^2 = h_T^2 \|\Delta u_h + f_h\|_{L^2(T)}^2 + \frac{1}{2} \sum_{E \in \partial T} h_E \|[\nabla u_h\cdot n]\|_{L^2(E)}^2$$
    with $h_\bullet$ the size of $\bullet$ and considering the Poisson problem.

    % Considering the Poisson problem with Dirichlet boundary conditions.

    % (en précisant que c'est le coût du solve qui est le plus important)
\end{frame}

\begin{frame}[noframenumbering]{Adaptive mesh refinement}	
    \textbf{Adaptive refinement loop} using Dorfler marking strategy.
    
    \begin{center}
        \textcolor{red}{\textbf{Additive Approach}}
        \vspace{2pt}

        \includegraphics[width=0.2\linewidth]{images/newlines/mesh/explications/add/p_h.png}
        \quad
        \includegraphics[width=0.2\linewidth]{images/newlines/mesh/explications/add/eta_h_add.png}
        \quad
        \includegraphics[width=0.17\linewidth]{images/newlines/mesh/explications/add/marking_add.png}
        \qquad
        \includegraphics[width=0.17\linewidth]{images/newlines/mesh/explications/add/refined_add.png}
    \end{center}

    \vspace{-10pt}
    $\cdots\hspace{1pt}\longrightarrow\hspace{8pt}
    \text{SOLVE}\hspace{18pt}\longrightarrow\hspace{6pt}
    \text{ESTIMATE}\hspace{8pt}\longrightarrow\hspace{14pt}
    \text{MARK}\hspace{14pt}\longrightarrow\hspace{8pt}
    \text{REFINE}\hspace{4pt}\longrightarrow\hspace{1pt}
    \cdots$

    \hspace{45pt}$\text{on }\textcolor{red}{p_h^+}\hspace{55pt}\eta_{res,T}$

    \vspace{8pt}
    \textbf{Local residual estimator (in $L^2$ norm):} Let $T$ be a cell of $\mathcal{T}_h$ .

    \vspace{-8pt}
    % $$\eta_{res,T}^2 = h_T^4 \|\textcolor{red}{\big((\Delta u_\theta)_h + \Delta p_h^+\big) + f_h}\|_{L^2(T)}^2 + \frac{1}{2} \sum_{E \in \partial T} h_E^2 \|\textcolor{red}{[\nabla p_h^+\cdot n]}\|_{L^2(E)}^2$$
    $$\eta_{res,T}^2 = h_T^2 \|\textcolor{red}{\big((\Delta u_\theta)_h + \Delta p_h^+\big) + f_h}\|_{L^2(T)}^2 + \frac{1}{2} \sum_{E \in \partial T} h_E \|\textcolor{red}{[\nabla p_h^+\cdot n]}\|_{L^2(E)}^2$$
    with $h_\bullet$ the size of $\bullet$ and considering the Poisson problem.
\end{frame}

\begin{frame}[noframenumbering]{Adaptive mesh refinement}	
    \textbf{Adaptive refinement loop} using Dorfler marking strategy.
    
    \begin{center}
        \textbf{Additive Approach \textcolor{red}{- No resolution}}
        \vspace{2pt}

        \includegraphics[width=0.2\linewidth]{images/newlines/mesh/explications/addnet/u_theta_h.png}
        \quad
        \includegraphics[width=0.2\linewidth]{images/newlines/mesh/explications/addnet/eta_h_addnet.png}
        \quad
        \includegraphics[width=0.17\linewidth]{images/newlines/mesh/explications/addnet/marking_addnet.png}
        \qquad
        \includegraphics[width=0.17\linewidth]{images/newlines/mesh/explications/addnet/refined_addnet.png}
    \end{center}

    \vspace{-10pt}
    $\cdots\longrightarrow
    \textcolor{red}{\text{INTERPOLATE}}\longrightarrow\hspace{6pt}
    \text{ESTIMATE}\hspace{8pt}\longrightarrow\hspace{14pt}
    \text{MARK}\hspace{14pt}\longrightarrow\hspace{8pt}
    \text{REFINE}\hspace{4pt}\longrightarrow\hspace{1pt}
    \cdots$

    \hspace{55pt}$\textcolor{red}{u_\theta}\hspace{55pt}\eta_{res,T}$

    \vspace{8pt}
    \textbf{Local residual estimator (in $L^2$ norm):} Let $T$ be a cell of $\mathcal{T}_h$ .

    \vspace{-8pt}
    % $$\eta_{res,T}^2 = h_T^4 \|\textcolor{red}{(\Delta u_\theta)_h + f_h}\|_{L^2(T)}^2$$
    $$\eta_{res,T}^2 = h_T^2 \|\textcolor{red}{(\Delta u_\theta)_h + f_h}\|_{L^2(T)}^2$$
    with $h_\bullet$ the size of $\bullet$ and considering the Poisson problem.
\end{frame}

\begin{frame}{Numerical results}
    \vspace{-10pt}
    \begin{center}
        \includegraphics[width=0.4\linewidth]{images/newlines/mesh/results/cvg.pdf}
        \includegraphics[width=0.4\linewidth]{images/newlines/mesh/results/times.pdf}
    \end{center}
    
    \vspace{-10pt}
    \footnotesize
    \warning \quad Results obtained on a laptop GPU (Time measurements polluted by external factors).
    
    \normalsize
    \vspace{5pt}
    \textbf{Ideas for improving results :} Additive approach (no resolution).

    \vspace{3pt}
    \begin{minipage}{0.1\linewidth}
        \begin{tikzpicture}[scale=1]
            \draw[->, thick] (0,1.8) -- (0.8,1);
            \node[above right] at (0.4,1.4) {\textbf{time}};

            \draw[->, thick] (0,0.8) -- (0.8,0);
            \node[above right] at (0.4,0.4) {\textbf{error}};
        \end{tikzpicture}
    \end{minipage} \hspace{5pt}
    \begin{minipage}{0.86\linewidth}
        \vspace{2pt}
        Interpolate only mesh points added in the refinement process. \\

        \vspace{5pt}
        Use another metric such as curvature, rather than residual error.

        % \vspace{-5pt}
        % $$\Delta u_\theta+f \qquad \ne \qquad u-u_\theta$$
    \end{minipage}
    % \begin{itemize}
    %     \item To improve execution times: \\ 
    %     Interpolate only mesh points added in the refinement process.
    %     % \item Cout du passage sur GPU.
    %     \item To improve the mesh : \\
    %     Use another metric such as curvature, rather than residual error. \\
    %     (The network residual ($\Delta u_\theta+f$) does not match the additive solution (the network error $u-u_\theta$).)

    % \end{itemize}

\end{frame}

\subsection{\filledstar Non linear PDEs}

\begin{frame}{Problem considered}	
    \textbf{Objective:} Extend the additive approach to non linear PDEs.

    \vspace{10pt}
    \textbf{Problem statement:} Considering the \textcolor{red}{non linear Poisson problem with Dirichlet BC}:
    \vspace{-5pt}
    \begin{equation*}
        \left\{
        \begin{aligned}
            -\text{div}\big((1+4u^4)\nabla u\big) & = f, \; &  & \text{in } \; \Omega, \\
            u         & = 1, \;  &  & \text{on } \; \partial\Omega.
        \end{aligned}
        \right.
    \end{equation*}

    with $\Omega=[-0.5\pi,0.5\pi]^2$ and $\mathcal{M}=[-0.5,0.5]^2$ ($p=2$).

    \vspace{5pt}
	\textbf{Analytical solution :}

	\vspace{-12pt}
	\begin{equation*}
		u(\bm{x};\bm{\mu})= 1 + \exp\left(-\frac{(x-\mu_1)^2+(y-\mu_2)^2}{2}\right)\sin(2x)\sin(2y)
	\end{equation*}
	\vspace{-5pt}
	
    \vspace{5pt}
	\textbf{PINN training:} Imposing BC exactly with a level-set function.

\end{frame}

\begin{frame}{Newton method}
    

    We want to solve the non linear system: \hfill \tiny $N_h$ : number of degrees of freedom.

    \normalsize
    \vspace{-10pt}
    \begin{equation}
        \label{eq:nonlinear}
        F(u) = 0 
    \end{equation}

    \vspace{-2pt}
    with $F:\mathbb{R}^{N_h} \to \mathbb{R}^{N_h}$ a non linear operator and $u\in\mathbb{R}^{N_h}$ the unknown vector.

    \begin{center}
        \small
        \begin{minipage}{0.9\linewidth}
            \begin{algorithm}[H]
                \SetAlgoLined
                \caption{Newton's method to solve \eqref{eq:nonlinear} \citep{newton_accel_2025}}
                \textbf{Initialization step:} set $u^{(0)} = u_0$\;
                \For{\( k \ge 0 \)}{
                    Solve the linear system \( F(u^{(k)}) + F'(u^{(k)}) \delta^{(k+1)} = 0 \) for \( \delta^{(k+1)} \)\;
                    Update \( u^{(k+1)} = u^{(k)} + \delta^{(k+1)} \)\;
                }
            \end{algorithm}
        \end{minipage}
    \end{center}

    \vspace{5pt}
    \textbf{Standard version:} \\
    Initialization with a constant value $u_0$. For instance, $u_0=1$.

    \vspace{5pt}
    \textbf{DeepPhysics version:} \citep{odot_deepphysics_2021} \\
    Initialization with a PINN solution $u_0=u_\theta$.
\end{frame}

\begin{frame}[noframenumbering]{Newton method}
    \vspace{5pt}
    We want to solve the non linear system: \hfill \tiny $N_h$ : number of degrees of freedom.

    \normalsize
    \vspace{-10pt}
    \begin{equation}
        F(\textcolor{red}{p_+ + u_\theta}) = 0 \tag{1}
    \end{equation}

    \vspace{-2pt}
    with $F:\mathbb{R}^{N_h} \to \mathbb{R}^{N_h}$ a non linear operator and $\textcolor{red}{p_+}\in\mathbb{R}^{N_h}$ the unknown vector.

    \begin{center}
        \small
        \begin{minipage}{0.9\linewidth}
            \begin{algorithm}[H]
                \SetAlgoLined
                \caption{\textcolor{red}{Additive approach} to solve \eqref{eq:nonlinear} }
                \textbf{Initialization step:} set \textcolor{red}{$p_+^{(0)} = 0$}\;
                \For{\( k \ge 0 \)}{
                    Solve the linear system \( F(\textcolor{red}{p_+^{(k)}+u_\theta}) + F'(\textcolor{red}{p_+^{(k)}+u_\theta}) \delta^{(k+1)} = 0 \) for \( \delta^{(k+1)} \)\;
                    Update \( \textcolor{red}{p_+^{(k+1)}} = \textcolor{red}{p_+^{(k)}} + \delta^{(k+1)} \)\;
                }
            \end{algorithm}
        \end{minipage}
    \end{center}

    \textbf{Advantage compared to DeepPhysics:}

    \begin{center}
    $u_\theta$ is not required to live in the same space as $p_+$.
    \end{center}
\end{frame}

\begin{frame}{Numerical results}	
    \vspace{-10pt}
    \begin{center}
        \begin{minipage}{0.54\linewidth}
            \includegraphics[width=\linewidth]{images/newlines/nonlinear/results/cvg_cropped.pdf}
        \end{minipage}
        \begin{minipage}{0.42\linewidth}
            \small
            \textbf{Number of iterations :}

            \begin{itemize}
                \item Standard Newton: $8$ iterations.
                \item DeepPhysics: $4$ iterations.
                \item Additive approach: $4$ iterations.
            \end{itemize}
        \end{minipage}
        
        \vspace{-6pt}
        \includegraphics[width=\linewidth]{images/newlines/nonlinear/results/times.pdf}
    \end{center}
\end{frame}

	\renewcommand{\insertsectionheadSubtitle}{
		\begin{itemize}
			\item Results obtained with a laptop GPU.
			\item The newton solver is the same for all methods (rtol$=10^{-10}$, atol$=10^{-10}$, max\_it$=30$).
			\item Additive approach : we consider $u_\theta$ in a $\mathbb{P}_3^2\times \mathbb{P}_2 \times \mathbb{P}_3$ continuous Lagrange FE space (defined on the current mesh).
		\end{itemize}}
	\section{Numerical results}
	\documentclass[french]{article}
\usepackage[T1]{fontenc}
\usepackage[utf8]{inputenc}
\usepackage[french]{babel}
\usepackage{amsmath}
\usepackage{mathtools}
\usepackage{color}
\usepackage[svgnames,dvipsnames]{xcolor} 
\usepackage{soul}
\usepackage{amssymb}
\usepackage{enumitem}
\usepackage{multicol}
\usepackage[left=2cm,right=2cm,top=2cm,bottom=2cm]{geometry}
\newcommand{\mathcolorbox}[2]{\colorbox{#1}{$\displaystyle #2$}}
\usepackage{pifont}
\usepackage{pst-all}
\usepackage{pstricks}
\usepackage{delarray}
\usepackage{setspace}
\usepackage{graphicx}
\usepackage{hyperref}
\usepackage{nicematrix}
\usepackage{listings}
\usepackage{float}

\hypersetup{
	colorlinks=true,
	linkcolor=blue,
	filecolor=magenta,      
	urlcolor=cyan,
	pdfpagemode=FullScreen,
}

\usepackage{amsthm}
\newtheorem*{Rem}{Remarque}

\newenvironment{conclusion}[1]{%
	\begin{center}\normalfont\textbf{Conclusion}\end{center}
	\begin{quotation} #1 \end{quotation}
}{%
	\vspace{1cm}
}

\newcommand\pythonstyle{\lstset{
	language=Python,
	basicstyle=\ttm,
	morekeywords={self},              % Add keywords here
	keywordstyle=\ttb\color{deepblue},
	emph={MyClass,__init__},          % Custom highlighting
	emphstyle=\ttb\color{deepred},    % Custom highlighting style
	stringstyle=\color{deepgreen},
	frame=tb,                         % Any extra options here
	showstringspaces=false
}}

\lstdefinestyle{Cpp}{
	language=C++,
	tabsize=3,
	basicstyle=\ttfamily,
	keywordstyle=\color{blue}\ttfamily,
	stringstyle=\color{red}\ttfamily,
	commentstyle=\color{green}\ttfamily,
	morecomment=[l][\color{magenta}]{\#}
}

\lstdefinestyle{Python}{
	language=Python,
	tabsize=3,
	basicstyle=\ttfamily,
	keywordstyle=\color{blue}\ttfamily,
	stringstyle=\color{red}\ttfamily,
	commentstyle=\color{green}\ttfamily,
	morecomment=[l][\color{magenta}]{\#}
}

\lstset{style=Cpp}

\setlength\parindent{0pt}
\usepackage[skip=2pt]{caption}

\usepackage{fontawesome}

\usepackage{lipsum}

\graphicspath{{images/}}

\usepackage[backend=biber,style=numeric,sorting=nyt]{biblatex}
\addbibresource{biblio.bib}



\begin{document}
	LECOURTIER Frédérique \hfill \today
	\begin{center}
		\Large\textbf{{Résultats}}
	\end{center}
	\tableofcontents
	\newpage

%	\nocite{*}

	\section{Étape 1 : Apprentissage d'une fonction Levelset $\phi$}
	
	\subsection{Génération des Inputs}
	
	On considère une courbe paramétrique $c(t)$ définissant un surface $\mathcal{S}$ (en 2D). En pratique, nous allons considérer les 3 formes définies par les courbes paramétriques, pour $t\in [0,1]$ :
	
	\begin{equation*}
		c_{circle}(t) = \begin{pmatrix}
			x_0 + r\cos(2\pi t) \\
			y_0 + r\sin(2\pi t)
		\end{pmatrix}, (x_0,y_0)=(0.5,0.5), r=\sqrt{2}/4
	\end{equation*}
	
	\begin{equation*}
		c_{bean}(t) = \begin{pmatrix}
			\sin(2\pi t) \times (\sin(2\pi t)^a+\cos(2\pi t)^b) \\
			-\cos(2\pi t) \times (\sin(2\pi t)^a+\cos(2\pi t)^b)
		\end{pmatrix}, a=3, b=5
	\end{equation*}

	\begin{equation*}
		c_{pumpkin}(t) = \begin{pmatrix}
			 \cos(2\pi t)+0.3\cos(6\pi t)+0.1\cos(10\pi t) \\
			 \sin(2\pi t)+0.3\sin(6\pi t)+0.1\sin(10\pi t)
		\end{pmatrix}
	\end{equation*}

	A partir de ces différentes courbes, on va générer un nombre $n_{bc}$ de points sur le surface $\mathcal{S}$ et on peut facilement calculer les normales à la surface en les points considérés. En considérants la dérivée $c'(t)$ de la courbe $c$ et $c'^{\perp}(t)$ définie comme la rotation de $c'$ d'un angle de $-\pi/2$, c'est-à-dire pour $\theta=\pi/2$, on a:
	\begin{equation*}
		c'^{\perp}(t) = \begin{pmatrix}
			\cos(\theta) & -\sin(\theta) \\
			\sin(\theta) & \cos(\theta)
		\end{pmatrix}c'(t)
	\end{equation*}
	et ainsi
	\begin{equation*}
		n(t)=\frac{c'^{\perp}(t)}{||c'^{\perp}(t)||}
	\end{equation*}
	
	\begin{minipage}{0.33\linewidth}
		\begin{figure}[H]
			\centering
			\includegraphics[width=\linewidth]{"levelset/circle/curve_circle.png"}
			\caption{Cercle.}
		\end{figure}
	\end{minipage}
	\begin{minipage}{0.33\linewidth}
		\begin{figure}[H]
			\centering
			\includegraphics[width=\linewidth]{"levelset/bean/curve_bean.png"}
			\caption{"Haricot".}
		\end{figure}
	\end{minipage}
	\begin{minipage}{0.33\linewidth}
		\begin{figure}[H]
			\centering
			\includegraphics[width=\linewidth]{"levelset/pumpkin/curve_pumpkin.png"}
			\caption{"Citrouille".}
		\end{figure}
	\end{minipage}

	\begin{Rem}
		Ce n'est que pour une question de simplicité qu'on considère une courbe paramétrique. En pratique, on peut partir uniquement d'un ensemble de points.
	\end{Rem}

	\subsection{Trouver une levelset}
	
	Pour déterminer $\phi$, on s'est basé sur l'article \cite{CLEMOT2023368} dont l'objectif est la génération d'une distance signée par la résolution de l'équation Eikonale dans le but de générer le squelette de géométrie complexe. Comme notre objectif n'est pas le même on a du remanier ce qui était présenté dans le bus que la fonction apprise soit utilisable dans l'apprentissage d'une solution à un problème donnée. On a alors considéré les loss suivantes présentées dans l'article afin d'obtenir une solution à l'équation Eikonal :
	\begin{equation*}
		\mathcal{L}_{eik}=\int_{\mathbb{R}^2} \left(1-||\nabla \phi(p)||\right)^2 dp
	\end{equation*}
	\begin{equation*}
		\mathcal{L}_{dir}=\int_{\partial\Omega} \phi^2 dp
	\end{equation*}
	\begin{equation*}
		\mathcal{L}_{neu}=\int_{\partial\Omega} 1-\frac{n(p)\cdot\nabla\phi(p)}{||n(p)||\;||\nabla\phi(p)||} dp
	\end{equation*}

	Dans notre cas, comme l'objectif est différent, on ne cherchera pas à approcher la sdf de la géométrie, on souhaite en fait déterminer une fonction levelset qui a certaines bonnes propriétés qui permettraient de l'utiliser par la suite. En particulier, on cherchera à limiter l'explosion des dérivées secondes et pour cela, on ajoutera la loss suivante : 
	\begin{equation*}
		\mathcal{L}_{lap}=\int_{\mathbb{R}^2} \Delta\phi^2 dp
	\end{equation*}

	Pour la résolution du problème, on considérera un PINNs dont le modèle sera défini par un simple MLP et on cherchera $\phi_\theta(x,y)$ avec la loss définie par
	\begin{equation*}
		\mathcal{L}=w_{eik}\mathcal{L}_{eik}+w_{dir}\mathcal{L}_{dir}+w_{neu}\mathcal{L}_{neu}+w_{lap}\mathcal{L}_{lap}
	\end{equation*}
	
	\begin{Rem}
		Pour l'instant, on entraîne un modèle par géométrie mais dans la suite on cherchera à construire un modèle paramétrique.
	\end{Rem}

	\subsection{Résultats}
	
	\textbf{Circle :}
	
	\begin{figure}[H]
		\centering
		\includegraphics[width=\linewidth]{"levelset/circle/config_circle.png"}
		\caption{Configuration.}
	\end{figure}
	
	\begin{minipage}{0.33\linewidth}
		\begin{figure}[H]
			\centering
			\includegraphics[width=\linewidth]{"levelset/circle/loss_circle.png"}
			\caption{Loss (tv=lap).}
		\end{figure}
	\end{minipage}
	\begin{minipage}{0.33\linewidth}
		\begin{figure}[H]
			\centering
			\includegraphics[width=\linewidth]{"levelset/circle/sol_circle.png"}
			\caption{Levelset $\phi_\theta$.}
		\end{figure}
	\end{minipage}
	\begin{minipage}{0.33\linewidth}
		\begin{figure}[H]
			\centering
			\includegraphics[width=\linewidth]{"levelset/circle/sol_mask_circle.png"}
			\caption{Levelset $\phi_\theta<0$.}
		\end{figure}
	\end{minipage}

	\begin{minipage}{0.56\linewidth}
		\begin{figure}[H]
			\centering
			\includegraphics[width=\linewidth]{"levelset/circle/bc_circle.png"}
			\caption{Loss (tv=lap).}
		\end{figure}
	\end{minipage}
	\begin{minipage}{0.43\linewidth}
		\begin{figure}[H]
			\centering
			\includegraphics[width=\linewidth]{"levelset/circle/derivees_circle.png"}
			\caption{Levelset $\phi_\theta$.}
		\end{figure}
	\end{minipage}
	
	\newpage
	\textbf{Bean :}
	
	\begin{figure}[H]
		\centering
		\includegraphics[width=\linewidth]{"levelset/bean/config_bean.png"}
		\caption{Configuration.}
	\end{figure}
	
	\begin{minipage}{0.33\linewidth}
		\begin{figure}[H]
			\centering
			\includegraphics[width=\linewidth]{"levelset/bean/loss_bean.png"}
			\caption{Loss (tv=lap).}
		\end{figure}
	\end{minipage}
	\begin{minipage}{0.33\linewidth}
		\begin{figure}[H]
			\centering
			\includegraphics[width=\linewidth]{"levelset/bean/sol_bean.png"}
			\caption{Levelset $\phi_\theta$.}
		\end{figure}
	\end{minipage}
	\begin{minipage}{0.33\linewidth}
		\begin{figure}[H]
			\centering
			\includegraphics[width=\linewidth]{"levelset/bean/sol_mask_bean.png"}
			\caption{Levelset $\phi_\theta<0$.}
		\end{figure}
	\end{minipage}
	
	\begin{minipage}{0.56\linewidth}
		\begin{figure}[H]
			\centering
			\includegraphics[width=\linewidth]{"levelset/bean/bc_bean.png"}
			\caption{Loss (tv=lap).}
		\end{figure}
	\end{minipage}
	\begin{minipage}{0.43\linewidth}
		\begin{figure}[H]
			\centering
			\includegraphics[width=\linewidth]{"levelset/bean/derivees_bean.png"}
			\caption{Levelset $\phi_\theta$.}
		\end{figure}
	\end{minipage}

	\textbf{Pumpkin :}
	
	\begin{figure}[H]
		\centering
		\includegraphics[width=\linewidth]{"levelset/pumpkin/config_pumpkin.png"}
		\caption{Configuration.}
	\end{figure}
	
	\begin{minipage}{0.33\linewidth}
		\begin{figure}[H]
			\centering
			\includegraphics[width=\linewidth]{"levelset/pumpkin/loss_pumpkin.png"}
			\caption{Loss (tv=lap).}
		\end{figure}
	\end{minipage}
	\begin{minipage}{0.33\linewidth}
		\begin{figure}[H]
			\centering
			\includegraphics[width=\linewidth]{"levelset/pumpkin/sol_pumpkin.png"}
			\caption{Levelset $\phi_\theta$.}
		\end{figure}
	\end{minipage}
	\begin{minipage}{0.33\linewidth}
		\begin{figure}[H]
			\centering
			\includegraphics[width=\linewidth]{"levelset/pumpkin/sol_mask_pumpkin.png"}
			\caption{Levelset $\phi_\theta<0$.}
		\end{figure}
	\end{minipage}
	
	\begin{minipage}{0.56\linewidth}
		\begin{figure}[H]
			\centering
			\includegraphics[width=\linewidth]{"levelset/pumpkin/bc_pumpkin.png"}
			\caption{Loss (tv=lap).}
		\end{figure}
	\end{minipage}
	\begin{minipage}{0.43\linewidth}
		\begin{figure}[H]
			\centering
			\includegraphics[width=\linewidth]{"levelset/pumpkin/derivees_pumpkin.png"}
			\caption{Levelset $\phi_\theta$.}
		\end{figure}
	\end{minipage}

	\section{Étape 2 : Résolution de Poisson}
	
	On considère le problème suivant
	\begin{equation*}
		\left\{\begin{aligned}
			&-\Delta u=1 \; \text{dans } \Omega \\
			&u=0 \; \text{sur } \partial\Omega
		\end{aligned}\right.
	\end{equation*}
	où $\Omega$ pourra représenter le cercle, le haricot ou la citrouille.
	
	On cherchera à entraîner un nouveau PINNs à apprendre $w$ tel que $u=\phi_\theta w$ avec $\phi_ \theta$ la levelset apprise précédemment.
	
	\textbf{Circle :}
	
	\begin{figure}[H]
		\centering
		\includegraphics[width=\linewidth]{"poisson/circle/config.png"}
		\caption{Configuration.}
	\end{figure}
	
	\begin{minipage}{0.48\linewidth}
		\begin{figure}[H]
			\centering
			\includegraphics[width=0.9\linewidth]{"poisson/circle/loss.png"}
			\caption{Loss.}
		\end{figure}
	\end{minipage}
	\begin{minipage}{0.48\linewidth}
		\begin{figure}[H]
			\centering
			\includegraphics[width=0.9\linewidth]{"poisson/circle/sol.png"}
			\caption{Sol $u_\theta$.}
		\end{figure}
	\end{minipage}
	
	\begin{figure}[H]
			\centering
			\includegraphics[width=\linewidth]{"poisson/circle/compare.png"}
			\caption{Comparaison avec une solution FEM sur-rafinée.}
	\end{figure}

	\textbf{Bean - 1 :}
	
	\begin{figure}[H]
		\centering
		\includegraphics[width=\linewidth]{"poisson/bean/config_1.png"}
		\caption{Configuration.}
	\end{figure}
	
	\begin{minipage}{0.48\linewidth}
		\begin{figure}[H]
			\centering
			\includegraphics[width=0.9\linewidth]{"poisson/bean/loss_1.png"}
			\caption{Loss.}
		\end{figure}
	\end{minipage}
	\begin{minipage}{0.48\linewidth}
		\begin{figure}[H]
			\centering
			\includegraphics[width=0.9\linewidth]{"poisson/bean/sol_1.png"}
			\caption{Sol $u_\theta$.}
		\end{figure}
	\end{minipage}
	
	\begin{figure}[H]
		\centering
		\includegraphics[width=\linewidth]{"poisson/bean/compare_1.png"}
		\caption{Comparaison avec une solution FEM sur-rafinée.}
	\end{figure}

	\newpage

	\textbf{Bean - 2 :}
	
	\begin{figure}[H]
		\centering
		\includegraphics[width=\linewidth]{"poisson/bean/config_2.png"}
		\caption{Configuration.}
	\end{figure}
	
	\begin{minipage}{0.48\linewidth}
		\begin{figure}[H]
			\centering
			\includegraphics[width=0.9\linewidth]{"poisson/bean/loss_2.png"}
			\caption{Loss.}
		\end{figure}
	\end{minipage}
	\begin{minipage}{0.48\linewidth}
		\begin{figure}[H]
			\centering
			\includegraphics[width=0.9\linewidth]{"poisson/bean/sol_2.png"}
			\caption{Sol $u_\theta$.}
		\end{figure}
	\end{minipage}
	
	\begin{figure}[H]
		\centering
		\includegraphics[width=\linewidth]{"poisson/bean/compare_2.png"}
		\caption{Comparaison avec une solution FEM sur-rafinée.}
	\end{figure}
	
	\textbf{Pumpkin :}
	
	\begin{figure}[H]
		\centering
		\includegraphics[width=\linewidth]{"poisson/pumpkin/config.png"}
		\caption{Configuration.}
	\end{figure}
	
	\begin{minipage}{0.48\linewidth}
		\begin{figure}[H]
			\centering
			\includegraphics[width=0.9\linewidth]{"poisson/pumpkin/loss.png"}
			\caption{Loss.}
		\end{figure}
	\end{minipage}
	\begin{minipage}{0.48\linewidth}
		\begin{figure}[H]
			\centering
			\includegraphics[width=0.9\linewidth]{"poisson/pumpkin/sol.png"}
			\caption{Sol $u_\theta$.}
		\end{figure}
	\end{minipage}
	
	\begin{figure}[H]
		\centering
		\includegraphics[width=\linewidth]{"poisson/pumpkin/compare.png"}
		\caption{Comparaison avec une solution FEM sur-rafinée.}
	\end{figure}

	\newpage
	\section{Étape 3 : Correction par addition}
	
	\subsection{Avec FEM}
	
	\textbf{Circle :}
	
	\begin{figure}[H]
		\centering
		\includegraphics[width=0.55\linewidth]{"correction/circle/corr_FEM.png"}
		\caption{Correction par addition avec FEM.}
	\end{figure}
	
	\textbf{Bean - 1 :}
	
	\begin{figure}[H]
		\centering
		\includegraphics[width=0.55\linewidth]{"correction/bean/corr_FEM_1.png"}
		\caption{Correction par addition avec FEM.}
	\end{figure}

	\textbf{Bean - 2 :}
	
	\begin{figure}[H]
		\centering
		\includegraphics[width=0.55\linewidth]{"correction/bean/corr_FEM_2.png"}
		\caption{Correction par addition avec FEM.}
	\end{figure}

	\textbf{Pumpkin :}
	
	\begin{figure}[H]
		\centering
		\includegraphics[width=0.55\linewidth]{"correction/pumpkin/corr_FEM.png"}
		\caption{Correction par addition avec FEM.}
	\end{figure}
	
	
	\newpage
	\subsection{Avec $\phi$-FEM}
	
	\textbf{Circle :}
	
	\begin{figure}[H]
		\centering
		\includegraphics[width=0.6\linewidth]{"correction/circle/corr_PhiFEM.png"}
		\caption{Correction par addition avec $\phi$-FEM.}
	\end{figure}

	\begin{figure}[H]
		\centering
		\includegraphics[width=0.6\linewidth]{"correction/circle/corr_PhiFEM_Omega.png"}
		\caption{Correction par addition avec $\phi$-FEM (projeté).}
	\end{figure}
	
	\newpage
	\textbf{Bean - 1 :}
	
	\begin{figure}[H]
		\centering
		\includegraphics[width=0.6\linewidth]{"correction/bean/corr_PhiFEM_1.png"}
		\caption{Correction par addition avec $\phi$-FEM.}
	\end{figure}
	
	\begin{figure}[H]
		\centering
		\includegraphics[width=0.6\linewidth]{"correction/bean/corr_PhiFEM_1_Omega.png"}
		\caption{Correction par addition avec $\phi$-FEM (projeté).}
	\end{figure}

	\newpage
	\textbf{Bean - 2 :}
	
	\begin{figure}[H]
		\centering
		\includegraphics[width=0.6\linewidth]{"correction/bean/corr_PhiFEM_2.png"}
		\caption{Correction par addition avec $\phi$-FEM.}
	\end{figure}
	
	\begin{figure}[H]
		\centering
		\includegraphics[width=0.6\linewidth]{"correction/bean/corr_PhiFEM_2_Omega.png"}
		\caption{Correction par addition avec $\phi$-FEM (projeté).}
	\end{figure}

	\newpage
	\textbf{Pumpkin :}
	
	\begin{figure}[H]
		\centering
		\includegraphics[width=0.6\linewidth]{"correction/pumpkin/corr_PhiFEM.png"}
		\caption{Correction par addition avec $\phi$-FEM.}
	\end{figure}
	
	\begin{figure}[H]
		\centering
		\includegraphics[width=0.6\linewidth]{"correction/pumpkin/corr_PhiFEM_Omega.png"}
		\caption{Correction par addition avec $\phi$-FEM (projeté).}
	\end{figure}

	\newpage
%	\section*{Bibliography}
	\printbibliography
\end{document}
	\renewcommand{\insertsectionheadSubtitle}{}
	% \section{New lines of research}

	\section*{Conclusion}

	\begin{frame}{Conclusion}
		\begin{itemize}
			\item The enriched approach provides the same results as the standard FEM method, but with \textbf{coarser meshes}. \\
			$\Rightarrow$ Reduction of the computational cost : DoFs, iterations, execution times.
			\item Theory on linear problems shows that it's the \textbf{derivatives} of the prior that are the most crucial. \\
			$\Rightarrow$ PINNs are good candidates for the enriched approach.
			\item The gains obtained on linear problems were much higher. \\
			$\Rightarrow$ \textbf{Improved training} of parametric PINN (or Neural Operators).
		\end{itemize}

		\flushright
		\begin{minipage}{0.05\linewidth}
			\flushright
			\rotatebox{90}{\textbf{Preprint (linear)}}
		\end{minipage} \; \begin{minipage}{0.28\linewidth}
		\includegraphics[width=\linewidth]{images/qrcode_paper.pdf}
		\end{minipage}
	\end{frame}

	\BackgroundBiblio
	
	{\setbeamertemplate{footline}{} 	
	\begin{frame}{References}
		\scriptsize
		\bibliography{biblio}
	\end{frame}
	}
	\addtocounter{framenumber}{-1} 
	
	\Background

	% \section{Appendix}
	
	\appendix
	
	\appendixsection{Finite element method (FEM)}\labelappendixframe{frame:appfem}

\begin{appendixframe}{Construction of the unknown vector}\labelappendixframe{frame:basis}

	Considering $(\phi_i)_{i=1}^{N_u}$, $(\psi_j)_{j=1}^{N_p}$ and $(\eta_k)_{k=1}^{N_T}$ the basis functions of the finite element spaces $V_h^{\, 0}$, $Q_h$ and $W_h$ respectively, we can write the discrete solutions as:
	\begin{equation*}
		\bm{u}_h(\bm{x}) = \sum_{i=1}^{N_u} \begin{pmatrix}
			u_i \\
			v_i
		\end{pmatrix} \phi_i(\bm{x}), \quad p_h(\bm{x}) = \sum_{j=1}^{N_p} p_j \psi_j(\bm{x}) \quad \text{and} \quad T_h(\bm{x}) = \sum_{k=1}^{N_T} T_k \eta_k(\bm{x}),
	\end{equation*}	
	with the unknown vectors for velocity, pressure and temperature defined by

	\vspace{-5pt}
	$$\vec{u} = \big(u_i\big)_{i=1}^{N_u} \in \mathbb{R}^{N_u}, \quad \vec{v} = \big(v_i\big)_{i=1}^{N_u} \in \mathbb{R}^{N_u},$$
	$$\vec{p} = \big(p_j\big)_{j=1}^{N_p} \in \mathbb{R}^{N_p} \; \text{ and } \; \vec{T} = \big(T_k\big)_{k=1}^{N_T} \in \mathbb{R}^{N_T}.$$

	\vspace{5pt}
	Considering $N_h = 2N_u + N_p + N_T$, we can define the global vector of unknowns as:
	\begin{equation*}
		\vec{U} = \big(\vec{u}, \vec{v}, \vec{p}, \vec{T}) \in \mathbb{R}^{N_h}.
	\end{equation*}
	and $F:\mathbb{R}^{N_h} \to \mathbb{R}^{N_h}$ the nonlinear operator associated to the weak formulation \eqref{eq:weak_pb}.
\end{appendixframe}

\appendixsection{PINN Initialization / Additive approach}\labelappendixframe{frame:comp}


\begin{appendixframe}{Comparison of the 2 approaches}
	Taking $U_\theta$ and $C_h^+$ in the same space, we have :
	$$F_\theta(\vec{C})=F(\vec{U}_\theta+\vec{C}),$$
	with $\vec{C}$ the correction vector and $\vec{U}_\theta$ the PINN vector (PINN evaluation at the dofs), both of size $N_h$.

	The first iteration of the additive approach :
	$$ F_\theta(\vec{C}^{(0)}) + F_\theta'(\vec{C}^{(0)}) \delta^{(1)} = 0 $$
	becomes (as $C^{(0)}=0$) :
	$$F(\vec{U}_\theta) + F'(\vec{U}_\theta)\delta^{(1)}=0,$$
	which is equivalent as the standard method with the PINN initialization.
\end{appendixframe}

%% AUTRES

\appendixsection{Results - Linear problem}\labelappendixframe{frame:linear}

\begin{appendixframe}{Problem considered} 
	\textbf{Problem statement:} Consider the Poisson problem with Dirichlet BC:
	\vspace{-5pt}
	\begin{equation*}
		\left\{
		\begin{aligned}
			-\Delta u & = f, \; &  & \text{in } \; \Omega \times \mathcal{M}, \\
			u         & =0, \;  &  & \text{on } \; \partial\Omega \times \mathcal{M},
		\end{aligned}
		\right.
		% \label{eq:Lap2D}\tag{$\mathcal{P}$}
	\end{equation*}

	\vspace{-5pt}
	with $\Omega=[-0.5 \pi, 0.5 \pi]^2$ and $\mathcal{M}=[-0.5,0.5]^2$ ($p=2$ parameters).
		
	\vspace{8pt}
	\textbf{Analytical solution :}

	\vspace{-5pt}
	\begin{equation*}
		% \label{eq:analytical_solution_Lap2D}
		u(\bm{x},\bm{\mu})=\exp\left(-\frac{(x-\mu_1)^2+(y-\mu_2)^2}{2}\right)\sin(2 x)\sin(2 y).
	\end{equation*}

	\vspace{12pt}
	\textbf{PINN training:} MLP of 5 layers; LBFGs optimizer (5000 epochs). \\
	Imposing the Dirichlet BC exactly in the PINN with the levelset $\varphi$ defined by
	$$\varphi(\bm{x})=(x+0.5\pi)(x-0.5\pi)(y+0.5\pi)(y-0.5\pi).$$
	
	\small\vspace{4pt}
	Training time : less than 1 hour on a laptop GPU.
\end{appendixframe}

\begin{appendixframe}{Numerical results}
	\hspace{-5pt}\begin{minipage}[t]{0.46\linewidth}
		\textbf{Error estimates :} 1 set of parameters.
		$$\bm{\mu}^{(1)}=(0.05, 0.22) $$
		\vspace{-35pt}
		\begin{figure}[H]
			\cvgFEMCorrAlldeg{images/appendix/linear_results/cvg/FEM_case1_v1_param1.csv}{images/appendix/linear_results/cvg/Corr_case1_v1_param1.csv}{1e-10}
		\end{figure}
	\end{minipage} \qquad \small
	\begin{minipage}[t]{0.48\linewidth}
	\end{minipage}
\end{appendixframe}

\begin{appendixframe}{Numerical results}[noframenumbering]
	\hspace{-5pt}\begin{minipage}[t]{0.46\linewidth}
		\textbf{Error estimates :} 1  set of parameters.
		$$\bm{\mu}^{(1)}=(0.05, 0.22) $$
		\vspace{-35pt}
		\begin{figure}[H]
			\cvgFEMCorrAlldeg{images/appendix/linear_results/cvg/FEM_case1_v1_param1.csv}{images/appendix/linear_results/cvg/Corr_case1_v1_param1.csv}{1e-10}
		\end{figure}
	\end{minipage} \qquad \small
	\begin{minipage}[t]{0.48\linewidth}
		\textbf{Gains achieved :} $n_p=50$ sets of parameters.
		$$\mathcal{S}=\left\{\bm{\mu}^{(1)},\dots,\bm{\mu}^{(n_p)}\right\}$$
		\vspace{-15pt}
		\begin{table}[H]
			\gainstableallq{images/appendix/linear_results/gains/Tab_stats_case1_v1.csv}
		\end{table}

		\normalsize\centering\vspace{-20pt}
		$$N=20$$

		\vspace{-5pt}
		Gain : $\| u-u_h\|_{L^2} / \| u-u_h^+\|_{L^2}$ \\
		
		\small\vspace{8pt}
		Cartesian mesh : $N^2$ nodes.
	\end{minipage}
\end{appendixframe}

\begin{appendixframe}{Numerical results}[noframenumbering]
	\hspace{-5pt}\begin{minipage}[t]{0.46\linewidth}
		\textbf{Error estimates :} 1 set of parameters.
		$$\bm{\mu}^{(1)}=(0.05, 0.22) $$
		\vspace{-35pt}
		\begin{figure}[H]
			\cvgFEMCorrAlldegLine{images/appendix/linear_results/cvg/FEM_case1_v1_param1.csv}{images/appendix/linear_results/cvg/Corr_case1_v1_param1.csv}{1e-10}
		\end{figure}
	\end{minipage} \qquad \small
	\begin{minipage}[t]{0.48\linewidth}
		\textbf{$N_\text{dofs}$ required to reach the same error $e$ :}

		\vspace{10pt}
		\begin{table}[H]
			\centering
			\coststableallq{images/appendix/linear_results/costs/TabDoFs_case1_v1_param1.csv}
		\end{table}
	\end{minipage}
\end{appendixframe}

\appendixsection{Data-driven vs Physics-Informed training}\labelappendixframe{frame:datavspinns}

\begin{appendixframe}{Problem considered}
	\textbf{Problem statement:} Consider the Poisson problem in 1D with Dirichlet BC:
	\vspace{-5pt}
	\begin{equation*}
		\left\{
		\begin{aligned}
			-\partial_{xx} u & = f, \; &  & \text{in } \; \Omega \times \mathcal{M}, \\
			u         & = 0, \;  &  & \text{on } \; \partial\Omega \times \mathcal{M},
		\end{aligned}
		\right.
		% \label{eq:Lap2D}\tag{$\mathcal{P}$}
	\end{equation*}

	\vspace{-5pt}
	with $\Omega=[0,1]^2$ and $\mathcal{M}=[0,1]^3$ ($p=3$ parameters).
		
	\vspace{4pt}
	\textbf{Analytical solution :} $\quad u(x;\bm{\mu})=\mu_1\sin(2\pi x)+\mu_2\sin(4\pi x)+\mu_3\sin(6\pi x) \,.$

	\vspace{4pt}
	\textbf{Construction of two priors:} MLP of 6 layers; Adam optimizer (10000 epochs). \\
	Imposing the Dirichlet BC exactly in the PINN with $\varphi(x)=x(x-1)$.

	\begin{itemize}
		\item \textbf{Physics-informed training:} $N_\text{col}=5000$ collocation points.
		$$J_r(\theta) \simeq
			\frac{1}{N_\text{col}} \sum_{i=1}^{N_\text{col}} \big| \partial_{xx}u_\theta(\bm{x}_\text{col}^{(i)};\bm{\mu}_\text{col}^{(i)}\big) + f\big(\bm{x}_\text{col}^{(i)};\bm{\mu}_\text{col}^{(i)}\big) \big|^2.$$
	
		\item \textbf{Data-driven training:}  $N_\text{data}=5000$ data.
		$$J_\text{data}(\theta) =
		\frac{1}{N_\text{data}}
		\sum_{i=1}^{N_\text{data}} \big| u_\theta^\text{data}(\bm{x}_\text{data}^{(i)};\bm{\mu}_\text{data}^{(i)}) - u(\bm{x}_\text{data}^{(i)};\bm{\mu}_\text{data}^{(i)}) \big|^2.$$
	\end{itemize}
\end{appendixframe}

\begin{appendixframe}{Priors derivatives}
	\vspace{-10pt}
	$$\bm{\mu}^{(1)}=(0.3,0.2,0.1)$$
	\begin{figure}[ht!]
		\centering
		\includegraphics[width=\linewidth]{images/appendix/datavspinns/standalone_solutions_and_errors_PINN.pdf}
	\end{figure}
	
	\begin{figure}[ht!]
		\centering
		\includegraphics[width=\linewidth]{images/appendix/datavspinns/standalone_solutions_and_errors_NN.pdf}
	\end{figure}
\end{appendixframe}

\begin{appendixframe}{Additive approach in $\mathbb{P}_1$}
	\vspace{-2pt}
	\textbf{1 set of parameters:} $\quad \bm{\mu}^{(1)}=(0.3,0.2,0.1)$
	
	\begin{table}[H]
		\centering
		\gainsbothNN{images/appendix/datavspinns/FEM_param1.csv}{images/appendix/datavspinns/compare_gains_param1.csv}
	\end{table}

	\vspace{6pt}
	\textbf{50 set of parameters:}

	\begin{table}[H]
		\centering
		\gainstableMult{images/appendix/datavspinns/Tab_stats_case1_degree1.csv}
	\end{table}

	\footnotesize
	$N$ : Nodes.
\end{appendixframe}
	
\end{document}
