%% Requires compilation with XeLaTeX or LuaLaTeX
\documentclass[compress,10pt,xcolor={table,dvipsnames},t]{beamer}
\usetheme{diapo}
\usepackage{amsmath}
\DeclareMathOperator*{\argmax}{arg\,max}
\DeclareMathOperator*{\argmin}{argmin}
\usepackage{amssymb}
\usepackage{xcolor}
\usepackage[bottom]{footmisc}
\usepackage{multirow}
\usepackage{setspace}
\usepackage{caption}
\usepackage{array,multirow,makecell}
\usepackage{pifont}
\usepackage{tikz}
\usepackage{paralist}
\usepackage{appendixnumberbeamer}
\usepackage[style=numeric,sorting=nyt,doi=false,url=false,maxbibnames=99,date=year]{biblatex}
\usepackage{etoolbox}
% box colorée dans équation
\usepackage[most]{tcolorbox}
\usepackage{tikz}
\usepackage{soul}
% pour l'indicatrice
\usepackage{dsfont}
\usepackage{cancel}

%%% configurer bibliographie

% Charger votre fichier de bibliographie
\addbibresource{biblio.bib}

% Définir les champs à ne pas afficher dans la bibliographie
\AtEveryBibitem{
	\clearlist{language}
	\clearfield{note}
	\clearfield{edition}
	\clearfield{series}
	\clearfield{url}
	\clearfield{urldate}
	\clearfield{pagetotal}
	\clearfield{pages}
	\clearfield{issn}
	\clearfield{doi}
	\clearfield{url}
	\clearfield{eprint}
}

% Définir le style des citations
\DeclareFieldFormat{title}{\textit{#1}}
\DeclareFieldFormat[article]{title}{#1}
\DeclareNameAlias{sortname}{last-first}
\DeclareFieldFormat{author}{#1.}

% Choisir les informations à afficher dans la bibliographie
\renewbibmacro{in:}{}
\renewbibmacro*{journal+issuetitle}{%
	\ifentrytype{article}{
		\usebibmacro{journal}
	}{%
		\printfield{title}%
	}%
}

\renewbibmacro*{date}{%
	\ifentrytype{misc}{%
		\printtext{\printdate}%
	}{%
		\printdate
	}%
}
%% fin configuration biblio

\setcellgapes{1pt}
\setlength{\parindent}{0pt}
\makegapedcells
\newcolumntype{R}[1]{>{\raggedleft\arraybackslash }b{#1}}
\newcolumntype{L}[1]{>{\raggedright\arraybackslash }b{#1}}
\newcolumntype{C}[1]{>{\centering\arraybackslash }b{#1}}
\renewcommand*{\bibfont}{\scriptsize}
\useoutertheme[subsection=false]{miniframes}
\makeatletter
%\patchcmd{\slideentry}{\advance\beamer@xpos by1\relax}{}{}{}
\def\beamer@subsectionentry#1#2#3#4#5{\advance\beamer@xpos by1\relax}%
\makeatother
\setbeamercolor*{mini frame}{fg=bulles,bg=bulles}
\hypersetup{
	colorlinks=true,
	urlcolor=blue,
	citecolor=other,
	linkcolor=title,
}

\title[PhiFEM]{How to work with complex geometries in PINNs ?}
%\subtitle{Macaron/Tonus retreat presentation}
\authors[Frédérique LECOURTIER]
\supervisors[Emmanuel FRANCK, Michel DUPREZ, Vanessa LLERAS]
\date{March 26, 2024}

\allowbreak

% u_chapeau (chapeau en couleur)
\usepackage{accents}
\newcommand{\uchapeau}[1]{\accentset{\textcolor{red}{\wedge}}{#1}}
\newcommand{\refappendix}[1]{\tikz[baseline=(char.base)]{\node[framednumber] (char) {\hyperlink{#1}{\small \textcolor{white}{Appendix \ref*{#1}}}};}}

\definecolor{appendix}{RGB}{180, 189, 138}
\tikzset{
	framednumber/.style={
		draw=appendix,% Couleur de la bordure
		fill=appendix, % Couleur de fond
		rounded corners, % Coins arrondis
		inner sep=2pt,  % Espace intérieur
	}
}

% numérotation et label des appendix
\newcounter{appendixframenumber}
\setcounter{appendixframenumber}{1}

\makeatletter
\newcommand{\labelappendixframe}[1]{%
	\protected@write\@auxout{}{%
		\string\newlabel{#1}{{\theappendixframenumber}{\thepage}}%
	}%
	\hypertarget{#1}{}
}	
\makeatother

%\newenvironment{appendixframe}[2][]{%
%	\begin{frame}[#1]{\appendixname~\theappendixframenumber~: #2}%
%	}{%
%	\end{frame}
%	\addtocounter{appendixframenumber}{1}
%}

% barre en couleur terme dans équation
\newcommand\Ccancel[2][black]{\renewcommand\CancelColor{\color{#1}}\cancel{#2}}

% chifrre romain dans le texte
\makeatletter
\newcommand*{\rom}[1]{\expandafter\@slowromancap\romannumeral #1@}
\makeatother

\begin{document}
	\nocite{*}
	
	\renewcommand{\inserttotalframenumber}{\pageref{lastslide}}
	
	{\setbeamertemplate{footline}{} 
		\begin{frame}
			\maketitle
		\end{frame}
	}
	\addtocounter{framenumber}{-1} 
	
	\AtBeginSection[]{
		{\setbeamertemplate{footline}{}
			\begin{frame}
				\vfill
				\centering
				\begin{beamercolorbox}[sep=5pt,shadow=true,rounded=true]{subtitle}
					\usebeamerfont{title}\insertsectionhead\par%
				\end{beamercolorbox}
				%\tableofcontents[sectionstyle=hide,subsectionstyle=show]
				
				%subsectionstyle=⟨style for current subsection⟩/⟨style for other subsections in current section⟩/⟨style for subsections in other sections⟩
				\tableofcontents[sectionstyle=hide,subsectionstyle=show/show/hide]
				\vfill
			\end{frame}
		}
		\addtocounter{framenumber}{-1} 
	}
	
	\AtBeginSubsection[]{
		{\setbeamertemplate{footline}{}
			\begin{frame}
				\vfill
				\centering
				\begin{beamercolorbox}[sep=5pt,shadow=true,rounded=true]{subtitle}
					\usebeamerfont{title}\insertsectionhead\par%
				\end{beamercolorbox}
				\tableofcontents[sectionstyle=hide,subsectionstyle=show/shaded/hide]
				\vfill
			\end{frame}
		}
		\addtocounter{framenumber}{-1} 
	}
	
	\begin{frame}{Problem considered}
	\vspace{5pt}
    \begin{minipage}{0.7\linewidth}
    	\textbf{Poisson problem with Dirichlet conditions :} \\
    	Find $u : \Omega \rightarrow \mathbb{R}^d (d=1,2,3)$ such that
    	\vspace{-10pt}
    	\begin{equation*}
    		\left\{\begin{aligned}
    			&-\Delta u(X) = f(X) \quad \text{in } \Omega, \\
    			&u(X) = g(X) \quad \text{on } \partial \Omega
    		\end{aligned}\right. \label{edp}
    	\end{equation*}
    \end{minipage}
	\begin{minipage}{0.26\linewidth}
		\vspace{-20pt}
		\pgfimage[width=\linewidth]{images/content/geometries.png}
	\end{minipage}
	
	with $\Delta$ the Laplace operator, $\Omega$ a smooth bounded open set and $\Gamma$ its boundary. 
	
	For the following examples, we will consider $f(X)=1$ and $g(X)=0$.
	
	\vspace{5pt}
	
	\textbf{Standard PINNs :} We are looking for $\theta_u$ such that
	\begin{equation*}
		\theta_u = \argmin_{\theta} w_{r}\; J_{r}(\theta)+w_{bc}\; J_{bc}(\theta)
	\end{equation*}
	where $w_{r}$ and $w_{bc}$ are the respective weights associated with
	\begin{equation*}
		J_{r} = \int_\Omega (\Delta u+f)^2 \; \text{ and } \; J_{bc} = \int_{\partial\Omega} (u-g)^2.
	\end{equation*}	
	
	\footnotesize
	\textit{Remark :} In practice, we use a Monte-Carlo method to discretize the cost function by random process.
\end{frame}

\begin{frame}{Simple geometry}
	\textbf{Claim on PINNs :} \textcolor{orange}{No mesh, so easy to go on complex geometry !}
	
	\centering
	\pgfimage[width=0.8\linewidth]{images/content/simple_geom/diapo.jpg}
	
	\flushleft
	\textbf{In practice :} Not so easy ! We need to find \textcolor{orange}{how to sample in the geometry}.
\end{frame}

\begin{frame}{Complex geometry}
	\begin{tabular}{c|c}
		\textbf{1st approach :} \textcolor{orange}{Mapping} & \textbf{2nd approach :} \textcolor{orange}{LevelSet function} \\
		\hline
		\begin{minipage}{0.44\linewidth}
			\textbf{Idea :} \\
			\ding{217} $\Omega_0$ a simple domain (as circle) \\
			\ding{217} $\Omega$ a target domain \\
			\ding{217} A mapping from $\Omega_0$ to $\Omega$ :
			$$\Omega=\phi(\Omega_0)$$
			
			\centering
			\pgfimage[width=0.95\linewidth]{images/content/complex_geom/mapping.jpg}
		\end{minipage} & \begin{minipage}{0.52\linewidth}
			\vspace{4pt}
			\begin{center}
				\pgfimage[width=0.6\linewidth]{images/content/complex_geom/levelset.png}
			\end{center}
			\vspace{-6pt}
			\textbf{Advantages :} \\
			\ding{217} Sample is easy in this case. \\
			\ding{217} Allow to impose in hard the BC :
			\vspace{-10pt}
			\begin{equation*}
				u_\theta(X)=\phi(X)w_\theta(X)+g(X)
			\end{equation*}
			
			\textbf{Natural LevelSet :} \\
			Signed Distance Function (SDF)
			
			\vspace{5pt}
			\textbf{Problem :} SDF is a $\mathcal{C}^0$ function  \\
			$\Rightarrow$ its derivatives explodes \\
			$\Rightarrow$ we \textcolor{orange}{need a regular levelset}
		\end{minipage}
	\end{tabular}
\end{frame}

\begin{frame}[allowframebreaks]{Construct smooth SDF}
		\textbf{1st solution :} \textcolor{orange}{Approximation theory} \cite{sukumar_exact_2022}
		
		$\Delta\phi$ can be singular at the boundary. Sampling at $\epsilon$ to it solve the problem.
		
		\begin{tabular}{c|c}
			\textbf{Polygonal domain} \refappendix{frame:PolygonalDomain} & \textbf{Curved domain} \refappendix{frame:CurvedDomain} \\
			\hline
			\begin{minipage}{0.48\linewidth}
				\flushright
				\pgfimage[width=0.8\linewidth]{images/content/approximation/polygone1.png} \\
				\flushleft
				\pgfimage[width=0.8\linewidth]{images/content/approximation/polygone2.png}
			\end{minipage} & \begin{minipage}{0.48\linewidth}
				\textbf{Minus :} Use of a parametric curve $c(t)$. \\
				\centering
				\pgfimage[width=0.7\linewidth]{images/content/approximation/bean.png}
				
				\pgfimage[width=\linewidth]{images/content/approximation/bean_poisson.png}
			\end{minipage}
		\end{tabular}
		
		\newpage
		
		\textbf{2nd solution :} \textcolor{orange}{Learn the levelset.} \cite{clemot_neural_2023} \\
		\ding{217} How make that ? with a \textcolor{orange}{PINNs}.
	
		\begin{tcolorbox}[
			colback=other, % Couleur de fond de la boîte
			colframe=other, % Couleur du cadre de la boîte
			arc=2mm, % Rayon de l'arrondi des coins
			boxrule=0.5pt, % Épaisseur du cadre de la boîte
			breakable, enhanced jigsaw,
			width=\linewidth,
			opacityback=0.1
			]
			
			If we have a boundary domain $\Gamma$, the SDF is solution to the Eikonal equation:
			
			\begin{minipage}{\linewidth}
				\centering
				$\left\{\begin{aligned}
					&||\nabla\phi(X)||=1, \; X\in\mathcal{O} \\
					&\phi(X)=0, \; X\in\Gamma \\
					&\nabla\phi(X)=n, \; X\in\Gamma
				\end{aligned}\right.$
			\end{minipage}
		
			with $\mathcal{O}$ a box which contains $\Omega$ completely and $n$ the exterior normal to $\Gamma$.
		\end{tcolorbox}
		
		\textbf{Advantage :} \textcolor{orange}{No need for parametric curves.}
		
		\begin{minipage}{0.48\linewidth}
			\centering
			\pgfimage[width=0.6\linewidth]{images/content/learn_levelset/points_normals.png}
		\end{minipage} 
		\begin{minipage}{0.48\linewidth}
			\ding{217} set of boundary points \\
			\ding{217} exterior normals at $\Gamma$ \\
			(evaluated at these points)
		\end{minipage}
\end{frame}

\begin{frame}{Learn LevelSet \rom{1}}
	\begin{minipage}{0.78\linewidth}
		\vspace{5pt}
		\textbf{Objective of the paper :} \\
		Learn topological Skeleton (by learning SDF) \quad \refappendix{frame:NeuralSkeleton}
	\end{minipage} \begin{minipage}{0.18\linewidth}	
		\vspace{-30pt}
		\hspace{70pt}\pgfimage[width=0.8\linewidth]{images/content/learn_levelset/skeleton_sans_background.png}
	\end{minipage}
	
	\ding{217} Skeleton correspond exactly to the gradient singularity \\
	\ding{217} Adding the following term in the loss \\
	\begin{minipage}{0.38\linewidth}
		\centering
		$\int_\mathcal{O} ||\nabla||\nabla\phi||(p)||dp$
		\flushright
		(Total Variation Regularization)
	\end{minipage} \quad \begin{minipage}{0.58\linewidth}	
		\vspace{-10pt}
		\flushright
		\pgfimage[width=0.85\linewidth]{images/content/learn_levelset/levelset.jpg}
	\end{minipage}

	\vspace{8pt}

%	\textbf{1st test :} Eikonal equation with TV Regularization \cite{clemot_neural_2023}
%	
%	\begin{minipage}{0.26\linewidth}
%		\pgfimage[width=\linewidth]{images/content/learn_levelset/tv_reg/levelset_loss.jpg}
%	\end{minipage} \quad \begin{minipage}{0.70\linewidth}	
%		\pgfimage[width=\linewidth]{images/content/learn_levelset/tv_reg/levelset.jpg}
%	\end{minipage}

	\begin{minipage}{0.65\linewidth}
		\textbf{1st test :} Eikonal equation with TV Regularization \cite{clemot_neural_2023}
		
		\vspace{25pt}
		\pgfimage[width=1.15\linewidth]{images/content/learn_levelset/tv_reg/levelset.jpg}
	\end{minipage} \begin{minipage}{0.32\linewidth}	
	\vspace{-55pt}
		\pgfimage[width=0.8\linewidth]{images/content/learn_levelset/tv_reg/levelset_loss.jpg}
	\end{minipage}
\end{frame}

\begin{frame}{Learn LevelSet \rom{1}}	
	\begin{minipage}{0.26\linewidth}
		\textbf{Sampling :}
		\begin{center}
			\pgfimage[width=0.8\linewidth]{images/content/learn_levelset/tv_reg/sampling_with_v.jpg}
		\end{center}
	\end{minipage} \quad \begin{minipage}{0.70\linewidth}
		\centering
		\pgfimage[width=0.8\linewidth]{images/content/learn_levelset/tv_reg/classical_pinns.png}
		
		\footnotesize\flushleft\vspace{-10pt}
		\qquad\qquad \textbf{Minus :} Costly boundary points generation.
	\end{minipage}
	
	\vspace{10pt}
	
	\textbf{PINNs - Impose BC in hard :} Looking for $u_\theta=\phi w_\theta$.
	\begin{center}
		\pgfimage[width=0.7\linewidth]{images/content/learn_levelset/tv_reg/levelset_derivatives.png}
	\end{center}
	
	\vspace{-10pt}
	Levelset derivatives explode.
\end{frame}

\begin{frame}{Learn LevelSet \rom{2}}
	\textbf{2nd test :} We replace the TV term by a penalization on the laplacian of the levelset
%	\begin{equation*}
%		J_{reg}=\int_{\mathcal{O}} |\Delta \phi|^2
%	\end{equation*}

	\begin{center}
		\pgfimage[width=0.6\linewidth]{images/content/learn_levelset/lap_reg/levelset_reg.jpg}
	\end{center}

	\begin{minipage}{0.28\linewidth}
		\centering
		\textbf{Sampling :}
		
		\pgfimage[width=0.7\linewidth]{images/content/learn_levelset/lap_reg/sampling_with_v.png}
	\end{minipage} \quad \begin{minipage}{0.68\linewidth}
		\centering
		\textbf{Dirichlet error on the boundary :} Looking for $u_\theta=\phi w_\theta$.
		
		\pgfimage[width=0.5\linewidth]{images/content/learn_levelset/lap_reg/dirichlet.png}
	\end{minipage}


\end{frame}

\begin{frame}{Learn LevelSet \rom{2}}
	\textbf{Derivatives :}

	\begin{center}
		\pgfimage[width=0.6\linewidth]{images/content/learn_levelset/lap_reg/levelset_derivatives.png}
	\end{center}
	
	 $\Rightarrow$ \textcolor{orange}{We can impose in hard boundary conditions}
	
	\textbf{PINNs - Impose BC in hard :}
	\begin{center}
		\pgfimage[width=0.7\linewidth]{images/content/learn_levelset/lap_reg/poisson_with_v.png}
	\end{center}
\end{frame}
	
	\begin{frame}[label={lastslide}]{Conclusion}
		\textbf{2 main questions :}
		\begin{itemize}[\ding{217}]
			\item How to sample in complex domains?
			\begin{itemize}
				\item Using mapping
				\item Using Levelset (Approximation theory/Learning)
			\end{itemize}
			\item How can we obtain a levelset that usable for imposing boundary conditions in hard ? \\
			By learning the Eikonal equation with penalisation of the levelset Laplacian
		\end{itemize}
		
		\textbf{To go further :} We can combine the option. \\
		(Mapping for the big domain. Level set for the hole.)
		\begin{minipage}{0.48\linewidth}
			\centering
			\pgfimage[width=0.9\linewidth]{images/oreille_loss.png}
		\end{minipage} \quad \begin{minipage}{0.48\linewidth}	
			\centering
			\pgfimage[width=0.9\linewidth]{images/oreille.png}
		\end{minipage}
		
	\end{frame}
	
	{\setbeamertemplate{footline}{} 
		\begin{frame}
			\vfill
			\centering
			\LARGE Thank you !
			\vfill
		\end{frame}
	
		\begin{frame}{Bibliography}
			\small
			% \vspace{30pt}
			% \setstretch{0.2}
			% \AtNextBibliography{\small}
			\printbibliography[heading=none]
		\end{frame}
	}
	\addtocounter{framenumber}{-2} 
	
	\appendix
	
	\begin{frame}{\appendixname~\insertframenumber~: Encoding - FEMs}\phantomsection\label{frame:encoding_fems}
	
	Pourquoi ?
\end{frame}

%\begin{frame}{Architecture of the FNO}
%    \begin{center}
%        \centering
%        \pgfimage[width=\linewidth]{images/more/FNO/FNO_schema.png}
%    \end{center}
%    \textbf{Input $X$} of shape (bs,ni,nj,nk) \qquad \qquad \textbf{Output $Y$} of shape (bs,ni,nj,1) \\
%    with bs the batch size, ni and nj the grid resolution and nk the number of channels.
%\end{frame}
%
%\begin{frame}{Description of the FNO architecture}
%    \begin{center}
%        \centering
%        \pgfimage[width=\linewidth]{images/more/FNO/FNO_schema_moitie1.png}
%    \end{center}
%    \begin{enumerate}[\ding{217}]
%        \item perform a $P$ transformation, to move to a space with more channels (to build a sufficiently rich representation of the data)
%        \item apply $L$ Fourier layers defined by
%        $$\mathcal{H}_\theta^l(\tilde{X})=\sigma\left(\mathcal{C}_\theta^l(\tilde{X})+\mathcal{B}_\theta^l(\tilde{X})\right),\; l=1,\dots,L$$
%        with $\tilde{X}$ the input of the current layer and
%        \begin{itemize}
%            \item $\sigma$ an activation function (ReLU or GELU)
%            \item $\mathcal{C}_\theta^l$ : convolution sublayer (convolution performed by Fast Fourier Transform)
%            \item $\mathcal{B}_\theta^l$ : "bias-sublayer"
%        \end{itemize}
%        \item return to the target dimension by performing a $Q$ transformation (in our case, the number of output channels is 1)
%    \end{enumerate}
%\end{frame}
%
%\begin{frame}{Fourier Layer Structure}
%    \setstretch{0.5}
%    \textbf{Convolution sublayer : } \quad $\mathcal{C}_\theta^l(X)=\mathcal{F}^{-1}(\mathcal{F}(X)\cdot\hat{W})$ \quad
%    \begin{minipage}{0.3\linewidth}
%        \vspace{-15pt}
%        \centering
%        \pgfimage[width=\linewidth]{images/more/FNO/FNO_schema_moitie2.png}
%    \end{minipage}
%    \begin{enumerate}[\ding{217}]
%        \item $\hat{W}$ : a trainable kernel
%        \item $\mathcal{F}$ : 2D Discrete Fourier Transform (DFT) defined by
%        \begin{equation*}
%            \mathcal{F}(X)_{ijk}=\frac{1}{ni}\frac{1}{nj}\sum_{i'=0}^{ni-1}\sum_{j'=0}^{nj-1}X_{i'j'k}e^{-2\sqrt{-1}\pi\left(\frac{ii'}{ni}+\frac{jj'}{nj}\right)}
%        \end{equation*}
%        $\mathcal{F}^{-1}$ : its inverse.
%        \item $(Y\cdot\hat{W})_{ijk}=\sum_{k'}Y_{ijk'}\hat{W}_{ijk'} \quad \Rightarrow \quad$ applied channel by channel
%    \end{enumerate} \; \\
%    \textbf{Bias-sublayer :} \quad  $\mathcal{B}_\theta^l(X)_{ijk}=\sum_{k'}X_{ijk}W_{k'k}+B_k$ \quad
%    \begin{minipage}{0.3\linewidth}
%        \vspace{-10pt}
%        \pgfimage[width=0.3\linewidth]{images/more/FNO/FNO_schema_moitie2_bis.png}
%    \end{minipage}
%    \begin{enumerate}[\ding{217}]
%        \item 2D convolution with a kernel of size 1
%        \item allowing channels to be mixed via a kernel without allowing interaction between pixels.
%    \end{enumerate}
%\end{frame}
%
%\begin{frame}{Dual method -  Poisson Problem}
%    \setstretch{0.5}		
%    \textbf{Problem :} Find $u$ on $\Omega_h$ and $p$ on $\Omega_h^\Gamma$ such that
%    \begin{align*}
%        \int_{\Omega_h}\nabla u\nabla v&-\int_{\partial\Omega_h}\frac{\partial u}{\partial n} v + \frac{\gamma}{h^2} \sum_{T\in\mathcal{T}_h^\Gamma}\int_T \left(u-\frac{1}{h}\phi p\right)\left(v-\frac{1}{h}\phi q\right) \\
%        &+ G_h(u,v) = \int_{\Omega_h}fv + G_h^{rhs}(v), \; \forall v \; \text{on } \Omega_h, \; q \; \text{on } \Omega_h^\Gamma
%    \end{align*}
%    with $\gamma$ an other positive stabilization parameter and $G_h$ and $G_h^{rhs}$ the stabilization terms defined previously.
%    
%    For the non homogeneous case, we replace
%    $$\int_T \left(u-\frac{1}{h}\phi p\right)\left(v-\frac{1}{h}\phi q\right) \quad \rightarrow \quad \int_T\left(u-\frac{1}{h}\phi p-g\right)\left(v-\frac{1}{h}\phi q\right)$$ 
%    by assuming $g$ is defined on $\Omega_h^\Gamma$
%\end{frame}
	
\end{document}
