\begin{frame}{\appendixname~\theappendixframenumber~: Polygonal domain \cite{sukumar_exact_2022}}\labelappendixframe{frame:PolygonalDomain}
	\begin{minipage}{0.68\linewidth}
		\begin{itemize}[\ding{217}]
			\item $X_i$, $i=1,\dots,n$ - coordinates of the polygon
			\item $\alpha_i$ - angle between $X_i$ and $X_{i+1}$
			\item $r_i=||X_i-X||$ - Euclidean distance between $X_i$ and $X$ \\
			\item $R_i=X_i-X$
		\end{itemize}
	\end{minipage}
	\begin{minipage}{0.28\linewidth}
		\centering
		\pgfimage[width=\linewidth]{images/more/polygon.png}
	\end{minipage}
	\vspace{-10pt}
	We define the SDF as 
	\begin{equation*}
		\phi(X)=\frac{2}{W(X)}
	\end{equation*}
	with
	\begin{equation*}
		W(X)=\sum_{i=1}^{n}\left(\frac{1}{r_i}+\frac{1}{r_{i+1}}\right)t_i \quad (r_{n+1}:=r_1)
	\end{equation*}
	and 
	\begin{equation*}
		t_i:=\tan\left(\frac{\alpha_i}{2}\right)=\frac{det(R_i,R_{i+1})}{r_ir_{i+1}+R_i\cdot R_{i+1}}
	\end{equation*}

	\footnotesize
	\textit{Remark :} The denominator vanishes when $\alpha_i=\pi$, (i.e. when $X$ lies on the boundary of the polygon), \\ but there $\phi_i(X)=0$.
\end{frame}
\addtocounter{appendixframenumber}{1}

\begin{frame}{\appendixname~\theappendixframenumber~: Curved domain \cite{sukumar_exact_2022}}\labelappendixframe{frame:CurvedDomain}
	Considering a nonconvex domain.
	\begin{itemize}[\ding{217}]
		\item $c(t)$ - parametrization of the curved boundary $\Gamma:[0,1]\rightarrow\mathbb{R}$
		\item $c'(t)$ - its tangent
		\item $c'^\perp(t)$ - rotating $c'(t)$ through 90° (clockwise)
	\end{itemize}
	\vspace{10pt}
	\begin{minipage}{0.68\linewidth}
		We define the SDF as 
		\begin{equation*}
			\phi(X)=\left(\frac{1}{W_p(X)}\right)^{1/p}
		\end{equation*}
		with
		\begin{equation*}
			W_p(X)=\int_0^1\frac{(c(t)-X)\cdot c'^\perp(t)}{||c(t)-X||^{2+p}}
		\end{equation*}
	\end{minipage}
	\begin{minipage}{0.28\linewidth}
		\centering
		\pgfimage[width=\linewidth]{images/more/curved.png}
	\end{minipage}
	\footnotesize
	
	\vspace{5pt}
	(Belyaev et al. \cite{belyaev_signed_2013} introduced $L_p$-distance fields ($p\ge 1$), which approximates the exact distance function.)
	
	\vspace{15pt}
	\textit{Remark :} For $X\in\Gamma$ (integral is singular), we set $\phi(X)=0$.
\end{frame}
\addtocounter{appendixframenumber}{1}

\begin{frame}{\appendixname~\theappendixframenumber~: Neural Skeleton}\labelappendixframe{frame:NeuralSkeleton}
	\textbf{Simple example :} Skeleton of the unit square.
	
	\begin{minipage}{\linewidth}
		\centering
		\pgfimage[width=0.7\linewidth]{images/more/skeleton.png}
	\end{minipage}
\end{frame}
\addtocounter{appendixframenumber}{1}