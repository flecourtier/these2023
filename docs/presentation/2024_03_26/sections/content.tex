\begin{frame}{Problem considered}
    \textbf{Poisson problem with homogeneous Dirichlet conditions :} \\
    Find $u : \Omega \rightarrow \mathbb{R}^d (d=1,2,3)$ such that
    \begin{equation*}
    	\left\{\begin{aligned}
    		&-\Delta u(X) = f \quad \text{in } \Omega, \\
    		&u(X) = g \quad \text{on } \partial \Omega
    	\end{aligned}\right. \label{edp}
    \end{equation*}
	with $\Delta$ the Laplace operator, $\Omega$ a smooth bounded open set and $\Gamma$ its boundary.
	
	\textbf{Standard PINNs :} We are looking for $\theta$ such that
	\begin{equation*}
		\theta_u = \argmin_{\theta} w_{r}\; J_{r}(\theta)+w_{bc}\; J_{bc}(\theta)
	\end{equation*}
	where $w_{r}$ and $w_{bc}$ are the respective weights associated with
	\begin{equation*}
		J_{r} = \int_\Omega (\Delta u+1)^2 \; \text{ and } \; J_{bc} = \int_{\partial\Omega} u^2.
	\end{equation*}	
	
	\footnotesize
	\textit{Remark :} In practice, we use a Monte-Carlo method to discretize the cost function by random process.
\end{frame}

\begin{frame}{Simple geometry}
	\textbf{Claim on PINNs :} \textcolor{orange}{No mesh, so easy to go on complex geometry !}
	
	\centering
	\pgfimage[width=0.8\linewidth]{images/simple_geom/diapo.jpg}
\end{frame}

\begin{frame}{Complex geometry}
	\textbf{In practice :} Not so easy ! We need to find \textcolor{orange}{how to sample in the geometry}.
	
	\begin{center}
		\begin{tabular}{c|c}
			\textbf{1st approach :} \textcolor{orange}{Mapping} & \textbf{2nd approach :} \textcolor{orange}{LevelSet function} \\
			\hline
			\begin{minipage}{0.48\linewidth}
				\textbf{Idea :} \\
				\ding{217} $\Omega_0$ a simple domain (as circle) \\
				\ding{217} $\Omega$ a target domain \\
				\ding{217} A mapping from $\Omega_0$ to $\Omega$
				$$\Omega=\phi(\Omega_0)$$
				
				\centering
				\pgfimage[width=0.8\linewidth]{images/complex_geom/mapping.jpg}
			\end{minipage} & \begin{minipage}{0.48\linewidth}
				\centering
				\pgfimage[width=0.6\linewidth]{images/complex_geom/levelset.png}
				
	%			\flushleft	
				\textbf{Advantages :} \\
				\ding{217} Sample is easy in this case. \\
				\ding{217} Allow to impose in hard the BC :
				\begin{equation*}
					u_\theta(X)=\phi(X)w_\theta(X)+g(X)
				\end{equation*}
				
				\textbf{Natural LevelSet :} \\
				Signed Distance Function (SDF)
			\end{minipage}
		\end{tabular}
		
	\end{center}
\end{frame}

\begin{frame}{LevelSet Approach}
	\textbf{Problem :} \\
	SDF is a $\mathcal{C}^0$ function $\Rightarrow$ its derivatives explodes $\Rightarrow$ We \textcolor{orange}{need a regular levelset}
	
	\textbf{How construct smooth SDF ?}
	\begin{itemize}[\ding{217}]
		\item \textbf{1st solution :} \textcolor{orange}{Approximation theory} \hl{ADD REFERENCE !!}
		
		$\Delta u$ can be singular at the boundary. Sampling at $\epsilon$ to it solve the problem.
		
		\item \textbf{2nd solution :} \textcolor{orange}{Learn the levelset.} How make that ? with a \textcolor{orange}{PINNs}
	\end{itemize}
\end{frame}

\begin{frame}{Approximation theory}
	TO COMPLETE !
\end{frame}

\begin{frame}{Learn LevelSet \rom{1}}
	TO COMPLETE !
\end{frame}

\begin{frame}{Learn LevelSet \rom{1}}
	TO COMPLETE !
\end{frame}

\begin{frame}{Learn LevelSet \rom{2}}
	TO COMPLETE !
\end{frame}

\begin{frame}{Learn LevelSet \rom{2}}
	TO COMPLETE !
\end{frame}