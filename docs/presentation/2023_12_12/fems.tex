\subsection{Standard FEM method}

\begin{frame}{Presentation of standard FEM method}	    
    \textbf{Variational Problem :} $\quad\text{Find } u\in V \; | \; a(u,v)=l(v), \;\forall v\in V$
    
    with $V$ - Hilbert space, $a$ - bilinear form, $l$ - linear form.

    \vspace{10pt}

    \begin{minipage}[t]{0.76\linewidth}
        \textbf{Approach Problem :} $\quad \text{Find } u_h\in V_h \; | \; a(u_h,v_h)=l(v_h), \;\forall v_h\in V_h$
    
        with $\textbullet$ $u_h\in V_h$ an approximate solution of $u$, 
        
        $\textbullet V_h\subset V, \; dim V_h=N_h<\infty, \; (\forall h>0)$ 
        
       $\Rightarrow$ Construct a piecewise continuous functions space
       \vspace{-5pt}
       \begin{equation*}
           V_h:=P_{C,h}^k=\{v_h\in C^0(\bar{\Omega}), \forall K\in\mathcal{T}_h, {v_h}_{|K}\in\mathbb{P}_k\}
       \end{equation*}

       where $\mathbb{P}_k$ is the vector space of polynomials of total degree $\le k$.
    \end{minipage} \hfill \begin{minipage}[t][][b]{0.2\linewidth}
        \vspace{-5pt}
        \centering
        \pgfimage[width=0.9\linewidth]{images/fems/FEM_triangle_mesh.png}
        
        \footnotesize
        $\mathcal{T}_h = \left\{K_1,\dots,K_{N_e}\right\}$

        \tiny
        ($N_e$ : number of elements)
    \end{minipage}

    

    \vspace{10pt}

    Finding an approximation of the PDE solution $\Rightarrow$ solving the following linear system:
    \begin{equation*}
        AU=b
    \end{equation*}
    with
    \begin{equation*}
        A=(a(\varphi_i,\varphi_j))_{1\le i,j\le N_h}, \quad U=(u_i)_{1\le i\le N_h} \quad \text{and} \quad b=(l(\varphi_j))_{1\le j\le N_h}
    \end{equation*}
    where $(\varphi_1,\dots,\varphi_{N_h})$ is a basis of $V_h$.
\end{frame}
	
% \begin{frame}{Presentation of standard FEM method}	
%     \fcolorbox{white}{yellow}{Réduire à une seule diapo rapide !}
    
%     \textbf{Variational Problem :} 
%     \begin{equation*}
%         \text{Find } u\in V \text{ such that } a(u,v)=l(v), \;\forall v\in V
%     \end{equation*}
%     where $V$ is a Hilbert space, $a$ is a bilinear form and $l$ is a linear form.
    
%     \textbf{Approach Problem :} 
%     \begin{equation*}
%         \text{Find } u_h\in V_h \text{ such that } a(u_h,v_h)=l(v_h), \;\forall v_h\in V
%     \end{equation*}
%     with $u_h$ an approximate solution in $V_h$, a finite-dimensional space dependent on $h$ such that $\quad V_h\subset V, \; dimV_h = N_h<\infty, \; (\forall h>0)$ 
    
%     As $u_h=\sum_{i=1}^{N_h}u_i\varphi_i$ with $(\varphi_1,\dots,\varphi_{N_h})$ a basis of $V_h$, finding an approximation of the PDE solution implies solving the following linear system:
%     \begin{equation*}
%         AU=b
%     \end{equation*}
%     with
%     \begin{equation*}
%         A=(a(\varphi_i,\varphi_j))_{1\le i,j\le N_h}, \quad U=(u_i)_{1\le i\le N_h} \quad \text{and} \quad b=(l(\varphi_j))_{1\le j\le N_h}
%     \end{equation*}
% \end{frame}

% \begin{frame}{In practice}
%     \begin{enumerate}[\ding{217}]
%         \item \begin{minipage}[t]{0.68\linewidth}
%             Construct a mesh of our $\Omega$ geometry with a family of elements (in 2D: triangle, rectangle; in 3D: tetrahedron, parallelepiped, prism) defined by
%             $$\mathcal{T}_h = \left\{K_1,\dots,K_{N_e}\right\}$$
%             where $N_e$ is the number of elements. \\
%         \end{minipage} \begin{minipage}[t][][b]{0.28\linewidth}
%             \centering
%             \qquad \pgfimage[width=0.8\linewidth]{images/fems/FEM_triangle_mesh.png}
%         \end{minipage}
%         (Importance of geometric quality)
%         \item Construct a space of piece-wise affine continuous functions, defined by
%         \begin{equation*}
%             V_h:=P_{C,h}^k=\{v_h\in C^0(\bar{\Omega}), \forall K\in\mathcal{T}_h, {v_h}_{|K}\in\mathbb{P}_k\}
%         \end{equation*}
%         where $\mathbb{P}_k$ is the vector space of polynomials of total degree less than or equal to $k$.
%     \end{enumerate}
    
% \end{frame}

\subsection{$\phi$-FEM method}

\begin{frame}{Problem}
    Let $u=\phi w+g$ such that
    $$\left\{\begin{aligned}
        -\Delta u &= f, \; \text{in } \Omega, \\
        u&=g, \; \text{on } \Gamma, \\
    \end{aligned}\right.$$
    where $\phi$ is the level-set function and $\Omega$ and $\Gamma$ are given by :
    \begin{center}
        \pgfimage[width=0.5\linewidth]{images/fems/PhiFEM_level_set.png}
    \end{center}
    The level-set function $\phi$ is supposed to be known on $\mathbb{R}^d$ and sufficiently smooth. \\
    For instance, the signed distance to $\Gamma$ is a good candidate.

    \vspace{5pt}

    \footnotesize
    \textit{Remark :} Thanks to $\phi$ and $g$, the conditions on the boundary are respected.
\end{frame}

\begin{frame}{Fictitious domain}
    \setstretch{0.5}

    \vspace{10pt}
    
    \begin{center}
        \begin{minipage}{0.43\linewidth}
            \centering
            \pgfimage[width=\linewidth]{images/fems/PhiFEM_domain.png}
        \end{minipage} \hfill
        \begin{minipage}{0.1\linewidth}
            \centering
            \pgfimage[width=\linewidth]{images/fems/PhiFEM_fleche.png} 
        \end{minipage} \hfill
        \begin{minipage}{0.43\linewidth}
            \centering
            \pgfimage[width=\linewidth]{images/fems/PhiFEM_domain_considered.png}
        \end{minipage}
    \end{center}

    \begin{enumerate}[\ding{217}]
        \item $\phi_h$ : approximation of $\phi$ \\ 
        \item $\Gamma_h=\{\phi_h=0\}$ : approximate boundary of $\Gamma$
        \item $\Omega_h$ : computational mesh
        \item $\partial\Omega_h$ : boundary of $\Omega_h$ ($\partial\Omega_h \ne \Gamma_h$)
    \end{enumerate}	
    
    % \begin{minipage}{0.6\linewidth}
    %     \begin{enumerate}[\ding{217}]
    %         \item $\mathcal{O}$ : fictitious domain such that $\Omega\subset\mathcal{O}$
    %         \item $\mathcal{T}_h^\mathcal{O}$ : simple quasi-uniform mesh on $\mathcal{O}$
    %         \item $\phi_h=I_{h,\mathcal{O}}^{(l)}(\phi)\in V_{h,\mathcal{O}}^{(l)}$ : approximation of $\phi$ \\ 
    %         with $I_{h,\mathcal{O}}^{(l)}$ the standard Lagrange interpolation operator on
    %         $$V_{h,\mathcal{O}}^{(l)}=\left\{v_h\in H^1(\mathcal{O}):v_{h|_T}\in\mathbb{P}_l(T) \;  \forall T\in\mathcal{T}_h^\mathcal{O}\right\}$$
    %         \item $\Gamma_h=\{\phi_h=0\}$ : approximate boundary of $\Gamma$
    %         \item $\mathcal{T}_h$ : sub-mesh of $\mathcal{T}_h^\mathcal{O}$ defined by
    %         $$\mathcal{T}_h=\left\{T\in \mathcal{T}_h^\mathcal{O}:T\cap\{\phi_h<0\}\ne\emptyset\right\}$$
    %         \item $\Omega_h$ : domain covered by the $\mathcal{T}_h$ mesh defined by
    %         $$\Omega_h=\left(\cup_{T\in\mathcal{T}_h}T\right)^O$$
    %         ($\partial\Omega_h$ its boundary)
    %     \end{enumerate}			
    % \end{minipage}
    
    \footnotesize
    \; \\
    \textit{Remark :} $n_{vert}$ will denote the number of vertices in each direction for $\mathcal{O}$
\end{frame}

\begin{frame}{Facets and Cells sets}

    \vspace{15pt}

    \begin{center}
        \begin{minipage}{0.48\linewidth}
            \centering
            \pgfimage[width=\linewidth]{images/fems/PhiFEM_boundary_cells.png}
        \end{minipage} \hfill
        \begin{minipage}{0.48\linewidth}
            \centering
            \pgfimage[width=\linewidth]{images/fems/PhiFEM_boundary_edges.png}
        \end{minipage}
    \end{center}

    \begin{enumerate}[\ding{217}]
        \item $\mathcal{T}_h^\Gamma$ : mesh elements cut by $\Gamma_h$
        \item $\mathcal{F}_h^\Gamma$ : collects the interior facets of $\mathcal{T}_h^\Gamma$ \\
        (either cut by $\Gamma_h$ or belonging to a cut mesh element)
    \end{enumerate}

    % \begin{minipage}{0.6\linewidth}
    %     \begin{enumerate}[\ding{217}]
    %         \item $\mathcal{T}_h^\Gamma\subset \mathcal{T}_h$ : contains the mesh elements cut by $\Gamma_h$, i.e. 
    %         \begin{equation*}
    %             \mathcal{T}_h^\Gamma=\left\{T\in\mathcal{T}_h:T\cap\Gamma_h\ne\emptyset\right\},
    %         \end{equation*}
    %         \item $\Omega_h^\Gamma$ : domain covered by the $\mathcal{T}_h^\Gamma$ mesh, i.e.
    %         \begin{equation*}
    %             \Omega_h^\Gamma=\left(\cup_{T\in\mathcal{T}_h^\Gamma}T\right)^O
    %         \end{equation*}
    %         \item $\mathcal{F}_h^\Gamma$ : collects the interior facets of $\mathcal{T}_h$ either cut by $\Gamma_h$ or belonging to a cut mesh element, i.e.
    %         \begin{align*}
    %             \mathcal{F}_h^\Gamma=\left\{E\;(\text{an internal facet of } \mathcal{T}_h) \text{ such that }\right. \\
    %             \left. \exists T\in \mathcal{T}_h:T\cap\Gamma_h\ne\emptyset \text{ and } E\in\partial T\right\}
    %         \end{align*}
    %     \end{enumerate}
    % \end{minipage}
\end{frame}

\begin{frame}{$\phi$-FEM Method - Poisson problem}
    \textbf{Approach Problem :} Find $w_h\in V_h^{(k)}$ such that 
    $$a_h(w_h,v_h) = l_h(v_h) \quad \forall v_h \in V_h^{(k)}$$
    where
    $$a_h(w,v)=\int_{\Omega_h} \nabla (\phi_h w) \cdot \nabla (\phi_h v) - \int_{\partial\Omega_h} \frac{\partial}{\partial n}(\phi_h w)\phi_h v+G_h(w,v),$$
    $$l_h(v)=\int_{\Omega_h} f \phi_h v + G_h^{rhs}(v) \qquad \qquad \color{white}\text{Stabilization terms}$$
    and 
    $$V_h^{(k)}=\left\{v_h\in H^1(\Omega_h):v_{h|_T}\in\mathbb{P}_k(T), \; \forall T\in\mathcal{T}_h\right\}.$$
    For the non homogeneous case, we replace
    $$u=\phi w \quad \rightarrow \quad u=\phi w+g$$ 
    by supposing that $g$ is currently given over the entire $\Omega_h$.
\end{frame}

\begin{frame}[noframenumbering]{$\phi$-FEM Method - Poisson problem}
    \textbf{Approach Problem :} Find $w_h\in V_h^{(k)}$ such that 
    $$a_h(w_h,v_h) = l_h(v_h) \quad \forall v_h \in V_h^{(k)}$$
    where
    $$a_h(w,v)=\int_{\Omega_h} \nabla (\phi_h w) \cdot \nabla (\phi_h v) - \int_{\partial\Omega_h} \frac{\partial}{\partial n}(\phi_h w)\phi_h v+\fcolorbox{blue}{white}{$G_h(w,v)$},$$
    $$l_h(v)=\int_{\Omega_h} f \phi_h v + \fcolorbox{blue}{white}{$G_h^{rhs}(v)$} \qquad \qquad \color{blue}\text{Stabilization terms}$$
    and 
    $$V_h^{(k)}=\left\{v_h\in H^1(\Omega_h):v_{h|_T}\in\mathbb{P}_k(T), \; \forall T\in\mathcal{T}_h\right\}.$$
    For the non homogeneous case, we replace
    $$u=\phi w \quad \rightarrow \quad u=\phi w+g$$ 
    by supposing that $g$ is currently given over the entire $\Omega_h$.
\end{frame}

\begin{frame}{Stabilization terms}
    \begin{center}
        \centering
        \pgfimage[width=\linewidth]{images/fems/PhiFEM_stab_terms.png}
    \end{center}
    \small
    \underline{1st term :} ensure continuity of the solution by penalizing gradient jumps. \\
    $\rightarrow$ Ghost penalty [Burman, 2010] \\
    \underline{2nd term :} require the solution to verify the strong form on $\Omega_h^\Gamma$. \\
    \normalsize
    \textbf{Purpose :} 
    \begin{enumerate}[\ding{217}]
        \item reduce the errors created by the "fictitious" boundary 
        \item ensure the correct condition number of the finite element matrix
        \item restore the coercivity of the bilinear scheme
    \end{enumerate}
\end{frame}
        
