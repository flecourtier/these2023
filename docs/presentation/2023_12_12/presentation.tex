%% Requires compilation with XeLaTeX or LuaLaTeX
\documentclass[compress,10pt,xcolor={table,dvipsnames},t]{beamer}
%\documentclass[compress]{beamer}
\usetheme{diapo}
\usepackage{amsmath}
\usepackage{amssymb}
\usepackage{xcolor}
\usepackage[bottom]{footmisc}
\usepackage{multirow}
\usepackage{setspace}
\usepackage{caption}
\usepackage{array,multirow,makecell}
% \usepackage[table]{xcolor}
\usepackage{pifont}
\usepackage{hyperref}
% \usepackage[utf8]{inputenc}
\setcellgapes{1pt}
\setlength{\parindent}{0pt}
\makegapedcells
\newcolumntype{R}[1]{>{\raggedleft\arraybackslash }b{#1}}
\newcolumntype{L}[1]{>{\raggedright\arraybackslash }b{#1}}
\newcolumntype{C}[1]{>{\centering\arraybackslash }b{#1}}
\usepackage{paralist}
\usepackage{appendixnumberbeamer}

\usepackage[backend=biber,style=numeric,sorting=nyt,doi=false,url=false]{biblatex}
\renewcommand*{\bibfont}{\scriptsize}


% Supprimer "In:" pour les articles
\renewbibmacro{in:}{}

% Supprimer les champs d'eprints
\AtEveryBibitem{\clearfield{arxiv}}
\AtEveryBibitem{\clearfield{eprint}}
\AtEveryBibitem{\clearfield{note}}
\AtEveryBibitem{\clearfield{eprintclass}}
\AtEveryBibitem{\clearfield{eprinttype}}

% Supprimer l'affichage de l'archive arXiv
% \AtEveryBibitem{\iffieldundef{eprinttype}{}{\clearfield{eprint}\clearfield{eprintclass}\clearfield{eprinttype}}}

% Supprimer les URL
\ExecuteBibliographyOptions{url=false}

% Charger votre fichier de bibliographie
\addbibresource{biblio.bib}

\useoutertheme[subsection=false]{miniframes}
\usepackage{etoolbox}
\makeatletter
\patchcmd{\slideentry}{\advance\beamer@xpos by1\relax}{}{}{}
\def\beamer@subsectionentry#1#2#3#4#5{\advance\beamer@xpos by1\relax}%
\makeatother

\setbeamercolor*{mini frame}{fg=bulles,bg=bulles}

\hypersetup{
	colorlinks=true,
	urlcolor=blue,
	citecolor=blue,
	linkcolor=title,
}

\title[PhiFEM]{Development of hybrid finite element/neural network methods to help create digital surgical twins}
\subtitle{Team meeting presentation}
\author[name]{Author : LECOURTIER Frédérique \\ Supervisors : DUPREZ Michel, FRANCK Emmanuel, LLERAS Vanessa}
\date{December 12, 2023}

\allowbreak

% to make animation
% \usepackage{animate}
\usepackage{tikz}
% \usetikzlibrary{calc,decorations,arrows,positioning,matrix}
 
% \pgfdeclaredecoration{ignore}{final}{
% 	\state{final}{}
% }
 
% \pgfdeclaremetadecoration{start}{initial}{
%     \state{initial}[width={0pt},next state=middle]{
%     	\decoration{moveto}
%     }
%     \state{middle}[width={\pgfmetadecoratedpathlength*\pgfdecorationsegmentlength},next state=final]{
%     	\decoration{curveto}
%     }
%     \state{final}{\decoration{ignore}}
% }
 
% \tikzset{
% 	start segment/.style={decoration={start,raise=2mm},decorate, segment length=#1},
% }

% u_chapeau (chapeau en couleur)
\usepackage{accents}
\newcommand{\uchapeau}[1]{\accentset{\textcolor{red}{\wedge}}{#1}}

% box colorée dans équation
\usepackage[most]{tcolorbox}

\begin{document}
	\nocite{*}

    \renewcommand{\inserttotalframenumber}{\pageref{lastslide}}

    {\setbeamertemplate{footline}{} 
    \begin{frame}
        \maketitle
    \end{frame}
    }
    \addtocounter{framenumber}{-1} 
 
	% \begin{frame}[plain]
	% 	\maketitle
	% \end{frame}
	
	\AtBeginSection[]{
		{\setbeamertemplate{footline}{}
        \begin{frame}
			\vfill
			\centering
			\begin{beamercolorbox}[sep=5pt,shadow=true,rounded=true]{subtitle}
				\usebeamerfont{title}\insertsectionhead\par%
			\end{beamercolorbox}
			%\tableofcontents[sectionstyle=hide,subsectionstyle=show]
			
            %subsectionstyle=⟨style for current subsection⟩/⟨style for other subsections in current section⟩/⟨style for subsections in other sections⟩
            \tableofcontents[sectionstyle=hide,subsectionstyle=show/show/hide]
			\vfill
		\end{frame}
        }
        \addtocounter{framenumber}{-1} 
	}

	\AtBeginSubsection[]{
        {\setbeamertemplate{footline}{}
		\begin{frame}
			\vfill
			\centering
			\begin{beamercolorbox}[sep=5pt,shadow=true,rounded=true]{subtitle}
				\usebeamerfont{title}\insertsectionhead\par%
			\end{beamercolorbox}
			\tableofcontents[sectionstyle=hide,subsectionstyle=show/shaded/hide]
			\vfill
		\end{frame}
        }
        \addtocounter{framenumber}{-1} 
	}

    % \begin{frame}{Test anmiation}
    %     \begin{center}
    %         \begin{animateinline}[autoplay,loop,controls]{3} 
    %     		% \multiframe{11}{rPos=0+0.1}{ 
    %     		% 	\begin{tikzpicture}
    %     		% 		\draw[start segment=\rPos,black!70, line width=2.5] (0,0) -- (1,0) -- (1,1) -- (0,1) --cycle ;
    %      	% 		\end{tikzpicture} 
    %     		% } 
    %         \animate<1-2>
    %         \multiframe{2}{rX=1+1}{
    %         \begin{center}
    %           \includegraphics[width=0.2\textwidth]{images/image\rX.png}
    %         \end{center}
    %         }
    %     	\end{animateinline} 
    %     \end{center}
    % \end{frame}

	\section{Introduction}
    \begin{frame}{Scientific context}
	\begin{minipage}{0.78\linewidth}
		\textbf{Context :} Create real-time digital twins of an organ (e.g. liver).
	\end{minipage}
	\begin{minipage}{0.18\linewidth}
		\vspace{-20pt}
		\includegraphics[width=0.95\linewidth]{images/intro/liver.png}
	\end{minipage}
	
	\vspace{1pt}
	\textbf{Objective :} Develop an hybrid \fcolorbox{red}{white}{finite element} / \fcolorbox{orange}{white}{neural network} method.
	
	\vspace{1pt}
	\small
	\hspace{130pt} \begin{minipage}{0.14\linewidth}
		\textcolor{red}{accurate}
	\end{minipage} \hspace{8pt} \begin{minipage}{0.3\linewidth}
		\textcolor{orange}{quick + parameterized}
	\end{minipage}

	\normalsize
	\vspace{5pt}
	\textbf{Parametric linear elliptic PDE :}
	For one or several  $\bm{\mu}\in \mathcal{M}$, find $u: \Omega\to \mathbb{R}$ such that
	\begin{equation*}
		% \label{eq:ob_pde}
		\mathcal{L}\big(u;\bm{x},\bm{\mu}\big) = f(\bm{x},\bm{\mu}),
	\end{equation*}
	where $\mathcal{L}$ is the parametric differential operator defined  by
	\begin{equation*}
		\mathcal{L}(\cdot;\bm{x},\bm{\mu}) : u \mapsto R(\bm{x},\bm{\mu}) u + C(\bm{\mu}) \cdot \nabla u - \frac{1}{\text{Pe}} \nabla \cdot (D(\bm{x},\bm{\mu}) \nabla u),
	\end{equation*}
	and some Dirichlet, Neumann or Robin BC (which can also depend on $\bm{\mu}$).
	
	\footnotesize
	\begin{table}[ht!]
		\centering
		\begin{tabular}{c|c}
			$\Omega$ & Spatial domain \\
			$d$ & Spatial dimension \\
			$\bm{x}=(x_1,\dots,x_d)$ & Spatial coordinates \\
			\hline
			$\mathcal{M}$ & Parameter space \\
			$p$ & Number of parameters \\
			$\bm{\mu}=(\mu_1,\ldots,\mu_p)$ & Parameter vector \\
		\end{tabular} \hspace{10pt}
		\begin{tabular}{c|c}
			$f$ & Right-hand side \\
			$R$ & Reaction coefficient \\
			$C$ & Convection coefficient \\
			$D$ & Diffusion matrix \\
			Pe & Péclet number \\
		\end{tabular}
	\end{table}
\end{frame}

\begin{frame}{Pipeline of the Enriched FEM}
	\begin{figure}[!ht]
		\centering
		\includegraphics[width=0.7\linewidth]{images/intro/pipeline/offline_v2.pdf}

		\includegraphics[width=0.7\linewidth]{images/intro/pipeline/online_v2.pdf}
	\end{figure}

	\textbf{Correction :} Enriched continuous Lagrange finite element approximation spaces
	using the PINN prediction.
\end{frame}

\begin{frame}{Physics-Informed Neural Networks}
	\textbf{Standard PINNs :} Find the optimal weights $\theta^\star$ that satisfy
	\begin{equation}
		\label{eq:opt_pb}
		\theta^\star = \argmin_{\theta}	\big( \omega_r \; J_r(\theta) + \omega_b \; J_b(\theta) \big),
	\end{equation}
	with the residual loss function and the boundary loss function defined by
	\begin{equation*}
		J_r(\theta) =
		\int_{\mathcal{M}}\int_{\Omega}
		\big| \mathcal{L}\big(u_\theta(\bm{x},\bm{\mu});\bm{x},\bm{\mu}\big)-f(\bm{x},\bm{\mu}) \big|^2 d\bm{x} d\bm{\mu},
	\end{equation*}
	\begin{equation*}
		J_b(\theta) =
		\int_{\mathcal{M}}\int_{\partial \Omega} \big| u_\theta(\bm{x},\bm{\mu}) - g(\bm{x},\bm{\mu}) \big|^2 d\bm{x} d\bm{\mu},
	\end{equation*}
	where $u_\theta$ is a neural network, $g$ is the Dirichlet BC. In \eqref{eq:opt_pb}, the weights $\omega_r$ and $\omega_b$ (hyperparameters) are used to balance the different terms of the loss function.

	\vspace{5pt}
	\textbf{Monte-Carlo method :} Discretize the cost functions by random process.
\end{frame}

\begin{frame}[noframenumbering]{Physics-Informed Neural Networks}
	\textbf{\textcolor{red}{Improved} PINNs\footcite{LagLikFot1998,FraMicNav2024} :} Find the optimal weights $\theta^\star$ that satisfy
	\begin{equation}
		\label{eq:opt_pb_nobc}
		\theta^\star = \argmin_{\theta}	\big( \omega_r \; J_r(\theta) + \Ccancel[red]{\omega_b \; J_b(\theta)} \big),
	\end{equation}
	with $\omega_r=1$ and the residual loss function defined by
	\begin{equation*}
		J_r(\theta) =
		\int_{\mathcal{M}}\int_{\Omega}
		\big| \mathcal{L}\big(u_\theta(\bm{x},\bm{\mu});\bm{x},\bm{\mu}\big)-f(\bm{x},\bm{\mu}) \big|^2 d\bm{x} d\bm{\mu},
	\end{equation*}
	\begin{minipage}{0.7\linewidth}
		where $u_\theta$ is a neural network defined by
		\begin{equation*}
			\textcolor{red}{u_{\theta}(\bm{x},\bm{\mu}) = \varphi(\bm{x}) w_{\theta}(\bm{x},\bm{\mu}) + g(\bm{x},\bm{\mu}),}
		\end{equation*}
		with $\varphi$ a level-set function, $w_\theta$ a NN and $g$ the Dirichlet BC. 
	\end{minipage}
	\begin{minipage}{0.28\linewidth}
		\vspace{-15pt}
		\includegraphics[width=0.95\linewidth]{images/intro/levelset.png}
	\end{minipage}

	\vspace{5pt}
	\textbf{Monte-Carlo method :} Discretize the residual cost function by random process.
	\vspace{15pt}
\end{frame}


\begin{frame}{Finite Element Method}
	TODO
\end{frame}

	\section{Finite Element Methods}
	\subsection{Standard FEM method}

\begin{frame}{Presentation of standard FEM method}	    
    \textbf{Variational Problem :} $\quad\text{Find } u\in V \; | \; a(u,v)=l(v), \;\forall v\in V$
    
    with $V$ - Hilbert space, $a$ - bilinear form, $l$ - linear form.

    \vspace{10pt}

    \begin{minipage}[t]{0.76\linewidth}
        \textbf{Approach Problem :} $\quad \text{Find } u_h\in V_h \; | \; a(u_h,v_h)=l(v_h), \;\forall v_h\in V_h$
    
        with $\textbullet$ $u_h\in V_h$ an approximate solution of $u$, 
        
        $\textbullet V_h\subset V, \; dim V_h=N_h<\infty, \; (\forall h>0)$ 
        
       $\Rightarrow$ Construct a piecewise continuous functions space
       \vspace{-5pt}
       \begin{equation*}
           V_h:=P_{C,h}^k=\{v_h\in C^0(\bar{\Omega}), \forall K\in\mathcal{T}_h, {v_h}_{|K}\in\mathbb{P}_k\}
       \end{equation*}

       where $\mathbb{P}_k$ is the vector space of polynomials of total degree $\le k$.
    \end{minipage} \hfill \begin{minipage}[t][][b]{0.2\linewidth}
        \vspace{-5pt}
        \centering
        \pgfimage[width=0.9\linewidth]{images/fems/FEM_triangle_mesh.png}
        
        \footnotesize
        $\mathcal{T}_h = \left\{K_1,\dots,K_{N_e}\right\}$

        \tiny
        ($N_e$ : number of elements)
    \end{minipage}

    

    \vspace{10pt}

    Finding an approximation of the PDE solution $\Rightarrow$ solving the following linear system:
    \begin{equation*}
        AU=b
    \end{equation*}
    with
    \begin{equation*}
        A=(a(\varphi_i,\varphi_j))_{1\le i,j\le N_h}, \quad U=(u_i)_{1\le i\le N_h} \quad \text{and} \quad b=(l(\varphi_j))_{1\le j\le N_h}
    \end{equation*}
    where $(\varphi_1,\dots,\varphi_{N_h})$ is a basis of $V_h$.
\end{frame}
	
% \begin{frame}{Presentation of standard FEM method}	
%     \fcolorbox{white}{yellow}{Réduire à une seule diapo rapide !}
    
%     \textbf{Variational Problem :} 
%     \begin{equation*}
%         \text{Find } u\in V \text{ such that } a(u,v)=l(v), \;\forall v\in V
%     \end{equation*}
%     where $V$ is a Hilbert space, $a$ is a bilinear form and $l$ is a linear form.
    
%     \textbf{Approach Problem :} 
%     \begin{equation*}
%         \text{Find } u_h\in V_h \text{ such that } a(u_h,v_h)=l(v_h), \;\forall v_h\in V
%     \end{equation*}
%     with $u_h$ an approximate solution in $V_h$, a finite-dimensional space dependent on $h$ such that $\quad V_h\subset V, \; dimV_h = N_h<\infty, \; (\forall h>0)$ 
    
%     As $u_h=\sum_{i=1}^{N_h}u_i\varphi_i$ with $(\varphi_1,\dots,\varphi_{N_h})$ a basis of $V_h$, finding an approximation of the PDE solution implies solving the following linear system:
%     \begin{equation*}
%         AU=b
%     \end{equation*}
%     with
%     \begin{equation*}
%         A=(a(\varphi_i,\varphi_j))_{1\le i,j\le N_h}, \quad U=(u_i)_{1\le i\le N_h} \quad \text{and} \quad b=(l(\varphi_j))_{1\le j\le N_h}
%     \end{equation*}
% \end{frame}

% \begin{frame}{In practice}
%     \begin{enumerate}[\ding{217}]
%         \item \begin{minipage}[t]{0.68\linewidth}
%             Construct a mesh of our $\Omega$ geometry with a family of elements (in 2D: triangle, rectangle; in 3D: tetrahedron, parallelepiped, prism) defined by
%             $$\mathcal{T}_h = \left\{K_1,\dots,K_{N_e}\right\}$$
%             where $N_e$ is the number of elements. \\
%         \end{minipage} \begin{minipage}[t][][b]{0.28\linewidth}
%             \centering
%             \qquad \pgfimage[width=0.8\linewidth]{images/fems/FEM_triangle_mesh.png}
%         \end{minipage}
%         (Importance of geometric quality)
%         \item Construct a space of piece-wise affine continuous functions, defined by
%         \begin{equation*}
%             V_h:=P_{C,h}^k=\{v_h\in C^0(\bar{\Omega}), \forall K\in\mathcal{T}_h, {v_h}_{|K}\in\mathbb{P}_k\}
%         \end{equation*}
%         where $\mathbb{P}_k$ is the vector space of polynomials of total degree less than or equal to $k$.
%     \end{enumerate}
    
% \end{frame}

\subsection{$\phi$-FEM method}

\begin{frame}{Problem}
    Let $u=\phi w+g$ such that
    $$\left\{\begin{aligned}
        -\Delta u &= f, \; \text{in } \Omega, \\
        u&=g, \; \text{on } \Gamma, \\
    \end{aligned}\right.$$
    where $\phi$ is the level-set function and $\Omega$ and $\Gamma$ are given by :
    \begin{center}
        \pgfimage[width=0.5\linewidth]{images/fems/PhiFEM_level_set.png}
    \end{center}
    The level-set function $\phi$ is supposed to be known on $\mathbb{R}^d$ and sufficiently smooth. \\
    For instance, the signed distance to $\Gamma$ is a good candidate.

    \vspace{5pt}

    \footnotesize
    \textit{Remark :} Thanks to $\phi$ and $g$, the conditions on the boundary are respected.
\end{frame}

\begin{frame}{Fictitious domain}
    \setstretch{0.5}

    \vspace{10pt}
    
    \begin{center}
        \begin{minipage}{0.43\linewidth}
            \centering
            \pgfimage[width=\linewidth]{images/fems/PhiFEM_domain.png}
        \end{minipage} \hfill
        \begin{minipage}{0.1\linewidth}
            \centering
            \pgfimage[width=\linewidth]{images/fems/PhiFEM_fleche.png} 
        \end{minipage} \hfill
        \begin{minipage}{0.43\linewidth}
            \centering
            \pgfimage[width=\linewidth]{images/fems/PhiFEM_domain_considered.png}
        \end{minipage}
    \end{center}

    \begin{enumerate}[\ding{217}]
        \item $\phi_h$ : approximation of $\phi$ \\ 
        \item $\Gamma_h=\{\phi_h=0\}$ : approximate boundary of $\Gamma$
        \item $\Omega_h$ : computational mesh
        \item $\partial\Omega_h$ : boundary of $\Omega_h$ ($\partial\Omega_h \ne \Gamma_h$)
    \end{enumerate}	
    
    % \begin{minipage}{0.6\linewidth}
    %     \begin{enumerate}[\ding{217}]
    %         \item $\mathcal{O}$ : fictitious domain such that $\Omega\subset\mathcal{O}$
    %         \item $\mathcal{T}_h^\mathcal{O}$ : simple quasi-uniform mesh on $\mathcal{O}$
    %         \item $\phi_h=I_{h,\mathcal{O}}^{(l)}(\phi)\in V_{h,\mathcal{O}}^{(l)}$ : approximation of $\phi$ \\ 
    %         with $I_{h,\mathcal{O}}^{(l)}$ the standard Lagrange interpolation operator on
    %         $$V_{h,\mathcal{O}}^{(l)}=\left\{v_h\in H^1(\mathcal{O}):v_{h|_T}\in\mathbb{P}_l(T) \;  \forall T\in\mathcal{T}_h^\mathcal{O}\right\}$$
    %         \item $\Gamma_h=\{\phi_h=0\}$ : approximate boundary of $\Gamma$
    %         \item $\mathcal{T}_h$ : sub-mesh of $\mathcal{T}_h^\mathcal{O}$ defined by
    %         $$\mathcal{T}_h=\left\{T\in \mathcal{T}_h^\mathcal{O}:T\cap\{\phi_h<0\}\ne\emptyset\right\}$$
    %         \item $\Omega_h$ : domain covered by the $\mathcal{T}_h$ mesh defined by
    %         $$\Omega_h=\left(\cup_{T\in\mathcal{T}_h}T\right)^O$$
    %         ($\partial\Omega_h$ its boundary)
    %     \end{enumerate}			
    % \end{minipage}
    
    \footnotesize
    \; \\
    \textit{Remark :} $n_{vert}$ will denote the number of vertices in each direction for $\mathcal{O}$
\end{frame}

\begin{frame}{Facets and Cells sets}

    \vspace{15pt}

    \begin{center}
        \begin{minipage}{0.48\linewidth}
            \centering
            \pgfimage[width=\linewidth]{images/fems/PhiFEM_boundary_cells.png}
        \end{minipage} \hfill
        \begin{minipage}{0.48\linewidth}
            \centering
            \pgfimage[width=\linewidth]{images/fems/PhiFEM_boundary_edges.png}
        \end{minipage}
    \end{center}

    \begin{enumerate}[\ding{217}]
        \item $\mathcal{T}_h^\Gamma$ : mesh elements cut by $\Gamma_h$
        \item $\mathcal{F}_h^\Gamma$ : collects the interior facets of $\mathcal{T}_h^\Gamma$ \\
        (either cut by $\Gamma_h$ or belonging to a cut mesh element)
    \end{enumerate}

    % \begin{minipage}{0.6\linewidth}
    %     \begin{enumerate}[\ding{217}]
    %         \item $\mathcal{T}_h^\Gamma\subset \mathcal{T}_h$ : contains the mesh elements cut by $\Gamma_h$, i.e. 
    %         \begin{equation*}
    %             \mathcal{T}_h^\Gamma=\left\{T\in\mathcal{T}_h:T\cap\Gamma_h\ne\emptyset\right\},
    %         \end{equation*}
    %         \item $\Omega_h^\Gamma$ : domain covered by the $\mathcal{T}_h^\Gamma$ mesh, i.e.
    %         \begin{equation*}
    %             \Omega_h^\Gamma=\left(\cup_{T\in\mathcal{T}_h^\Gamma}T\right)^O
    %         \end{equation*}
    %         \item $\mathcal{F}_h^\Gamma$ : collects the interior facets of $\mathcal{T}_h$ either cut by $\Gamma_h$ or belonging to a cut mesh element, i.e.
    %         \begin{align*}
    %             \mathcal{F}_h^\Gamma=\left\{E\;(\text{an internal facet of } \mathcal{T}_h) \text{ such that }\right. \\
    %             \left. \exists T\in \mathcal{T}_h:T\cap\Gamma_h\ne\emptyset \text{ and } E\in\partial T\right\}
    %         \end{align*}
    %     \end{enumerate}
    % \end{minipage}
\end{frame}

\begin{frame}{$\phi$-FEM Method - Poisson problem}
    \textbf{Approach Problem :} Find $w_h\in V_h^{(k)}$ such that 
    $$a_h(w_h,v_h) = l_h(v_h) \quad \forall v_h \in V_h^{(k)}$$
    where
    $$a_h(w,v)=\int_{\Omega_h} \nabla (\phi_h w) \cdot \nabla (\phi_h v) - \int_{\partial\Omega_h} \frac{\partial}{\partial n}(\phi_h w)\phi_h v+G_h(w,v),$$
    $$l_h(v)=\int_{\Omega_h} f \phi_h v + G_h^{rhs}(v) \qquad \qquad \color{white}\text{Stabilization terms}$$
    and 
    $$V_h^{(k)}=\left\{v_h\in H^1(\Omega_h):v_{h|_T}\in\mathbb{P}_k(T), \; \forall T\in\mathcal{T}_h\right\}.$$
    For the non homogeneous case, we replace
    $$u=\phi w \quad \rightarrow \quad u=\phi w+g$$ 
    by supposing that $g$ is currently given over the entire $\Omega_h$.
\end{frame}

\begin{frame}[noframenumbering]{$\phi$-FEM Method - Poisson problem}
    \textbf{Approach Problem :} Find $w_h\in V_h^{(k)}$ such that 
    $$a_h(w_h,v_h) = l_h(v_h) \quad \forall v_h \in V_h^{(k)}$$
    where
    $$a_h(w,v)=\int_{\Omega_h} \nabla (\phi_h w) \cdot \nabla (\phi_h v) - \int_{\partial\Omega_h} \frac{\partial}{\partial n}(\phi_h w)\phi_h v+\fcolorbox{blue}{white}{$G_h(w,v)$},$$
    $$l_h(v)=\int_{\Omega_h} f \phi_h v + \fcolorbox{blue}{white}{$G_h^{rhs}(v)$} \qquad \qquad \color{blue}\text{Stabilization terms}$$
    and 
    $$V_h^{(k)}=\left\{v_h\in H^1(\Omega_h):v_{h|_T}\in\mathbb{P}_k(T), \; \forall T\in\mathcal{T}_h\right\}.$$
    For the non homogeneous case, we replace
    $$u=\phi w \quad \rightarrow \quad u=\phi w+g$$ 
    by supposing that $g$ is currently given over the entire $\Omega_h$.
\end{frame}

\begin{frame}{Stabilization terms}
    \begin{center}
        \centering
        \pgfimage[width=\linewidth]{images/fems/PhiFEM_stab_terms.png}
    \end{center}
    \small
    \underline{1st term :} ensure continuity of the solution by penalizing gradient jumps. \\
    $\rightarrow$ Ghost penalty [Burman, 2010] \\
    \underline{2nd term :} require the solution to verify the strong form on $\Omega_h^\Gamma$. \\
    \normalsize
    \textbf{Purpose :} 
    \begin{enumerate}[\ding{217}]
        \item reduce the errors created by the "fictitious" boundary 
        \item ensure the correct condition number of the finite element matrix
        \item restore the coercivity of the bilinear scheme
    \end{enumerate}
\end{frame}
        


    \section{Internship results}
	\subsection{Correction Methods}

% \begin{frame}{Correction Methods}
%     We are given $u_\theta$ the FNO prediction (for the problem under consideration).
    
%     \begin{minipage}{0.48\linewidth}
%         \textbf{By adding :}
    
%         We will consider
%         \begin{equation*}
%             \tilde{u}=u_\theta+\tilde{C}
%         \end{equation*}

%         We want $\tilde{C}: \Omega \rightarrow \mathbb{R}^d$ such that
%         \begin{equation}
%             \left\{\begin{aligned}
%                 -\Delta \tilde{C}&=\tilde{f}, \; &&\text{on } \Omega, \\
%                 \tilde{C}&=0, \; &&\text{in } \Gamma.
%             \end{aligned}\right. \tag{$\mathcal{C}_{+}$} %\label{corr_add}
%         \end{equation}
%         with $\tilde{f}=f+\Delta u_\theta$ and $\tilde{C}=\phi C$ for the $\phi$-FEM method.
        
%         \small
%         \textit{Remark :} In practice, it may be useful to integrate by parts the term containing $\Delta u_\theta$.
%     \end{minipage} \quad
%     \begin{minipage}{0.48\linewidth}
%         \textbf{By multiplying :}
        
%         \small
%         \textcolor{white}{Find $\hat{u} : \Omega \rightarrow \mathbb{R}^d$ such that
% 		\begin{equation}
% 			\left\{
% 			\begin{aligned}
% 				-\Delta \hat{u} = f, \; &&\text{in } \; \Omega, \\
% 				\hat{u}=g+m, \; &&\text{on } \; \Gamma,
% 			\end{aligned}
% 			\right. \tag{$\mathcal{P}^\mathcal{M}$} \label{pb_reh}
% 		\end{equation}
% 		with $\hat{u}=u+m$ ($m$ a constant).}

%         \normalsize
%         We will consider
%         \begin{equation*}
%             \tilde{u}=u_\theta C
%         \end{equation*}
%         \textcolor{white}{with $\hat{u_\theta}=u_\theta+m$.}
        
%         We want $C: \Omega \rightarrow \mathbb{R}^d$ such that
%         \begin{equation*}
%             \left\{\begin{aligned}
%                 &-\Delta (u_\theta C)=f, \; &&\text{on } \Omega, \\
%                 &C=1, \; &&\text{on } \; \Gamma.
%             \end{aligned}\right. \tag{$\mathcal{C}_\times$} %\label{corr_mult}
%         \end{equation*}
%     \end{minipage}
% \end{frame}

\begin{frame}{Correction Methods}
    We are given $u_\theta$ the FNO prediction (for the problem under consideration).
    
    \begin{minipage}{0.48\linewidth}
        \textbf{By adding :}
    
        We will consider
        \begin{equation*}
            \tilde{u}=u_\theta+\fcolorbox{blue}{white}{$\tilde{C}$}\textcolor{blue}{\approx 0}
        \end{equation*}

        We want $\tilde{C}: \Omega \rightarrow \mathbb{R}^d$ such that
        \begin{equation}
            \left\{\begin{aligned}
                -\Delta \tilde{C}&=\tilde{f}, \; &&\text{in } \Omega, \\
                \tilde{C}&=0, \; &&\text{on } \Gamma.
            \end{aligned}\right. \tag{$\mathcal{C}_{+}$} %\label{corr_add}
        \end{equation}
        with $\tilde{f}=f+\Delta u_\theta$ and $\tilde{C}=\phi C$ for the $\phi$-FEM method.
        
        \small
        \textit{Remark :} In practice, it may be useful to integrate by parts the term containing $\Delta u_\theta$.
    \end{minipage} \quad
    \begin{minipage}{0.48\linewidth}
        \textbf{By multiplying :}
        
        \small
        \textcolor{white}{Find $\hat{u} : \Omega \rightarrow \mathbb{R}^d$ such that
		\begin{equation}
			\left\{
			\begin{aligned}
				-\Delta \hat{u} = f, \; &&\text{in } \; \Omega, \\
				\hat{u}=g+m, \; &&\text{on } \; \Gamma,
			\end{aligned}
			\right. \tag{$\mathcal{P}^\mathcal{M}$} %\label{pb_reh}
		\end{equation}
		with $\hat{u}=u+m$ ($m$ a constant).}

        \normalsize
        We will consider
        \begin{equation*}
            \tilde{u}=u_\theta \fcolorbox{blue}{white}{$C$}\textcolor{blue}{\approx 1}
        \end{equation*}
        \textcolor{white}{with $\hat{u_\theta}=u_\theta+m$.}
        
        We want $C: \Omega \rightarrow \mathbb{R}^d$ such that
        \begin{equation*}
            \left\{\begin{aligned}
                &-\Delta (u_\theta C)=f, \; &&\text{on } \Omega, \\
                &C=1, \; &&\text{on } \; \Gamma.
            \end{aligned}\right. \tag{$\mathcal{C}_\times$} \label{corr_mult}
        \end{equation*}
    \end{minipage}
\end{frame}

\begin{frame}[noframenumbering]{Correction Methods}
    We are given $u_\theta$ the FNO prediction (for the problem under consideration).
    
    \begin{minipage}{0.48\linewidth}
        \textbf{By adding :}
    
        We will consider
        \begin{equation*}
            \tilde{u}=u_\theta+\tilde{C}
        \end{equation*}

        We want $\tilde{C}: \Omega \rightarrow \mathbb{R}^d$ such that
        \begin{equation}
            \left\{\begin{aligned}
                -\Delta \tilde{C}&=\tilde{f}, \; &&\text{in } \Omega, \\
                \tilde{C}&=0, \; &&\text{on } \Gamma.
            \end{aligned}\right. \tag{$\mathcal{C}_{+}$} \label{corr_add}
        \end{equation}
        with $\tilde{f}=f+\Delta u_\theta$ and $\tilde{C}=\phi C$ for the $\phi$-FEM method.
        
        \small
        \textit{Remark :} In practice, it may be useful to integrate by parts the term containing $\Delta u_\theta$.
    \end{minipage} \quad
    \begin{minipage}{0.48\linewidth}
        \textbf{By multiplying \textcolor{red}{- elevated problem} :}
        
        \small
        \textcolor{red}{Find $\hat{u} : \Omega \rightarrow \mathbb{R}^d$ such that
		\begin{equation}
			\left\{
			\begin{aligned}
				-\Delta \hat{u} = f, \; &&\text{in } \; \Omega, \\
				\hat{u}=g+m, \; &&\text{on } \; \Gamma,
			\end{aligned}
			\right. \tag{$\mathcal{P}^{\mathcolor{red}{\mathcal{M}}}$} \label{pb_reh}
		\end{equation}
		with $\hat{u}=u+m$ ($m$ a constant).}

        \normalsize
        We will consider
        \begin{equation*}
            \tilde{u}=\uchapeau{u_\theta} C
        \end{equation*}
        \textcolor{red}{with $\hat{u_\theta}=u_\theta+m$.}
        
        We want $C: \Omega \rightarrow \mathbb{R}^d$ such that
        \begin{equation*}
            \left\{\begin{aligned}
                &-\Delta (\uchapeau{u_\theta} C)=f, \; &&\text{in } \Omega, \\
                &C=1, \; &&\text{on } \; \Gamma.
            \end{aligned}\right. \tag{$\mathcal{C}_\times^{\mathcolor{red}{\mathcal{M}}}$} \label{corr_mult_reh}
        \end{equation*}
    \end{minipage}
\end{frame}

\subsection{Results - with FNO}

\begin{frame}{Explanation}
    \vspace{5pt}
    \begin{minipage}{0.48\linewidth}
        \textbf{Train a FNO :}
        
        \pgfimage[width=\linewidth]{images/internship/FNO/Training.png}
    \end{minipage} \quad
    \begin{minipage}{0.48\linewidth}
        \textbf{Correct the predictions of the FNO :}
        
        \pgfimage[width=\linewidth]{images/internship/FNO/Correction.png}
    \end{minipage}

    \vspace{15pt}
    \textbf{Some important points on the FNO :}
    \begin{enumerate}[\ding{217}]        
        \item widely used in PDE solving and constitute an active field of research
        \item FNO are Neural Operator networks : Unlike standard neural networks, which learn using inputs and outputs of fixed dimensions, neural operators \textbf{learn operators, which are mappings between spaces of functions}.
        \item \textbf{Mesh resolution independent :} can be evaluated at almost any data resolution without the need for retraining
    \end{enumerate}
\end{frame}

\begin{frame}{Correction on a FNO prediction - $\phi$-FEM}
    We consider an unknown solution on the circle with $f$ Gaussian \eqref{sol1}, $n_{vert}=63$, $n_{data}=1000$ (including validation sample) and $n_{test}=100$. 

    \vspace{10pt}
    \begin{minipage}{0.38\linewidth}
        Training on 4000 epochs (bs=64,lr=0.01) :
        \centering
        \pgfimage[width=0.9\linewidth]{images/internship/FNO/misfits_f_gaussienne.png}
    \end{minipage}
    \begin{minipage}{0.58\linewidth}
        Correction with the different methods :
        \centering
        \pgfimage[width=\linewidth]{images/internship/FNO/corr_boxplot.png}
    \end{minipage}	
    
    \vspace{5pt}
    \footnotesize
    \textbf{Remark :} We should try to reduce the resolution for correction, maybe we will gain in the time-to-error ratio.
\end{frame}

\subsection{Other results}

% \begin{frame}{Precision of the prediction - FEM}    
%     We consider the trigonometric solution on the circle \eqref{sol2} with

%     \begin{minipage}{0.4\linewidth}
%         $\; \textbullet$ $u_{ex}$ : the exact solution of \eqref{pb_initial}
        
%         \textcolor{white}{$\; \textbullet$ $u_\theta$ : a disturbed solution of }
        
%         $(S,f,\varphi)=(0.5,1,0)$
        
%         \textcolor{white}{$(S_p,f_p,\varphi_p)=(0.5,2,0)$}
        
%         \textbf{Exact solution :} $u_\theta=u_{ex}\in\mathbb{P}^{10}$

%         Correction with FEM ($n_{vert}=32$) :
        
%         \pgfimage[width=\linewidth]{images/internship/precision/exact_sol/exact_fem_circle_nvert32_fleche}
%     \end{minipage}
    
% \end{frame}

% \begin{frame}[noframenumbering]{Precision of the prediction - FEM}    
%     We consider the trigonometric solution on the circle \eqref{sol2} with

%     \begin{minipage}{0.4\linewidth}
%         $\; \textbullet$ $u_{ex}$ : the exact solution of \eqref{pb_initial}
        
%         \textcolor{white}{$\; \textbullet$ $u_\theta$ : a disturbed solution of }
        
%         $(S,f,\varphi)=(0.5,1,0)$
        
%         \textcolor{white}{$(S_p,f_p,\varphi_p)=(0.5,2,0)$}
        
%         \textbf{Exact solution :} $u_\theta=u_{ex}\in\mathbb{P}^{10}$

%         Correction with FEM ($n_{vert}=32$) :
        
%         \pgfimage[width=\linewidth]{images/internship/precision/exact_sol/exact_fem_circle_nvert32_fleche}

%         \textcolor{red}{Correction with FEM ($n_{vert}=100$) :}
        
%         \pgfimage[width=\linewidth]{images/internship/precision/exact_sol/exact_fem_circle_nvert100_colorbox}
%     \end{minipage}	
% \end{frame}

% \begin{frame}[noframenumbering]{Precision of the prediction - FEM}    
%     We consider the trigonometric solution on the circle \eqref{sol2} with

%     \begin{minipage}{0.4\linewidth}
%         $\; \textbullet$ $u_{ex}$ : the exact solution of \eqref{pb_initial}
        
%         \textcolor{red}{$\; \textbullet$ $u_\theta$ : a disturbed solution}
        
%         $(S,f,\varphi)=(0.5,1,0)$
        
%         \textcolor{red}{$(S_p,f_p,\varphi_p)=(0.5,2,0)$}
        
%         \textbf{Exact solution :} $u_\theta=u_{ex}\in\mathbb{P}^{10}$
        
%         Correction with FEM ($n_{vert}=32$) :

%         \centering
%         \pgfimage[width=0.9\linewidth]{images/internship/precision/exact_sol/exact_fem_circle_nvert32}

%         \raggedright
%         Correction with FEM ($n_{vert}=100$) :

%         \centering
%         \pgfimage[width=0.9\linewidth]{images/internship/precision/exact_sol/exact_fem_circle_nvert100}
%     \end{minipage} \quad
%     \fcolorbox{red}{white}{\begin{minipage}{0.54\linewidth}
%         \textbf{Disturbed solution :} $u_\theta=u_{ex}+\epsilon P\in\mathbb{P}^{k}$
        
%         with $\epsilon$ a real number and $P$ the perturbation.
        
%         Correction by adding with FEM ($n_{vert}=32$) :

%         \centering
%         \pgfimage[width=0.8\linewidth]{images/internship/precision/disturbed_sol/disturbed_fem_circle_nvert32_add}

%         \raggedright
%         Results for $k=10$ :
        
%         \pgfimage[width=\linewidth]{images/internship/precision/disturbed_sol/norms_P10_line}

%         with $\tilde{u}$ the solution obtained with the correction by addition, apply on $u_\theta$. 
%     \end{minipage}}

%     \footnotesize
%     \textit{Remark :} In practice, the expression of $u_{ex}$ will be chosen as the expression of $P$ (with the perturbation parameters $(S_p,f_p,\varphi_p)$). Consider $\varphi_p=0$ so that $P=0$ on $\Gamma$ and therefore $\tilde{\phi}=0$ on $\Gamma$.
% \end{frame}

\begin{frame}{Precision of the prediction - FEM}    
    We consider the trigonometric solution on the circle \eqref{sol2} with
    $$u_{ex}(x,y)=S\sin\left(8\pi f\left((x-0.5)^2+(y-0.5)^2\right)+\varphi\right)$$
    with $S=0.5$ and $\varphi=0$.

    \textbf{Exact solution :} Testing different correction methods for different frequencies.

    $$u_\theta=u_{ex}\in\mathbb{P}^{10} \; \rightarrow \; \tilde{u}\in\mathbb{P}^1$$

    \begin{center}     
        Correction with FEM ($n_{vert}=100$) :
        
        \pgfimage[width=0.6\linewidth]{images/internship/precision/exact_sol/exact_fem_circle_nvert100_colorbox}
    \end{center}
\end{frame}

\begin{frame}{Precision of the prediction - FEM}  
    We consider $(S,f,\varphi)=(0.5,1,0)$.

    \textbf{Disturbed solution :} Testing different $\epsilon$ and different degree $k$.
    $$u_\theta=u_{ex}+\epsilon P\in\mathbb{P}^{k} \; \rightarrow \; \tilde{u}\in\mathbb{P}^1$$
    with $\epsilon$ a real number and $P$ a perturbation.
    
    \begin{minipage}{0.58\linewidth}        
        Correction \eqref{corr_add} with FEM ($n_{vert}=32$) :

        \centering
        \pgfimage[width=\linewidth]{images/internship/precision/disturbed_sol/disturbed_fem_circle_nvert32_add}
    \end{minipage} \hfill
    \begin{minipage}{0.38\linewidth}        
        \vspace{-5pt}
        Results for $k=10$ :

        \centering
        \pgfimage[width=0.5\linewidth]{images/internship/precision/disturbed_sol/norms_P10}
    \end{minipage}

    \footnotesize
    \textit{Remark :} $P(x,y)=S_p\sin\left(8\pi f_p\left((x-0.5)^2+(y-0.5)^2\right)+\varphi_p\right)$ with $(S_p,f_p,\varphi_p)=(0.5,2,0)$
\end{frame}

\begin{frame}{Theoretical results - FEM}
    \textbf{Correction by multiplication on the elevated problem :} We consider 
        
    \begin{itemize}
        \item \textcolor{red}{$\hat{u_{ex}}=u_{ex}+m$} : the exact solution of \eqref{pb_reh}
        \item \textcolor{red}{$\hat{u_\theta} = u_\theta+m$} : a disturbed solution of \eqref{pb_reh}.
        \item $\tilde{u_h}=\hat{u_\theta}C_h$ : the approximate solution of \eqref{corr_mult_reh}
    \end{itemize}

    % We show that $\qquad \left|\left|\hat{u_{ex}}-\tilde{u_h}\right|\right|_0\le ch^{k+1}||\hat{u_\theta}||_\infty\left|C\right|_{k+1}$
        
    \begin{minipage}{0.48\linewidth}
        \setstretch{0.5}
        \textbf{1.} When m tends to infinity : 
            \begin{center}
                solution of \eqref{corr_mult_reh} $\rightarrow$ solution of \eqref{corr_add}
            \end{center}

            \small
            \textbf{Results :} $n_{vert}=32$, $\epsilon=0.001$ \\
            
            \centering
            \pgfimage[width=0.6\linewidth]{images/internship/theo_results/first/fig_fem_circle.png}
    \end{minipage} \;
    \begin{minipage}{0.48\linewidth}
        \setstretch{0.5}
        \textbf{2.} For $m$ sufficiently large : $C_{ex}=\hat{u_{ex}}/\hat{u_\theta}$
            \begin{equation*}
                \left|\left|C_{ex}-C_h\right|\right|_{0,\Omega}\le ch^{k+1}\epsilon\left|\left|P''\right|\right|_{0,\Omega}
            \end{equation*}

            \small
            \textbf{Results :} $n_{vert}=32$, $\epsilon=0.001$, $f_p=2$ \\
    
            \centering
            \pgfimage[width=0.6\linewidth]{images/internship/theo_results/second/fig_fem_circle.png}
    \end{minipage}
    
    
\end{frame}

    \section{PhD results}
	\begin{frame}{Explanation}
    \textbf{Context :} Need $u_\theta\in\mathbb{P}^k$ with $k$ of high degree

    \begin{center}
        \begin{minipage}{0.28\linewidth}
            \centering
            FNO \\
            (on a regular grid) 
        \end{minipage} $\rightarrow$ \begin{minipage}{0.35\linewidth}
            \centering
            NN which can predict \\
            solution at any point
        \end{minipage}
    \end{center}

    \textbf{Solutions :}

    \vspace{2pt}
    
    \begin{minipage}{0.48\linewidth}
        \; \\
        \textbf{1. MLP} - Multi-Layer Perceptron \\
        (= Fully connected)

        \centering
        \pgfimage[width=0.9\linewidth]{images/phd/MLP_schema.png}

        \raggedright
        \textit{Problem :} As the prediction is injected into an FEM solver, the accuracy of the derivatives is very important.
    \end{minipage} \quad
    \begin{minipage}{0.48\linewidth}
        \textbf{2. PINNs} - MLP with a physical loss
        \begin{center}
            $loss = mse(\Delta (\phi(x_i,y_i)w_{\theta,i})+f_i)$

            \vspace{1.5pt}
        
            \pgfimage[width=0.5\linewidth]{images/phd/PINNs_explanation.png}
        \end{center}

        with $(x_i,y_i)\in\mathcal{O}$.

        \textit{Remark :} We impose exact boundary conditions. \\
    \end{minipage}
\end{frame}

\begin{frame}{PINNs Training}
    We consider the solution on the circle defined in \\eqref{sol3} and defined by
    \begin{equation*}
        u_{ex}(x,y)=\phi(x,y)\sin(x)\exp(y)
    \end{equation*}
    We train a PINNs with 4 layers of 20 neurons over 10000 epochs (with $n_{pts}=2000$ points selected uniformly over $\mathcal{O}$).

    \centering
    \pgfimage[width=0.9\linewidth]{images/phd/solution_config0.png}

    {\fontencoding{U}\fontfamily{futs}\selectfont\char 49\relax} We consider a single problem ($f$ fixed) on a single geometry ($\phi$ fixed).

    \raggedright    
    $||u_{ex}-u_\theta||_{0,\Omega}^{(rel)}\approx 2.81e-3$
\end{frame}

\begin{frame}{Derivatives}
    \; \\
    
    \centering
    \pgfimage[width=0.75\linewidth]{images/phd/derivatives_x.png}
\end{frame}

\begin{frame}{Correction by addition}
    $u_\theta\in\mathbb{P}^{10} \; \rightarrow \; \tilde{u}\in\mathbb{P}^1$

    \begin{minipage}{0.5\linewidth}
        \centering
        \pgfimage[width=\linewidth]{images/phd/time_precision.png}
    
        \raggedright
        FEM / $\phi$-FEM : $n_{vert}\in\{8,16,32,64,128\}$
        
        Corr : $n_{vert}\in\{5,10,15,20,25,30\}$
    \end{minipage} \quad
    \begin{minipage}{0.46\linewidth}
        \centering
        \pgfimage[width=\linewidth]{images/phd/results_time_1e-4.png}

        \small\raggedright
        \begin{enumerate}[\textbullet]
        	\item \textbf{mesh} - FEM : construct the mesh \\
        	($\phi$-FEM : construct cell/facet sets) \\
        	\item \textbf{u\_PINNs} - get $u_\theta$ in $\mathbb{P}^{10}$ freedom degrees \\
        	\item \textbf{assemble} - assemble the FE matrix \\
        	\item \textbf{solve} - resolve the linear system
        \end{enumerate}
    \end{minipage}

    \small
    \textit{Remark :} The stabilisation parameter $\sigma$ of the $\phi$-FEM method has a major impact on the error obtained.
\end{frame}
	
	\section{Conclusion} %perspectives
	
	\begin{frame}[label={lastslide}]{Conclusion}
        \textbf{Observations :}
        
        \textbf{1.} Correction by addition seems to be the best choice \\
        (based on theoretical results obtained with FEM)

        \textbf{2.} We need a high degree prediction ($u_\theta\in\mathbb{P}^{10}$) \\
        $\Rightarrow$ no longer use FNO (needs NN defined at any point)

        \textbf{3.} We need to approximate the derivatives of the solution precisely \\
        $\Rightarrow$ no longer use simple MLP, replaced by a PINNs

        \vspace{15pt}

        \textbf{What's next ?}

        \textbf{1.} Consider multiple problems (varying $f$)

        \textbf{2.} Consider multiple and more complex geometry (varying $\phi$)

        \textbf{3.} Replace PINNs with a Neural Operator
	\end{frame}

	\section{Bibliography}
	
    {\setbeamertemplate{footline}{} 
    \begin{frame}{Bibliography}
		\small
        % \vspace{30pt}
        % \setstretch{0.2}
        % \AtNextBibliography{\small}
        \printbibliography[heading=none]
	\end{frame}
    }
    \addtocounter{framenumber}{-1} 

    \appendix

    \section{Mesh-based methods}

\begin{frame}{\appendixname~\theappendixframenumber~: Encoding - FEMs}\labelappendixframe{frame:encoding_fems}
	We want to project $f$ onto the vector subspace $V_N$ so that $f_\theta = p_{V_N}(f)$ \\	
	then $\forall i \in \{1,\dots,N\}$, we have
	\begin{align*}
		&\quad \langle f_\theta - f, \varphi_i \rangle = 0 \\
		\iff &\quad \langle f_\theta, \varphi_i \rangle = \langle f, \varphi_i \rangle \quad  \\
		\iff &\quad \sum_{j=1}^N(\theta_f)_j \langle \varphi_j, \varphi_i\rangle = \langle f, \varphi_i \rangle \\ 
		\iff &\quad M \theta_f = b(f) \\
		\iff &\quad \theta_f = M^{-1} b(f)
	\end{align*}	
	with 
	\begin{align*}
		M_{ij} &= \langle \varphi_i, \varphi_j\rangle = \int_{\Omega} \varphi_i(x) \varphi_j(x) \, dx \\
		b_i(f) &= \langle f, \varphi_i \rangle = \int_{\Omega} f(x) \varphi_i(x) \, dx
	\end{align*}	
\end{frame}
\addtocounter{appendixframenumber}{1}

\begin{frame}[allowframebreaks]{\appendixname~\theappendixframenumber~: Energetic form} \labelappendixframe{frame:minpb_galerkin}
	Let's compute the gradient of $J$ with respect to $v$ with
	\begin{equation*}
		J(v)=J_{in}(v)+J_{bc}(v)=\left(\frac{1}{2}\int_\Omega L(v)v - \int_\Omega fv\right) + \left(\frac{1}{2}\int_\Omega R_{bc}(v)^2\right)
	\end{equation*}

	\begin{itemize}[\textbullet]
		\item First, let's calculate the differential of $J_{in}$ with respect to $v$.
		\begin{align*}
			J_{in}(v+\epsilon h)=\frac{1}{2} \int_{\Omega} (A\nabla(v+\epsilon h)) \cdot \nabla(v+\epsilon h) + c(v+\epsilon h)^2 - \int_{\Omega} f(v+\epsilon h)
		\end{align*}
		
		By bilinearity of the scalar product and by symmetry of $A$, we finally obtain
		\begin{equation*}
			\mathcal{D}J_{in}(v)\cdot h = \lim_{\epsilon\rightarrow 0}\frac{J_{in}(v+\epsilon h)-J_{in}(v)}{\epsilon} = \int_{\Omega} (-\nabla\cdot(A\nabla v) + cv - f)h
		\end{equation*}
		
		And thus
		\begin{equation*}
			\nabla_v \; J_{in}(v) = L(v) - f = R_{in}(v)
		\end{equation*}
	
		\newpage
		
		\item In the same way, we can compute the differential of $J_{bc}$ with respect to $v$.
		\begin{align*}
			J_{bc}(v+\epsilon h)=\frac{1}{2} \int_{\Omega} v^2+2\epsilon vh +\epsilon^2 h^2 - 2vg - 2\epsilon hg+g^2
		\end{align*}
		
		Then
		\begin{align*}
			\mathcal{D}J_{bc}(v)\cdot h =  \lim_{\epsilon\rightarrow 0}\frac{J_{bc}(v+\epsilon h)-J_{bc}(v)}{\epsilon} = \int_{\Omega} v^2 - hg
		\end{align*}
		
		And thus
		\begin{align*}
			\nabla_v \; J_{bc}(v) = (v-g) = R_{bc}(v) 
		\end{align*}
	\end{itemize}
	
	Finally
	\begin{equation*}
		\nabla_v \; J(v) = \nabla_v \; J_{i}(v) + \nabla_v \; J_{bc}(v) = R(v)
	\end{equation*}
\end{frame}
\addtocounter{appendixframenumber}{1}

\begin{frame}{\appendixname~\theappendixframenumber~: Galerkin Projection}\labelappendixframe{frame:galerkin_proj}
	Let's compute the gradient of $J$ with respect to $\theta$ with
	\begin{equation*}
		J(\theta)=J_{in}(\theta)=\frac{1}{2}\int_\Omega L(u_\theta)v_\theta - \int_\Omega fv_\theta
	\end{equation*}
	First, we define
	\begin{equation*}
		v_\theta=\sum_{i=1}^{N} \theta_i \varphi_i=\theta\cdot\varphi \qquad \text{and} \qquad v_{\theta+\epsilon h}=(\theta+\epsilon h)\cdot\varphi=v_\theta+\epsilon v_h
	\end{equation*}
	Then since $A$ is symmetric
	\begin{equation*}
		\mathcal{D}J(\theta)\cdot h =\int_\Omega R(v_\theta)v_h =\sum_{i=1}^N h_i\int_\Omega R(v_\theta)\varphi_i
	\end{equation*}
	Finally
	\begin{align*}
		\nabla_\theta \; J(\theta) = \left(\int_\Omega R(v_\theta)\varphi_i\right)_{i=1,\dots,N}
	\end{align*}
\end{frame}
\addtocounter{appendixframenumber}{1}

\begin{frame}[allowframebreaks]{\appendixname~\theappendixframenumber~: Least-Square form}\labelappendixframe{frame:minpb_leastsquare}
	Let's compute the gradient of $J$ with respect to $v$ with
	\begin{equation*}
		J(v)=J_{in}(v)+J_{bc}(v)=\left(\frac{1}{2}\int_\Omega R_{in}(v)^2\right)=\left(\frac{1}{2}\int_\Omega R_{bc}(v)^2\right)
	\end{equation*}
	\begin{itemize}[\textbullet]
		\item First, let's calculate the differential of $J_{in}$ with respect to $v$.
		\begin{align*}
			\mathcal{D}J_{in}(v)\cdot h &= \langle \nabla\cdot(A\nabla h), \nabla\cdot(A\nabla v) - cv +f \rangle+\langle ch, -\nabla\cdot(A\nabla v) + cv - f \rangle \\ 
			&= -\langle \nabla\cdot(A\nabla h), R_{in}(v) \rangle+\langle ch, R_{in}(v)\rangle \\ 
			&= \langle -\nabla\cdot(A\nabla R_{in}(v))+cR_{in}(v), h \rangle \\
			&= \langle L(R_{in}(v)), h \rangle
		\end{align*}
		And thus		
		\begin{equation*}
			\nabla_v \; J_{in}(v) = L(R_{in}(v))
		\end{equation*}
	
		\newpage
		
		\item In the same way, we can compute the differential of $J_{bc}$ with respect to $v$.
		\begin{align*}
			J_{bc}(v+\epsilon h)=\frac{1}{2} \int_{\Omega} v^2+2\epsilon vh +\epsilon^2 h^2 - 2vg - 2\epsilon hg+g^2
		\end{align*}
		
		Then
		\begin{align*}
			\mathcal{D}J_{bc}(v)\cdot h =  \lim_{\epsilon\rightarrow 0}\frac{J_{bc}(v+\epsilon h)-J_{bc}(v)}{\epsilon} = \int_{\Omega} v^2 - hg
		\end{align*}
		
		And thus
		\begin{align*}
			\nabla_v \; J_{bc}(v) = (v-g) = R_{bc}(v) 
		\end{align*}
	\end{itemize}
	
	Finally
	\begin{equation*}
		\nabla_v \; J(v) = L(R(v))\mathds{1}_\Omega + (v-g)\mathds{1}_{\partial\Omega}
	\end{equation*}
	
\end{frame}
\addtocounter{appendixframenumber}{1}

\begin{frame}{\appendixname~\arabic{appendixframenumber}~: LS Galerkin Projection}\labelappendixframe{frame:leastsquare_proj}
	Let's compute the gradient of $J$ with respect to $\theta$ with
	\begin{equation*}
		J(\theta)=J_{in}(\theta)=\frac{1}{2}\int_\Omega (L(u_\theta) - f)^2
	\end{equation*}
	First, we define
	\begin{equation*}
		v_\theta=\sum_{i=1}^{N} \theta_i \varphi_i=\theta\cdot\varphi \qquad \text{and} \qquad v_{\theta+\epsilon h}=(\theta+\epsilon h)\cdot\varphi=v_\theta+\epsilon v_h
	\end{equation*}
	Then since $A$ is symmetric
	\begin{equation*}
		\mathcal{D}J(\theta)\cdot h = \int_\Omega L(R(v_\theta))v_h = \sum_{i=1}^N h_i\int_\Omega L(R(v_\theta))\varphi_i
	\end{equation*}
	Finally
	\begin{align*}
		\nabla_\theta J(\theta) = \left(\int_\Omega L(R(v_\theta))\varphi_i\right)_{i=1,\dots,N}
	\end{align*}
\end{frame}
\addtocounter{appendixframenumber}{1}

\section{Physically Informed Learning}

\begin{frame}{\appendixname~\theappendixframenumber~: ADAM Method}\labelappendixframe{frame:adam}
	\hl{A compléter !}
\end{frame}
\addtocounter{appendixframenumber}{1}
	
\end{document}
