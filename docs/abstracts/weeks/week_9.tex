\textbf{Réunions :}
\begin{enumerate}[label=\textbullet]
	\item \textit{Lundi} - Journée d'intégration Inria - Présentations :
	\begin{itemize}
		\item Victor Michel-Dansac, chercheur dans l'équipe Tonus "Sur la simulation des tsunamis et l'apprentissage"
		\item Cécile Pierrot, chargée de recherche dans l'équipe Caramba : "La lettre chiffrée de Charles Quint"
		\item Vincent Loechner, maître de conférence dans l'équipe Camus " MICROCARD: Numerical modeling of cardiac electrophysiology at the cellular scale"
		\item Christophe Vuillot, chargé de recherche dans l'équipe Mocqua : "Comment faire un ordinateur quantique"
		\item Valentina Scarponi, doctorante dans l'équipe Mimesis: "Insertion de catheter par réalité augmentée à l'aide de fibres optiques"
	\end{itemize}
	\item \textit{Mardi matin} - Réunion d'équipe - Présentation de Boyang
	\item \textit{Mercredi matin} - Réunion (Emmanuel + Michel)
	\item \textit{Mercredi après-midi} - Réunion (Stéphane + Michel + Nicola + Pablo + Killian)
	\item \textit{Jeudi après-midi} - Formation Fidle Seq 3
	\item \textit{Vendredi matin} - TP d'Informatique L2S3
\end{enumerate}
\textbf{Fait durant la semaine :}
\begin{enumerate}[label=\textbullet]
	\item rangement du code (correction et dérivées)
	\item lancement de modèle sur le carré
	\item lancement du modèle sur le cercle avec une nouvelle solution (plus gentille)
	\item considération d'une nouvelle géométrie - géométrie aléatoire (en utilisant le code partagé par Killian)
	\item début des diapos pour la présentation du 12/12/2023 (à la réunion d'équipe)
\end{enumerate}
\textbf{A faire :}
\begin{enumerate}[label=\textbullet]
	\item 
\end{enumerate}