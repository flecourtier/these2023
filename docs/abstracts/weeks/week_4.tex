\textbf{Réunions :}
\begin{enumerate}[label=\textbullet]
	\item \textit{Lundi après-midi} - Réunion (Michel + Vanessa)
	\item \textit{Mardi matin} - Réunion d'équipe - Présentation de Diwei
	\item \textit{Vendredi matin} - TP d'Informatique L2S3
\end{enumerate}
\textbf{Fait durant la semaine :}
\begin{enumerate}[label=\textbullet]
	\item Organisation de la partie Correction avec sauvegarde des images - script qui lance la correction à partir d’un modèle donnée
	\item Modification du excel avec ajout des résultats de correction...
	\item Rectification problème modèle avec variation du terme source f - re-lancement des entraînements
	\item Entraînement du modèle à prédire la solution $u=\phi w$ sur $\mathcal{O}$ -> Correction avec $\phi$-FEM
	\item Recalage de la levelset (avec méthode de Killian) -> Entrainement du modèle sur $u$ (il n'y a que dans ce cas que le sampling au bord est utilisé) -> Correction avec FEM
	\item \textbf{Suivi hebdomadaire} avec les résultats obtenus depuis le début
	\item Préparation d'un document pour la \textbf{réunion} de Lundi prochain avec les nouveaux résultats obtenus
\end{enumerate}

\textbf{A faire :}
\begin{enumerate}[label=\textbullet]
	\item tester méthode de Newton (proposé par Emmanuel par mail) pour recalage de la levelset ?
	\item lecture article 2301.05187 (WIRE)
\end{enumerate}