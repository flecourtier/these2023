\documentclass[french]{article}
\usepackage[T1]{fontenc}
\usepackage[utf8]{inputenc}
\usepackage[french]{babel}
\usepackage{amsmath}
\usepackage{mathtools}
\usepackage{color}
\usepackage[svgnames,dvipsnames]{xcolor} 
\usepackage{soul}
\usepackage{amssymb}
\usepackage{enumitem}
\usepackage{multicol}
\usepackage[left=2cm,right=2cm,top=2cm,bottom=2cm]{geometry}
\newcommand{\mathcolorbox}[2]{\colorbox{#1}{$\displaystyle #2$}}
\usepackage{pifont}
\usepackage{pst-all}
\usepackage{pstricks}
\usepackage{delarray}
\usepackage{setspace}
\usepackage{graphicx}
\usepackage{hyperref}
\usepackage{nicematrix}
\usepackage{listings}
\usepackage{float}

\hypersetup{
	colorlinks=true,
	linkcolor=blue,
	filecolor=magenta,      
	urlcolor=cyan,
	pdfpagemode=FullScreen,
}

\usepackage{amsthm}
\newtheorem*{Rem}{Remarque}

\newenvironment{conclusion}[1]{%
	\begin{center}\normalfont\textbf{Conclusion}\end{center}
	\begin{quotation} #1 \end{quotation}
}{%
	\vspace{1cm}
}

\setlength\parindent{0pt}

\begin{document}
	LECOURTIER Frédérique \hfill \today
	\begin{center}
		\Large\textbf{{Problème considéré}}\\
	\end{center}
	
	\textbf{Géométrie :} On considère $\Omega$ comme étant un cercle de rayon $r$ et de centre $(x_0,y_0)$. Pour simplifier, on va considérer que $\Omega$ est entièrement contenu dans un carré $\mathcal{O}$ (Figure \ref{geom_circle}). 
	\begin{figure}[H]
		\centering
		\includegraphics[width=0.3\linewidth]{"geom_circle.png"}
		\caption{Representation of the first domain considered : the Circle.}
		\label{geom_circle}
	\end{figure}
	\begin{Rem}
		On notera qu'un choix simple peut être de prendre le carré $[x_0-r-\epsilon,x_0+r+\epsilon]\times[y_0-r-\epsilon,y_0+r+\epsilon]$ où $\epsilon>0$ est un paramètre fixé dans l but que $\Omega$ soit entièrement compris dans $\mathcal{O}$.
	\end{Rem}
	
	\textbf{EDP :} On considère le problème de Poisson avec condition de Dirichlet homogène, définie par
	
	Trouver $u : \Omega \rightarrow \mathbb{R}^d (d=1,2,3)$ tel que
	\begin{equation}
		\left\{
		\begin{aligned}
			-\Delta u &= f, \; &&\text{dans } \; \Omega, \\
			u&=g, \; &&\text{sur } \; \partial\Omega,
		\end{aligned}
		\right. \tag{$\mathcal{P}$} \label{pb_initial}
	\end{equation}
	avec $\Delta$ l'opérateur de Laplace.

	\textbf{Problème :} On considère la solution analytique $u_{ex}$ à ce problème, définie par
	\begin{equation*}
		u_{ex}(x,y)=S\times\sin\left(\frac{1}{r^2}\pi((x-x_0)^2+(y-y_0)^2)\right)
	\end{equation*}

	\begin{Rem}
		On voit bien que sur le cercle, le problème est bien homogène.
	\end{Rem}

	Ce qui nous fournit le terme source $f$, définie par
	\begin{equation*}
		f(x,y)=\frac{4}{r^4}\pi^2S((x-x_0)^2+(y-y_0)^2)\sin\left(\frac{1}{r^2}\pi((x-x_0)^2+(y-y_0)^2)\right)-\frac{4}{r^2}\pi S \cos\left(\frac{1}{r^2}\pi((x-x_0)^2+(y-y_0)^2)\right)
	\end{equation*}
\end{document}